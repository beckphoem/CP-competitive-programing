\documentclass{beamer}
\usepackage[utf8]{inputenc}
\usepackage[T5]{fontenc} % Bắt buộc để hiển thị tiếng Việt
\usepackage[vietnamese]{babel}
\usepackage{tcolorbox}
\usepackage{listings}
\usepackage{xcolor}
\usepackage{booktabs}
\usetheme{Madrid}

\definecolor{codegreen}{rgb}{0,0.6,0}
\definecolor{codegray}{rgb}{0.5,0.5,0.5}
\definecolor{codepurple}{rgb}{0.58,0,0.82}
\definecolor{backcolour}{rgb}{0.95,0.95,0.92}

\lstdefinestyle{mystyle}{
    backgroundcolor=\color{backcolour},   
    commentstyle=\color{codegreen},
    keywordstyle=\color{magenta},
    numberstyle=\tiny\color{codegray},
    stringstyle=\color{codepurple},
    basicstyle=\ttfamily\scriptsize,
    breakatwhitespace=false,         
    breaklines=true,                 
    captionpos=b,                    
    keepspaces=true,                 
    numbers=left,                    
    numbersep=4pt,                  
    showspaces=false,                
    showstringspaces=false,
    showtabs=false,                  
    tabsize=2,
    escapechar=@
}

\lstset{style=mystyle}

\title[Đệ quy Codeforces]{Chiến lược rèn luyện tư duy đệ quy}
\subtitle{Hệ thống bài tập Codeforces cấp độ Newbie và Pupil}
\author{Slide Learning CPP}
\date{\today}

\begin{document}

\begin{frame}
  \titlepage
\end{frame}

\begin{frame}{Tổng quan về Đệ quy}
    \begin{itemize}
        \item Đệ quy là triết lý giải quyết vấn đề dựa trên việc chia nhỏ bài toán lớn thành các bài toán con đồng dạng.
        \item Quan trọng đối với cấp độ \textbf{Newbie (Rating 800-1100)}.
        \item Nền tảng cho: Quy hoạch động (DP), Quay lui (Backtracking), Đồ thị.
    \end{itemize}
    \begin{alertblock}{Cơ chế hoạt động}
        Dựa trên cấu trúc ngăn xếp (stack). Mỗi lời gọi hàm tạo ra một \texttt{stack\_frame} mới. Cần kiểm soát để tránh \textbf{stack overflow}.
    \end{alertblock}
\end{frame}

\begin{frame}{Phân loại năng lực Đệ quy}
    \begin{table}[]
        \centering
        \begin{tabular}{l|l}
            \textbf{Rating} & \textbf{Đặc điểm tư duy} \\ \hline
            800 - 900 & Đệ quy tuyến tính, mô phỏng vòng lặp \\
            900 - 1000 & Phân nhánh đơn giản, xử lý chuỗi \\
            1100 - 1400 & Quay lui cơ bản, duyệt cây \\
            1500+ & Kết hợp tối ưu hóa (DP, Chia để trị) \\
        \end{tabular}
    \end{table}
\end{frame}

% Bài 1
\begin{frame}{1. In ấn đệ quy (Print Recursion)}
    \textbf{Link}: \href{https://codeforces.com/group/MWSDmqGsZm/contest/223339/problem/A}{CF - 223339A}
    \begin{block}{Yêu cầu}
        Cho số nguyên $N$. In ra dòng chữ "I love Recursion" đúng $N$ lần bằng đệ quy.
    \end{block}
    \begin{exampleblock}{Chiến lược}
        \begin{itemize}
            \item \textbf{Base case}: Nếu $N = 0$, kết thúc.
            \item \textbf{Thực hiện}: In dòng chữ trước, sau đó gọi đệ quy $N-1$.
        \end{itemize}
    \end{exampleblock}
\end{frame}

% Bài 2
\begin{frame}{2. In số từ 1 đến N (Print from 1 to N)}
    \textbf{Link}: \href{https://codeforces.com/group/MWSDmqGsZm/contest/223339/problem/B}{CF - 223339B}
    \begin{block}{Yêu cầu}
        Cho số nguyên $N$, in các số từ 1 đến $N$, mỗi số trên một dòng.
    \end{block}
    \begin{exampleblock}{Chiến lược}
        \textbf{Đệ quy ngược}: Đặt lệnh in sau lời gọi đệ quy $N-1$. Quá trình giải phóng ngăn xếp (unwinding) sẽ in theo thứ tự từ 1 đến $N$.
    \end{exampleblock}
\end{frame}

% Bài 3
\begin{frame}{3. In số từ N về 1 (Print from N to 1)}
    \textbf{Link}: \href{https://codeforces.com/group/MWSDmqGsZm/contest/223339/problem/C}{CF - 223339C}
    \begin{block}{Yêu cầu}
        Cho số nguyên $N$, in các số từ $N$ về 1 trên cùng một dòng.
    \end{block}
    \begin{exampleblock}{Chiến lược}
        In giá trị $N$ hiện tại ngay lập tức, sau đó mới gọi đệ quy cho $N-1$.
    \end{exampleblock}
\end{frame}

% Bài 4
\begin{frame}{4. In các chữ số (Print Digits)}
    \textbf{Link}: \href{https://codeforces.com/group/MWSDmqGsZm/contest/223339/problem/D}{CF - 223339D}
    \begin{block}{Yêu cầu}
        Cho số nguyên không âm $N$, in ra các chữ số từ trái sang phải.
    \end{block}
    \begin{exampleblock}{Chiến lược}
        Gọi đệ quy với $N / 10$ để tiến tới chữ số đầu bên trái, sau đó in $N \% 10$ khi hàm quay lui.
    \end{exampleblock}
\end{frame}

% Bài 5
\begin{frame}{5. Chuyển đổi hệ cơ số (Base Conversion)}
    \textbf{Link}: \href{https://codeforces.com/group/MWSDmqGsZm/contest/223339/problem/E}{CF - 223339E}
    \begin{block}{Yêu cầu}
        In biểu diễn nhị phân của số nguyên $N$.
    \end{block}
    \begin{exampleblock}{Chiến lược}
        Gọi đệ quy $N / 2$ và in $N \% 2$ để tự động đảo ngược thứ tự các số dư về đúng dạng nhị phân.
    \end{exampleblock}
\end{frame}

% Bài 6
\begin{frame}{6. In mảng tại chỉ số chẵn (Print Even Indices)}
    \textbf{Link}: \href{https://codeforces.com/group/MWSDmqGsZm/contest/223339/problem/F}{CF - 223339F}
    \begin{block}{Yêu cầu}
        Cho mảng $A$, in các phần tử ở vị trí chỉ số chẵn theo thứ tự đảo ngược.
    \end{block}
    \begin{exampleblock}{Chiến lược}
        Sử dụng đệ quy tiến tới cuối mảng. Khi quay lui, kiểm tra nếu chỉ số là chẵn thì thực hiện in.
    \end{exampleblock}
\end{frame}

% Bài 7 & 8
\begin{frame}{7 \& 8. Vẽ kim tự tháp (Pyramid)}
    \textbf{Link}: CF - 223339G \& 223339H
    \begin{block}{Yêu cầu}
        Vẽ kim tự tháp thuận và ngược với độ cao $N$.
    \end{block}
    \begin{exampleblock}{Chiến lược}
        \begin{itemize}
            \item \textbf{Thuận}: In dòng hiện tại ($N - i$ khoảng trắng, $2i-1$ dấu sao) rồi gọi đệ quy.
            \item \textbf{Ngược}: Gọi đệ quy đến dòng sâu hơn trước khi thực hiện in dòng hiện tại.
        \end{itemize}
    \end{exampleblock}
\end{frame}

% Bài 9
\begin{frame}{9. Đếm nguyên âm (Count Vowels)}
    \textbf{Link}: \href{https://codeforces.com/group/MWSDmqGsZm/contest/223339/problem/I}{CF - 223339I}
    \begin{block}{Yêu cầu}
        Đếm số lượng nguyên âm (a, e, i, o, u) trong chuỗi.
    \end{block}
    \begin{exampleblock}{Chiến lược}
        Duyệt qua từng ký tự bằng chỉ số $i$. Nếu là nguyên âm trả về $1 + \text{gọi đệ quy}$, ngược lại trả về $0 + \text{gọi đệ quy}$.
    \end{exampleblock}
\end{frame}

% Bài 10
\begin{frame}{10. Tính giai thừa (Factorial)}
    \textbf{Link}: \href{https://codeforces.com/group/MWSDmqGsZm/contest/223339/problem/J}{CF - 223339J}
    \begin{block}{Yêu cầu}
        Tính giá trị $N!$.
    \end{block}
    \begin{exampleblock}{Chiến lược}
        Công thức truy hồi: $f(n) = n \times f(n-1)$. Lưu ý dùng kiểu dữ liệu \texttt{long long}.
    \end{exampleblock}
\end{frame}

% Bài 11 & 12
\begin{frame}{11 \& 12. Max và Sum của mảng}
    \textbf{Link}: CF - 223339K \& 223339L
    \begin{block}{Yêu cầu}
        Tìm phần tử lớn nhất và tính tổng các phần tử trong mảng.
    \end{block}
    \begin{exampleblock}{Chiến lược}
        \begin{itemize}
            \item \textbf{Max}: So sánh $A[i]$ với kết quả đệ quy của phần còn lại.
            \item \textbf{Sum}: Trả về $A[i] + \text{sum}(i+1)$.
        \end{itemize}
    \end{exampleblock}
\end{frame}

% Bài 15
\begin{frame}{15. Số Fibonacci (Fibonacci)}
    \textbf{Link}: \href{https://codeforces.com/group/MWSDmqGsZm/contest/223339/problem/O}{CF - 223339O}
    \begin{block}{Yêu cầu}
        Tìm số Fibonacci thứ $n$.
    \end{block}
    \begin{exampleblock}{Chiến lược}
        Sử dụng công thức $f(n) = f(n-1) + f(n-2)$. Có thể thảo luận về \textbf{Memoization} để tối ưu.
    \end{exampleblock}
\end{frame}

% Bài 18
\begin{frame}{18. Mảng đối xứng (Palindrome Array)}
    \textbf{Link}: \href{https://codeforces.com/group/MWSDmqGsZm/contest/223339/problem/R}{CF - 223339R}
    \begin{block}{Yêu cầu}
        Kiểm tra mảng có phải là Palindrome không.
    \end{block}
    \begin{exampleblock}{Chiến lược}
        So sánh cặp $A[left]$ và $A[right]$. Nếu khác trả về NO, nếu giống tiếp tục thu hẹp vào bên trong.
    \end{exampleblock}
\end{frame}

% Bài 20
\begin{frame}{20. Tổ hợp (Combination)}
    \textbf{Link}: \href{https://codeforces.com/group/MWSDmqGsZm/contest/223339/problem/T}{CF - 223339T}
    \begin{block}{Yêu cầu}
        Tính giá trị $C(n, r)$.
    \end{block}
    \begin{exampleblock}{Chiến lược}
        Dựa trên tính chất tam giác Pascal: $C(n, r) = C(n-1, r-1) + C(n-1, r)$.
    \end{exampleblock}
\end{frame}

\begin{frame}{Lời kết và Lộ trình}
    \begin{itemize}
        \item \textbf{Giai đoạn 1}: Mô phỏng vòng lặp (In ấn).
        \item \textbf{Giai đoạn 2}: Xử lý mảng/chuỗi & giá trị trả về.
        \item \textbf{Giai đoạn 3}: Đệ quy đa chiều & Chia để trị.
    \end{itemize}
    \begin{tcolorbox}[colback=green!5,colframe=green!40!black,title=Lời khuyên]
        Nắm vững đệ quy là chìa khóa để tiến tới Quy hoạch động và Đồ thị sau này.
    \end{tcolorbox}
\end{frame}

\end{document}