\documentclass{beamer}
\usepackage[utf8]{inputenc}
\usepackage[T5]{fontenc} % Bắt buộc để hiển thị tiếng Việt
\usepackage[vietnamese]{babel}
\usepackage{tcolorbox}
\usepackage{listings}
\usepackage{xcolor}
\usepackage{booktabs}
\usetheme{Madrid}

\definecolor{codegreen}{rgb}{0,0.6,0}
\definecolor{codegray}{rgb}{0.5,0.5,0.5}
\definecolor{codepurple}{rgb}{0.58,0,0.82}
\definecolor{backcolour}{rgb}{0.95,0.95,0.92}

\lstdefinestyle{mystyle}{
    backgroundcolor=\color{backcolour},   
    commentstyle=\color{codegreen},
    keywordstyle=\color{magenta},
    numberstyle=\tiny\color{codegray},
    stringstyle=\color{codepurple},
    basicstyle=\ttfamily\scriptsize,
    breakatwhitespace=false,         
    breaklines=true,                 
    captionpos=b,                    
    keepspaces=true,                 
    numbers=left,                    
    numbersep=4pt,                  
    showspaces=false,                
    showstringspaces=false,
    showtabs=false,                  
    tabsize=2,
    escapechar=@
}

\lstset{style=mystyle}

\title{F. Print Even Indices}
\subtitle{Huấn luyện viên tư duy thuật toán}
\author{Micro-Chunks Learning}
\date{\today}

\begin{document}

\begin{frame}
    \titlepage
\end{frame}

\begin{frame}{Bước 1: Tiếp nhận \& Phẫu thuật (Briefing)}
    \begin{block}{Tóm tắt bài toán}
        Bạn có một danh sách các số. Nhiệm vụ của bạn là:
        \begin{enumerate}
            \item Chỉ nhìn vào những vị trí (chỉ số) \textbf{chẵn} (0, 2, 4...).
            \item In những số đó theo \textbf{thứ tự ngược lại} (từ cuối lên đầu).
            \item \textbf{Điều kiện bắt buộc:} Phải dùng \textbf{Đệ quy (Recursion)}.
        \end{enumerate}
    \end{block}

    \begin{exampleblock}{Lộ trình tư duy (Chunks)}
        \begin{itemize}
            \item \textbf{Chunk 1:} Xác định vị trí chẵn.
            \item \textbf{Chunk 2:} Hiểu về Đệ quy - "Búp bê Nga".
            \item \textbf{Chunk 3:} Xử lý thứ tự ngược \& Điểm dừng.
            \item \textbf{Chunk 4:} Tổng kết thuật toán.
        \end{itemize}
    \end{exampleblock}
\end{frame}

\begin{frame}{Chunk 1: Xác định vị trí chẵn (Even Indices)}
    \begin{itemize}
        \item Mảng là một dãy các ngăn tủ đánh số từ $0$ đến $N-1$.
        \item Ví dụ với $N=4$, mảng: \texttt{1 4 2 7}
        \begin{itemize}
            \item Ngăn 0: số 1 (\textbf{Chẵn})
            \item Ngăn 1: số 4 (Lẻ)
            \item Ngăn 2: số 2 (\textbf{Chẵn})
            \item Ngăn 3: số 7 (Lẻ)
        \end{itemize}
    \end{itemize}

    \begin{alertblock}{Quy luật}
        Để nhảy từ vị trí chẵn này sang vị trí chẵn tiếp theo, chúng ta thực hiện bước nhảy: \texttt{index = index + 2}.
    \end{alertblock}

    \begin{exampleblock}{Thử thách tư duy}
        Nếu mảng có 7 phần tử ($0$ đến $6$), các chỉ số cần quan tâm là gì?
        \pause
        \textbf{Đáp án:} 0, 2, 4, và 6.
    \end{exampleblock}
\end{frame}

\begin{frame}{Chunk 2: Đệ quy – "Hành động lùi lại"}
    Đệ quy giống như mở búp bê Nga: Mở con lớn $\rightarrow$ gặp con nhỏ hơn $\rightarrow$ ... $\rightarrow$ \textbf{Base Case} (con nhỏ nhất).

    \begin{block}{Cơ chế hoạt động}
        \begin{enumerate}
            \item \textbf{Đi tiếp:} Gọi chính nó với chỉ số tiếp theo (\texttt{index + 2}).
            \item \textbf{Quay về \& Thực hiện:} In ra màn hình khi quay ngược trở lại.
        \end{enumerate}
    \end{block}

    \begin{exampleblock}{Thử thách: Đặt lệnh \texttt{cout << A[i]} ở đâu?}
        A. Đặt \textbf{trước} khi gọi hàm đệ quy tiếp theo.\\
        B. Đặt \textbf{sau} khi gọi hàm đệ quy tiếp theo.
        \pause
        \vfill
        \textbf{Đáp án B:} Để chờ các hàm sau chạy xong (chạm đến cuối mảng) rồi mới in ngược lại.
    \end{exampleblock}
\end{frame}

\begin{frame}{Chunk 3: Điểm dừng (Base Case)}
    \begin{alertblock}{Tại sao cần điểm dừng?}
        Để tránh lỗi \textbf{Stack Overflow} (Tràn bộ nhớ) do gọi hàm vô tận.
    \end{alertblock}

    \begin{exampleblock}{Thử thách tư duy}
        Điều kiện dừng nào an toàn nhất cho mảng có $N$ phần tử?
        \begin{itemize}
            \item A. Khi \texttt{index < N}.
            \item B. Khi \texttt{index >= N}.
            \item C. Khi \texttt{index == N}.
        \end{itemize}
        \pause
        \textbf{Đáp án B:} \texttt{index >= N} đảm bảo chúng ta dừng lại ngay khi vượt quá phạm vi mảng.
    \end{exampleblock}
\end{frame}

\begin{frame}{Chunk 4: Luồng hoạt động của thuật toán}
    Giả sử $N=3$, mảng có chỉ số $0, 2$:
    
    \begin{itemize}
        \item \texttt{printEven(0)} gọi \texttt{printEven(2)}
        \item \texttt{printEven(2)} gọi \texttt{printEven(4)}
        \item \texttt{printEven(4)} thỏa mãn \texttt{index >= 3} $\rightarrow$ \textbf{Dừng}.
        \item Quay về \texttt{printEven(2)}: In \texttt{A[2]}.
        \item Quay về \texttt{printEven(0)}: In \texttt{A[0]}.
    \end{itemize}

    \begin{exampleblock}{Câu hỏi khởi động}
        Trong hàm \texttt{main}, ta gọi hàm lần đầu với tham số nào?
        \pause
        \textbf{Đáp án:} \texttt{printEven(0, N, A)}.
    \end{exampleblock}
\end{frame}

\begin{frame}[fragile]{Cấu trúc mã nguồn C++}
\begin{lstlisting}[language=C++]
void printEven(int index, int N, long long A[]) {
    // 1. Diem dung (Base Case)
    if (index >= N) {
        return;
    }

    // 2. Loi goi de quy (Buoc nhay)
    // Di sau vao cac chi so tiep theo truoc
    printEven(index + 2, N, A);

    // 3. Hanh dong (In ra)
    // Khi ham quay nguoc tro lai, ta moi in gia tri
    cout << A[index] << " ";
}
\end{lstlisting}

\begin{block}{Lưu ý kỹ thuật}
    \begin{itemize}
        \item \textbf{Kiểu dữ liệu:} Dùng \texttt{long long} cho giá trị mảng.
        \item \textbf{Giới hạn:} $N$ lên tới $10^3$, mảng khai báo \texttt{A[1005]}.
    \end{itemize}
\end{block}
\end{frame}

\begin{frame}{Tổng kết \& Bước tiếp theo}
    \begin{exampleblock}{Bạn đã nắm vững:}
        \begin{itemize}
            \item Cách nhảy bước đôi trong mảng.
            \item Cơ chế Stack của đệ quy để đảo ngược dữ liệu.
            \item Cách xác định điều kiện dừng an toàn.
        \end{itemize}
    \end{exampleblock}

    \begin{block}{Bước tiếp theo}
        1. Thử tự viết code hoàn chỉnh dựa trên cấu trúc trên.\\
        2. Tìm hiểu cách truyền mảng/vector tối ưu vào hàm.
    \end{block}
\end{frame}

\end{document}