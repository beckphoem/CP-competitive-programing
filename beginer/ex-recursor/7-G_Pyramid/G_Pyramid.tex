\documentclass{beamer}
\usepackage[utf8]{inputenc}
\usepackage[T5]{fontenc} % Bắt buộc để hiển thị tiếng Việt
\usepackage[vietnamese]{babel}
\usepackage{tcolorbox}
\usepackage{listings}
\usepackage{xcolor}
\usepackage{booktabs}
\usetheme{Madrid}

\definecolor{codegreen}{rgb}{0,0.6,0}
\definecolor{codegray}{rgb}{0.5,0.5,0.5}
\definecolor{codepurple}{rgb}{0.58,0,0.82}
\definecolor{backcolour}{rgb}{0.95,0.95,0.92}

\lstdefinestyle{mystyle}{
    backgroundcolor=\color{backcolour},   
    commentstyle=\color{codegreen},
    keywordstyle=\color{magenta},
    numberstyle=\tiny\color{codegray},
    stringstyle=\color{codepurple},
    basicstyle=\ttfamily\scriptsize,
    breakatwhitespace=false,         
    breaklines=true,                 
    captionpos=b,                    
    keepspaces=true,                 
    numbers=left,                    
    numbersep=4pt,                  
    showspaces=false,                
    showstringspaces=false,
    showtabs=false,                  
    tabsize=2,
    escapechar=@
}

\lstset{style=mystyle}

\title{Huấn luyện Tư duy Thuật toán}
\subtitle{Bài toán G. Pyramid}
\author{Slide Learning C++}
\date{2026}

\begin{document}

\begin{frame}
    \titlepage
\end{frame}

\begin{frame}{Chào mừng bạn!}
    \begin{block}{Giới thiệu}
        Tôi đã sẵn sàng nhập vai \textbf{Huấn luyện viên Tư duy Thuật toán} của bạn. Chúng ta sẽ cùng nhau "mổ xẻ" bài toán \textbf{G. Pyramid}.
    \end{block}
    
    \begin{exampleblock}{Mục tiêu}
        Luyện tập tư duy \textbf{Đệ quy (Recursion)} và khả năng quan sát quy luật hình học.
    \end{exampleblock}
\end{frame}

% --- Bước 1 ---
\begin{frame}{Bước 1: Tiếp nhận \& Phẫu thuật (Briefing)}
    \textbf{Tóm tắt đề bài:} Vẽ hình kim tự tháp bằng các dấu sao \texttt{*} có độ cao đúng bằng $N$ dòng.
    
    \begin{alertblock}{Ràng buộc đặc biệt}
        Đề bài yêu cầu bắt buộc phải sử dụng \textbf{Đệ quy}.
    \end{alertblock}

    \textbf{Lộ trình tư duy:}
    \begin{enumerate}
        \item Giải mã cấu trúc mỗi dòng (Khoảng trống và Dấu sao).
        \item Tìm công thức liên hệ giữa số dòng và số lượng ký tự.
        \item Xây dựng cơ chế đệ quy (Điểm dừng và Bước nhảy).
    \end{enumerate}
\end{frame}

% --- Bước 2 ---
\begin{frame}[fragile]{Bước 2: Giải mã cấu trúc Kim tự tháp}
    Hãy nhìn kỹ vào ví dụ với $N = 3$:
\begin{verbatim}
  * (Dòng 1)
 *** (Dòng 2)
***** (Dòng 3)
\end{verbatim}

    \begin{block}{Ẩn dụ hóa}
        Hãy tưởng tượng mỗi dòng là một hàng gạch. Để hàng trên cùng nằm ở đỉnh, bạn phải đặt những "viên gạch tàng hình" (khoảng trống) ở phía trước.
    \end{block}
\end{frame}

\begin{frame}{Thử thách tư duy: Quy luật khoảng trống}
    Nhìn vào ví dụ $N = 3$:
    \begin{itemize}
        \item Dòng 1: Có bao nhiêu dấu cách?
        \item Dòng 2: Có bao nhiêu dấu cách?
        \item Dòng 3: Có bao nhiêu dấu cách?
    \end{itemize}
    
    \pause
    \begin{exampleblock}{Quy luật}
        Số lượng dấu cách sẽ \textbf{giảm dần} từ trên xuống dưới.
    \end{exampleblock}
\end{frame}

\begin{frame}{Bảng phân tích (Với $N = 3$)}
    \begin{table}[]
        \centering
        \begin{tabular}{@{}ccc@{}}
            \toprule
            Dòng ($i$) & Số dấu cách & Số dấu sao (\texttt{*}) \\ \midrule
            1          & 2           & 1                      \\
            2          & 1           & 3                      \\
            3          & 0           & 5                      \\ \bottomrule
        \end{tabular}
    \end{table}
    
    \textbf{Quan sát:}
    \begin{itemize}
        \item Dấu cách giảm dần ($2 \to 1 \to 0$).
        \item Dấu sao tăng dần theo số lẻ ($1 \to 3 \to 5$).
    \end{itemize}
\end{frame}

\begin{frame}{Thử thách: Khi $N = 4$}
    Nếu tổng độ cao $N = 4$:
    \begin{enumerate}
        \item Ở \textbf{Dòng 1}, cần bao nhiêu dấu cách?
        \item Ở \textbf{Dòng 4}, có bao nhiêu dấu sao (theo quy luật số lẻ)?
    \end{enumerate}
    
    \pause
    \begin{exampleblock}{Đáp án}
        \begin{itemize}
            \item Dòng 1: 3 dấu cách.
            \item Dòng 4: $2 \times 4 - 1 = 7$ dấu sao.
        \end{itemize}
    \end{exampleblock}
\end{frame}

\begin{frame}{Công thức tổng quát cho Khoảng trống}
    \begin{block}{Công thức}
        Nếu tổng độ cao là $N$, thì ở dòng thứ $i$, số \textbf{khoảng trống} là:
        $$\text{Spaces} = N - i$$
    \end{block}

    \textbf{Kiểm tra với $N = 3$:}
    \begin{itemize}
        \item Dòng $i=1$: $3 - 1 = 2$ (Đúng!)
        \item Dòng $i=3$: $3 - 3 = 0$ (Đúng!)
    \end{itemize}
\end{frame}

% --- Bước 3 ---
\begin{frame}{Bước 3: Công thức tính số dấu sao}
    Dãy số lẻ của chúng ta: $1, 3, 5, 7, \dots$
    
    \begin{table}[]
        \centering
        \begin{tabular}{@{}ccc@{}}
            \toprule
            Dòng ($i$) & Số dấu sao & Phân tích          \\ \midrule
            1          & 1          & $2 \times 1 - 1$   \\
            2          & 3          & $2 \times 2 - 1$   \\
            3          & 5          & $2 \times 3 - 1$   \\ \bottomrule
        \end{tabular}
    \end{table}

    \begin{exampleblock}{Kết luận}
        Công thức tính \textbf{số dấu sao} ở dòng thứ $i$ là:
        $$\text{Stars} = 2 \times i - 1$$
    \end{exampleblock}
\end{frame}

% --- Bước 4 ---
\begin{frame}{Bước 4: Xây dựng cơ chế Đệ quy}
    Chúng ta cần xác định:
    \begin{enumerate}
        \item \textbf{Điểm dừng (Base Case):} Khi nào hàm ngừng gọi chính nó?
        \item \textbf{Bước nhảy (Recursive Step):} Gọi lại chính nó với giá trị nào tiếp theo?
    \end{enumerate}

    \begin{block}{Hàm mẫu}
        \texttt{void solve(int current\_row, int total\_rows)}
    \end{block}
\end{frame}

\begin{frame}{Tư duy Đệ quy "Ngược"}
    Để vẽ kim tự tháp cao $N$ tầng:
    \begin{enumerate}
        \item "Vẽ giúp tôi kim tự tháp cao $N-1$ tầng trước."
        \item "Sau đó, tôi sẽ tự vẽ dòng cuối cùng (dòng thứ $N$)."
    \end{enumerate}

    \begin{alertblock}{Trường hợp cơ sở}
        Khi $N=1$, chỉ cần in ra 1 dấu sao và dừng lại.
    \end{alertblock}
\end{frame}

\begin{frame}{Thử thách cuối: Điều kiện dừng}
    Nếu dùng \textbf{Đệ quy Tiến} từ \texttt{i = 1} đến \texttt{n}:
    \begin{itemize}
        \item Công việc: In $N-i$ cách và $2i-1$ sao.
        \item Bước nhảy: \texttt{solve(i + 1, n)}.
    \end{itemize}

    \textbf{Câu hỏi:} Khi nào chúng ta nên \textbf{ngừng lại} để không gọi \texttt{solve(i + 1, n)} nữa?
    
    \pause
    \begin{exampleblock}{Đáp án}
        Khi \texttt{i > n}.
    \end{exampleblock}
\end{frame}

\end{document}