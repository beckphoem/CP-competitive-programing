\documentclass{beamer}
\usepackage[utf8]{inputenc}
\usepackage[T5]{fontenc} % Bắt buộc để hiển thị tiếng Việt
\usepackage[vietnamese]{babel}
\usepackage{tcolorbox}
\usepackage{listings}
\usepackage{xcolor}
\usepackage{booktabs}
\usetheme{Madrid}

\definecolor{codegreen}{rgb}{0,0.6,0}
\definecolor{codegray}{rgb}{0.5,0.5,0.5}
\definecolor{codepurple}{rgb}{0.58,0,0.82}
\definecolor{backcolour}{rgb}{0.95,0.95,0.92}

\lstdefinestyle{mystyle}{
    backgroundcolor=\color{backcolour},   
    commentstyle=\color{codegreen},
    keywordstyle=\color{magenta},
    numberstyle=\tiny\color{codegray},
    stringstyle=\color{codepurple},
    basicstyle=\ttfamily\scriptsize,
    breakatwhitespace=false,         
    breaklines=true,                 
    captionpos=b,                    
    keepspaces=true,                 
    numbers=left,                    
    numbersep=4pt,                  
    showspaces=false,                
    showstringspaces=false,
    showtabs=false,                  
    tabsize=2,
    escapechar=@
}

\lstset{style=mystyle}

\title{Huấn luyện tư duy thuật toán}
\subtitle{Bài toán: Print from 1 to N (Recursion)}
\author{Slide Learning C++}
\date{\today}

\begin{document}

\begin{frame}
    \titlepage
\end{frame}

\begin{frame}{Tiếp nhận \& Phẫu thuật (Briefing)}
    \begin{block}{Tóm tắt đề bài}
        Máy tính cho bạn một số nguyên dương $N$. Nhiệm vụ của bạn là in ra các số từ $1$ đến $N$, mỗi số nằm trên một dòng riêng biệt.
    \end{block}

    \begin{alertblock}{Yêu cầu bắt buộc}
        Phải sử dụng \textbf{Đệ quy (Recursion)} để giải quyết.
    \end{alertblock}

    \textbf{Lộ trình tư duy:}
    \begin{enumerate}
        \item \textbf{Chunk 1:} Hiểu về đệ quy qua hình ảnh thực tế.
        \item \textbf{Chunk 2:} Xác định điểm dừng (Base case).
        \item \textbf{Chunk 3:} Quy luật "Gửi thông điệp" và thứ tự thực hiện.
    \end{enumerate}
\end{frame}

\begin{frame}{Chunk 1: Đệ quy là gì? (Ẩn dụ "Búp bê Nga")}
    \begin{itemize}
        \item Tưởng tượng Đệ quy giống như bộ \textbf{Búp bê Nga (Matryoshka)}. 
        \item Khi mở một con búp bê lớn, bên trong lại có một con búp bê nhỏ hơn y hệt nó.
        \item Quá trình tiếp diễn cho đến khi chạm đến con búp bê nhỏ nhất.
    \end{itemize}

    \begin{exampleblock}{Định nghĩa trong lập trình}
        Đệ quy là một hàm \textbf{tự gọi lại chính nó} nhưng với một phiên bản bài toán nhỏ hơn (giá trị truyền vào giảm dần).
    \end{exampleblock}

    \textbf{Ví dụ:} \texttt{in\_so(5)} nhờ \texttt{in\_so(4)} làm giúp phần việc phía trước, sau đó nó mới làm phần của mình.
\end{frame}

\begin{frame}{Thử thách tư duy: Điểm dừng}
    \begin{block}{Câu hỏi}
        Nếu chúng ta cứ "nhờ" mãi như vậy (\texttt{in\_so(4)} nhờ \texttt{in\_so(3)},...), đến số mấy thì chúng ta phải dừng lại để tránh việc máy tính chạy mãi không nghỉ?
    \end{block}
    
    \pause
    \begin{exampleblock}{Giải đáp: Điểm dừng (Base case)}
        Điểm dừng quan trọng nhất chính là \textbf{số 1} (hoặc số 0 tùy cách cài đặt).
        \begin{itemize}
            \item Vì đề bài yêu cầu in từ $1$ đến $N$.
            \item Nếu tiếp tục lùi xuống các số âm, chúng ta sẽ làm sai yêu cầu đề bài.
        \end{itemize}
    \end{exampleblock}
\end{frame}

\begin{frame}{Chunk 2: Quy luật "Gửi thông điệp"}
    Có hai cách để chúng ta sắp xếp công việc trong đệ quy:
    
    \begin{columns}
        \begin{column}{0.5\textwidth}
            \begin{block}{1. Làm xong rồi mới gọi}
                In số hiện tại ra, sau đó mới nhờ "đệ tử" tiếp theo làm phần còn lại.
            \end{block}
        \end{column}
        \begin{column}{0.5\textwidth}
            \begin{block}{2. Gọi xong mới làm}
                Nhờ "đệ tử" làm hết mọi việc phía trước, khi nào xong mới quay lại in số của mình.
            \end{block}
        \end{column}
    \end{columns}

    \vspace{0.5cm}
    \textbf{Trắc nghiệm:} Để in theo thứ tự tăng dần ($1 \rightarrow N$), ta chọn cách nào?
    \begin{itemize}
        \item \textbf{A.} In số $N$ trước, sau đó mới gọi \texttt{print(N-1)}.
        \item \textbf{B.} Gọi \texttt{print(N-1)} trước, sau đó mới in số $N$.
    \end{itemize}
    \pause
    \textbf{Đáp án: B} - Vì ta muốn các số nhỏ hơn hiện ra trước.
\end{frame}

\begin{frame}{Phân tích luồng thực thi}
    Cấu trúc của hàm đệ quy:
    \begin{enumerate}
        \item \textbf{Kiểm tra điểm dừng:} Nếu $N < 1$, thoát ra.
        \item \textbf{Gọi đệ quy:} Nhờ hàm giải quyết bài toán với $N - 1$.
        \item \textbf{Thực hiện công việc:} In số $N$ ra màn hình.
    \end{enumerate}

    \begin{exampleblock}{Diễn biến khi $N=3$}
        \begin{itemize}
            \item \texttt{print(3)} gọi \texttt{print(2)}
            \item \texttt{print(2)} gọi \texttt{print(1)}
            \item \texttt{print(1)} gọi \texttt{print(0)} $\rightarrow$ \textbf{Dừng!}
            \item Quay lại \texttt{print(1)}: In ra \textbf{1}
            \item Quay lại \texttt{print(2)}: In ra \textbf{2}
            \item Quay lại \texttt{print(3)}: In ra \textbf{3}
        \end{itemize}
    \end{exampleblock}
\end{frame}

\begin{frame}{Cơ chế Ngăn xếp (Stack)}
    \begin{itemize}
        \item \textbf{Giai đoạn "Vào" (Calling phase):} Máy tính "tạm dừng" các hàm và xếp chúng vào một chiếc hộp (Stack).
        \item \textbf{Giai đoạn "Ra" (Returning phase):} Khi chạm điểm dừng, máy tính lấy các hàm ra theo thứ tự ngược lại (từ trên xuống).
    \end{itemize}

    \begin{alertblock}{Ghi nhớ}
        Lệnh in đặt \textbf{sau} lời gọi đệ quy sẽ được thực hiện trong giai đoạn "Ra", tạo nên thứ tự tăng dần.
    \end{alertblock}
\end{frame}

\begin{frame}[fragile]{Chunk 3: Chốt thuật toán}
    Dưới đây là cấu trúc logic (Mã giả) chúng ta đã xây dựng:

\begin{lstlisting}[language=C++, caption=Mã giả cho bài toán Print 1 to N]
Hàm Print_Numbers(N):
    // 1. Diem dung (Base case)
    Neu N == 0: 
        Ket thuc 
    
    // 2. Goi de quy truoc de in cac so nho hon
    Print_Numbers(N - 1)  
    
    // 3. Thuc hien in so hien tai (Giai doan "Ra")
    In N ra man hinh
\end{lstlisting}

    \textbf{Kết quả với $N=5$:}
    \texttt{1 2 3 4 5} (mỗi số một dòng).
\end{frame}

\begin{frame}{Thử thách mở rộng}
    \begin{block}{Câu hỏi mở rộng}
        Nếu bài toán yêu cầu in \textbf{ngược lại} từ $N$ về $1$ ($5, 4, 3, 2, 1$), bạn sẽ thay đổi vị trí của lệnh \textbf{In N} như thế nào?
    \end{block}

    \begin{itemize}
        \item \textbf{A.} Giữ nguyên (In $N$ sau khi gọi đệ quy).
        \item \textbf{B.} Đảo lên trên (In $N$ trước khi gọi đệ quy).
    \end{itemize}

    \pause
    \begin{exampleblock}{Giải đáp}
        \textbf{Đáp án B:} In trước khi gọi đệ quy sẽ thực hiện lệnh in ngay trong giai đoạn "Vào", giúp in từ số lớn đến số bé.
    \end{exampleblock}
\end{frame}

\end{document}