\documentclass{beamer}
\usepackage[utf8]{inputenc}
\usepackage[T5]{fontenc} % Bắt buộc để hiển thị tiếng Việt
\usepackage[vietnamese]{babel}
\usepackage{tcolorbox}
\usepackage{listings}
\usepackage{xcolor}
\usepackage{booktabs}
\usetheme{Madrid}

\definecolor{codegreen}{rgb}{0,0.6,0}
\definecolor{codegray}{rgb}{0.5,0.5,0.5}
\definecolor{codepurple}{rgb}{0.58,0,0.82}
\definecolor{backcolour}{rgb}{0.95,0.95,0.92}

\lstdefinestyle{mystyle}{
    backgroundcolor=\color{backcolour},   
    commentstyle=\color{codegreen},
    keywordstyle=\color{magenta},
    numberstyle=\tiny\color{codegray},
    stringstyle=\color{codepurple},
    basicstyle=\ttfamily\scriptsize,
    breakatwhitespace=false,         
    breaklines=true,                 
    captionpos=b,                    
    keepspaces=true,                 
    numbers=left,                    
    numbersep=4pt,                  
    showspaces=false,                
    showstringspaces=false,
    showtabs=false,                  
    tabsize=2,
    escapechar=@
}

\lstset{style=mystyle}

\title{Coach Tư duy Thuật toán}
\subtitle{Mổ xẻ Đệ quy: Bài toán Print from N to 1}
\author{Slide Learning CPP}
\date{\today}

\begin{document}

\begin{frame}
    \titlepage
\end{frame}

\begin{frame}{Phẫu thuật đề bài (Briefing)}
    \begin{block}{Cốt lõi bài toán}
        \begin{itemize}
            \item \textbf{Dữ liệu vào:} Một số nguyên $n$ (tối đa 1000).
            \item \textbf{Nhiệm vụ:} In dãy số giảm dần từ $n$ về 1.
            \item \textbf{Ràng buộc:} 
                \begin{itemize}
                    \item Phải dùng \textbf{Đệ quy}.
                    \item Các số cách nhau bởi 1 khoảng trắng.
                    \item \textbf{Không} có khoảng trắng ở cuối cùng.
                \end{itemize}
        \end{itemize}
    \end{block}
\end{frame}

\begin{frame}{Lộ trình tư duy (Micro-Chunks)}
    \begin{enumerate}
        \item \textbf{Chunk 1:} Hiểu về "Lời nguyền" của Đệ quy (Điểm dừng \& Bước lùi).
        \item \textbf{Chunk 2:} Xử lý logic in ấn (In trước hay gọi đệ quy trước?).
        \item \textbf{Chunk 3:} Giải quyết cái bẫy "Khoảng trắng thừa".
    \end{enumerate}
\end{frame}

\begin{frame}{Chunk 1: Điểm dừng (Base Case)}
    \begin{block}{Khái niệm}
        Tưởng tượng đệ quy như đi xuống cầu thang tối. Nếu không có sàn nhà (điểm dừng), bạn sẽ bị lỗi \texttt{Stack Overflow}.
    \end{block}

    \begin{alertblock}{Thử thách tư duy}
        Khi $n$ giảm xuống giá trị nào thì chúng ta nên \textbf{dừng hẳn}?
        \begin{itemize}
            \item A. Khi $n = 1$
            \item B. Khi $n = 0$
            \item C. Khi $n = -1$
        \end{itemize}
    \end{alertblock}
    
    \pause
    \textbf{Đáp án: B ($n=0$)}. Vì tại $n=1$ ta vẫn cần in số, khi xuống 0 mới thực sự hết việc.
\end{frame}

\begin{frame}{Chunk 2: Thứ tự thực hiện}
    \begin{block}{Hai lựa chọn logic}
        \begin{itemize}
            \item \textbf{Cách 1:} In số $n$ hiện tại, sau đó mới gọi đệ quy cho $n-1$.
            \item \textbf{Cách 2:} Gọi đệ quy cho đến hết rồi mới in $n$ trên đường quay về.
        \end{itemize}
    \end{block}

    \begin{exampleblock}{Thử thách tư duy}
        Để dãy số in ra theo đúng thứ tự $n \to 1$, bạn chọn cách nào?
    \end{exampleblock}

    \pause
    \textbf{Đáp án: Cách 1}. In ngay lập tức rồi mới "đi sâu" vào bước tiếp theo sẽ tạo ra dãy giảm dần.
\end{frame}

\begin{frame}{Chunk 3: Cái bẫy "Khoảng trắng thừa"}
    \begin{alertblock}{Vấn đề}
        Lệnh \texttt{cout << n << " ";} sẽ để lại khoảng trắng sau số 1.
    \end{alertblock}

    \begin{block}{Giải pháp (Lựa chọn A)}
        Sử dụng điều kiện để quyết định in dấu cách:
        \begin{itemize}
            \item Nếu $n > 1$: In số $n$ kèm một dấu cách.
            \item Nếu $n = 1$: Chỉ in đúng số 1.
        \end{itemize}
    \end{block}
\end{frame}

\begin{frame}[fragile]{Tổng kết Thuật toán (Pseudocode)}
    \begin{exampleblock}{Mã giả hàm \texttt{printNumbers(n)}}
        \begin{lstlisting}[language=C++]
void printNumbers(int n) {
    // 1. Diem dung
    if (n <= 0) return;

    // 2. Than ham & Xu ly khoang trang
    cout << n;
    if (n > 1) {
        cout << " ";
    }

    // 3. Goi de quy
    printNumbers(n - 1);
}
        \end{lstlisting}
    \end{exampleblock}
\end{frame}

\begin{frame}{Thử thách cuối cùng: Chạy tay (Dry Run)}
    \begin{block}{Với trường hợp $n = 2$}
        \begin{enumerate}
            \item Vào hàm với $n=2$. In \textbf{2}.
            \item Vì $2 > 1$, in thêm \textbf{dấu cách}.
            \item Gọi đệ quy với $n=1$. In \textbf{1}.
            \item Vì $1$ không lớn hơn $1$, \textbf{không} in dấu cách.
            \item Gọi đệ quy với $n=0$. Gặp điểm dừng $\to$ Thoát.
        \end{enumerate}
    \end{block}
    \centering
    \textbf{Kết quả: "2 1" - Thành công!}
\end{frame}

\begin{frame}
    \centering
    \Huge \textcolor{blue}{Bạn đã sẵn sàng tự mình "chốt hạ" bài này bằng code chưa? 😊}
\end{frame}

\end{document}