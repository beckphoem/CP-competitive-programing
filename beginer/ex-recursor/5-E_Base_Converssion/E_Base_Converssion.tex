\documentclass{beamer}
\usepackage[utf8]{inputenc}
\usepackage[T5]{fontenc} % Bắt buộc để hiển thị tiếng Việt
\usepackage[vietnamese]{babel}
\usepackage{tcolorbox}
\usepackage{listings}
\usepackage{xcolor}
\usepackage{booktabs}
\usetheme{Madrid}

\definecolor{codegreen}{rgb}{0,0.6,0}
\definecolor{codegray}{rgb}{0.5,0.5,0.5}
\definecolor{codepurple}{rgb}{0.58,0,0.82}
\definecolor{backcolour}{rgb}{0.95,0.95,0.92}

\lstdefinestyle{mystyle}{
    backgroundcolor=\color{backcolour},   
    commentstyle=\color{codegreen},
    keywordstyle=\color{magenta},
    numberstyle=\tiny\color{codegray},
    stringstyle=\color{codepurple},
    basicstyle=\ttfamily\scriptsize,
    breakatwhitespace=false,         
    breaklines=true,                 
    captionpos=b,                    
    keepspaces=true,                 
    numbers=left,                    
    numbersep=4pt,                  
    showspaces=false,                
    showstringspaces=false,
    showtabs=false,                  
    tabsize=2,
    escapechar=@
}

\lstset{style=mystyle}

\title{Huấn luyện viên tư duy thuật toán}
\subtitle{Bài toán: Base Conversion (Chuyển đổi cơ số)}
\author{Slide Learning C++}
\date{\today}

\begin{document}

\begin{frame}
    \titlepage
\end{frame}

\begin{frame}{Lời chào dẫn nhập}
    \begin{tcolorbox}[colback=blue!5,colframe=blue!75,title=Mục tiêu bài học]
        Chào mừng bạn! Chúng ta sẽ cùng nhau "mổ xẻ" bài toán chuyển đổi cơ số theo phong cách \textbf{Learning How to Learn}, tập trung vào bản chất logic và hình ảnh hóa thay vì chỉ nhìn vào code.
    \end{tcolorbox}
\end{frame}

\begin{frame}{1. Phẫu thuật đề bài (Briefing)}
    \begin{block}{Cốt lõi vấn đề}
        \begin{itemize}
            \item \textbf{Nhiệm vụ:} Nhập số thập phân (cơ số 10), in ra dạng Nhị phân (cơ số 2).
            \item \textbf{Điều kiện bắt buộc:} Sử dụng \textbf{Đệ quy (Recursion)}.
            \item \textbf{Dữ liệu:} Nhiều bộ thử nghiệm, $N$ có thể lên tới $10^{18}$.
        \end{itemize}
    \end{block}

    \begin{exampleblock}{Lộ trình tư duy}
        1. Hiểu bản chất "chia để trị".\\
        2. Xây dựng cấu trúc đệ quy (Điểm dừng \& Hành động).\\
        3. Xử lý thứ tự in ấn (Tránh bẫy in ngược).
    \end{exampleblock}
\end{frame}

\begin{frame}{2. Chunk 1: Cơ chế "Rút tỉa"}
    \begin{itemize}
        \item Hệ nhị phân chỉ hiểu hai trạng thái: \textbf{Dư lẻ (1)} hoặc \textbf{Chẵn đôi (0)}.
        \item \textbf{Quy tắc:}
        \begin{enumerate}
            \item Lấy số đó chia cho 2.
            \item Ghi lại số dư (0 hoặc 1).
            \item Lấy phần nguyên tiếp tục chia cho đến khi bằng 0.
        \end{enumerate}
    \end{itemize}

    \begin{alertblock}{Bẫy logic (The Trap)}
        Nếu vừa chia vừa in ngay, kết quả sẽ bị \textbf{ngược} (Ví dụ 6 ra \texttt{011} thay vì \texttt{110}).
    \end{alertblock}
\end{frame}

\begin{frame}{Thử thách tư duy (Mental Check)}
    \begin{block}{Câu hỏi}
        Nếu $N = 13$, hãy thực hiện các bước chia cho 2. Liệt kê các số dư thu được theo thứ tự từ bước đầu đến cuối?
    \end{block}
    \pause
    \begin{exampleblock}{Đáp án \& Giải thích}
        $13 \div 2 = 6$ dư \textbf{1} (đơn vị $2^0$)\\
        $6 \div 2 = 3$ dư \textbf{0} (hàng $2^1$)\\
        $3 \div 2 = 1$ dư \textbf{1} (hàng $2^2$)\\
        $1 \div 2 = 0$ dư \textbf{1} (hàng $2^3$)\\
        $\rightarrow$ Kết quả: \textbf{1101}. Chữ số cuối cùng tìm được lại có giá trị lớn nhất.
    \end{exampleblock}
\end{frame}

\begin{frame}{Mảnh ghép 2: Cơ chế Stack trong Đệ quy}
    \begin{itemize}
        \item Đệ quy sử dụng cơ chế \textbf{Stack (Ngăn xếp)}: Vào sau - Ra trước.
        \item \textbf{Bước gọi:} Chia cho 2 đến khi chạm đáy ($N=0$).
        \item \textbf{Bước in:} Chỉ thực hiện \textbf{sau khi} lệnh gọi đệ quy quay trở về.
    \end{itemize}
    
    \begin{exampleblock}{Mô phỏng đệ quy}
        1. \texttt{convert(13)} gọi \texttt{convert(6)}\\
        2. \texttt{convert(6)} gọi \texttt{convert(3)}\\
        3. \texttt{convert(3)} gọi \texttt{convert(1)}\\
        4. \texttt{convert(1)} gọi \texttt{convert(0)} $\rightarrow$ Chạm đáy!
    \end{exampleblock}
\end{frame}

\begin{frame}{Mảnh ghép 3: Thiết kế hàm đệ quy}
    \begin{block}{Thành phần quan trọng}
        \begin{enumerate}
            \item \textbf{Điểm dừng (Base Case):} Khi nào không chia nữa?
            \item \textbf{Bước đệ quy (Recursive Step):} Gọi lại chính nó với $N/2$.
        \end{enumerate}
    \end{block}

    \begin{alertblock}{Thử thách lựa chọn}
        Nên đặt lệnh in ở đâu để đúng thứ tự?
        \begin{itemize}
            \item \textbf{Phương án A:} In số dư $\rightarrow$ Gọi \texttt{convert(N/2)}
            \item \textbf{Phương án B:} Gọi \texttt{convert(N/2)} $\rightarrow$ In số dư
        \end{itemize}
    \end{alertblock}
    \pause
    \textbf{Đáp án:} Phương án B. Lệnh in sau lời gọi đệ quy sẽ chờ "bung" ngược từ đáy lên.
\end{frame}

\begin{frame}{Mô phỏng "Vụ nổ ngược"}
    \begin{table}[]
        \centering
        \begin{tabular}{@{}lll@{}}
            \toprule
            \textbf{Tầng} & \textbf{Trạng thái} & \textbf{Hành động tiếp theo} \\ \midrule
            Tầng 1 & \texttt{convert(13)} & Gọi \texttt{convert(6)}, chờ in $13\%2=1$ \\
            Tầng 2 & \texttt{convert(6)} & Gọi \texttt{convert(3)}, chờ in $6\%2=0$ \\
            Tầng 3 & \texttt{convert(3)} & Gọi \texttt{convert(1)}, chờ in $3\%2=1$ \\
            Tầng 4 & \texttt{convert(1)} & Gọi \texttt{convert(0)}, chờ in $1\%2=1$ \\
            Tầng 5 & \texttt{convert(0)} & \textbf{Chạm đáy!} Kết thúc gọi. \\ \bottomrule
        \end{tabular}
    \end{table}
    \pause
    Thứ tự in thực tế: $1 \rightarrow 1 \rightarrow 0 \rightarrow 1$.
\end{frame}

\begin{frame}{Mảnh ghép cuối cùng: Điểm dừng (Base Case)}
    \begin{block}{Điều kiện dừng}
        Khi nào chúng ta ngừng chia?
        \begin{itemize}
            \item A. Khi $N = 1$
            \item B. Khi $N = 0$
            \item C. Khi $N$ âm
        \end{itemize}
    \end{block}
    \pause
    \textbf{Đáp án: B.} Khi $N=0$, không còn gì để chia.
    
    \begin{tcolorbox}[colback=orange!5,colframe=orange!75]
        \textbf{Lưu ý:} Nếu đề cho $N=0$ ngay từ đầu, cần xử lý riêng để in ra số \texttt{0}.
    \end{tcolorbox}
\end{frame}

\begin{frame}[fragile]{Tổng kết Logic \& Mã giả}
    \begin{lstlisting}[language=C++, caption=Cấu trúc hàm đệ quy]
void convert(long long n) {
    // 1. Diem dung (Base Case)
    if (n == 0) return;

    // 2. Goi de quy (Di sau vao stack)
    convert(n / 2);

    // 3. In ket qua (Thuc hien khi quay lui)
    cout << n % 2;
}
    \end{lstlisting}
    
    \begin{exampleblock}{Kết quả với N = 3}
        1. \texttt{convert(1)} chạy trước $\rightarrow$ in \texttt{1}.\\
        2. \texttt{convert(3)} chạy sau $\rightarrow$ in \texttt{1}.\\
        Kết quả: \texttt{11}.
    \end{exampleblock}
\end{frame}

\begin{frame}{Bước cuối cùng: Thực hành}
    \begin{tcolorbox}[colback=green!5,colframe=green!75]
        Bạn đã nắm vững "trái tim" của đệ quy. Hãy thử hiện thực hóa nó bằng ngôn ngữ lập trình của bạn!
    \end{tcolorbox}
    
    \begin{itemize}
        \item Bạn muốn tôi hỗ trợ viết mã hoàn chỉnh cho bài này?
        \item Hay bạn muốn chuyển sang một thử thách đệ quy khó hơn?
    \end{itemize}
\end{frame}

\end{document}