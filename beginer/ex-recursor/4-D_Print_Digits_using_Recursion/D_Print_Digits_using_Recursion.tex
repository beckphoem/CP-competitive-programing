\documentclass{beamer}
\usepackage[utf8]{inputenc}
\usepackage[T5]{fontenc} 
\usepackage[vietnamese]{babel}
\usepackage{tcolorbox}
\usepackage{listings}
\usepackage{xcolor}
\usepackage{booktabs}
\usetheme{Madrid}

\definecolor{codegreen}{rgb}{0,0.6,0}
\definecolor{codegray}{rgb}{0.5,0.5,0.5}
\definecolor{codepurple}{rgb}{0.58,0,0.82}
\definecolor{backcolour}{rgb}{0.95,0.95,0.92}

\lstdefinestyle{mystyle}{
    backgroundcolor=\color{backcolour},   
    commentstyle=\color{codegreen},
    keywordstyle=\color{magenta},
    numberstyle=\tiny\color{codegray},
    stringstyle=\color{codepurple},
    basicstyle=\ttfamily\scriptsize,
    breakatwhitespace=false,         
    breaklines=true,                 
    captionpos=b,                    
    keepspaces=true,                 
    numbers=left,                    
    numbersep=4pt,                  
    showspaces=false,                
    showstringspaces=false,
    showtabs=false,                  
    tabsize=2,
    escapechar=@
}

\lstset{style=mystyle}

\title{D. Print Digits using Recursion}
\subtitle{Huấn luyện viên tư duy thuật toán}
\author{Learning How to Learn}
\date{\today}

\begin{document}

\begin{frame}
    \titlepage
\end{frame}

\begin{frame}{Bước 1: Phẫu thuật đề bài (Briefing)}
    \begin{block}{Nhiệm vụ cốt lõi}
        Cho một số tự nhiên $N$, hãy in ra từng chữ số của nó, cách nhau bởi một khoảng trắng.
    \end{block}
    
    \begin{itemize}
        \item \textbf{Ràng buộc đặc biệt:} Bắt buộc phải sử dụng \textbf{Đệ quy (Recursion)}.
        \item \textbf{Dữ liệu:} Có $T$ bộ test, mỗi số $N$ có thể lên tới $10^{18}$ (khoảng 10-19 chữ số).
    \end{itemize}

    \begin{block}{Lộ trình tư duy}
        \begin{enumerate}
            \item Cách lấy ra chữ số cuối cùng của một số.
            \item Hiểu về "Thứ tự" trong đệ quy (In từ trái sang phải).
            \item Điểm dừng (Base case) để tránh tràn bộ nhớ đệm.
        \end{enumerate}
    \end{block}
\end{frame}

\begin{frame}{Bước 2: Mảnh ghép 1 - Nghệ thuật "Tách đuôi"}
    Hãy tưởng tượng số $N$ giống như một đoàn tàu có nhiều toa:
    
    \begin{itemize}
        \item \textbf{Phép chia lấy dư (\% 10)}: Lấy toa cuối cùng.
        \begin{itemize}
            \item $39 \% 10 = 9$ (Toa cuối).
        \end{itemize}
        \item \textbf{Phép chia nguyên (/ 10)}: Vứt bỏ toa cuối, giữ lại phần còn lại.
        \begin{itemize}
            \item $39 / 10 = 3$ (Phần còn lại).
        \end{itemize}
    \end{itemize}

    \begin{alertblock}{Bẫy tư duy}
        Nếu chỉ dùng \texttt{\% 10} liên tục, bạn sẽ nhận được thứ tự \textbf{ngược} (phải sang trái). Ví dụ $39$ sẽ thành $9 \ 3$ trong khi đề bài yêu cầu $3 \ 9$.
    \end{alertblock}
\end{frame}

\begin{frame}{Thử thách tư duy}
    \begin{exampleblock}{Câu hỏi}
        Giả sử chúng ta có số $N = 121$. Nếu thực hiện phép toán: \texttt{N / 10} rồi lại lấy kết quả đó \texttt{(N / 10) \% 10}, ta sẽ thu được chữ số nào?
    \end{exampleblock}
    
    \pause
    \begin{block}{Giải đáp}
        \begin{itemize}
            \item $121 / 10 = 12$
            \item $12 \% 10 = \mathbf{2}$
            \item Đây là cách chúng ta truy cập dần vào các chữ số ở giữa.
        \end{itemize}
    \end{block}
\end{frame}

\begin{frame}{Điểm dừng và Thứ tự in}
    \begin{block}{Điều kiện dừng}
        Khi $N$ chỉ còn một chữ số (nghĩa là $N < 10$), ta chạm đến "đáy" của đệ quy. In chữ số đó ra và dừng lại.
    \end{block}

    \begin{itemize}
        \item \textbf{Trường hợp $N=0$}: Cần in ra chữ số $0$.
        \item \textbf{Khoảng trắng}: Cần xử lý để không bị dư dấu cách gây lỗi \textit{Presentation Error}.
    \end{itemize}

    \textbf{Luồng chạy của số 121:}
    \begin{enumerate}
        \item \texttt{printDigits(121)} gọi \texttt{printDigits(12)}.
        \item \texttt{printDigits(12)} gọi \texttt{printDigits(1)}.
        \item \texttt{printDigits(1)}: In \textbf{1}.
        \item Quay lại \texttt{printDigits(12)}: In \textbf{2}.
        \item Quay lại \texttt{printDigits(121)}: In \textbf{1}.
    \end{enumerate}
\end{frame}

\begin{frame}[fragile]{Bước 3: Tổng kết Thuật toán}
    \begin{block}{Hàm đệ quy \texttt{solve(n)}}
        \begin{itemize}
            \item \textbf{Bước dừng}: Nếu $n < 10$, in $n$ và kết thúc.
            \item \textbf{Bước đệ quy}:
            \begin{enumerate}
                \item Gọi \texttt{solve(n / 10)} để xử lý các số phía trước.
                \item Sau khi hàm con kết thúc, in một dấu cách.
                \item In chữ số cuối cùng: \texttt{n \% 10}.
            \end{enumerate}
        \end{itemize}
    \end{block}

    \begin{exampleblock}{Ví dụ với $n=39$}
        \begin{itemize}
            \item \texttt{solve(39)} gọi \texttt{solve(3)}.
            \item \texttt{solve(3)} in \textbf{3}.
            \item \texttt{solve(39)} tiếp tục: in \textbf{dấu cách}, sau đó in \textbf{9}.
            \item Kết quả: \texttt{3 9}
        \end{itemize}
    \end{exampleblock}
\end{frame}

\begin{frame}{Câu hỏi cuối cùng}
    \begin{block}{Kiểm tra kiến thức}
        Trong phần Input, $N$ có thể lên tới $10^{18}$. Trong C++, kiểu dữ liệu nào phù hợp nhất để tránh tràn số?
    \end{block}

    \begin{itemize}
        \item A. \texttt{int}
        \item B. \texttt{long long}
        \item C. \texttt{char}
    \end{itemize}

    \pause
    \begin{exampleblock}{Đáp án đúng: B}
        Kiểu \texttt{long long} có thể chứa giá trị lên tới khoảng $9 \times 10^{18}$, phù hợp với yêu cầu đề bài.
    \end{exampleblock}
\end{frame}

\end{document}