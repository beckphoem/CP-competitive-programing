\documentclass{beamer}
\usepackage[utf8]{inputenc}
\usepackage[T5]{fontenc} % Bắt buộc để hiển thị tiếng Việt
\usepackage[vietnamese]{babel}
\usepackage{tcolorbox}
\usepackage{listings}
\usepackage{xcolor}
\usepackage{booktabs}
\usetheme{Madrid}

\definecolor{codegreen}{rgb}{0,0.6,0}
\definecolor{codegray}{rgb}{0.5,0.5,0.5}
\definecolor{codepurple}{rgb}{0.58,0,0.82}
\definecolor{backcolour}{rgb}{0.95,0.95,0.92}

\lstdefinestyle{mystyle}{
    backgroundcolor=\color{backcolour},   
    commentstyle=\color{codegreen},
    keywordstyle=\color{magenta},
    numberstyle=\tiny\color{codegray},
    stringstyle=\color{codepurple},
    basicstyle=\ttfamily\scriptsize,
    breakatwhitespace=false,         
    breaklines=true,                 
    captionpos=b,                    
    keepspaces=true,                 
    numbers=left,                    
    numbersep=4pt,                  
    showspaces=false,                
    showstringspaces=false,
    showtabs=false,                  
    tabsize=2,
    escapechar=@
}

\lstset{style=mystyle}

\title{Phẫu thuật bài toán: Nearest Interesting Number}
\subtitle{Algorithmic Coach - Codeforces 1183/A}
\author{Slide Learning CPP}
\date{\today}

\begin{document}

\maketitle

\begin{frame}
\frametitle{Chào mừng bạn!}
\begin{block}{Triết lý học tập}
Chúng ta sẽ cùng chinh phục thử thách trên Codeforces theo phương pháp \textbf{"Learning How to Learn"}. Thay vì những dòng code khô khan, chúng ta sẽ "phẫu thuật" bài toán để tìm ra chiến thuật tối ưu.
\end{block}
\end{frame}

\begin{frame}
\frametitle{1. Phẫu thuật đề bài (Deconstruct)}
\begin{block}{Cốt lõi vấn đề}
\begin{itemize}
    \item \textbf{Khái niệm "Số thú vị":} Một số có tổng các chữ số chia hết cho 4.
    \item \textbf{Nhiệm vụ:} Cho số nguyên $n$, tìm số nguyên $a$ nhỏ nhất sao cho:
    \begin{enumerate}
        \item $a \geq n$
        \item $a$ là một "số thú vị".
    \end{enumerate}
\end{itemize}
\end{block}

\begin{exampleblock}{Lộ trình tư duy}
\begin{itemize}
    \item \textbf{Chunk 1:} Cách tính tổng các chữ số.
    \item \textbf{Chunk 2:} Chiến thuật tìm kiếm (Dùng công thức hay tiến lên?).
    \item \textbf{Chunk 3:} Xử lý bẫy và chốt thuật toán.
\end{itemize}
\end{exampleblock}
\end{frame}

\begin{frame}
\frametitle{2. Mảnh ghép 1: Chiếc máy quét tổng (Digit Sum)}
\begin{block}{Ẩn dụ: Túi tiền}
Hãy tưởng tượng số $n$ như một túi tiền có các tờ tiền mệnh giá khác nhau. Để tính tổng, bạn phải lấy từng tờ ra và cộng lại.
\end{block}

\begin{alertblock}{Bẫy logic}
Phân biệt giá trị số (ví dụ: 43) và tổng các chữ số ($4 + 3 = 7$). Vì 7 không chia hết cho 4, số 43 không "thú vị".
\end{alertblock}

\begin{exampleblock}{Thử thách tư duy}
Giả sử ta có số $n = 91$:
\begin{enumerate}
    \item Tổng các chữ số của $n$ là bao nhiêu?
    \item $n$ có phải là "số thú vị" không?
\end{enumerate}
\pause
\textbf{Đáp án:} Tổng là $9 + 1 = 10$. Không thú vị vì 10 không chia hết cho 4.
\end{exampleblock}
\end{frame}

\begin{frame}
\frametitle{3. Mảnh ghép 2: Chiến thuật "Bước từng bước"}
\begin{block}{Câu hỏi}
Cần công thức siêu việt để "nhảy" thẳng đến đáp án hay chỉ cần đi bộ?
\end{block}

\begin{exampleblock}{Ẩn dụ: Giao hàng trên phố}
Bạn đứng ở cửa nhà số $n$. Hãy tìm ngôi nhà "màu xanh" (thú vị) đầu tiên bằng cách kiểm tra $n, n+1, n+2 \dots$
\end{exampleblock}

\pause
\begin{block}{Thực hành với $n = 432$}
\begin{itemize}
    \item $432$: $4+3+2 = 9$ (Không)
    \item $433$: $4+3+3 = 10$ (Không)
    \pause
    \item $434$: $4+3+4 = 11$ (Không)
    \pause
    \item $435$: $4+3+5 = 12$ (Đúng! $12 \vdots 4$)
\end{itemize}
\textbf{Kết luận:} Khoảng cách giữa các số thú vị rất ngắn (thường $\leq 5$ bước).
\end{block}
\end{frame}

\begin{frame}[fragile]
\frametitle{4. Mảnh ghép 3: Tổng kết thuật toán (The Blueprint)}
\begin{block}{Chiến thuật: Vòng lặp "Thử và Sai"}
Vì khoảng cách nhỏ, vòng lặp đơn giản là tối ưu nhất.
\end{block}

\begin{lstlisting}[language=Python, caption=Mã giả tư duy]
Ham TinhTongChuSo(x):
    tong = 0
    Trong khi x > 0:
        tong = tong + (x chia lay du cho 10)
        x = x chia lay nguyen cho 10
    Tra ve tong

Nhap n
Trong khi TinhTongChuSo(n) khong chia het cho 4:
    n = n + 1

In ra n
\end{lstlisting}
\end{frame}

\begin{frame}
\frametitle{5. Thử thách cuối cùng (Final Boss)}
\begin{alertblock}{Trường hợp bẫy: $n = 99$}
Hãy cẩn thận khi bước từ 99 sang 100!
\end{alertblock}

\begin{itemize}
    \item $99 \rightarrow 9+9 = 18$ (Không)
    \pause
    \item $100 \rightarrow 1+0+0 = 1$ (Không)
    \pause
    \item $101 \rightarrow 1+0+1 = 2$ (Không)
    \pause
    \item $102 \rightarrow 1+0+2 = 3$ (Không)
    \pause
    \item $103 \rightarrow 1+0+3 = 4$ (\textbf{Thú vị!})
\end{itemize}

\begin{block}{Kết luận}
Dù số chữ số có thay đổi, logic "thử từng số" vẫn hoạt động hoàn hảo.
\end{block}
\end{frame}

\begin{frame}
\frametitle{Tổng kết}
\begin{itemize}
    \item \textbf{Kỹ thuật:} Sử dụng vòng lặp `while` kết hợp hàm tính tổng chữ số.
    \item \textbf{Lưu ý:} Khi viết hàm \texttt{TinhTongChuSo}, hãy dùng biến tạm để tránh làm mất giá trị gốc của \texttt{n}.
\end{itemize}

\begin{exampleblock}{Bước tiếp theo}
Bạn có muốn tôi hỗ trợ chuyển ý tưởng này sang C++ hay Python không, hay bạn muốn tự mình "múa phím" thử xem sao?
\end{exampleblock}
\end{frame}

\end{document}