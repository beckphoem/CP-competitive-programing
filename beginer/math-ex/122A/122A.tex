\documentclass{beamer}
\usepackage[utf8]{inputenc}
\usepackage[T5]{fontenc} % Bắt buộc để hiển thị tiếng Việt
\usepackage[vietnamese]{babel}
\usepackage{tcolorbox}
\usepackage{listings}
\usepackage{xcolor}
\usepackage{booktabs}
\usetheme{Madrid}

\definecolor{codegreen}{rgb}{0,0.6,0}
\definecolor{codegray}{rgb}{0.5,0.5,0.5}
\definecolor{codepurple}{rgb}{0.58,0,0.82}
\definecolor{backcolour}{rgb}{0.95,0.95,0.92}

\lstdefinestyle{mystyle}{
    backgroundcolor=\color{backcolour},   
    commentstyle=\color{codegreen},
    keywordstyle=\color{magenta},
    numberstyle=\tiny\color{codegray},
    stringstyle=\color{codepurple},
    basicstyle=\ttfamily\scriptsize,
    breakatwhitespace=false,         
    breaklines=true,                 
    captionpos=b,                    
    keepspaces=true,                 
    numbers=left,                    
    numbersep=4pt,                  
    showspaces=false,                
    showstringspaces=false,
    showtabs=false,                  
    tabsize=2,
    escapechar=@
}

\lstset{style=mystyle}

\title{Phân tích bài toán: 122A - Lucky Division}
\subtitle{Tư duy thuật toán từng bước}
\author{Codeforces Solving}
\date{\today}

\begin{document}

% Slide Tiêu đề
\begin{frame}
    \titlepage
\end{frame}

% Bước 1: Phẫu thuật đề bài
\begin{frame}{Bước 1: Phẫu thuật đề bài (Briefing)}
    \begin{block}{Các định nghĩa cốt lõi}
        \begin{enumerate}
            \item \textbf{Số may mắn (Lucky Number):} Là những số chỉ chứa các chữ số $4$ và $7$. Ví dụ: $47, 744, 4$.
            \item \textbf{Số suýt may mắn (Almost Lucky Number):} Một số $n$ được gọi là "suýt may mắn" nếu nó \textbf{chia hết} cho bất kỳ một "số may mắn" nào đó.
            \item \textbf{Nhiệm vụ:} Nhập $n$ ($1 \le n \le 1000$). Kiểm tra xem $n$ có phải là "số suýt may mắn" hay không.
        \end{enumerate}
    \end{block}

    \begin{exampleblock}{Lộ trình tư duy}
        \begin{itemize}
            \item \textbf{Chunk 1:} Nhận diện "Số may mắn" trong phạm vi cho phép.
            \item \textbf{Chunk 2:} Kiểm tra tính chia hết (Chiến thuật "Bắn hạ mục tiêu").
            \item \textbf{Chunk 3:} Tổng kết và chốt phương án.
        \end{itemize}
    \end{exampleblock}
\end{frame}

% Chunk 1: Nhận diện số may mắn
\begin{frame}{Chunk 1: Nhận diện "Số may mắn"}
    Vì đề bài giới hạn $n$ trong khoảng từ $1$ đến $1000$, chúng ta chỉ cần quan tâm các số may mắn trong khoảng này.
    
    \vspace{0.5cm}
    Liệt kê theo số lượng chữ số:
    \begin{itemize}
        \item \textbf{Có 1 chữ số:} $4, 7$
        \item \textbf{Có 2 chữ số:} $44, 47, 74, 77$
        \item \textbf{Có 3 chữ số:} $444, 447, 474, 477, 744, 747, 774, 777$
    \end{itemize}
\end{frame}

% Thử thách tư duy 1
\begin{frame}{Thử thách tư duy: Số nào KHÔNG phải "Lucky"?}
    Trong các số sau đây, số nào \textbf{KHÔNG PHẢI} là "số may mắn"?
    
    \begin{itemize}
        \item A. $47$
        \item B. $744$
        \item C. $467$
        \item D. $7$
    \end{itemize}

    \pause
    \vspace{1cm}
    \begin{alertblock}{Đáp án}
        \textbf{C. 467}. \\
        Giải thích: Con số $6$ đã "tố cáo" nó không phải là số may mắn (chỉ được chứa $4$ và $7$).
    \end{alertblock}
\end{frame}

% Chunk 2: Chiến thuật
\begin{frame}{Chunk 2: Chiến thuật "Bắn hạ mục tiêu"}
    Để kiểm tra $n$ có phải là "số suýt may mắn" hay không, ta xem $n$ có chia hết cho \textbf{bất kỳ} số may mắn nào trong danh sách dưới đây không:
    
    \vspace{0.5cm}
    \texttt{List = \{4, 7, 44, 47, 74, 77, 444, 447, 474, 477, 744, 747, 774, 777\}}
    
    \vspace{0.5cm}
    \begin{alertblock}{Cạm bẫy (Trap)}
        Chỉ cần chia hết cho \textbf{MỘT} số trong danh sách trên là đủ để kết luận \texttt{YES}. Không cần phải chia hết cho tất cả!
    \end{alertblock}
\end{frame}

% Thử thách tư duy 2
\begin{frame}{Thử thách tư duy: Kiểm tra số 47}
    Giả sử người dùng nhập vào số \textbf{47}.
    
    \begin{enumerate}
        \item 47 có chia hết cho số nào trong danh sách số may mắn không?
        \item Kết quả cuối cùng bài này nên in ra là \texttt{YES} hay \texttt{NO}?
    \end{enumerate}

    \pause
    \vspace{0.5cm}
    \begin{exampleblock}{Kết quả}
        \textbf{YES}. \\
        Giải thích: $47$ chia hết cho $47$ (chính nó là một số may mắn). Theo định nghĩa, $47$ là "số suýt may mắn".
    \end{exampleblock}
\end{frame}

% Chunk 3: Ví dụ số 79
\begin{frame}{Chunk 3: Kiểm tra số 79}
    Nếu nhập vào số \textbf{79}. Hãy nhìn lại danh sách:
    
    \texttt{4, 7, 44, 47, 74, 77, ...}
    
    \vspace{0.5cm}
    Theo bạn, với $n = 79$, máy tính sẽ in ra \texttt{YES} hay \texttt{NO}?
    
    \pause
    \vspace{0.5cm}
    \begin{alertblock}{Kết quả}
        \textbf{NO}. \\
        Giải thích: $79$ không chia hết cho $4$, không chia hết cho $7$, và cũng không chia hết cho bất kỳ số nào lớn hơn trong danh sách.
    \end{alertblock}
\end{frame}

% Mã giả (Pseudocode) - Slide quan trọng
\begin{frame}[fragile]{Mã giả (Tư duy lập trình)}
    Chiến thuật "Duyệt danh sách" được thể hiện như sau (Tiếng Việt không dấu):

\begin{lstlisting}[language=Python, caption=Mã giả giải thuật]
BUOC 1: Tao mot mang (danh sach) chua cac so: 
        4, 7, 44, 47, 74, 77, 444, 447, 474, 477, 744, 747...

BUOC 2: Nhap so nguyen n tu ban phim.

BUOC 3: Tao bien "kiem_tra" va dat la False (chua tim thay).

BUOC 4: Chay vong lap qua tung so "x" trong danh sach o Buoc 1:
        Neu n chia cho x du 0:
           Dat "kiem_tra" thanh True.
           Dung vong lap (break).

BUOC 5: Neu "kiem_tra" la True:
           In ra "YES".
        Nguoc lai:
           In ra "NO".
\end{lstlisting}
\end{frame}

% Câu hỏi cuối cùng
\begin{frame}{Câu hỏi chốt thuật toán}
    Trong Bước 4, tại sao chúng ta nên dùng lệnh "Dừng vòng lặp" (\texttt{break}) ngay khi tìm thấy một số mà $n$ chia hết?

    \begin{enumerate}
        \item Để tiết kiệm thời gian chạy của máy tính.
        \item Vì nếu không dừng, máy tính sẽ bị lỗi.
        \item Vì đề bài yêu cầu chỉ được chia hết cho đúng một số duy nhất.
    \end{enumerate}

    \pause
    \vspace{1cm}
    \begin{exampleblock}{Đáp án chính xác}
        \textbf{1. Để tiết kiệm thời gian chạy của máy tính.} \\
        Khi đã tìm thấy một ước số may mắn, điều kiện đề bài đã thỏa mãn. Việc kiểm tra tiếp là dư thừa.
    \end{exampleblock}
\end{frame}

\end{document}