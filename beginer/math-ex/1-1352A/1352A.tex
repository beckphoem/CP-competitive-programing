\documentclass{beamer}
\usepackage[utf8]{inputenc}
\usepackage[T5]{fontenc} % Bắt buộc để hiển thị tiếng Việt
\usepackage[vietnamese]{babel}
\usepackage{tcolorbox}
\usepackage{listings}
\usepackage{xcolor}
\usepackage{booktabs}
\usetheme{Madrid}

\definecolor{codegreen}{rgb}{0,0.6,0}
\definecolor{codegray}{rgb}{0.5,0.5,0.5}
\definecolor{codepurple}{rgb}{0.58,0,0.82}
\definecolor{backcolour}{rgb}{0.95,0.95,0.92}

\lstdefinestyle{mystyle}{
    backgroundcolor=\color{backcolour},   
    commentstyle=\color{codegreen},
    keywordstyle=\color{magenta},
    numberstyle=\tiny\color{codegray},
    stringstyle=\color{codepurple},
    basicstyle=\ttfamily\scriptsize,
    breakatwhitespace=false,         
    breaklines=true,                 
    captionpos=b,                    
    keepspaces=true,                 
    numbers=left,                    
    numbersep=4pt,                  
    showspaces=false,                
    showstringspaces=false,
    showtabs=false,                  
    tabsize=2,
    escapechar=@
}

\lstset{style=mystyle}

\title[Sum of Round Numbers]{Huấn luyện Tư duy Thuật toán}
\subtitle{Codeforces 1352A - Sum of Round Numbers}
\author{Slide Learning CPP}
\date{\today}

\begin{document}

\begin{frame}
    \titlepage
\end{frame}



\begin{frame}{Bước 1: Phẫu thuật đề bài (Deconstruct)}
    \begin{itemize}
        \item \textbf{Định nghĩa "Số Tròn" (Round Number):} Là số chỉ có đúng \textbf{một} chữ số khác 0.
        \begin{itemize}
            \item Ví dụ đúng: 5, 400, 70, 9000.
            \item Ví dụ sai: 110, 42, 1001 (vì có từ 2 chữ số khác 0 trở lên).
        \end{itemize}
        \item \textbf{Nhiệm vụ:} Cho số nguyên $n$, phân tích thành tổng của \textbf{ít nhất} các số tròn.
    \end{itemize}
    
    \begin{exampleblock}{Lộ trình tư duy}
        \begin{enumerate}
            \item Hiểu cách tách số thành hàng đơn vị, chục, trăm...
            \item Lọc ra các "nguyên liệu" là số tròn.
            \item Đếm và trình bày kết quả.
        \end{enumerate}
    \end{exampleblock}
\end{frame}

\begin{frame}{Chunk 1: Hiểu về "Giá trị vị trí" (Place Value)}
    Hãy tưởng tượng số $n$ giống như một \textbf{số tiền} bạn đang có trong ví:
    \begin{itemize}
        \item Ví dụ số $9876$:
        \item $9876 = 9000 + 800 + 70 + 6$
    \end{itemize}
    
    \begin{block}{Ẩn dụ}
        Mỗi vị trí đại diện cho một "tờ tiền" có mệnh giá khác nhau. Số 0 nghĩa là bạn không có tờ tiền ở mệnh giá đó.
    \end{block}

    \begin{alertblock}{Thử thách tư duy}
        Nếu có số $50302$, bạn sẽ tách thành những số tròn nào?
    \end{alertblock}
    \pause
    \textbf{Đáp án:} $50000, 300, 2$.
\end{frame}

\begin{frame}{Chunk 2: Chiến thuật "Quét sạch" (The Scanning Strategy)}
    Sử dụng một chiếc \textbf{máy quét} từ phải sang trái (từ hàng đơn vị lên):
    \begin{enumerate}
        \item \textbf{Lần quét 1:} Lấy chữ số cuối. Nhân với $1$. Nếu $>0$, bỏ vào "giỏ".
        \item \textbf{Lần quét 2:} Lấy chữ số tiếp theo. Nhân với $10$. Nếu $>0$, bỏ vào "giỏ".
        \item \textbf{Lần quét 3:} Lấy chữ số tiếp theo. Nhân với $100$. Nếu $>0$, bỏ vào "giỏ".
    \end{enumerate}

    \begin{alertblock}{Bẫy logic (The Trap)}
        Đừng cố biến số thành chuỗi (string). Hãy dùng toán học:
        \begin{itemize}
            \item \texttt{n \% 10}: Lấy chữ số cuối.
            \item \texttt{n / 10}: Bỏ chữ số cuối.
        \end{itemize}
    \end{alertblock}
\end{frame}

\begin{frame}{Thử thách quét số 703}
    Giả sử ta dùng toán học để "quét" số \textbf{703}:
    \begin{itemize}
        \item \textbf{Bước 1:} $703 \pmod{10} = 3$. Mệnh giá: $1$. Số tròn: $3 \times 1 = 3$.
        \item \textbf{Bước 2:} $n$ thành $70$. $70 \pmod{10} = 0$. Mệnh giá: $10$.
    \end{itemize}
    \pause
    \begin{exampleblock}{Kết quả Bước 2}
        Ở bước này ta gặp số $0$, nên \textbf{không} nhặt gì bỏ vào giỏ. Số $70$ sẽ biến thành $7$ và mệnh giá nâng lên $100$ để quét tiếp.
    \end{exampleblock}
\end{frame}

\begin{frame}{Chunk 3: Tổng kết thuật toán}
    \begin{enumerate}
        \item Nhập số lượng bộ test $t$.
        \item Với mỗi số $n$:
        \begin{itemize}
            \item Tạo một \textbf{danh sách} trống.
            \item Biến \texttt{he\_so} = 1.
            \item \textbf{Vòng lặp} khi $n > 0$:
            \begin{enumerate}
                \item \texttt{chu\_so = n \% 10}.
                \item Nếu \texttt{chu\_so > 0}: Thêm \texttt{(chu\_so * he\_so)} vào danh sách.
                \item \texttt{n = n / 10}.
                \item \texttt{he\_so = he\_so * 10}.
            \end{enumerate}
        \end{itemize}
        \item In ra số lượng phần tử và các số trong danh sách.
    \end{enumerate}
\end{frame}

\begin{frame}{🏆 Thử thách cuối cùng (The Final Boss)}
    Vận hành thuật toán với số \textbf{4002}:
    \begin{itemize}
        \item Trong "giỏ" của bạn sẽ có bao nhiêu số tròn?
        \item Đó là những số nào?
    \end{itemize}
    \pause
    \begin{exampleblock}{Đáp án}
        Có \textbf{2} số tròn: \textbf{2} (hàng đơn vị) và \textbf{4000} (hàng nghìn).
    \end{exampleblock}
\end{frame}

\begin{frame}[fragile]{Từ Tư duy sang Ngôn ngữ (Pseudocode)}
\begin{lstlisting}[language=Python, caption=Mã giả Tiếng Việt]
Doc vao so luong bo test (t)
Lap lai t lan:
    Doc vao so n
    Tao mot mang de chua cac so tron
    Bien he_so = 1
    
    Trong khi n > 0:
        chu_so = n % 10
        Neu chu_so != 0:
            Luu (chu_so * he_so) vao mang
        n = n / 10
        he_so = he_so * 10
    
    In ra kich thuoc cua mang
    In ra cac phan tu trong mang
\end{lstlisting}
\end{frame}

\begin{frame}{Lời kết và Gợi ý kỹ thuật}
    \begin{itemize}
        \item \textbf{Trong C++:} Dùng \texttt{std::vector<int>} để lưu kết quả vì số lượng số tròn không cố định.
        \item \textbf{Trong Python:} Sử dụng một \texttt{list} đơn giản.
    \end{itemize}
    
    \begin{block}{Bước tiếp theo}
        Bạn có muốn thử tự viết code cho bài này không? Hay chúng ta sẽ sang bài tiếp theo: \textbf{"Way Too Long Words (71A)"}?
    \end{block}
\end{frame}

\end{document}