\documentclass{beamer}
\usepackage[utf8]{inputenc}
\usepackage[T5]{fontenc} % Bắt buộc để hiển thị tiếng Việt
\usepackage[vietnamese]{babel}
\usepackage{tcolorbox}
\usepackage{listings}
\usepackage{xcolor}
\usepackage{booktabs}
\usetheme{Madrid}

\definecolor{codegreen}{rgb}{0,0.6,0}
\definecolor{codegray}{rgb}{0.5,0.5,0.5}
\definecolor{codepurple}{rgb}{0.58,0,0.82}
\definecolor{backcolour}{rgb}{0.95,0.95,0.92}

\lstdefinestyle{mystyle}{
    backgroundcolor=\color{backcolour},   
    commentstyle=\color{codegreen},
    keywordstyle=\color{magenta},
    numberstyle=\tiny\color{codegray},
    stringstyle=\color{codepurple},
    basicstyle=\ttfamily\scriptsize,
    breakatwhitespace=false,         
    breaklines=true,                 
    captionpos=b,                    
    keepspaces=true,                 
    numbers=left,                    
    numbersep=4pt,                  
    showspaces=false,                
    showstringspaces=false,
    showtabs=false,                  
    tabsize=2,
    escapechar=@
}

\lstset{style=mystyle}

\title[Codeforces 1766A]{Huấn luyện viên Tư duy Thuật toán}
\subtitle{Phân tích bài toán Extremely Round (1766A)}
\author{Slide Learning C++}
\date{\today}

\begin{document}

\begin{frame}
  \titlepage
\end{frame}

\begin{frame}{Giới thiệu}
  \begin{itemize}
    \item Chào mừng bạn đến với chương trình huấn luyện tư duy thuật toán.
    \item Chúng ta sẽ "mổ xẻ" bài toán theo triết lý \textbf{Micro-chunking} để nắm chắc gốc rễ vấn đề.
  \end{itemize}
  \begin{block}{Mục tiêu}
    Giải quyết bài toán đếm số "cực tròn" một cách tối ưu.
  \end{block}
\end{frame}

\begin{frame}{Bước 1: Phẫu thuật đề bài (Deconstruct)}
  \begin{block}{Định nghĩa: Số "Extremely Round"}
    Là số nguyên dương mà trong các chữ số của nó, \textbf{chỉ có duy nhất một chữ số khác 0}.
  \end{block}
  
  \begin{exampleblock}{Ví dụ}
    \begin{itemize}
      \item \textbf{Đúng:} 5, 400, 9000.
      \item \textbf{Sai:} 11, 405, 9100 (có từ 2 chữ số khác 0 trở lên).
    \end{itemize}
  \end{exampleblock}

  \begin{alertblock}{Nhiệm vụ}
    Cho số nguyên $n$, đếm số lượng số "cực tròn" từ 1 đến $n$ ($n \le 999,999$).
  \end{alertblock}
\end{frame}

\begin{frame}{Lộ trình tư duy}
  Chúng ta sẽ đi qua 3 mảnh kiến thức (Chunks):
  \vfill
  \begin{enumerate}
    \item \textbf{Chunk 1:} Nhận diện quy luật của các số "cực tròn".
    \item \textbf{Chunk 2:} Cách đếm thông minh không cần duyệt từng số.
    \item \textbf{Chunk 3:} Xử lý các bẫy logic và rút ra công thức.
  \end{enumerate}
\end{frame}

\begin{frame}{Chunk 1: Truy tìm quy luật "Cực tròn"}
  Hãy tưởng tượng mỗi hàng (đơn vị, chục, trăm...) là một "ngăn tủ":
  \begin{itemize}
    \item \textbf{Hàng 1 chữ số:} 1, 2, ..., 9 (9 số).
    \item \textbf{Hàng 2 chữ số:} 10, 20, ..., 90 (9 số).
    \item \textbf{Hàng 3 chữ số:} 100, 200, ..., 900 (9 số).
  \end{itemize}
  
  \begin{exampleblock}{Thử thách tư duy}
    Nếu $n = 35$, có bao nhiêu số cực tròn nhỏ hơn hoặc bằng 35?
    \pause
    \begin{itemize}
      \item Hàng đơn vị: 1, 2, 3, 4, 5, 6, 7, 8, 9 (9 số)
      \item Hàng chục: 10, 20, 30 (3 số)
      \item \textbf{Tổng cộng: 12 số.}
    \end{itemize}
  \end{exampleblock}
\end{frame}

\begin{frame}{Chunk 2: Tổng quát hóa quy luật}
  Với mỗi "bậc" (chữ số), ta luôn có tối đa \textbf{9 số} cực tròn.
  Để đếm đến $n$, ta chia làm 2 phần:
  \begin{enumerate}
    \item \textbf{Phần nguyên:} Những "bậc" đã đi qua hết hoàn toàn.
    \item \textbf{Phần dư:} Những số cực tròn ở bậc hiện tại của $n$.
  \end{enumerate}

  \begin{exampleblock}{Ví dụ với $n = 4567$}
    \begin{itemize}
      \item Bậc đơn vị, chục, trăm: $3 \times 9 = 27$ số.
      \item Bậc nghìn (hiện tại): 1000, 2000, 3000, 4000 (4 số).
      \item \textbf{Tổng cộng:} $27 + 4 = 31$ số.
    \end{itemize}
  \end{exampleblock}
\end{frame}

\begin{frame}{Chunk 3: Rút ra công thức}
  Gọi $s$ là số chữ số của $n$, và $d$ là chữ số đầu tiên của $n$.
  
  \begin{block}{Mối liên hệ}
    \begin{itemize}
      \item Số bậc đầy đủ là $(s - 1)$.
      \item Mỗi bậc đầy đủ có 9 số cực tròn.
      \item Bậc cuối cùng có đúng $d$ số.
    \end{itemize}
  \end{block}

  \begin{tcolorbox}[colback=blue!5,colframe=blue!75,title=Công thức tổng quát]
    $$\text{Result} = (s - 1) \times 9 + d$$
  \end{tcolorbox}
\end{frame}

\begin{frame}{Thử nghiệm giới hạn}
  \begin{exampleblock}{Kiểm tra với $n = 999,999$}
    \begin{itemize}
      \item Số chữ số $s = 6$.
      \item Chữ số đầu tiên $d = 9$.
      \item Áp dụng: $(6 - 1) \times 9 + 9 = 5 \times 9 + 9 = 54$.
    \end{itemize}
    \textbf{Giải thích:} Có 6 bậc (từ đơn vị đến trăm nghìn), mỗi bậc đóng góp 9 số.
  \end{exampleblock}
\end{frame}

\begin{frame}[fragile]{Bước 3: Hiện thực hóa thành thuật toán}
  \begin{block}{Cách lấy $d$ và $s$ bằng toán học}
    Sử dụng biến tạm \texttt{go\_n} để không làm mất giá trị gốc của $n$.
  \end{block}

\begin{lstlisting}[language=Python, caption=Mã giả logic xử lý]
Nhap t (so bo test)
Lap lai t lan:
    Nhap n
    go_n = n
    s = do_dai_chuoi(n) # Hoac dem trong khi chia 10
    
    # Tim chu so dau tien d
    Trong khi go_n >= 10:
        go_n = go_n // 10
    d = go_n
    
    In ra (s - 1) * 9 + d
\end{lstlisting}
\end{frame}

\begin{frame}{Thử thách thực hành}
  \begin{alertblock}{Câu hỏi suy luận}
    Nếu $n = 7$ (số có 1 chữ số):
    \begin{itemize}
      \item Vòng lặp \texttt{while (go\_n >= 10)} có chạy không?
      \item Giá trị $d$ cuối cùng là bao nhiêu?
      \item Công thức $(s-1) \times 9 + d$ có trả về đúng 7 không?
    \end{itemize}
  \end{alertblock}
  \pause
  \begin{exampleblock}{Đáp án}
    \begin{itemize}
      \item Vòng lặp không chạy. $d = 7$.
      \item $s = 1 \Rightarrow (1-1) \times 9 + 7 = 7$.
      \item \textbf{Kết luận:} Thuật toán chạy đúng cho cả số có 1 chữ số!
    \end{itemize}
  \end{exampleblock}
\end{frame}

\end{document}