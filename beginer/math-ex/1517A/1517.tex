\documentclass{beamer}
\usepackage[utf8]{inputenc}
\usepackage[T5]{fontenc} % Bắt buộc để hiển thị tiếng Việt
\usepackage[vietnamese]{babel}
\usepackage{tcolorbox}
\usepackage{listings}
\usepackage{xcolor}
\usepackage{booktabs}
\usetheme{Madrid}

\definecolor{codegreen}{rgb}{0,0.6,0}
\definecolor{codegray}{rgb}{0.5,0.5,0.5}
\definecolor{codepurple}{rgb}{0.58,0,0.82}
\definecolor{backcolour}{rgb}{0.95,0.95,0.92}

\lstdefinestyle{mystyle}{
    backgroundcolor=\color{backcolour},   
    commentstyle=\color{codegreen},
    keywordstyle=\color{magenta},
    numberstyle=\tiny\color{codegray},
    stringstyle=\color{codepurple},
    basicstyle=\ttfamily\scriptsize,
    breakatwhitespace=false,         
    breaklines=true,                 
    captionpos=b,                    
    keepspaces=true,                 
    numbers=left,                    
    numbersep=4pt,                  
    showspaces=false,                
    showstringspaces=false,
    showtabs=false,                  
    tabsize=2,
    escapechar=@
}

\lstset{style=mystyle}

\title{Codeforces 1517A - Sum of 2050}
\author{Slide Learning C++}
\date{\today}

\begin{document}

\begin{frame}
    \titlepage
\end{frame}

\begin{frame}{Lời chào}
    \begin{block}{Đồng hành cùng bạn}
        Chào bạn! Tôi đã sẵn sàng đồng hành cùng bạn để "mổ xẻ" tư duy thuật toán cho bài \textbf{Codeforces 1517A - Sum of 2050}.
    \end{block}
    
    \begin{itemize}
        \item Chúng ta sẽ không bắt đầu bằng những dòng code khô khan.
        \item Hãy cùng biến những con số thành những mảnh ghép logic!
    \end{itemize}
\end{frame}

\begin{frame}{🏗️ PHÁT THẢO LỘ TRÌNH (BRIEFING)}
    \textbf{Tóm tắt đề bài:}
    \begin{itemize}
        \item Bạn có túi vô tận các số dạng $2050 \times 10^k$ (như $2050, 20500, 205000 \dots$).
        \item Nhiệm vụ: Lấy ra ít số nhất sao cho tổng bằng $n$.
        \item Nếu không thể: Trả về -1.
    \end{itemize}

    \begin{exampleblock}{Lộ trình tư duy}
        \begin{enumerate}
            \item \textbf{Chunk 1:} Tấm vé thông hành (Điều kiện cần và đủ).
            \item \textbf{Chunk 2:} Chiến thuật "đổi tiền" (Phân rã số $n$).
            \item \textbf{Chunk 3:} Tìm ra con số tối ưu (Tổng kết quy luật).
        \end{enumerate}
    \end{exampleblock}
\end{frame}

\begin{frame}{🧩 CHUNK 1: TẤM VÉ THÔNG HÀNH}
    \textbf{Logic:}
    Mọi số hạng đều có dạng $2050 \times 10^k$, nghĩa là chúng đều chia hết cho $2050$. Do đó, tổng của chúng ($n$) cũng phải chia hết cho $2050$.

    \begin{alertblock}{Bẫy (Trap)}
        Nhiều bạn sẽ cố gắng lấy $n$ trừ đi số 2050 lớn nhất có thể, rồi lại trừ tiếp... nhưng nếu ngay từ đầu $n$ đã không chia hết cho 2050 thì mọi nỗ lực đều vô ích.
    \end{alertblock}
\end{frame}

\begin{frame}{❓ Thử thách tư duy (Mental Check)}
    Trong các trường hợp sau, trường hợp nào chúng ta có thể khẳng định ngay là \textbf{không thể} (trả về -1)?
    
    \begin{itemize}
        \item A. $n = 4100$
        \item B. $n = 205$
        \item C. $n = 2051$
        \item D. Cả B và C
    \end{itemize}

    \pause
    \begin{block}{Đáp án: D}
        Cả B ($n < 2050$) và C ($n$ không chia hết cho 2050) đều không thể tạo ra được.
    \end{block}
\end{frame}

\begin{frame}{🧩 CHUNK 2: CHIẾN THUẬT "ĐỔI TIỀN"}
    \textbf{Logic:}
    Giả sử $n$ chia hết cho 2050. Gọi $q = n / 2050$.
    Bài toán trở thành: Cần ít nhất bao nhiêu số có dạng $10^k$ để cộng lại vừa bằng $q$?

    \begin{block}{Ẩn dụ: Đổi tiền}
        Bạn có $q$ đô la, ngân hàng chỉ có các tờ mệnh giá $1, 10, 100, 1000 \dots$
        Để dùng \textbf{ít tờ nhất}, bạn sẽ đổi các tờ mệnh giá lớn nhất có thể.
    \end{block}

    \textbf{Ví dụ:} Với $q = 123$, ta cần:
    \begin{itemize}
        \item 1 tờ mệnh giá 100.
        \item 2 tờ mệnh giá 10.
        \item 3 tờ mệnh giá 1.
        \item Tổng cộng: $1 + 2 + 3 = 6$ tờ.
    \end{itemize}
\end{frame}

\begin{frame}{❓ Thử thách tư duy (Mental Check)}
    Nếu sau khi chia $n$ cho $2050$, bạn nhận được thương số $q = 4021$.
    Theo logic trên, bạn cần ít nhất bao nhiêu số hạng?

    \begin{itemize}
        \item A. 3
        \item B. 7
        \item C. 4
        \item D. 4021
    \end{itemize}

    \pause
    \begin{block}{Đáp án: B}
        Tổng các chữ số: $4 + 0 + 2 + 1 = 7$.
    \end{block}
\end{frame}

\begin{frame}{🧩 CHUNK 3: TỔNG KẾT THUẬT TOÁN}
    \begin{enumerate}
        \item \textbf{Kiểm tra tính chia hết}: Nếu $n \% 2050 \neq 0$, kết quả là $-1$.
        \item \textbf{Tính thương số}: $q = n / 2050$.
        \item \textbf{Tính kết quả}: Tổng các chữ số của $q$.
    \end{enumerate}

    \begin{alertblock}{Kỹ thuật lập trình}
        Để tách chữ số của $q$:
        \begin{itemize}
            \item Dùng \texttt{q \% 10} để lấy chữ số hàng đơn vị.
            \item Dùng \texttt{q / 10} để bỏ chữ số đã lấy.
        \end{itemize}
    \end{alertblock}
\end{frame}

\begin{frame}{❓ Thử thách cuối cùng}
    Tính nhanh cho trường hợp $n = 6150$:
    \begin{enumerate}
        \item $n$ có chia hết cho 2050 không? Nếu có, $q$ bằng bao nhiêu?
        \item Tổng các chữ số của $q$ là bao nhiêu?
    \end{enumerate}

    \pause
    \begin{block}{Kết quả}
        \begin{itemize}
            \item $6150 / 2050 = 3$.
            \item Tổng các chữ số là $3$.
            \item In ra: $3$.
        \end{itemize}
    \end{block}
\end{frame}

\begin{frame}[fragile]{📝 MÃ GIẢ (PSEUDOCODE)}
\begin{lstlisting}[language=Python]
Nhap n
Neu n khong chia het cho 2050:
    In ra -1
Nguoc lai:
    Gan q = n / 2050
    Gan tong_chu_so = 0
    
    Trong khi q > 0:
        tong_chu_so = tong_chu_so + (q chia lay du cho 10)
        q = q chia lay nguyen cho 10
        
    In ra tong_chu_so
\end{lstlisting}

\begin{itemize}
    \item \texttt{q \% 10} lấy số ở hàng đơn vị.
    \item \texttt{q / 10} vứt bỏ hàng đơn vị để tiếp tục vòng lặp.
\end{itemize}
\end{frame}

\begin{frame}{🏁 BƯỚC CUỐI CÙNG}
    Bạn muốn thực hiện bước nào tiếp theo?
    
    \begin{enumerate}
        \item \textbf{Chuyển sang code C++ hoặc Python}: Giải quyết vấn đề kiểu dữ liệu lớn ($n \le 10^{18}$).
        \item \textbf{Phân tích độ phức tạp (Big O)}: Tại sao cách này lại chạy rất nhanh.
        \item \textbf{Thử thách bài mới}: Chuyển sang bài \textbf{71A - Way Too Long Words}.
    \end{enumerate}
    
    \begin{block}{Lưu ý}
        Vì $n$ có thể lên đến $10^{18}$, hãy sử dụng kiểu dữ liệu \texttt{long long} trong C++ nhé!
    \end{block}
\end{frame}

\end{document}