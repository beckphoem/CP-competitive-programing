\documentclass{beamer}
\usepackage[utf8]{inputenc}
\usepackage[T5]{fontenc} % Bắt buộc để hiển thị tiếng Việt
\usepackage[vietnamese]{babel}
\usepackage{tcolorbox}
\usepackage{listings}
\usepackage{xcolor}
\usepackage{booktabs}
\usetheme{Madrid}

\definecolor{codegreen}{rgb}{0,0.6,0}
\definecolor{codegray}{rgb}{0.5,0.5,0.5}
\definecolor{codepurple}{rgb}{0.58,0,0.82}
\definecolor{backcolour}{rgb}{0.95,0.95,0.92}

\lstdefinestyle{mystyle}{
    backgroundcolor=\color{backcolour},   
    commentstyle=\color{codegreen},
    keywordstyle=\color{magenta},
    numberstyle=\tiny\color{codegray},
    stringstyle=\color{codepurple},
    basicstyle=\ttfamily\scriptsize,
    breakatwhitespace=false,         
    breaklines=true,                 
    captionpos=b,                    
    keepspaces=true,                 
    numbers=left,                    
    numbersep=4pt,                  
    showspaces=false,                
    showstringspaces=false,
    showtabs=false,                  
    tabsize=2,
    escapechar=@
}

\lstset{style=mystyle}

\title[Cấu tạo số trong CP]{Phân tích hệ thống bài tập cấu tạo số trong lập trình thi đấu}
\subtitle{Từ nền tảng cấu trúc điều khiển đến tư duy thuật toán bậc thấp}
\author{Slide Learning C++}
\date{Rank 800 - 1000}

\begin{document}

\begin{frame}
    \titlepage
\end{frame}

\begin{frame}{Nền tảng lý thuyết}
    \begin{block}{Biểu diễn số học}
    Một số nguyên dương $n$ có thể được phân rã thành:
    $$n = a_k \cdot 10^k + a_{k-1} \cdot 10^{k-1} + \dots + a_1 \cdot 10^1 + a_0 \cdot 10^0$$
    \end{block}
    
    \begin{itemize}
        \item \textbf{Phép toán trích xuất:} $a_0 = n \pmod{10}$.
        \item \textbf{Phép toán loại bỏ:} $n = n / 10$.
        \item \textbf{Cấu trúc:} Sử dụng \texttt{while} để tách chữ số khi chưa biết trước độ dài.
    \end{itemize}
\end{frame}

\begin{frame}[fragile]{1. Sum of Round Numbers (CF 1352A)}
    \textbf{Link:} \url{https://codeforces.com/problemset/problem/1352/A}
    
    \begin{block}{Đề bài}
    Phân tích $n$ ($n \le 10^4$) thành tổng ít nhất các "số tròn" (dạng $d \cdot 10^k$).
    \end{block}

    \begin{exampleblock}{Hướng giải}
    \begin{itemize}
        \item Sử dụng vòng lặp \texttt{while} để tách từng chữ số.
        \item Duyệt qua các vị trí $i$ từ hàng đơn vị.
        \item Nếu chữ số $a_i > 0$, lưu giá trị $a_i \cdot 10^i$ vào mảng kết quả.
    \end{itemize}
    \end{exampleblock}
\end{frame}

\begin{frame}[fragile]{2. Nearest Interesting Number (CF 1183A)}
    \textbf{Link:} \url{https://codeforces.com/problemset/problem/1183/A}
    
    \begin{block}{Đề bài}
    Tìm số thú vị nhỏ nhất $a \ge n$ sao cho tổng các chữ số của nó chia hết cho 4.
    \end{block}

    \begin{exampleblock}{Hướng giải}
    \begin{itemize}
        \item Bắt đầu vòng lặp \texttt{while(true)} từ giá trị $n$.
        \item \textbf{Bước 1:} Dùng vòng lặp con tính tổng các chữ số của số hiện tại.
        \item \textbf{Bước 2:} Dùng \texttt{if} để kiểm tra \texttt{sum \% 4 == 0}.
        \item Nếu thỏa mãn thì dừng và in kết quả.
    \end{itemize}
    \end{exampleblock}
\end{frame}

\begin{frame}[fragile]{3. Digits Sum (CF 1553A)}
    \textbf{Link:} \url{https://codeforces.com/problemset/problem/1553A}
    
    \begin{block}{Đề bài}
    Đếm số nguyên $x \le n$ sao cho tổng chữ số của $x+1$ nhỏ hơn tổng chữ số của $x$.
    \end{block}

    \begin{alertblock}{Phân tích quy luật}
    \begin{itemize}
        \item Quy luật: Tổng chữ số chỉ giảm khi có hiện tượng nhớ (ví dụ từ 9 lên 10).
        \item Chỉ các số kết thúc bằng chữ số 9 mới thỏa mãn điều kiện này.
        \item \textbf{Công thức:} $(n + 1) / 10$.
    \end{itemize}
    \end{alertblock}
\end{frame}

\begin{frame}[fragile]{4. Distinct Digits (CF 1228A)}
    \textbf{Link:} \url{https://codeforces.com/problemset/problem/1228/A}
    
    \begin{block}{Đề bài}
    Tìm một số $x$ trong đoạn $[l, r]$ có các chữ số đôi một khác nhau.
    \end{block}

    \begin{exampleblock}{Hướng giải}
    \begin{itemize}
        \item Duyệt \texttt{for} từ $l$ đến $r$.
        \item Với mỗi số, dùng \texttt{while} để tách từng chữ số.
        \item Sử dụng mảng đánh dấu \texttt{bool \text{appeared}[10]} để kiểm tra xem chữ số đã xuất hiện chưa.
        \item Nếu tìm thấy số thỏa mãn, in ra và kết thúc chương trình.
    \end{itemize}
    \end{exampleblock}
\end{frame}

\begin{frame}[fragile]{5. Ordinary Numbers (CF 1520B)}
    \textbf{Link:} \url{https://codeforces.com/problemset/problem/1520/B}
    
    \begin{block}{Đề bài}
    Đếm số lượng số chỉ gồm một loại chữ số (1, 22, 333,...) trong đoạn $[1, n]$.
    \end{block}

    \begin{exampleblock}{Hướng giải}
    \begin{itemize}
        \item Thay vì duyệt đến $10^9$, ta chủ động sinh ra các số này.
        \item Dùng 2 vòng lặp lồng nhau:
        \begin{itemize}
            \item Vòng ngoài cho chữ số từ 1 đến 9.
            \item Vòng trong tạo số dạng $d, dd, ddd, \dots$
        \end{itemize}
        \item Đếm các số sinh ra mà $\le n$.
    \end{itemize}
    \end{exampleblock}
\end{frame}

\begin{frame}[fragile]{6. Div. 7 (CF 1633A)}
    \textbf{Link:} \url{https://codeforces.com/problemset/problem/1633/A}
    
    \begin{block}{Đề bài}
    Thay đổi ít nhất chữ số của $n$ để được số mới chia hết cho 7.
    \end{block}

    \begin{exampleblock}{Hướng giải}
    \begin{itemize}
        \item Nếu $n \pmod{7} == 0$, giữ nguyên.
        \item Nếu không, thử thay đổi chữ số hàng đơn vị.
        \item Duyệt \texttt{for} từ 0 đến 9 để thay thế chữ số cuối.
        \item Kiểm tra số mới, nếu chia hết cho 7 thì dừng lại.
        \item \textbf{Lưu ý:} Đảm bảo số mới không làm thay đổi số lượng chữ số ban đầu.
    \end{itemize}
    \end{exampleblock}
\end{frame}

\begin{frame}{Tổng kết tư duy}
    \begin{itemize}
        \item \textbf{Kỹ năng tách số:} Là xương sống của các bài toán cấu tạo số.
        \item \textbf{Độ phức tạp:} $O(\log_{10} n)$ cho mỗi lần xử lý chữ số, cực kỳ tối ưu.
        \item \textbf{Tiếp cận:} 
        \pause
        \item Luôn kiểm tra các trường hợp biên (số 0, số có toàn chữ số 9).
        \pause
        \item Cân nhắc giữa việc duyệt số và việc sinh số.
    \end{itemize}
\end{frame}

\end{document}