\documentclass{beamer}
\usepackage[utf8]{inputenc}
\usepackage[T5]{fontenc} % Bắt buộc để hiển thị tiếng Việt
\usepackage[vietnamese]{babel}
\usepackage{tcolorbox}
\usepackage{listings}
\usepackage{xcolor}
\usepackage{booktabs}
\usetheme{Madrid}

\definecolor{codegreen}{rgb}{0,0.6,0}
\definecolor{codegray}{rgb}{0.5,0.5,0.5}
\definecolor{codepurple}{rgb}{0.58,0,0.82}
\definecolor{backcolour}{rgb}{0.95,0.95,0.92}

\lstdefinestyle{mystyle}{
    backgroundcolor=\color{backcolour},   
    commentstyle=\color{codegreen},
    keywordstyle=\color{magenta},
    numberstyle=\tiny\color{codegray},
    stringstyle=\color{codepurple},
    basicstyle=\ttfamily\scriptsize,
    breakatwhitespace=false,         
    breaklines=true,                 
    captionpos=b,                    
    keepspaces=true,                 
    numbers=left,                    
    numbersep=4pt,                  
    showspaces=false,                
    showstringspaces=false,
    showtabs=false,                  
    tabsize=2,
    escapechar=@
}

\lstset{style=mystyle}

\title[Primary Task]{Huấn luyện viên tư duy thuật toán}
\subtitle{Codeforces 2000A - Primary Task}
\author{Slide Learning C++}
\date{\today}

\begin{document}

\begin{frame}
  \titlepage
\end{frame}

\begin{frame}{Khởi đầu: Phương pháp Micro-Chunks}
  \begin{block}{Mục tiêu}
    Chúng ta sẽ không sa đà vào việc viết code ngay lập tức. Thay vào đó, chúng ta sẽ "phẫu thuật" từng lớp của bài toán theo phương pháp \textbf{Micro-Chunks} (chia nhỏ tư duy).
  \end{block}
  
  \begin{itemize}
    \item \textbf{Mảnh ghép 1:} Cách tách số $n$ thành "Tiền tố" và "Số mũ".
    \item \textbf{Mảnh ghép 2:} Kiểm tra "Tiền tố" có đúng chuẩn không.
    \item \textbf{Mảnh ghép 3:} Xử lý các điều kiện của "Số mũ".
  \end{itemize}
\end{frame}

\begin{frame}{Bước 1: Phẫu thuật đề bài (Deconstruct)}
  \begin{exampleblock}{Định dạng bắt buộc: $10^a$}
    Số nguyên $n$ phải có dạng: \textbf{10...}, với các điều kiện:
  \end{exampleblock}
  
  \begin{enumerate}
    \item Bắt đầu bằng hai chữ số \textbf{"10"}.
    \item Phần còn lại (số mũ $a$) phải thỏa mãn:
    \begin{itemize}
      \item Giá trị của $a$ phải \textbf{lớn hơn hoặc bằng 2} ($a \ge 2$).
      \item $a$ \textbf{không được bắt đầu bằng chữ số 0}.
    \end{itemize}
  \end{enumerate}
\end{frame}

\begin{frame}{Chunk 1: Định dạng cơ bản (The Structure)}
  \begin{block}{Thử thách tư duy}
    Dựa trên quy tắc: "Bắt đầu bằng 10" và "Có phần số mũ $a$". Trong các số sau, số nào \textbf{vi phạm}?
    \begin{enumerate}
      \item \texttt{101}
      \item \texttt{2010}
      \item \texttt{10}
      \item \texttt{1015}
    \end{enumerate}
  \end{block}

  \pause
  
  \begin{alertblock}{Đáp án & Giải thích}
    \begin{itemize}
      \item \textbf{Số 2 (\texttt{2010})}: Vi phạm vì bắt đầu bằng "20".
      \item \textbf{Số 3 (\texttt{10})}: Vi phạm vì "cụt đuôi" — không có phần số mũ $a$.
    \end{itemize}
  \end{alertblock}
\end{frame}

\begin{frame}{Chunk 2: Điều kiện của "Số mũ" (The Exponent)}
  \begin{block}{2 Điều kiện khắt khe}
    \begin{enumerate}
      \item \textbf{Giá trị}: $a \ge 2$.
      \item \textbf{Hình thức}: Không có số 0 vô nghĩa ở đầu (ví dụ: "05" là sai).
    \end{enumerate}
  \end{block}

  \textbf{Ví dụ:}
  \begin{itemize}
    \item Số A: \texttt{10} + \texttt{1} $\rightarrow$ $a=1$ (Loại vì $1 < 2$).
    \item Số B: \texttt{10} + \texttt{02} $\rightarrow$ Bắt đầu bằng \texttt{0} (Loại).
    \item Số C: \texttt{10} + \texttt{14} $\rightarrow$ \textbf{Hợp lệ}.
  \end{itemize}
\end{frame}

\begin{frame}{Chunk 3: Cách "tóm" lấy phần số mũ}
  \begin{block}{Tư duy lập trình}
    Coi số $n$ là một \textbf{Chuỗi ký tự (String)} $s$:
    \begin{itemize}
      \item \textbf{Tiền tố}: 2 ký tự đầu tiên ($s[0], s[1]$).
      \item \textbf{Số mũ}: Các ký tự từ vị trí thứ 3 trở đi.
    \end{itemize}
  \end{block}

  \begin{exampleblock}{Phân tích số $n = 1005$}
    \begin{enumerate}
      \item Tiền tố: \texttt{10} (Hợp lệ).
      \item Số mũ: Chuỗi \texttt{05}.
      \item Vi phạm: Bắt đầu bằng số \texttt{0}.
    \end{enumerate}
    \textbf{Kết luận:} $1005$ không phải là Primary Task.
  \end{exampleblock}
\end{frame}

\begin{frame}[fragile]{Chunk 4: Tổng kết thuật toán (The Logic Flow)}
  \begin{block}{Hoàn thiện Logic}
    Hãy điền vào chỗ trống để hoàn thiện quy trình:
  \end{block}

  \begin{itemize}
    \item \textbf{B1:} Nếu độ dài chuỗi $s \le$ \dots \pause \textbf{2} \dots thì loại.
    \item \textbf{B2:} Kiểm tra $s[0], s[1]$ có phải là \dots \pause \textbf{"10"} \dots không?
    \item \textbf{B3:} Lấy phần còn lại gọi là \texttt{expo\_str}.
    \begin{itemize}
      \item Nếu ký tự đầu của \texttt{expo\_str} là \dots \pause \textbf{'0'} \dots thì loại.
      \item Chuyển \texttt{expo\_str} sang số $a$. Nếu $a <$ \dots \pause \textbf{2} \dots thì loại.
    \end{itemize}
    \item \textbf{B4:} Nếu vượt qua tất cả, in ra "YES".
  \end{itemize}
\end{frame}

\begin{frame}{Lời kết}
  \begin{block}{Bước tiếp theo}
    Bạn đã nắm vững logic của bài toán! Bạn có muốn tôi hỗ trợ chuyển đổi các bước logic này thành mã nguồn C++ hoàn chỉnh không?
  \end{block}
\end{frame}

\end{document}