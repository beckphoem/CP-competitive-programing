\documentclass{beamer}
\usepackage[utf8]{inputenc}
\usepackage[T5]{fontenc} % Bắt buộc để hiển thị tiếng Việt
\usepackage[vietnamese]{babel}
\usepackage{tcolorbox}
\usepackage{listings}
\usepackage{xcolor}
\usepackage{booktabs}
\usetheme{Madrid}

\definecolor{codegreen}{rgb}{0,0.6,0}
\definecolor{codegray}{rgb}{0.5,0.5,0.5}
\definecolor{codepurple}{rgb}{0.58,0,0.82}
\definecolor{backcolour}{rgb}{0.95,0.95,0.92}

\lstdefinestyle{mystyle}{
    backgroundcolor=\color{backcolour},   
    commentstyle=\color{codegreen},
    keywordstyle=\color{magenta},
    numberstyle=\tiny\color{codegray},
    stringstyle=\color{codepurple},
    basicstyle=\ttfamily\scriptsize,
    breakatwhitespace=false,         
    breaklines=true,                 
    captionpos=b,                    
    keepspaces=true,                 
    numbers=left,                    
    numbersep=4pt,                  
    showspaces=false,                
    showstringspaces=false,
    showtabs=false,                  
    tabsize=2,
    escapechar=@
}

\lstset{style=mystyle}

\title[Fair Numbers]{Giải bài 1411B - Fair Numbers}
\author{Slide Learning C++}
\date{\today}

\begin{document}

\begin{frame}
    \titlepage
\end{frame}

\begin{frame}{Tiếp nhận \& Phẫu thuật bài toán}
    \begin{block}{Dữ kiện cốt lõi}
        Tìm số nguyên $x$ nhỏ nhất sao cho $x \ge n$ và $x$ là một \textbf{"số công bằng" (fair number)}.
    \end{block}
    
    \begin{exampleblock}{Định nghĩa Số công bằng}
        Một số được gọi là "công bằng" nếu nó \textbf{chia hết cho tất cả các chữ số khác 0} của chính nó.
    \end{exampleblock}
    
    \begin{itemize}
        \item \textbf{Ví dụ 1:} $120$ là số công bằng vì $120 \vdots 1$ và $120 \vdots 2$ (số 0 bỏ qua).
        \item \textbf{Ví dụ 2:} $7$ là số công bằng vì $7 \vdots 7$.
        \item \textbf{Ví dụ 3:} $23$ \textbf{không} phải số công bằng vì $23$ không chia hết cho $3$.
    \end{itemize}
\end{frame}

\begin{frame}{Lộ trình tư duy (Micro-chunking)}
    Chúng ta sẽ chia bài toán thành 3 phần chính:
    \begin{enumerate}
        \item \textbf{Chunk 1:} Hiểu cách kiểm tra một số có "công bằng" hay không.
        \item \textbf{Chunk 2:} Chiến thuật tìm số $x$ nhỏ nhất (Duyệt hay dùng công thức?).
        \item \textbf{Chunk 3:} Ước lượng giới hạn (Liệu tìm kiếm có quá lâu không?).
    \end{enumerate}
\end{frame}

\begin{frame}{Chunk 1: Kiểm tra tính "Công bằng"}
    Để biết một số $x$ có chia hết cho tất cả các chữ số của nó hay không:
    \begin{enumerate}
        \item \textbf{Tách từng chữ số} của $x$.
        \item \textbf{Kiểm tra điều kiện:} Với mỗi chữ số $d \neq 0$, liệu $x \pmod d = 0$?
    \end{enumerate}
    
    \begin{block}{Ẩn dụ}
        Vị vua $x$ chỉ "công bằng" nếu có thể chia đều ngân khố cho tất cả các quan đại thần (chữ số) của mình (ngoại trừ quan tên "Không").
    \end{block}
\end{frame}

\begin{frame}{Thử thách tư duy}
    \begin{exampleblock}{Câu hỏi}
        Kiểm tra số $123$. Các chữ số là: $1, 2, 3$.
    \end{exampleblock}
    \begin{itemize}
        \item $123 \vdots 1$? \pause \textbf{Có}
        \item $123 \vdots 2$? \pause \textbf{Không} ($123$ là số lẻ)
        \item $123 \vdots 3$? \pause \textbf{Có} ($1+2+3=6 \vdots 3$)
    \end{itemize}
    \pause
    \begin{alertblock}{Kết luận}
        Số $123$ \textbf{không} phải là số công bằng vì không chia hết cho 2.
    \end{alertblock}
\end{frame}

\begin{frame}{Chunk 2: Tại sao duyệt trâu (Brute Force) lại nhanh?}
    BCNN của các số từ $\{1, 2, 3, 4, 5, 6, 7, 8, 9\}$ là:
    \[ 2^3 \times 3^2 \times 5 \times 7 = 2520 \]
    
    \begin{block}{Ý nghĩa con số 2520}
        Cứ trong khoảng $2520$ số liên tiếp, \textbf{chắc chắn} sẽ có ít nhất một số chia hết cho tất cả các chữ số từ 1 đến 9.
    \end{block}
    
    Vì khoảng cách tối đa chỉ là 2520 (thực tế còn nhỏ hơn), việc dùng vòng lặp \texttt{while} để tăng dần $n$ là cực kỳ hiệu quả!
\end{frame}

\begin{frame}[fragile]{Chunk 3: Tổng kết thuật toán \& Mã giả}
    \begin{block}{Thuật toán}
        \begin{enumerate}
            \item Đọc số $n$.
            \item Chừng nào $n$ chưa phải số công bằng, tăng $n$ lên 1.
        \end{enumerate}
    \end{block}

    \begin{lstlisting}[language=C++, caption=Mã giả kiểm tra số công bằng]
bool kiem_tra_cong_bang(long long x) {
    long long temp = x;
    while (temp > 0) {
        int d = temp % 10;
        if (d != 0 && x % d != 0) {
            return false;
        }
        temp /= 10;
    }
    return true;
}
    \end{lstlisting}
\end{frame}

\begin{frame}{Thử thách cuối cùng}
    \begin{exampleblock}{Câu hỏi}
        Nếu đề bài cho $n=23$, máy tính sẽ thực hiện thế nào?
    \end{exampleblock}
    \begin{itemize}
        \item \textbf{A.} Kiểm tra 23 (sai), tăng lên 24. Kiểm tra 24...
        \item \textbf{B.} Nhảy một phát đến 2520.
    \end{itemize}
    \pause
    \textbf{Đáp án: A}.
    Kiểm tra $24$: $24 \vdots 2$ và $24 \vdots 4$. \textbf{Dừng lại!} Chỉ mất 2 bước.
\end{frame}

\begin{frame}{Bước cuối: Hiện thực hóa \& Lưu ý}
    Đề bài cho $n \le 10^{18}$. 
    
    \begin{alertblock}{Lưu ý quan trọng về kiểu dữ liệu}
        Bắt buộc dùng \texttt{long long} trong C++ để chứa $n$.
        \begin{itemize}
            \item \texttt{int}: Tối đa $\approx 2 \times 10^9$
            \item \texttt{long long}: Tối đa $\approx 9 \times 10^{18}$
        \end{itemize}
    \end{alertblock}

    \begin{block}{Tóm tắt các bước}
        1. Đọc $n$ kiểu \texttt{long long}. \\
        2. Viết hàm \texttt{isFair(x)}. \\
        3. Vòng lặp: \texttt{while(!isFair(n)) n++;} \\
        4. In kết quả $n$.
    \end{block}
\end{frame}

\begin{frame}{Bạn muốn tiếp tục thế nào?}
    \begin{enumerate}
        \item \textbf{Thử thách mới:} Một bài toán khác có tư duy tương tự.
        \item \textbf{Nâng cấp:} Phân tích bài khó hơn (Quy hoạch động/Tham lam).
        \item \textbf{Hỗ trợ Code:} Chuyển mã giả thành code C++ hoàn chỉnh.
    \end{enumerate}
\end{frame}

\end{document}