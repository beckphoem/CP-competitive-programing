\documentclass{beamer}
\usepackage[utf8]{inputenc}
\usepackage[T5]{fontenc} % Bắt buộc để hiển thị tiếng Việt
\usepackage[vietnamese]{babel}
\usepackage{tcolorbox}
\usepackage{listings}
\usepackage{xcolor}
\usepackage{booktabs}
\usetheme{Madrid}

\definecolor{codegreen}{rgb}{0,0.6,0}
\definecolor{codegray}{rgb}{0.5,0.5,0.5}
\definecolor{codepurple}{rgb}{0.58,0,0.82}
\definecolor{backcolour}{rgb}{0.95,0.95,0.92}

\lstdefinestyle{mystyle}{
    backgroundcolor=\color{backcolour},   
    commentstyle=\color{codegreen},
    keywordstyle=\color{magenta},
    numberstyle=\tiny\color{codegray},
    stringstyle=\color{codepurple},
    basicstyle=\ttfamily\scriptsize,
    breakatwhitespace=false,         
    breaklines=true,                 
    captionpos=b,                    
    keepspaces=true,                 
    numbers=left,                    
    numbersep=4pt,                  
    showspaces=false,                
    showstringspaces=false,
    showtabs=false,                  
    tabsize=2,
    escapechar=@
}

\lstset{style=mystyle}

\title{Codeforces 1684A - Digit Minimization}
\subtitle{Huấn luyện viên Tư duy Thuật toán}
\author{Slide Learning C++}
\date{\today}

\begin{document}

\begin{frame}
  \titlepage
\end{frame}

\begin{frame}{Lời mở đầu}
  \begin{block}{Triết lý}
    Chào bạn! Tôi đã sẵn sàng nhập vai \textbf{Huấn luyện viên Tư duy Thuật toán} theo triết lý "Learning How to Learn". Chúng ta sẽ cùng nhau "mổ xẻ" bài toán để tìm ra bản chất nhé.
  \end{block}
\end{frame}

\begin{frame}{Bước 1: Phẫu thuật đề bài (Deconstruct)}
  \textbf{Cốt lõi của vấn đề:}
  \begin{itemize}
    \item \textbf{Dữ liệu vào:} Một số nguyên dương $n$.
    \item \textbf{Người 1:} Chọn hai vị trí khác nhau trong số $n$ và hoán đổi (swap).
    \item \textbf{Người 2:} Xóa chữ số cuối cùng bên phải của số đó.
    \item \textbf{Kết thúc:} Trò chơi dừng khi số chỉ còn \textbf{đúng một chữ số}.
    \item \textbf{Mục tiêu:} Tìm chữ số cuối cùng còn sót lại là \textbf{nhỏ nhất có thể}.
  \end{itemize}
\end{frame}

\begin{frame}{Lộ trình tư duy (Chunking)}
  \begin{enumerate}
    \item \textbf{Chunk 1:} Quan sát quy luật khi số chỉ có đúng 2 chữ số.
    \item \textbf{Chunk 2:} Quan sát quy luật khi số có từ 3 chữ số trở lên.
    \item \textbf{Chunk 3:} Tổng kết chiến thuật tối ưu.
  \end{enumerate}
  \vspace{0.5cm}
  \begin{exampleblock}{Sẵn sàng?}
    Bạn đã sẵn sàng mổ xẻ mảnh ghép đầu tiên chưa?
  \end{exampleblock}
\end{frame}

\begin{frame}{Chunk 1: Trường hợp số có đúng 2 chữ số}
  \begin{alertblock}{Bẫy logic}
    Bạn \textbf{bắt buộc} phải hoán đổi trước khi bị xóa. Không có quyền lựa chọn "không hoán đổi".
  \end{alertblock}
  
  \textbf{Ví dụ: Số $n = 12$}
  \begin{enumerate}
    \item Sau khi Người 1 hoán đổi, số trở thành bao nhiêu?
    \item Sau đó Người 2 xóa chữ số cuối, kết quả cuối cùng là gì?
  \end{enumerate}
  
  \pause
  \begin{block}{Giải đáp}
    Kết quả là: \textbf{2}. Với số có 2 chữ số, đáp án luôn là \textbf{chữ số thứ hai} (hàng đơn vị).
  \end{block}
\end{frame}

\begin{frame}{Chunk 2: Khi số có từ 3 chữ số trở lên}
  \textbf{Ví dụ: Số $n = 312$}
  \begin{itemize}
    \item Người 1 có thể chọn \textbf{bất kỳ} 2 vị trí nào để hoán đổi.
    \item Người 2 xóa chữ số cuối cùng bên phải.
  \end{itemize}
  
  \begin{exampleblock}{Thử thách tư duy}
    Nếu mục tiêu là giữ lại số \textbf{1} đến cuối cùng, bạn nên đưa số 1 về vị trí nào để nó không bao giờ bị "bay màu"?
  \end{exampleblock}
  
  \pause
  \begin{block}{Chiến thuật}
    Đưa chữ số nhỏ nhất về một vị trí an toàn (không phải vị trí cuối) cho đến tận bước cuối cùng.
  \end{block}
\end{frame}

\begin{frame}{Tổng kết chiến thuật}
  \begin{table}[]
    \centering
    \begin{tabular}{@{}lll@{}}
    \toprule
    \textbf{Trường hợp} & \textbf{Chiến thuật} & \textbf{Kết quả} \\ \midrule
    Số có 2 chữ số & Bị ép buộc hoán đổi & Chữ số thứ 2 \\
    Số có 3+ chữ số & Tự do điều khiển & Chữ số nhỏ nhất \\ \bottomrule
    \end{tabular}
  \end{table}
\end{frame}

\begin{frame}[fragile]{Bước 3: Tổng kết \& Chốt thuật toán}
  \begin{block}{Mã giả (Pseudocode)}
    \begin{lstlisting}[language=Pascal]
    1. nhap vao chuoi ky tu S (dai dien cho so n)
    2. neu do dai cua S bang 2:
           in ra ky tu S[1] (ky tu thu hai)
    3. nguoc lai (neu do dai lon hon 2):
           tim ky tu nho nhat trong chuoi S
           in ra ky tu nho nhat do
    \end{lstlisting}
  \end{block}
\end{frame}

\begin{frame}{Thử thách cuối cùng}
  Hãy thử áp dụng thuật toán cho 2 test case này:
  \begin{enumerate}
    \item $n = 249$
    \item $n = 42$
  \end{enumerate}
  
  \pause
  \begin{exampleblock}{Đáp án}
    \begin{itemize}
      \item Với $n = 249$: Đáp án là \textbf{2} (nhỏ nhất).
      \item Với $n = 42$: Đáp án là \textbf{2} (vị trí thứ hai).
    \end{itemize}
  \end{exampleblock}
\end{frame}

\begin{frame}{Kết luận}
  \begin{block}{Tiếp theo}
    Bạn đã hoàn toàn làm chủ được tư duy bài toán này! Bạn có muốn mình giúp chuyển mã giả này sang một ngôn ngữ lập trình cụ thể (C++, Python...) hay mổ xẻ bài toán khác thú vị hơn?
  \end{block}
\end{frame}

\end{document}