\documentclass{beamer}
\usepackage[utf8]{inputenc}
\usepackage[T5]{fontenc} % Bắt buộc để hiển thị tiếng Việt
\usepackage[vietnamese]{babel}
\usepackage{tcolorbox}
\usepackage{listings}
\usepackage{xcolor}
\usepackage{booktabs}
\usetheme{Madrid}

\definecolor{codegreen}{rgb}{0,0.6,0}
\definecolor{codegray}{rgb}{0.5,0.5,0.5}
\definecolor{codepurple}{rgb}{0.58,0,0.82}
\definecolor{backcolour}{rgb}{0.95,0.95,0.92}

\lstdefinestyle{mystyle}{
    backgroundcolor=\color{backcolour},   
    commentstyle=\color{codegreen},
    keywordstyle=\color{magenta},
    numberstyle=\tiny\color{codegray},
    stringstyle=\color{codepurple},
    basicstyle=\ttfamily\scriptsize,
    breakatwhitespace=false,         
    breaklines=true,                 
    captionpos=b,                    
    keepspaces=true,                 
    numbers=left,                    
    numbersep=4pt,                  
    showspaces=false,                
    showstringspaces=false,
    showtabs=false,                  
    tabsize=2,
    escapechar=@
}

\lstset{style=mystyle}

\title[Codeforces 102B]{Codeforces 102B - Sum of Digits}
\subtitle{Coach Tư duy Thuật toán - Phong cách Micro-Chunks}
\author{Gemini AI}
\date{\today}

\begin{document}

\begin{frame}
    \titlepage
\end{frame}

\begin{frame}{Lộ trình học tập}
    \begin{block}{Mục tiêu}
        "Mổ xẻ" bài toán tập trung hoàn toàn vào chiến thuật và logic thay vì chỉ nhìn vào code.
    \end{block}
    
    \textbf{Lộ trình 3 giai đoạn:}
    \begin{itemize}
        \item \textbf{Chunk 1:} Xử lý con số "khổng lồ" (Dữ liệu đầu vào).
        \item \textbf{Chunk 2:} Cơ chế "nén" số (Tính tổng các chữ số).
        \item \textbf{Chunk 3:} Điều kiện dừng và bộ đếm.
    \end{itemize}
\end{frame}

\begin{frame}{Bước 1: Phẫu thuật đề bài (Deconstruction)}
    \begin{itemize}
        \item \textbf{Input:} Một số nguyên cực lớn $n$. (Có thể có tới $10^5$ chữ số!).
        \item \textbf{Hành động:} Thay thế số đó bằng \textbf{tổng các chữ số} của nó.
        \item \textbf{Lặp lại:} Tiếp tục cho đến khi số còn lại chỉ có \textbf{duy nhất 1 chữ số}.
        \item \textbf{Mục tiêu:} Đếm số lần thực hiện hành động thay thế.
    \end{itemize}
\end{frame}

\begin{frame}{Chunk 1: Đối mặt với "Con quái vật" số lớn}
    \begin{alertblock}{Vấn đề}
        Số $n$ có thể có $10^5$ chữ số. Kiểu dữ liệu \texttt{int} hay \texttt{long long} chỉ chứa được tối đa khoảng 10 đến 19 chữ số.
    \end{alertblock}

    \begin{block}{Thử thách tư duy}
        Để lưu trữ một "đoàn tàu" có $10^5$ toa (chữ số), chúng ta nên dùng kiểu dữ liệu nào?
        \begin{enumerate}[A.]
            \item Biến kiểu số thực (Float/Double).
            \item Chuỗi ký tự (String/Array of Characters).
            \item Ép kiểu về số nguyên lớn (BigInt).
        \end{enumerate}
    \end{block}
    \pause
    \textbf{Đáp án: B. Chuỗi ký tự (String)}
\end{frame}

\begin{frame}{Chunk 2: Cơ chế "nén" số}
    \begin{exampleblock}{Ví dụ: n = "47"}
        \begin{itemize}
            \item \textbf{Lần 1:} $4 + 7 = 11$. (Thực hiện 1 lần).
            \item \textbf{Lần 2:} $1 + 1 = 2$. (Thực hiện lần 2).
            \item \textbf{Dừng lại:} Vì số 2 chỉ có 1 chữ số. Kết quả: 2 lần.
        \end{itemize}
    \end{exampleblock}

    \begin{tcolorbox}[colback=blue!5!white,colframe=blue!75!black,title=Lưu ý kỹ thuật]
        Để lấy giá trị số từ ký tự (ASCII), ta dùng công thức: \texttt{'7' - '0'} = 7.
    \end{tcolorbox}
\end{frame}

\begin{frame}{Thử thách Chunk 2}
    \begin{block}{Câu hỏi}
        Nếu đầu vào là số $n = 991$.
        \begin{enumerate}
            \item Lần nén thứ nhất sẽ biến số này thành số bao nhiêu?
            \item Sau lần đó, đã thỏa mãn điều kiện dừng chưa?
        \end{enumerate}
    \end{block}
    \pause
    \begin{exampleblock}{Giải đáp}
        \begin{enumerate}
            \item $9 + 9 + 1 = 19$.
            \item Chưa dừng (vì 19 có 2 chữ số).
        \end{enumerate}
    \end{exampleblock}
\end{frame}

\begin{frame}{Chunk 3: Điều kiện dừng và Bộ đếm}
    Sử dụng vòng lặp \texttt{while} để xử lý khi chưa biết rõ số bước:
    \begin{itemize}
        \item \textbf{Điều kiện dừng:} Khi chiều dài chuỗi bằng 1.
        \item \textbf{Bộ đếm (\texttt{count}):} Tăng lên 1 sau mỗi lần tính tổng thành công.
    \end{itemize}

    \begin{alertblock}{Cảnh báo bẫy (Edge Case)}
        Nếu số $n$ ngay từ đầu chỉ có 1 chữ số (ví dụ $n = 5$)?
        \begin{enumerate}[A.]
            \item 0 lần.
            \item 1 lần.
            \item Không thể thực hiện.
        \end{enumerate}
    \end{alertblock}
    \pause
    \textbf{Đáp án: A. 0 lần.} (Sử dụng \texttt{while} kiểm tra trước khi làm sẽ tránh được lỗi này).
\end{frame}

\begin{frame}[fragile]{Tổng kết thuật toán (Pseudocode)}
    \begin{lstlisting}[language=C++, caption=Bản thiết kế thuật toán]
1. Doc du lieu n duoi dang @\textbf{string}@.
2. Khoi tao bien @\texttt{dem = 0}@.
3. Vong lap @\textbf{while (n.length() > 1)}@:
   a. Tinh tong cac chu so trong chuoi n.
   b. Bien doi Tong vua tinh ve lai dang @\textbf{string}@ va gan cho n.
   c. Tang @\texttt{dem}@ len 1.
4. In ra @\texttt{dem}@.
    \end{lstlisting}

    \begin{block}{Bước tiếp theo}
        Bạn có muốn thử tự viết code dựa trên bản thiết kế này không? Tôi có thể hỗ trợ phần chuyển đổi số thành chuỗi hoặc xử lý vòng lặp!
    \end{block}
\end{frame}

\end{document}