\documentclass{beamer}
\usepackage[utf8]{inputenc}
\usepackage[T5]{fontenc} % Bắt buộc để hiển thị tiếng Việt
\usepackage[vietnamese]{babel}
\usepackage{tcolorbox}
\usepackage{listings}
\usepackage{xcolor}
\usepackage{booktabs}
\usetheme{Madrid}

\definecolor{codegreen}{rgb}{0,0.6,0}
\definecolor{codegray}{rgb}{0.5,0.5,0.5}
\definecolor{codepurple}{rgb}{0.58,0,0.82}
\definecolor{backcolour}{rgb}{0.95,0.95,0.92}

\lstdefinestyle{mystyle}{
    backgroundcolor=\color{backcolour},   
    commentstyle=\color{codegreen},
    keywordstyle=\color{magenta},
    numberstyle=\tiny\color{codegray},
    stringstyle=\color{codepurple},
    basicstyle=\ttfamily\scriptsize,
    breakatwhitespace=false,         
    breaklines=true,                 
    captionpos=b,                    
    keepspaces=true,                 
    numbers=left,                    
    numbersep=4pt,                  
    showspaces=false,                
    showstringspaces=false,
    showtabs=false,                  
    tabsize=2,
    escapechar=@
}

\lstset{style=mystyle}

\title[Codeforces 1811A]{Coach Tư duy Thuật toán}
\subtitle{Codeforces 1811A - Insert Digit}
\author{Slide Learning C++}
\date{\today}

\begin{document}

\begin{frame}
    \titlepage
\end{frame}

\begin{frame}{Lời mở đầu}
    \begin{block}{Chào bạn!}
        Tôi đã nắm rõ vai trò là \textbf{Coach Tư duy Thuật toán} của bạn. Chúng ta sẽ cùng nhau "mổ xẻ" các bài toán theo phong cách \textbf{Micro-Chunks}, tập trung hoàn toàn vào tư duy logic và hình ảnh hóa vấn đề trước khi chạm vào code.
    \end{block}
    Hãy bắt đầu với bài \textbf{Codeforces 1811A - Insert Digit}.
\end{frame}

\begin{frame}{Bước 1: Phẫu thuật đề bài (Deconstruct)}
    Loại bỏ những yếu tố rườm rà, đây là cốt lõi của vấn đề:
    \begin{itemize}
        \item \textbf{Dữ liệu cho sẵn:} Một số nguyên rất lớn ($n$ chữ số) và một chữ số lẻ $d$ (từ 0 đến 9).
        \item \textbf{Nhiệm vụ:} Chèn chữ số $d$ vào \textbf{bất kỳ vị trí nào} để tạo ra một số mới \textbf{lớn nhất có thể}.
        \item \textbf{Lộ trình tư duy:}
        \begin{enumerate}
            \item Hiểu quy tắc so sánh hai số lớn.
            \item Xác định thời điểm "vàng" để chèn chữ số $d$.
            \item Xử lý trường hợp chữ số $d$ "về bét".
        \end{enumerate}
    \end{itemize}
\end{frame}

\begin{frame}{Bước 2: Mảnh ghép tư duy 1 - Quy tắc "Ai đứng trước là vua"}
    \begin{block}{Ẩn dụ}
        Những người đứng ở đầu hàng (bên trái) có quyền lực cao nhất. Bạn có một viên kim cương (chữ số $d$) và muốn đặt nó vào hàng sao cho tổng giá trị là lớn nhất.
    \end{block}
    \begin{alertblock}{Bẫy logic}
        Không phải lúc nào chèn vào cuối hay sau chữ số lớn nhất cũng là tối ưu. Ta cần so sánh trực tiếp $d$ với từng "người" từ trái sang phải.
    \end{alertblock}
\end{frame}

\begin{frame}{Thử thách tư duy 1}
    Giả sử số ban đầu là $764$ và chữ số cần chèn là $5$.
    Vị trí nào cho ra kết quả lớn nhất?
    \begin{itemize}
        \item A. Chèn vào đầu: 5764
        \item B. Chèn vào sau số 7: 7564
        \item C. Chèn vào sau số 6: 7654
        \item D. Chèn vào cuối: 7645
    \end{itemize}
    \pause
    \begin{exampleblock}{Đáp án: C. 7654}
        Ở hàng trăm, số $6$ lớn hơn số $5$. Vì vậy, $7645$ và $7654$ sẽ lớn hơn các số bắt đầu bằng $75...$
    \end{exampleblock}
\end{frame}

\begin{frame}{Quy luật rút ra}
    \begin{tcolorbox}[colback=blue!5,colframe=blue!75,title=Chiến thuật]
        Chúng ta sẽ duyệt từ trái sang phải, nếu thấy chữ số $d$ của mình \textbf{lớn hơn} chữ số đang xét, thì đó chính là "thời điểm vàng" để chèn vào.
    \end{tcolorbox}
    
    \textbf{Thử thách tư duy 2:} Nếu số $445$ và $d = 6$. Bạn chèn vào đâu?
    \begin{itemize}
        \item A. Trước số 4 đầu tiên.
        \item B. Giữa hai số 4.
        \item C. Sau số 5.
    \end{itemize}
    \pause
    \textbf{Đáp án: A} (Vì $6 > 4$, chèn ngay lập tức để có $6445$).
\end{frame}

\begin{frame}{Bước 3: Mảnh ghép tư duy 3 - Khi nào thì "Về bét"?}
    Trường hợp đặc biệt: Duyệt từ đầu đến cuối mà \textbf{không có chữ số nào nhỏ hơn $d$}.
    
    \textbf{Ví dụ:} $887$, $d = 3$.
    \begin{itemize}
        \item Duyệt $8: 3 > 8$ (Sai)
        \item Duyệt $8: 3 > 8$ (Sai)
        \item Duyệt $7: 3 > 7$ (Sai)
    \end{itemize}
    
    \textbf{Thử thách tư duy 3:} Ta nên đặt số $3$ ở đâu?
    \begin{itemize}
        \item A. Đặt ở ngay đầu số.
        \item B. Đặt ở cuối cùng của số.
        \item C. Không đặt vào nữa.
    \end{itemize}
    \pause
    \textbf{Đáp án: B} (Kết quả: $8873$).
\end{frame}

\begin{frame}[fragile]{Bước 4: Chốt thuật toán (The Master Plan)}
    \begin{enumerate}
        \item \textbf{Duyệt:} Từng chữ số của $s$ từ trái sang phải.
        \item \textbf{So sánh:} Nếu chữ số $s[i] < d$, chèn $d$ vào trước $s[i]$.
        \item \textbf{Dừng:} In phần còn lại và kết thúc.
        \item \textbf{Trường hợp "Về bét":} In $d$ ở cuối cùng.
    \end{enumerate}

\begin{lstlisting}[language=text, caption=Ma gia thuan Viet]
Doc vao so luong test case
Voi moi test case:
    Doc n va d, xau s
    Bien 'da_chen' = False
    Duyet tung ky tu c trong chuoi s:
        Neu 'da_chen' la False VA d > c:
            In ra d
            Bien 'da_chen' = True
        In ra c
    Neu 'da_chen' van la False:
        In ra d
\end{lstlisting}
\end{frame}

\begin{frame}{Thử thách cuối cùng}
    Số $543$ và $d = 4$. Kết quả theo thuật toán là gì?
    \begin{itemize}
        \item A. 5443
        \item B. 4543
        \item C. 5434
    \end{itemize}
    \pause
    \begin{exampleblock}{Giải thích}
        1. Xét số 5: $4 > 5$ (Sai). \\
        2. Xét số 4: $4 > 4$ (Sai - lưu ý dấu $>$ chứ không phải $\ge$). \\
        3. Xét số 3: $4 > 3$ (Đúng). Chèn ngay! $\rightarrow$ \textbf{5443}.
    \end{exampleblock}
\end{frame}

\begin{frame}[fragile]{Bước 5: Hiện thực hóa ý tưởng (Implementation)}
    Vì số $s$ có thể lên tới $2 \cdot 10^5$ chữ số, ta dùng \textbf{String}.

\begin{lstlisting}[language=text]
Nhap so luong bo test (t)
Lap t lan:
    Nhap n, d, xau s
    Tao mot bien co 'da_chen' = false
    Duyet i tu 0 den n-1:
        Neu 'da_chen' == false VA d > (s[i] - '0'):
            In ra d
            'da_chen' = true
        In ra s[i]
    Neu 'da_chen' van la false:
        In ra d
\end{lstlisting}

    \textbf{Câu hỏi chốt hạ:} Tại sao cần \texttt{s[i] - '0'}?
    \begin{itemize}
        \item A. Biến ký tự (char) thành giá trị số để so sánh.
        \item B. Để xóa số 0 ở đầu.
        \item C. Quy tắc bắt buộc của Codeforces.
    \end{itemize}
    \pause
    \textbf{Đáp án: A}. Bạn có muốn tôi hướng dẫn code C++ hay Python không?
\end{frame}

\end{document}

