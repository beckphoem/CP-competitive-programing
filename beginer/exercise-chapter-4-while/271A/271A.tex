\documentclass{beamer}
\usepackage[utf8]{inputenc}
\usepackage[T5]{fontenc} % Bắt buộc để hiển thị tiếng Việt
\usepackage[vietnamese]{babel}
\usepackage{tcolorbox}
\usepackage{listings}
\usepackage{xcolor}
\usepackage{booktabs}
\usepackage{amsmath}

\usetheme{Madrid}

% Cấu hình màu sắc cho code/pseudocode
\definecolor{codegreen}{rgb}{0,0.6,0}
\definecolor{codegray}{rgb}{0.5,0.5,0.5}
\definecolor{codepurple}{rgb}{0.58,0,0.82}
\definecolor{backcolour}{rgb}{0.95,0.95,0.92}

\lstdefinestyle{mystyle}{
    backgroundcolor=\color{backcolour},   
    commentstyle=\color{codegreen},
    keywordstyle=\color{magenta},
    numberstyle=\tiny\color{codegray},
    stringstyle=\color{codepurple},
    basicstyle=\ttfamily\footnotesize,
    breakatwhitespace=false,         
    breaklines=true,                 
    captionpos=b,                    
    keepspaces=true,                 
    numbers=left,                    
    numbersep=5pt,                  
    showspaces=false,                
    showstringspaces=false,
    showtabs=false,                  
    tabsize=2
}
\lstset{style=mystyle}

% Thông tin tiêu đề
\title{Giải bài toán Codeforces 271A}
\subtitle{Beautiful Year - Tư duy Vét cạn \& Tách số}
\author{Slide Learning C++}
\date{\today}

\begin{document}

% Slide 1: Tiêu đề
\begin{frame}
    \titlepage
\end{frame}

% Slide 2: Giới thiệu
\begin{frame}{Giới thiệu}
    Chào bạn! Hôm nay chúng ta sẽ cùng "mổ xẻ" bài toán \textbf{Codeforces 271A - Beautiful Year}.
    
    \vspace{0.5cm}
    Đây là một bài toán kinh điển để rèn luyện tư duy \textbf{Brute Force (Vét cạn)} cơ bản.
    
    \begin{block}{Mục tiêu bài học}
        Hiểu bản chất thuật toán tìm kiếm và kỹ thuật xử lý các chữ số của một số nguyên.
    \end{block}
\end{frame}

% Slide 3: Bước 1 - Phẫu thuật đề bài
\begin{frame}{Bước 1: Phẫu thuật đề bài (Deconstruct)}
    Lõi của bài toán này gồm 2 quy tắc vàng:

    \begin{enumerate}
        \item \textbf{Input:} Một con số biểu thị năm (gọi là $y$).
        \item \textbf{Mục tiêu:} Tìm một con số \textbf{năm X} sao cho:
        \begin{itemize}
            \item $X$ phải \textbf{lớn hơn hẳn} $y$ (Strictly larger).
            \item $X$ là số \textbf{nhỏ nhất} thỏa mãn điều kiện.
            \item \textbf{Quan trọng nhất:} Tất cả các chữ số trong $X$ phải \textbf{khác nhau} (Distinct digits).
        \end{itemize}
    \end{enumerate}

    \begin{exampleblock}{Lộ trình tư duy}
        \begin{itemize}
            \item \textbf{Chunk 1:} Hiểu thế nào là "Năm đẹp".
            \item \textbf{Chunk 2:} Chiến thuật tìm kiếm (Brute Force).
            \item \textbf{Chunk 3:} Kỹ thuật tách số (Machine Logic).
        \end{itemize}
    \end{exampleblock}
\end{frame}

% Slide 4: Chunk 1 - Định nghĩa
\begin{frame}{Chunk 1: Định nghĩa "Năm đẹp"}
    Một năm được gọi là "đẹp" nếu như không có bất kỳ chữ số nào xuất hiện lặp lại.

    \begin{alertblock}{Ẩn dụ: Mã PIN hỏng bàn phím}
        Hãy nghĩ đến một mã PIN 4 số. Nếu bàn phím dính mực, bạn chỉ được phép ấn mỗi phím số \textbf{đúng 1 lần duy nhất}.
        \begin{itemize}
            \item Ấn \textbf{1998}: Số 9 bị ấn 2 lần $\to$ \textbf{Thua}.
            \item Ấn \textbf{2013}: Các số 2, 0, 1, 3 đều khác nhau $\to$ \textbf{Thắng}.
        \end{itemize}
    \end{alertblock}
\end{frame}

% Slide 5: Thử thách tư duy 1
\begin{frame}{Thử thách tư duy (Mental Check)}
    \textbf{Đề bài:} Giả sử năm đầu vào là $y = 1000$.
    \vspace{0.2cm}
    
    Tôi có 3 ứng cử viên cho "Năm đẹp tiếp theo". Hãy chọn đáp án đúng nhất:
    
    \begin{itemize}
        \item \textbf{A.} $1020$
        \item \textbf{B.} $1998$
        \item \textbf{C.} $1023$
    \end{itemize}

    \vspace{0.5cm}
    \pause
    
    \begin{block}{Đáp án: C (1023)}
        \begin{itemize}
            \item \textbf{A (1020):} Loại vì lặp số $0$.
            \item \textbf{B (1998):} Loại vì lặp số $9$, hơn nữa số này quá lớn (không phải nhỏ nhất).
            \item \textbf{C (1023):} Các số $1, 0, 2, 3$ hoàn toàn riêng biệt $\to$ \textbf{Đúng!}
        \end{itemize}
    \end{block}
\end{frame}

% Slide 6: Chunk 2 - Chiến thuật tìm kiếm
\begin{frame}{Chunk 2: Chiến thuật Brute Force}
    Đừng cố dùng công thức toán học phức tạp. Hãy dùng chiến thuật \textbf{"Gõ cửa từng nhà"}.

    \begin{exampleblock}{Tư duy chiến thuật}
        Tưởng tượng bạn đang đứng ở năm $y$. Để tìm năm đẹp nhỏ nhất lớn hơn $y$:
        \begin{enumerate}
            \item Bước lên 1 năm ($y = y + 1$).
            \item Kiểm tra xem năm mới này có "đẹp" không?
            \item \textbf{Nếu ĐẸP:} Dừng lại và Hét lên "Tìm thấy rồi!" (Print).
            \item \textbf{Nếu XẤU (trùng số):} Quay lại bước 1.
        \end{enumerate}
    \end{exampleblock}
    Vòng lặp này sẽ chạy mãi (Loop) cho đến khi tìm thấy đích đến.
\end{frame}

% Slide 7: Thử thách tư duy 2
\begin{frame}{Thử thách tư duy (Mental Check)}
    Chạy thử bằng "cơm" (chạy bằng não). Giả sử input là \textbf{198}.
    
    \vspace{0.3cm}
    \textbf{Quy trình "Gõ cửa từng nhà":}
    \begin{itemize}
        \item \textbf{Bước 1:} Tăng lên $199$. Đẹp hay xấu?
        \item \textbf{Bước 2:} Tăng lên $200$. Đẹp hay xấu?
        \item \textbf{Bước 3:} Tăng lên $201$. Đẹp hay xấu?
    \end{itemize}

    \pause
    \vspace{0.5cm}
    \begin{block}{Kết quả phân tích}
        \begin{itemize}
            \item $199$ $\to$ Trùng 9 $\to$ \textbf{Loại}.
            \item $200$ $\to$ Trùng 0 $\to$ \textbf{Loại}.
            \item $201$ $\to$ 2, 0, 1 khác nhau $\to$ \textbf{Chọn!}
        \end{itemize}
    \end{block}
\end{frame}

% Slide 8: Chunk 3 - Kỹ thuật Mổ xẻ con số
\begin{frame}{Chunk 3: Kỹ thuật tách số (Digit Extraction)}
    Máy tính lưu số $2013$ là một khối nguyên vẹn. Để kiểm tra trùng lặp, ta phải "chặt" số ra làm 4 phần: $a, b, c, d$.
    
    \vspace{0.3cm}
    \textbf{Công cụ toán học:}
    \begin{enumerate}
        \item \textbf{Chia lấy nguyên (/):} Loại bỏ số phía sau.
        \item \textbf{Chia lấy dư (\%):} Lấy số cuối cùng.
    \end{enumerate}

    \begin{exampleblock}{Ví dụ: Năm 1987 (abcd)}
        \begin{itemize}
            \item $a = 1987 / 1000 \to \mathbf{1}$
            \item $b = (1987 / 100) \% 10 \to 19 \% 10 \to \mathbf{9}$
            \item $c = (1987 / 10) \% 10 \to 198 \% 10 \to \mathbf{8}$
            \item $d = 1987 \% 10 \to \mathbf{7}$
        \end{itemize}
    \end{exampleblock}
\end{frame}

% Slide 9: Bẫy logic
\begin{frame}{Cảnh báo: Bẫy Logic (The Trap)}
    Khi so sánh các chữ số $a, b, c, d$, một sai lầm chết người là chỉ so sánh "hàng xóm" (ví dụ so $a$ với $b$, $b$ với $c$...).

    \begin{alertblock}{Ví dụ sai lầm: Số 1213}
        \begin{itemize}
            \item $1 \neq 2$ (OK)
            \item $2 \neq 1$ (OK)
            \item $1 \neq 3$ (OK)
            \item \textbf{Nhưng:} Số đầu ($1$) trùng với số thứ ba ($1$).
        \end{itemize}
    \end{alertblock}
    
    \textbf{Quy tắc đúng:} Mỗi chữ số phải đi "so găng" với \textbf{TẤT CẢ} các chữ số còn lại.
\end{frame}

% Slide 10: Thử thách tư duy 3
\begin{frame}{Thử thách tư duy (Mental Check)}
    Với 4 biến $a, b, c, d$. Để đảm bảo \textbf{không có bất kỳ cặp nào trùng nhau}, cần bao nhiêu phép so sánh ($!=$)?

    \begin{itemize}
        \item \textbf{A.} 3 phép so sánh
        \item \textbf{B.} 4 phép so sánh
        \item \textbf{C.} 6 phép so sánh
    \end{itemize}

    \pause
    \vspace{0.5cm}
    \begin{block}{Đáp án: C (6 phép so sánh)}
        Logic tổ hợp (Chọn 2 trong 4):
        \begin{enumerate}
            \item $a \neq b$, $a \neq c$, $a \neq d$
            \item $b \neq c$, $b \neq d$
            \item $c \neq d$
        \end{enumerate}
    \end{block}
\end{frame}

% Slide 11: Pseudocode
\begin{frame}[fragile]{Bản thiết kế thuật toán (Pseudocode)}
\begin{lstlisting}[language=C++, basicstyle=\small\ttfamily]
NHAP y;

LAP VO TAN (while true):
    y = y + 1; // Tang nam len 1
    
    // Tach so
    a = y / 1000;
    b = (y / 100) % 10;
    c = (y / 10) % 10;
    d = y % 10;
    
    // Kiem tra tat ca cac cap
    NEU (a!=b VA a!=c VA a!=d VA 
         b!=c VA b!=d VA 
         c!=d):
        IN ra y;
        DUNG vong lap (break);
    KET THUC NEU
KET THUC LAP
\end{lstlisting}
\end{frame}

% Slide 12: Tổng kết
\begin{frame}{Tổng kết}
    Chúng ta đã hoàn thành "Bản thiết kế" cho bài toán Beautiful Year:
    
    \begin{enumerate}
        \item \textbf{Vòng lặp:} Tăng dần năm lên ($y++$).
        \item \textbf{Tách số:} Dùng toán tử $/$ và $\%$ để lấy $a, b, c, d$.
        \item \textbf{Điều kiện:} So sánh 6 cặp để đảm bảo khác nhau hoàn toàn.
    \end{enumerate}

    \vspace{1cm}
    \begin{center}
        \textbf{Bạn đã sẵn sàng để viết Code chưa?}
        \\ \small (Hãy thử code bằng C++ hoặc Python dựa trên Pseudocode vừa rồi nhé!)
    \end{center}
\end{frame}

\end{document}