\documentclass{beamer}
\usepackage[utf8]{inputenc}
\usepackage[T5]{fontenc} % Quan trọng để hiển thị tiếng Việt đúng
\usepackage[vietnamese]{babel}
\usepackage{tcolorbox}
\usepackage{listings}
\usepackage{xcolor}
\usepackage{booktabs}

\usetheme{Madrid}
\usecolortheme{default}

% Định nghĩa màu sắc code
\definecolor{codegreen}{rgb}{0,0.6,0}
\definecolor{codegray}{rgb}{0.5,0.5,0.5}
\definecolor{codepurple}{rgb}{0.58,0,0.82}
\definecolor{backcolour}{rgb}{0.95,0.95,0.92}

\lstdefinestyle{mystyle}{
    backgroundcolor=\color{backcolour},   
    commentstyle=\color{codegreen},
    keywordstyle=\color{magenta},
    numberstyle=\tiny\color{codegray},
    stringstyle=\color{codepurple},
    basicstyle=\ttfamily\scriptsize,
    breakatwhitespace=false,         
    breaklines=true,                 
    captionpos=b,                    
    keepspaces=true,                 
    numbers=left,                    
    numbersep=4pt,                  
    showspaces=false,                
    showstringspaces=false,
    showtabs=false,                  
    tabsize=2,
    escapechar=@
}

\lstset{style=mystyle}

% Thông tin bài giảng
\title[CF 492A - Vanya and Cubes]{Codeforces 492A - Vanya and Cubes}
\subtitle{Phân tích tư duy và Giải thuật}
\author{Coach Tư Duy}
\date{\today}

\begin{document}

% Slide 1: Title
\begin{frame}
    \titlepage
\end{frame}

% Slide 2: Briefing
\begin{frame}{Bước 1: Tiếp nhận \& Phẫu thuật (Briefing)}
    \begin{block}{Tóm tắt đề bài}
        Bạn có $n$ viên gạch. Xây kim tự tháp cao nhất có thể với quy luật:
        \begin{itemize}
            \item Tầng 1 (trên cùng): 1 viên.
            \item Tầng 2: $1 + 2$ viên.
            \item Tầng 3: $1 + 2 + 3$ viên.
            \item ...
            \item Tầng $i$: $1 + 2 + ... + i$ viên.
        \end{itemize}
    \end{block}

    \vspace{0.5cm}
    \textbf{Lộ trình tư duy:}
    \begin{enumerate}
        \item \textbf{Chunk 1:} Giá tiền của từng tầng (Mỗi tầng tốn bao nhiêu?).
        \item \textbf{Chunk 2:} Tính tổng hóa đơn (Tổng gạch đã dùng).
        \item \textbf{Chunk 3:} Điểm dừng (Khi nào hết gạch?).
    \end{enumerate}
\end{frame}

% Slide 3: Chunk 1 - Logic
\begin{frame}{Bước 2: Chunk 1 - Giá của từng tầng}
    \begin{block}{Logic: Quy tắc cộng dồn (Rolling Sum)}
        Mỗi tầng giống như một "hòn tuyết lăn". Số gạch cần cho tầng $i$ bằng số gạch tầng trước cộng thêm $i$.
        \begin{itemize}
            \item Tầng 1: 1 viên.
            \item Tầng 2: (Tầng 1) + 2 = 3 viên.
            \item Tầng 3: (Tầng 2) + 3 = 6 viên.
        \end{itemize}
    \end{block}

    \begin{alertblock}{Lưu ý (Trap)}
        Đừng nhầm lẫn giữa:
        \begin{enumerate}
            \item Số gạch của \textbf{riêng tầng đó}.
            \item Tổng số gạch của \textbf{cả kim tự tháp}.
        \end{enumerate}
        Ở đây ta đang tính số gạch cho \textbf{riêng tầng đó}.
    \end{alertblock}
\end{frame}

% Slide 4: Chunk 1 - Mental Check
\begin{frame}{Thử thách tư duy: Chunk 1}
    \textbf{Câu hỏi:}
    Giả sử muốn xây \textbf{Tầng thứ 4} (Level 4). Chỉ riêng tầng này cần bao nhiêu viên gạch?
    
    \vspace{0.5cm}
    \begin{itemize}
        \item[A.] 4 viên
        \item[B.] 6 viên
        \item[C.] 10 viên
    \end{itemize}

    \pause
    \vspace{0.5cm}
    \begin{exampleblock}{Đáp án}
        \textbf{C. 10 viên.} \\
        \textit{Giải thích:} Tầng 3 tốn 6 viên. Tầng 4 sẽ tốn $6 + 4 = 10$ viên.
    \end{exampleblock}
\end{frame}

% Slide 5: Chunk 2 - Total Cost
\begin{frame}{Bước 2: Chunk 2 - Tính tổng "Hóa đơn"}
    \begin{block}{Logic: Tổng chi phí}
        Để xây tháp cao $h$ tầng, bạn cần tổng số gạch của tất cả các tầng cộng lại.
        $$ \text{Total} = \text{Gạch Tầng 1} + \text{Gạch Tầng 2} + ... + \text{Gạch Tầng } h $$
    \end{block}

    \begin{alertblock}{Bẫy lớn nhất!}
        Phải lấy \textbf{TỔNG số gạch đã dùng từ đầu} so sánh với $n$.
        \\ \textbf{KHÔNG ĐƯỢC} lấy số gạch của riêng tầng đang xây để so sánh.
    \end{alertblock}
\end{frame}

% Slide 6: Chunk 2 - Mental Check
\begin{frame}{Thử thách tư duy: Chunk 2}
    \textbf{Dữ liệu:} Tầng 1 (1 viên), Tầng 2 (3 viên), Tầng 3 (6 viên).
    
    \vspace{0.2cm}
    \textbf{Câu hỏi:} Để xây kim tự tháp hoàn chỉnh cao \textbf{3 tầng}, tổng cộng cần bao nhiêu gạch?

    \vspace{0.3cm}
    \begin{itemize}
        \item[A.] 6 viên
        \item[B.] 9 viên ($1+2+6$)
        \item[C.] 10 viên ($1+3+6$)
    \end{itemize}

    \pause
    \vspace{0.3cm}
    \begin{exampleblock}{Đáp án}
        \textbf{C. 10 viên.} \\
        \textit{Giải thích:} Tổng = 1 (Tầng 1) + 3 (Tầng 2) + 6 (Tầng 3) = 10 viên.
    \end{exampleblock}
\end{frame}

% Slide 7: Chunk 3 - Stop Condition
\begin{frame}{Bước 2: Chunk 3 - Điểm dừng (Simulation)}
    \begin{block}{Chiến thuật: Thử và Sai (Simulation)}
        Chúng ta không biết tháp cao bao nhiêu, nên sẽ xây từng tầng và kiểm tra túi tiền ($n$).
    \end{block}

    \vspace{0.3cm}
    \textbf{Quy trình lặp (Loop):}
    \begin{enumerate}
        \item Tính gạch cho tầng tiếp theo.
        \item \textbf{Kiểm tra:} Tổng gạch đang có ($n$) có đủ trả cho tầng này không?
        \begin{itemize}
            \item \textbf{Đủ:} Xây tiếp (Chiều cao + 1, Trừ bớt gạch).
            \item \textbf{Thiếu:} DỪNG LẠI NGAY!
        \end{itemize}
    \end{enumerate}

    \begin{alertblock}{Bẫy: Quá trớn (Off-by-one error)}
        Chỉ khi \textbf{chắc chắn đủ gạch} mới được đếm tầng đó vào chiều cao!
    \end{alertblock}
\end{frame}

% Slide 8: Chunk 3 - Mental Check (Trace)
\begin{frame}{Thử thách tư duy: Chạy bằng cơm ($n=25$)}
    Hãy kiểm tra với $n=25$. Kết quả chiều cao là bao nhiêu?

    \begin{table}[]
        \begin{tabular}{@{}lllll@{}}
            \toprule
            \textbf{Tầng} & \textbf{Chi phí tầng} & \textbf{Tổng chi phí} & \textbf{Trạng thái ($n=25$)} \\ \midrule
            1             & 1                     & 1                     & Đủ (Còn 24)                  \\
            2             & 3                     & $1+3=4$               & Đủ (Còn 21)                  \\
            3             & 6                     & $4+6=10$              & Đủ (Còn 15)                  \\
            4             & 10                    & $10+10=20$            & Đủ (Còn 5)                   \\ 
            \textbf{5}    & \textbf{15}           & \textbf{35}           & \textbf{THIẾU!}              \\ \bottomrule
        \end{tabular}
    \end{table}

    \pause
    \begin{exampleblock}{Kết quả}
        \textbf{Đáp án: A. 4 tầng.} \\
        (Tầng 5 cần tổng 35 viên, ta chỉ có 25 viên $\to$ Không xây được).
    \end{exampleblock}
\end{frame}

% Slide 9: Solution Construction
\begin{frame}[fragile]{Bước 3: Tổng kết \& Mã giả (Pseudocode)}
    \begin{block}{Giải thuật (Algorithm)}
        \begin{enumerate}
            \item \textbf{Chuẩn bị:} 
                \begin{itemize}
                    \item \verb|height = 0| (Chiều cao)
                    \item \verb|current_level_cost = 0| (Gạch tầng hiện tại)
                    \item \verb|n| (Input)
                \end{itemize}
            \item \textbf{Vòng lặp (While):}
                \begin{itemize}
                    \item Phí tầng kế: \verb|next_cost = current_level_cost + (height + 1)|
                    \item \textbf{Kiểm tra:} 
                    \item Nếu \verb|n >= next_cost|: 
                        \begin{itemize}
                             \item \verb|n = n - next_cost|
                             \item \verb|current_level_cost = next_cost|
                             \item \verb|height++|
                        \end{itemize}
                    \item Nếu \verb|n < next_cost|: \textbf{BREAK}.
                \end{itemize}
            \item \textbf{In kết quả:} \verb|height|.
        \end{enumerate}
    \end{block}
\end{frame}

% Slide 10: Skeleton Code
\begin{frame}[fragile]{Mẫu Code (Skeleton)}
    \begin{lstlisting}[language=C++]
#include <iostream>
using namespace std;

int main() {
    int n;
    cin >> n;

    int height = 0;
    int current_level_cost = 0; 
    
    // Bat dau xay dung
    while (true) {
        // 1. Tinh so gach can cho tang tiep theo
        int next_level = height + 1;
        int cost_for_next = current_level_cost + next_level;

        // 2. Kiem tra xem co du gach khong?
        if (n >= cost_for_next) {
            // ??? (Dien code vao day: Tru gach, tang chieu cao)
            // Cap nhat current_level_cost
        } else {
            // ??? (Dien code vao day: Dung lai)
        }
    }

    cout << height << endl;
    return 0;
}
    \end{lstlisting}
\end{frame}

% Slide 11: Next Step
\begin{frame}{Next Step}
    Mọi thứ đã nằm trong đầu bạn. Bạn muốn làm gì tiếp theo?

    \vspace{0.5cm}
    \begin{enumerate}
        \item \textbf{Tự viết code:} Mở IDE lên và code ngay bằng C++/Python.
        \item \textbf{Cần hỗ trợ:} Sử dụng mẫu Skeleton Code ở slide trước.
    \end{enumerate}
    
    \vspace{1cm}
    \centering \textit{"Logic trước, Code sau - Đó là cách của Master."}
\end{frame}

\end{document}