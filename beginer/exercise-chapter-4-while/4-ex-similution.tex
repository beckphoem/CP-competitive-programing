\documentclass{beamer}

% --- USER PACKAGES & PREAMBLE ---
\usepackage[utf8]{inputenc}
\usepackage[vietnamese]{babel}
\usepackage{tcolorbox}
\usepackage{listings}
\usepackage{xcolor}
\usepackage{hyperref} % Added for clickable links

\usetheme{Madrid}
\usecolortheme{default}

% Colors for code blocks (kept as requested, though not used for text slides)
\definecolor{codegreen}{rgb}{0,0.6,0}
\definecolor{codegray}{rgb}{0.5,0.5,0.5}
\definecolor{codepurple}{rgb}{0.58,0,0.82}
\definecolor{backcolour}{rgb}{0.95,0.95,0.92}

\lstdefinestyle{mystyle}{
    backgroundcolor=\color{backcolour},   
    commentstyle=\color{codegreen},
    keywordstyle=\color{magenta},
    numberstyle=\tiny\color{codegray},
    stringstyle=\color{codepurple},
    basicstyle=\ttfamily\scriptsize,
    breakatwhitespace=false,         
    breaklines=true,                 
    captionpos=b,                    
    keepspaces=true,                 
    numbers=left,                    
    numbersep=4pt,                  
    showspaces=false,                
    showstringspaces=false,
    showtabs=false,                  
    tabsize=2,
    escapechar=@
}

\lstset{style=mystyle}

% --- TITLE INFO ---
\title[Codeforces Training]{Tuyển tập 20 Bài Tập Simulation \& Constructive Algorithms}
\subtitle{Dành cho Rank Newbie (800 - 900 Elo)}
\author{Học C++ cùng Codeforces}
\date{}

% --- DOCUMENT START ---
\begin{document}

% Slide 1: Title
\begin{frame}
    \titlepage
\end{frame}

% Slide 2: Table of Contents (Optional)
\begin{frame}{Mục lục}
    \tableofcontents
\end{frame}

% --- SECTION 1 ---
\section{Phần 1: Bài 1 - 10}

% Problem 1
\begin{frame}{1. Bear and Big Brother (791A)}
    \begin{tcolorbox}[colback=white, colframe=blue!75!black, title=Thông tin bài tập]
        \textbf{Link:} \url{https://codeforces.com/problemset/problem/791/A}
        
        \vspace{0.5cm}
        \textbf{Tóm tắt:} Gấu Limak nặng $a$ và Bob nặng $b$ ($a \le b$). Sau mỗi năm, Limak tăng cân gấp 3, Bob tăng cân gấp 2. Hỏi sau bao nhiêu năm Limak sẽ nặng hơn Bob?
    \end{tcolorbox}
\end{frame}

% Problem 2
\begin{frame}{2. Vanya and Cubes (492A)}
    \begin{tcolorbox}[colback=white, colframe=blue!75!black, title=Thông tin bài tập]
        \textbf{Link:} \url{https://codeforces.com/problemset/problem/492/A}
        
        \vspace{0.5cm}
        \textbf{Tóm tắt:} Xây kim tự tháp từ $n$ khối lập phương. Tầng 1 cần 1 khối, tầng 2 cần 3 khối ($1+2$), tầng $i$ cần tổng các số từ 1 đến $i$. Tìm chiều cao tối đa xây được.
    \end{tcolorbox}
\end{frame}

% Problem 3
\begin{frame}{3. Wrong Subtraction (977A)}
    \begin{tcolorbox}[colback=white, colframe=blue!75!black, title=Thông tin bài tập]
        \textbf{Link:} \url{https://codeforces.com/problemset/problem/977/A}
        
        \vspace{0.5cm}
        \textbf{Tóm tắt:} Giảm số $n$ đi $k$ lần. Quy tắc: nếu tận cùng là 0 thì chia 10, nếu khác 0 thì trừ 1.
    \end{tcolorbox}
\end{frame}

% Problem 4
\begin{frame}{4. Elephant (617A)}
    \begin{tcolorbox}[colback=white, colframe=blue!75!black, title=Thông tin bài tập]
        \textbf{Link:} \url{https://codeforces.com/problemset/problem/617/A}
        
        \vspace{0.5cm}
        \textbf{Tóm tắt:} Con voi cần đi đến điểm $x$. Mỗi bước đi được 1, 2, 3, 4 hoặc 5 bước. Tìm số bước tối thiểu.
    \end{tcolorbox}
\end{frame}

% Problem 5
\begin{frame}{5. Soldier and Bananas (546A)}
    \begin{tcolorbox}[colback=white, colframe=blue!75!black, title=Thông tin bài tập]
        \textbf{Link:} \url{https://codeforces.com/problemset/problem/546/A}
        
        \vspace{0.5cm}
        \textbf{Tóm tắt:} Mua $w$ quả chuối. Quả thứ $i$ giá $i \times k$. Tính số tiền phải vay.
    \end{tcolorbox}
\end{frame}

% Problem 6
\begin{frame}{6. Tram (116A)}
    \begin{tcolorbox}[colback=white, colframe=blue!75!black, title=Thông tin bài tập]
        \textbf{Link:} \url{https://codeforces.com/problemset/problem/116/A}
        
        \vspace{0.5cm}
        \textbf{Tóm tắt:} Tàu điện qua $n$ trạm. Mỗi trạm có người lên và xuống. Tìm sức chứa tối thiểu cần thiết (số khách cực đại tại một thời điểm).
    \end{tcolorbox}
\end{frame}

% Problem 7
\begin{frame}{7. Beautiful Year (271A)}
    \begin{tcolorbox}[colback=white, colframe=blue!75!black, title=Thông tin bài tập]
        \textbf{Link:} \url{https://codeforces.com/problemset/problem/271/A}
        
        \vspace{0.5cm}
        \textbf{Tóm tắt:} Tìm năm nhỏ nhất lớn hơn năm $y$ đã cho mà các chữ số của nó đôi một khác nhau.
    \end{tcolorbox}
\end{frame}

% Problem 8
\begin{frame}{8. Near Lucky Number (110A)}
    \begin{tcolorbox}[colback=white, colframe=blue!75!black, title=Thông tin bài tập]
        \textbf{Link:} \url{https://codeforces.com/problemset/problem/110/A}
        
        \vspace{0.5cm}
        \textbf{Tóm tắt:} Đếm số lượng chữ số may mắn (4 và 7) trong số nguyên $n$. Kiểm tra xem số lượng đó có phải số may mắn không.
    \end{tcolorbox}
\end{frame}

% Problem 9
\begin{frame}{9. Word (59A)}
    \begin{tcolorbox}[colback=white, colframe=blue!75!black, title=Thông tin bài tập]
        \textbf{Link:} \url{https://codeforces.com/problemset/problem/59/A}
        
        \vspace{0.5cm}
        \textbf{Tóm tắt:} Đếm số chữ hoa và thường. Nếu hoa $>$ thường thì in toàn bộ hoa, ngược lại in thường.
    \end{tcolorbox}
\end{frame}

% Problem 10
\begin{frame}{10. Translation (41A)}
    \begin{tcolorbox}[colback=white, colframe=blue!75!black, title=Thông tin bài tập]
        \textbf{Link:} \url{https://codeforces.com/problemset/problem/41/A}
        
        \vspace{0.5cm}
        \textbf{Tóm tắt:} Kiểm tra chuỗi $t$ có phải là đảo ngược của chuỗi $s$ không.
    \end{tcolorbox}
\end{frame}

% --- SECTION 2 ---
\section{Phần 2: Bài 11 - 20}

% Problem 11
\begin{frame}{11. Anton and Danik (734A)}
    \begin{tcolorbox}[colback=white, colframe=blue!75!black, title=Thông tin bài tập]
        \textbf{Link:} \url{https://codeforces.com/problemset/problem/734/A}
        
        \vspace{0.5cm}
        \textbf{Tóm tắt:} Đếm số trận thắng của Anton ('A') và Danik ('D'). So sánh xem ai thắng nhiều hơn.
    \end{tcolorbox}
\end{frame}

% Problem 12
\begin{frame}{12. Queue at the School (266B)}
    \begin{tcolorbox}[colback=white, colframe=blue!75!black, title=Thông tin bài tập]
        \textbf{Link:} \url{https://codeforces.com/problemset/problem/266/B}
        
        \vspace{0.5cm}
        \textbf{Tóm tắt:} Hàng đợi có nam (B) và nữ (G). Mỗi giây, nếu B đứng trước G thì đổi chỗ. Mô phỏng sau $t$ giây.
    \end{tcolorbox}
\end{frame}

% Problem 13
\begin{frame}{13. In Search of an Easy Problem (1030A)}
    \begin{tcolorbox}[colback=white, colframe=blue!75!black, title=Thông tin bài tập]
        \textbf{Link:} \url{https://codeforces.com/problemset/problem/1030/A}
        
        \vspace{0.5cm}
        \textbf{Tóm tắt:} Khảo sát độ khó. Nếu có ít nhất 1 người nói khó (1) thì bài toán là KHÓ.
    \end{tcolorbox}
\end{frame}

% Problem 14
\begin{frame}{14. George and Accommodation (467A)}
    \begin{tcolorbox}[colback=white, colframe=blue!75!black, title=Thông tin bài tập]
        \textbf{Link:} \url{https://codeforces.com/problemset/problem/467/A}
        
        \vspace{0.5cm}
        \textbf{Tóm tắt:} Đếm số phòng ký túc xá còn trống ít nhất 2 chỗ.
    \end{tcolorbox}
\end{frame}

% Problem 15
\begin{frame}{15. Stones on the Table (266A)}
    \begin{tcolorbox}[colback=white, colframe=blue!75!black, title=Thông tin bài tập]
        \textbf{Link:} \url{https://codeforces.com/problemset/problem/266/A}
        
        \vspace{0.5cm}
        \textbf{Tóm tắt:} Đếm số lượng đá cần bỏ bớt để không có hai viên đá màu giống nhau nằm cạnh nhau.
    \end{tcolorbox}
\end{frame}

% Problem 16
\begin{frame}{16. Magnets (344A)}
    \begin{tcolorbox}[colback=white, colframe=blue!75!black, title=Thông tin bài tập]
        \textbf{Link:} \url{https://codeforces.com/problemset/problem/344/A}
        
        \vspace{0.5cm}
        \textbf{Tóm tắt:} Đếm số nhóm nam châm liên kết với nhau (do hút nhau 01-10). Nhóm mới hình thành khi cực thay đổi.
    \end{tcolorbox}
\end{frame}

% Problem 17
\begin{frame}{17. Drinks (200B)}
    \begin{tcolorbox}[colback=white, colframe=blue!75!black, title=Thông tin bài tập]
        \textbf{Link:} \url{https://codeforces.com/problemset/problem/200/B}
        
        \vspace{0.5cm}
        \textbf{Tóm tắt:} Tính nồng độ phần trăm trung bình của cocktail từ $n$ loại nước.
    \end{tcolorbox}
\end{frame}

% Problem 18
\begin{frame}{18. Hulk (705A)}
    \begin{tcolorbox}[colback=white, colframe=blue!75!black, title=Thông tin bài tập]
        \textbf{Link:} \url{https://codeforces.com/problemset/problem/705/A}
        
        \vspace{0.5cm}
        \textbf{Tóm tắt:} In ra chuỗi cảm xúc "I hate that I love that..." theo lớp $n$.
    \end{tcolorbox}
\end{frame}

% Problem 19
\begin{frame}{19. Calculating Function (486A)}
    \begin{tcolorbox}[colback=white, colframe=blue!75!black, title=Thông tin bài tập]
        \textbf{Link:} \url{https://codeforces.com/problemset/problem/486/A}
        
        \vspace{0.5cm}
        \textbf{Tóm tắt:} Tính $f(n) = -1 + 2 - 3 + \dots + (-1)^n n$.
    \end{tcolorbox}
\end{frame}

% Problem 20
\begin{frame}{20. Police Recruits (427A)}
    \begin{tcolorbox}[colback=white, colframe=blue!75!black, title=Thông tin bài tập]
        \textbf{Link:} \url{https://codeforces.com/problemset/problem/427/A}
        
        \vspace{0.5cm}
        \textbf{Tóm tắt:} Nhận sự kiện: số dương là tuyển cảnh sát, -1 là có tội phạm. Đếm số tội phạm không bị bắt (khi không có cảnh sát).
    \end{tcolorbox}
\end{frame}

\begin{frame}
    \begin{center}
        \Huge \textcolor{blue!75!black}{\textbf{Happy Coding!}}
    \end{center}
\end{frame}

\end{document}