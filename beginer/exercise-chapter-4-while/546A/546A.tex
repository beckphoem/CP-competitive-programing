\documentclass{beamer}
\usepackage[utf8]{inputenc}
\usepackage[T5]{fontenc} % Bắt buộc để hiển thị tiếng Việt
\usepackage[vietnamese]{babel}
\usepackage{tcolorbox}
\usepackage{listings}
\usepackage{xcolor}
\usepackage{booktabs}
\usepackage{amsmath}

\usetheme{Madrid}

% Cấu hình màu sắc cho code (nếu cần dùng sau này)
\definecolor{codegreen}{rgb}{0,0.6,0}
\definecolor{codegray}{rgb}{0.5,0.5,0.5}
\definecolor{codepurple}{rgb}{0.58,0,0.82}
\definecolor{backcolour}{rgb}{0.95,0.95,0.92}

\lstdefinestyle{mystyle}{
    backgroundcolor=\color{backcolour},   
    commentstyle=\color{codegreen},
    keywordstyle=\color{magenta},
    numberstyle=\tiny\color{codegray},
    stringstyle=\color{codepurple},
    basicstyle=\ttfamily\footnotesize,
    breakatwhitespace=false,         
    breaklines=true,                 
    captionpos=b,                    
    keepspaces=true,                 
    numbers=left,                    
    numbersep=5pt,                  
    showspaces=false,                
    showstringspaces=false,
    showtabs=false,                  
    tabsize=2
}
\lstset{style=mystyle}

% Thông tin slide
\title{Codeforces 546A - Soldier and Bananas}
\subtitle{Tư duy thuật toán \& Cấp số cộng}
\author{Coach Tư duy thuật toán}
\date{\today}

\begin{document}

% --- Slide Tiêu đề ---
\begin{frame}
    \titlepage
\end{frame}

% --- Slide Giới thiệu ---
\begin{frame}{Giới thiệu}
    \begin{block}{Lời chào từ Coach}
        Chào mừng bạn. Tôi là Coach Tư duy thuật toán của bạn đây. \\
        Đã nhận yêu cầu: \textbf{Codeforces 546A - Soldier and Bananas}.
    \end{block}

    \vspace{1em}

    \begin{itemize}
        \item \textbf{Đánh giá:} Bài toán nhập môn kinh điển (Rate 800).
        \item \textbf{Kiến thức cốt lõi:} 
        \begin{itemize}
            \item Cấp số cộng (Arithmetic Progression).
            \item Xử lý số âm (Negative handling).
        \end{itemize}
    \end{itemize}
    
    Chúng ta sẽ không code vội. Hãy mổ xẻ nó trước.
\end{frame}

% --- Slide Bước 1: Briefing ---
\begin{frame}{BƯỚC 1: TIẾP NHẬN \& PHẪU THUẬT}
    \begin{block}{Tóm tắt đề bài (Ngôn ngữ con người)}
        Bài toán này thực chất là một bài toán đi chợ với quy luật giá tăng dần:
        \begin{itemize}
            \item Bạn muốn mua $w$ quả chuối.
            \item Quả thứ \textbf{1} giá: $1k$ đô.
            \item Quả thứ \textbf{2} giá: $2k$ đô.
            \item Quả thứ \textbf{3} giá: $3k$ đô.
            \item ... Cứ thế tăng lên theo cấp số nhân với $k$.
            \item Trong túi bạn đang có sẵn $n$ đô.
        \end{itemize}
        \textbf{Mục tiêu:} Tính số tiền bạn \textbf{phải vay thêm} để mua đủ $w$ quả.
    \end{block}
\end{frame}

% --- Slide Input & Roadmap ---
\begin{frame}{Input \& Lộ trình tư duy}
    \begin{exampleblock}{Các nhân vật chính (Input)}
        \begin{itemize}
            \item $k$: Giá gốc (hệ số giá).
            \item $n$: Tiền có sẵn (ví tiền).
            \item $w$: Số lượng cần mua.
        \end{itemize}
    \end{exampleblock}

    \vspace{1em}

    \textbf{Lộ trình tư duy (Roadmap):}
    Để giải quyết bài này, chúng ta cần đi qua 3 mảnh ghép (Chunks):
    \begin{enumerate}
        \item \textbf{Chunk 1:} Hiểu quy luật tính tiền (Tư duy quy nạp).
        \item \textbf{Chunk 2:} Tính tổng hóa đơn (Tư duy tối ưu toán học).
        \item \textbf{Chunk 3:} Tính số tiền nợ \& Bẫy Logic (Edge Cases).
    \end{enumerate}
\end{frame}

% --- Slide Chunk 1: Quy luật ---
\begin{frame}{CHUNK 1: QUY LUẬT "CẦU THANG GIÁ CẢ"}
    \begin{exampleblock}{Mô hình Cầu thang}
        Giá chuối không cố định mà tăng dần như cầu thang:
        \begin{itemize}
            \item Bậc 1 (Quả thứ 1): Giá gốc là $1$ lần $k$.
            \item Bậc 2 (Quả thứ 2): Giá tăng lên $2$ lần $k$.
            \item Bậc 3 (Quả thứ 3): Giá tăng lên $3$ lần $k$.
            \item ...
            \item Bậc $i$ (Quả thứ $i$): Giá là $i \times k$.
        \end{itemize}
    \end{exampleblock}
    Càng mua nhiều, giá từng quả càng đắt đỏ.
\end{frame}

% --- Slide Mental Check 1 (Quiz) ---
\begin{frame}{THỬ THÁCH TƯ DUY (MENTAL CHECK)}
    \begin{alertblock}{Câu hỏi}
        Giả sử hệ số giá $k = 3$. \\
        Tôi muốn bạn tính giá tiền của \textbf{riêng quả chuối thứ 4} (chỉ quả đó thôi, không phải tổng).
        
        \begin{itemize}
            \item A. 3 đô
            \item B. 9 đô
            \item C. 12 đô
        \end{itemize}
    \end{alertblock}

    \pause
    
    \begin{block}{Đáp án: C}
        Chính xác! Giá quả thứ 4 là $4 \times 3 = 12$. \\
        Công thức: Giá quả thứ $i = i \times k$.
    \end{block}
\end{frame}

% --- Slide Chunk 2: Tính tổng ---
\begin{frame}{CHUNK 2: TÍNH TỔNG HÓA ĐƠN (SUMMATION)}
    Công thức tính tổng tiền cho $w$ quả:
    $$ Total = 1k + 2k + 3k + \dots + wk $$

    \begin{block}{Nhìn kỹ nhé (Pattern Recognition)}
        Nếu ta đặt $k$ ra ngoài làm nhân tử chung, ta sẽ có:
        $$ Total = k \times (1 + 2 + 3 + \dots + w) $$
    \end{block}

    Cụm $(1 + 2 + \dots + w)$ là tổng các số tự nhiên liên tiếp.
    \begin{itemize}
        \item \textit{Trong toán học:} Đây là cấp số cộng.
        \item \textit{Trong lập trình:} Dùng vòng lặp hoặc công thức $\frac{w(w+1)}{2}$.
    \end{itemize}
\end{frame}

% --- Slide Mental Check 2 (Quiz) ---
\begin{frame}{THỬ THÁCH TƯ DUY (MENTAL CHECK)}
    \begin{alertblock}{Câu hỏi}
        Vẫn giữ $k = 3$. Lần này bạn muốn mua \textbf{4 quả} ($w = 4$). \\
        Hãy tính \textbf{Tổng số tiền} bạn phải trả cho cả 4 quả này là bao nhiêu?
        
        \begin{itemize}
            \item A. 12
            \item B. 30
            \item C. 24
        \end{itemize}
    \end{alertblock}

    \pause

    \begin{block}{Đáp án: B}
        Quá chuẩn! Bạn đã cộng dồn rất đúng: \\
        $3 \times (1 + 2 + 3 + 4) = 3 \times 10 = 30$.
    \end{block}
\end{frame}

% --- Slide Chunk 3: Bẫy logic ---
\begin{frame}{CHUNK 3: ĐIỀU KIỆN VAY MƯỢN \& CÁI BẪY SỐ ÂM}
    Logic thông thường:
    $$ \text{Tiền thiếu} = \text{Tổng chi phí} - \text{Tiền đang có} $$
    $$ \text{Result} = Total - n $$

    Nhưng cuộc đời không phải lúc nào cũng cần vay mượn.

    \begin{exampleblock}{Tình huống thực tế}
        Nếu tiền trong túi ($n$) lớn hơn hóa đơn ($Total$), bạn không cần vay ai cả. Số tiền nợ phải là 0, chứ không phải số âm.
    \end{exampleblock}
\end{frame}

% --- Slide Trap Check (Quiz) ---
\begin{frame}{THỬ THÁCH TƯ DUY (TRAP CHECK)}
    \begin{alertblock}{Câu hỏi}
        Giả sử hóa đơn là \textbf{30 đô}. Bạn có \textbf{40 đô} trong ví ($n=40$).
        Máy tính cần in ra (số tiền phải vay) là bao nhiêu?
        
        \begin{itemize}
            \item A. -10 (Vì $30 - 40 = -10$)
            \item B. 0 (Không cần vay gì cả)
            \item C. 10 (Vẫn vay cho chắc?)
        \end{itemize}
    \end{alertblock}

    \pause

    \begin{block}{Đáp án: B}
        Tuyệt vời! Bạn đã tránh được cái "bẫy" phổ biến nhất. \\
        \textbf{Tại sao lại là 0?} Không ai nói "Tôi phải vay âm 10 đô" cả. Nếu $Total \le n$, kết quả là 0.
    \end{block}
\end{frame}

% --- Slide Tổng kết thuật toán ---
\begin{frame}{TỔNG KẾT CHIẾN THUẬT (THE ALGORITHM)}
    \begin{enumerate}
        \item \textbf{Nhập dữ liệu (Input):} Lấy 3 số $k, n, w$.
        
        \item \textbf{Tính Tổng hóa đơn (Total Cost):}
        \begin{itemize}
            \item Cách 1 (Loop): Vòng lặp từ 1 đến $w$, cộng dồn.
            \item Cách 2 (Math - Khuyên dùng): 
            $$ Total = k \times \frac{w \times (w + 1)}{2} $$
        \end{itemize}

        \item \textbf{Tính tiền nợ (Result):}
        \begin{itemize}
            \item Tính độ lệch: $diff = Total - n$.
            \item \textbf{Quyết định (Decision):}
            \begin{itemize}
                \item Nếu $diff > 0 \rightarrow$ In ra $diff$.
                \item Nếu $diff \le 0 \rightarrow$ In ra $0$.
            \end{itemize}
        \end{itemize}
    \end{enumerate}

    \begin{block}{Công thức Pro (One-liner)}
        $$ \text{Result} = \max(0, Total - n) $$
    \end{block}
\end{frame}

% --- Slide Kết thúc ---
\begin{frame}{BƯỚC TIẾP THEO}
    \begin{exampleblock}{Hành động ngay}
        Bây giờ bạn đã có đủ tư duy logic trong đầu. \\
        \vspace{0.5em}
        \textbf{Nhiệm vụ của bạn:} Hãy viết code (C++, Python hoặc ngôn ngữ bạn chọn) và nộp thử.
    \end{exampleblock}

    \vspace{1em}
    Bạn có muốn dán đoạn code của bạn vào đây để tôi review không?
\end{frame}

\end{document}