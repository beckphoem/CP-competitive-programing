\documentclass{beamer}
\usepackage[utf8]{inputenc}
\usepackage[T5]{fontenc} % Bắt buộc để hiển thị tiếng Việt
\usepackage[vietnamese]{babel}
\usepackage{tcolorbox}
\usepackage{listings}
\usepackage{xcolor}
\usepackage{booktabs}
\usetheme{Madrid}

% Cấu hình hiển thị code/giả mã
\lstset{
    basicstyle=\ttfamily\small,
    keywordstyle=\color{blue}\bfseries,
    stringstyle=\color{red},
    commentstyle=\color{green!60!black},
    numbers=left,
    numberstyle=\tiny,
    stepnumber=1,
    numbersep=5pt,
    backgroundcolor=\color{white},
    showspaces=false,
    showstringspaces=false,
    frame=single,
    tabsize=2,
    breaklines=true
}

\title{Codeforces 110A - Nearly Lucky Number}
\subtitle{Rèn luyện kỹ năng đọc hiểu và cài đặt (Implementation)}
\author{Slide Learning CPP}
\date{\today}

\begin{document}

% Slide Tiêu đề
\begin{frame}
    \titlepage
\end{frame}

% Slide Giới thiệu
\begin{frame}{Giới thiệu}
    \begin{block}{Mục tiêu bài học}
        Hôm nay chúng ta sẽ cùng "mổ xẻ" bài toán \textbf{Codeforces 110A - Nearly Lucky Number}.
        Đây là bài toán tuyệt vời để:
        \begin{itemize}
            \item Rèn luyện kỹ năng đọc hiểu định nghĩa.
            \item Tránh các bẫy tư duy lối mòn khi xử lý số học.
        \end{itemize}
    \end{block}
\end{frame}

% Bước 1: Tiếp nhận
\begin{frame}{Bước 1: Tiếp nhận \& Phẫu thuật (Briefing)}
    Hãy nhìn vào lõi vấn đề dưới ngôn ngữ "người thường":

    \begin{exampleblock}{Định nghĩa 1: Số May Mắn (Lucky Number)}
        Là số dương mà trong biểu diễn thập phân \textbf{CHỈ} chứa các chữ số \textbf{4} và \textbf{7}.
        \begin{itemize}
            \item \textbf{Đúng:} 4, 7, 47, 744, 477...
            \item \textbf{Sai:} 5, 17, 467...
        \end{itemize}
    \end{exampleblock}

    \begin{alertblock}{Định nghĩa 2: Số Gần May Mắn (Nearly Lucky Number)}
        Là số nguyên $n$ bất kỳ, sao cho \textbf{số lượng} các chữ số may mắn có trong $n$ phải là một \textbf{Số May Mắn}.
    \end{alertblock}
\end{frame}

% Lộ trình tư duy
\begin{frame}{Lộ trình tư duy (Chunks)}
    Chúng ta chia bài toán thành 2 mảnh ghép:

    \begin{enumerate}
        \item \textbf{Chunk 1: Sàng lọc và Đếm.}
        \begin{itemize}
            \item Tìm xem trong $n$ có bao nhiêu chữ số 4 và 7.
        \end{itemize}
        
        \item \textbf{Chunk 2: Thẩm định kết quả.}
        \begin{itemize}
            \item Kiểm tra xem con số vừa đếm được có phải là số may mắn hay không.
        \end{itemize}
    \end{enumerate}
\end{frame}

% Chunk 1: Logic
\begin{frame}{Chunk 1: Sàng lọc và Đếm}
    \begin{exampleblock}{Logic (Tư duy ẩn dụ)}
        Tưởng tượng số nguyên $n$ là một \textbf{đoàn tàu dài}. Mỗi toa là một chữ số.
        Bạn là người soát vé chỉ quan tâm đến \textbf{"Khách VIP"}:
        \begin{itemize}
            \item Khách VIP: Số \textbf{4} và Số \textbf{7}.
            \item Các số khác (0, 1, 2, 3, 5...): Lờ đi.
        \end{itemize}
        \textbf{Hành động:} Cầm một cái bấm đếm số. Cứ gặp 1 khách VIP $\rightarrow$ Bấm "Tạch" (cộng thêm 1).
    \end{exampleblock}
\end{frame}

% Cảnh báo quan trọng (Trap)
\begin{frame}{Cảnh báo quan trọng!}
    \begin{alertblock}{Bẫy (Trap)}
        Đề bài cho $n$ có thể lên tới $10^{18}$ (18 chữ số).
        \begin{itemize}
            \item Dùng toán học (chia lấy dư, chia nguyên) vẫn được nhưng phức tạp.
            \item \textbf{Tư duy đơn giản hơn:} Đoàn tàu bản chất là một \textbf{Chuỗi ký tự (String)}.
        \end{itemize}
    \end{alertblock}
    
    Việc duyệt qua một chuỗi ký tự dễ hình dung hơn nhiều so với xử lý số nguyên lớn.
\end{frame}

% Thử thách tư duy
\begin{frame}[fragile]{Thử thách tư duy (Mental Check)}
    Giả sử đoàn tàu có nội dung: \texttt{n = 10047704}
    
    \textbf{Câu hỏi:}
    \begin{enumerate}
        \item Bạn tìm thấy những "khách VIP" nào?
        \item Kết thúc chuyến tàu, biến đếm hiển thị số mấy?
    \end{enumerate}

    \pause
    \vspace{0.5cm}
    \begin{block}{Phân tích sai lầm thường gặp}
        \begin{enumerate}
            \item \textbf{Nhầm lẫn "Khách VIP":} Số 1 không phải là số may mắn. Chỉ đếm 4 và 7.
            \item \textbf{Nhầm lẫn ĐẾM và CỘNG:} Chúng ta đếm \textbf{số lượng} (tăng 1 đơn vị), không cộng giá trị của các con số lại.
        \end{enumerate}
    \end{block}
\end{frame}

% Làm lại Chunk 1
\begin{frame}{Làm lại Chunk 1 (Sửa sai)}
    Xét lại: \texttt{n = 10047704}
    
    \begin{enumerate}
        \item \texttt{1} $\rightarrow$ Bỏ qua.
        \item \texttt{0} $\rightarrow$ Bỏ qua.
        \item \texttt{0} $\rightarrow$ Bỏ qua.
        \item \texttt{4} $\rightarrow$ VIP! $\rightarrow$ Bấm đếm: \textbf{1}.
        \item \texttt{7} $\rightarrow$ VIP! $\rightarrow$ Bấm đếm: \textbf{2}.
        \item \texttt{7} $\rightarrow$ VIP! $\rightarrow$ Bấm đếm: \textbf{3}.
        \item \texttt{0} $\rightarrow$ Bỏ qua.
        \item \texttt{4} $\rightarrow$ VIP! $\rightarrow$ Bấm đếm: \textbf{4}.
    \end{enumerate}
    
    \textbf{Kết quả cuối cùng trên máy đếm là: 4.}
\end{frame}

% Thử thách lại (Re-Test)
\begin{frame}{Thử thách lại (Re-Test)}
    Để chắc chắn bạn đã nắm được việc \textbf{ĐẾM}, hãy xem xét các trường hợp sau:

    \begin{block}{Trường hợp 1: Input $n = 47$}
        \begin{itemize}
            \item Số lượng chữ số may mắn là bao nhiêu?
            \item \pause \textbf{Đáp án: 2}. (Số 2 không phải là số may mắn $\rightarrow$ \textbf{NO}).
        \end{itemize}
    \end{block}
    \pause
    
    \begin{block}{Trường hợp 2: Input $n$ toàn số 7 (10 chữ số 7)}
        \begin{itemize}
            \item Số lượng chữ số may mắn là bao nhiêu?
            \item \pause \textbf{Đáp án: 10}. (Số 10 chứa số 1 và 0 $\rightarrow$ \textbf{NO}).
        \end{itemize}
    \end{block}
    
    \pause
    Logic đã hoàn thiện: Chỉ khi số lượng là \textbf{4} hoặc \textbf{7} thì mới in \textbf{YES}.
\end{frame}

% Chunk 3: Blueprint
\begin{frame}[fragile]{Chunk 3: Tổng kết \& Mã giả (Blueprint)}
    Chúng ta xử lý $n$ dưới dạng \textbf{Chuỗi ký tự (String)}.

    \begin{exampleblock}{Kịch bản chương trình}
        \begin{enumerate}
            \item \textbf{Nhập liệu:} Đọc cả dòng vào biến chuỗi \texttt{s}.
            \item \textbf{Chuẩn bị:} Tạo biến đếm \texttt{dem = 0}.
            \item \textbf{Duyệt (Loop):} 
            \begin{lstlisting}[language=C++, basicstyle=\ttfamily\footnotesize]
FOR mỗi ky_tu trong s:
    IF ky_tu == '4' OR ky_tu == '7':
        dem = dem + 1
            \end{lstlisting}
            \item \textbf{Phán quyết:}
            \begin{lstlisting}[language=C++, basicstyle=\ttfamily\footnotesize]
IF dem == 4 OR dem == 7:
    PRINT "YES"
ELSE:
    PRINT "NO"
            \end{lstlisting}
        \end{enumerate}
    \end{exampleblock}
\end{frame}

% Bước cuối: Hiện thực hóa
\begin{frame}{Bước cuối: Hiện thực hóa}
    \begin{block}{Nhiệm vụ của bạn}
        Hãy viết code hoàn chỉnh dựa trên kịch bản trên bằng ngôn ngữ sở trường (C++, Python, Java...).
    \end{block}

    \begin{alertblock}{Lưu ý nhỏ}
        \begin{itemize}
            \item Input là chuỗi (String), không phải số nguyên.
            \item In hoa \textbf{YES} và \textbf{NO} đúng yêu cầu.
        \end{itemize}
    \end{alertblock}
    
    \centering
    \vspace{1cm}
    \textbf{Chúc các bạn thành công!}
\end{frame}

\end{document}