\documentclass{beamer}
\usepackage[utf8]{inputenc}
\usepackage[vietnamese]{babel}
\usepackage{tcolorbox}
\usepackage{listings}
\usepackage{xcolor}

\usetheme{Madrid}
\usecolortheme{default}

% Colors for code blocks
\definecolor{codegreen}{rgb}{0,0.6,0}
\definecolor{codegray}{rgb}{0.5,0.5,0.5}
\definecolor{codepurple}{rgb}{0.58,0,0.82}
\definecolor{backcolour}{rgb}{0.95,0.95,0.92}

\lstdefinestyle{mystyle}{
    backgroundcolor=\color{backcolour},   
    commentstyle=\color{codegreen},
    keywordstyle=\color{magenta},
    numberstyle=\tiny\color{codegray},
    stringstyle=\color{codepurple},
    basicstyle=\ttfamily\scriptsize,
    breakatwhitespace=false,         
    breaklines=true,                 
    captionpos=b,                    
    keepspaces=true,                 
    numbers=left,                    
    numbersep=4pt,                  
    showspaces=false,                
    showstringspaces=false,
    showtabs=false,                  
    tabsize=2,
    escapechar=@
}

\lstset{style=mystyle}

% Meta information
\title{791A - Bear and Big Brother}
\subtitle{Tư Duy Thuật Toán & Kỹ Năng Mô Phỏng}
\author{Coach Tư Duy Thuật Toán}
\date{\today}

\begin{document}

% Slide 1: Title
\begin{frame}
    \titlepage
\end{frame}

% Slide 2: Introduction
\begin{frame}{Giới thiệu bài toán}
    \begin{block}{Tổng quan}
        Đây là bài toán nhập môn kinh điển về dạng \textbf{"Mô phỏng" (Simulation)}.
        \begin{itemize}
            \item \textbf{Mục tiêu:} Rèn luyện thói quen tư duy chuẩn xác.
            \item \textbf{Phương pháp:} Phẫu thuật \& Dẫn dắt (Step-by-step).
        \end{itemize}
    \end{block}

    \begin{alertblock}{Lưu ý}
        Đừng để tiêu đề "nhẹ nhàng" đánh lừa. Hãy nhìn bài toán dưới lăng kính logic trần trụi.
    \end{alertblock}
\end{frame}

% Slide 3: Step 1 - Briefing
\begin{frame}{Bước 1: Tiếp nhận \& Phẫu thuật (Briefing)}
    \textbf{Tóm tắt đề bài (Ngôn ngữ lập trình):}
    
    \begin{itemize}
        \item \textbf{Biến số:} $a$ (Limak), $b$ (Bob).
        \item \textbf{Ban đầu:} $a \le b$.
        \item \textbf{Quy luật tăng trưởng (mỗi năm):}
        \begin{itemize}
            \item Limak: $a = a \times 3$
            \item Bob: $b = b \times 2$
        \end{itemize}
        \item \textbf{Mục tiêu:} Tìm số năm nhỏ nhất để $a > b$ (Lớn hơn hẳn).
    \end{itemize}

    \vspace{0.5cm}
    \textbf{Lộ trình tư duy (3 Micro-Chunks):}
    \begin{enumerate}
        \item Cơ chế tăng trưởng.
        \item Điểm gãy (Điều kiện dừng).
        \item Tư duy biên (Edge Cases).
    \end{enumerate}
\end{frame}

% Slide 4: Chunk 1 - The Growth
\begin{frame}{Chunk 1: Cơ chế tăng trưởng}
    \begin{tcolorbox}[colback=blue!5!white,colframe=blue!75!black,title=Logic Mô Phỏng]
        Hãy tưởng tượng một cuộc đua xe:
        \begin{itemize}
            \item \textbf{Xe Limak:} Xuất phát sau nhưng động cơ tên lửa ($\times 3$).
            \item \textbf{Xe Bob:} Xuất phát trước nhưng chạy chậm hơn ($\times 2$).
        \end{itemize}
    \end{tcolorbox}

    \textbf{Phương pháp:} 
    Không dùng công thức toán phức tạp. Hãy để máy tính đóng vai trọng tài, đếm từng năm trôi qua.
    
    \begin{alertblock}{Bẫy tư duy (Trap)}
        Đừng tính khoảng cách chênh lệch (1.5 lần). Hãy nghĩ đơn giản: \textbf{Sau 1 năm, giá trị cũ bị thay thế bởi giá trị mới.}
    \end{alertblock}
\end{frame}

% Slide 5: Mental Check - Example 1
\begin{frame}{Thử thách tư duy: Ví dụ mẫu}
    \textbf{Giả sử ban đầu:} $a = 4$ (Limak), $b = 7$ (Bob).
    
    \vspace{0.5cm}
    
    \textbf{Sau Năm thứ 1:}
    \begin{itemize}
        \item $a_{new} = 4 \times 3 = 12$
        \item $b_{new} = 7 \times 2 = 14$
    \end{itemize}
    
    \vspace{0.2cm}
    $\rightarrow$ So sánh: $12 \le 14$ (Limak vẫn thua).
    
    \vspace{0.2cm}
    \textbf{Kết luận:} Cuộc đua chưa kết thúc. Tiếp tục sang năm sau.
\end{frame}

% Slide 6: Chunk 2 - Loop Condition
\begin{frame}{Chunk 2: Điều kiện dừng (The Break Point)}
    \begin{tcolorbox}[title=Tư duy Vòng lặp (While Loop)]
        Sử dụng chiến thuật: \textbf{"Lặp cho đến khi..."}
        \begin{enumerate}
            \item Kiểm tra: Limak có thắng chưa?
            \item Nếu chưa ($a \le b$): Tăng biến đếm `nam`, cập nhật cân nặng.
            \item Nếu rồi ($a > b$): \textbf{DỪNG NGAY}.
        \end{enumerate}
    \end{tcolorbox}

    \begin{alertblock}{Lỗi phổ biến nhất}
        Điều kiện dừng là $a > b$.
        Nếu $a = b$, cuộc đua vẫn \textbf{PHẢI TIẾP TỤC}.
    \end{alertblock}
\end{frame}

% Slide 7: Mental Check - Example 2
\begin{frame}{Thử thách tư duy: Tiếp tục ví dụ}
    Tiếp tục với $a = 12, b = 14$ (Sau năm 1).
    
    \vspace{0.5cm}
    \textbf{Sau Năm thứ 2:}
    \begin{itemize}
        \item $a_{new} = 12 \times 3 = 36$
        \item $b_{new} = 14 \times 2 = 28$
    \end{itemize}
    
    \vspace{0.2cm}
    $\rightarrow$ So sánh: $36 > 28$.
    
    \vspace{0.2cm}
    \textbf{Quyết định:} 
    \begin{itemize}
        \item Limak đã lớn hơn hẳn Bob.
        \item Dừng vòng lặp.
        \item Kết quả: \textbf{2 năm}.
    \end{itemize}
\end{frame}

% Slide 8: Chunk 3 - Edge Cases
\begin{frame}{Chunk 3: Tư duy biên (Edge Cases)}
    \textbf{Trường hợp "nguy hiểm":} Hai bạn bằng cân nhau ngay từ đầu.
    
    \vspace{0.2cm}
    \textbf{Input:} \texttt{1 1} ($a=1, b=1$)
    
    \begin{block}{Phân tích}
        \begin{itemize}
            \item Ban đầu: $1 = 1 \rightarrow$ Chưa thỏa mãn điều kiện $a > b$.
            \item Vòng lặp chạy lần 1:
            \begin{itemize}
                \item $a = 1 \times 3 = 3$
                \item $b = 1 \times 2 = 2$
            \end{itemize}
            \item Kiểm tra: $3 > 2 \rightarrow$ \textbf{Thỏa mãn}.
        \end{itemize}
    \end{block}
    
    $\Rightarrow$ Kết quả là \textbf{1 năm} (Không phải 0).
\end{frame}

% Slide 9: Algorithm Recipe
\begin{frame}[fragile]{Tổng kết thuật toán (Algorithm Recipe)}
    Đây là "bản thiết kế" trước khi viết code:
    
    \begin{enumerate}
        \item \textbf{Đầu vào:} Nhập 2 số $a, b$.
        \item \textbf{Khởi tạo:} Biến đếm \texttt{nam = 0}.
        \item \textbf{Vòng lặp (While Loop):}
            \begin{lstlisting}[language=C++]
while (a <= b) {
    nam++;      // Tang nam
    a = a * 3;  // Limak tang truong
    b = b * 2;  // Bob tang truong
}
            \end{lstlisting}
        \item \textbf{Kết thúc:} In ra giá trị \texttt{nam}.
    \end{enumerate}
\end{frame}

% Slide 10: Next Steps
\begin{frame}{Bước tiếp theo: Hiện thực hóa}
    Bạn đã nắm vững thuật toán. Hãy chọn bước tiếp theo:
    
    \begin{itemize}
        \item[1.] \textbf{Tự viết code:} Code bằng C++/Python và tự kiểm tra.
        \item[2.] \textbf{Code khung sườn:} Điền vào chỗ trống.
        \item[3.] \textbf{Xem đáp án mẫu:} Tham khảo cách viết chuẩn (Clean Code).
    \end{itemize}
    
    \vspace{1cm}
    \centering
    \textit{"Code is logic made tangible."}
\end{frame}

\end{document}