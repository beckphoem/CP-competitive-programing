\documentclass{beamer}
\usepackage[utf8]{inputenc}
\usepackage[T5]{fontenc} % Bắt buộc để hiển thị tiếng Việt
\usepackage[vietnamese]{babel}
\usepackage{tcolorbox}
\usepackage{listings}
\usepackage{xcolor}
\usepackage{booktabs}
\usepackage{hyperref}
\usetheme{Madrid}

\definecolor{codegreen}{rgb}{0,0.6,0}
\definecolor{codegray}{rgb}{0.5,0.5,0.5}
\definecolor{codepurple}{rgb}{0.58,0,0.82}
\definecolor{backcolour}{rgb}{0.95,0.95,0.92}

\lstdefinestyle{mystyle}{
    backgroundcolor=\color{backcolour},   
    commentstyle=\color{codegreen},
    keywordstyle=\color{magenta},
    numberstyle=\tiny\color{codegray},
    stringstyle=\color{codepurple},
    basicstyle=\ttfamily\scriptsize,
    breakatwhitespace=false,         
    breaklines=true,                 
    captionpos=b,                    
    keepspaces=true,                 
    numbers=left,                    
    numbersep=4pt,                  
    showspaces=false,                
    showstringspaces=false,
    showtabs=false,                  
    tabsize=2,
    escapechar=@
}

\lstset{style=mystyle}

\title{Luyện tập C++: Set và Unordered Set}
\subtitle{Danh sách bài tập Codeforces Elo 800 - 1000}
\author{Slide Learning CPP}
\date{\today}

\begin{document}

\begin{frame}
    \titlepage
\end{frame}

\begin{frame}{Giới thiệu}
    \begin{block}{Mục tiêu}
        Việc luyện tập theo nhóm chủ đề giúp tạo ra các "chunk" kiến thức vững chắc. Danh sách này tập trung vào \texttt{set} và \texttt{unordered\_set}.
    \end{block}
    
    \begin{itemize}
        \item \textbf{\texttt{std::set}}: Tự động sắp xếp, phần tử duy nhất.
        \item \textbf{\texttt{std::unordered\_set}}: Không sắp xếp, phần tử duy nhất, tốc độ nhanh hơn.
    \end{itemize}
\end{frame}

% Bài 1
\begin{frame}{1. Boy or Girl (236A)}
    \textbf{Link:} \href{https://codeforces.com/problemset/problem/236/A}{Codeforces 236A}
    \begin{block}{Tóm tắt đề bài}
        Đếm số ký tự khác nhau trong tên người dùng.
    \end{block}
    \begin{exampleblock}{Chiến lược giải quyết}
        Đưa các ký tự của chuỗi vào một \texttt{set<char>}. Sau đó kiểm tra \texttt{set.size()} là số chẵn hay số lẻ để đưa ra kết luận.
    \end{exampleblock}
\end{frame}

% Bài 2
\begin{frame}{2. Is your horseshoe on the other hoof? (228A)}
    \textbf{Link:} \href{https://codeforces.com/problemset/problem/228/A}{Codeforces 228A}
    \begin{block}{Tóm tắt đề bài}
        Cho 4 màu móng ngựa, cần mua thêm ít nhất bao nhiêu móng để có đủ 4 màu khác nhau?
    \end{block}
    \begin{exampleblock}{Chiến lược giải quyết}
        Đưa 4 giá trị màu vào \texttt{set}. Số lượng móng cần mua thêm sẽ bằng $4 - \texttt{set.size()}$.
    \end{exampleblock}
\end{frame}

% Bài 3
\begin{frame}{3. Pangram (520A)}
    \textbf{Link:} \href{https://codeforces.com/problemset/problem/520/A}{Codeforces 520A}
    \begin{block}{Tóm tắt đề bài}
        Kiểm tra một chuỗi có chứa đủ 26 chữ cái tiếng Anh (không phân biệt hoa thường) hay không.
    \end{block}
    \begin{exampleblock}{Chiến lược giải quyết}
        Chuyển tất cả ký tự về chữ thường (lowercase), bỏ vào một \texttt{set<char>}. Kiểm tra xem \texttt{set.size()} có bằng 26 hay không.
    \end{exampleblock}
\end{frame}

% Bài 4
\begin{frame}{4. Registration System (4C)}
    \textbf{Link:} \href{https://codeforces.com/problemset/problem/4/C}{Codeforces 4C}
    \begin{block}{Tóm tắt đề bài}
        Quản lý tên người dùng. Nếu tên đã tồn tại, thêm số thứ tự tăng dần vào sau tên.
    \end{block}
    \begin{alertblock}{Chiến lược giải quyết}
        Sử dụng \texttt{map<string, int>} để lưu trữ và đếm số lần xuất hiện của mỗi tên người dùng. Đây là bài tập chuyển tiếp từ \texttt{set} sang \texttt{map}.
    \end{alertblock}
\end{frame}

% Bài 5
\begin{frame}{5. Anton and Letters (443A)}
    \textbf{Link:} \href{https://codeforces.com/problemset/problem/443/A}{Codeforces 443A}
    \begin{block}{Tóm tắt đề bài}
        Đếm các chữ cái khác nhau xuất hiện trong một tập hợp có định dạng \texttt{\{a, b, c\}}.
    \end{block}
    \begin{exampleblock}{Chiến lược giải quyết}
        Xử lý chuỗi đầu vào để lọc lấy các ký tự chữ cái, bỏ chúng vào \texttt{set} rồi in ra \texttt{set.size()}.
    \end{exampleblock}
\end{frame}

% Bài 6
\begin{frame}{6. I Wanna Be the Guy (469A)}
    \textbf{Link:} \href{https://codeforces.com/problemset/problem/469/A}{Codeforces 469A}
    \begin{block}{Tóm tắt đề bài}
        Hai người chơi có các danh sách màn chơi họ có thể vượt qua. Liệu cả hai có thể cùng nhau vượt qua tất cả $n$ màn chơi không?
    \end{block}
    \begin{exampleblock}{Chiến lược giải quyết}
        Thực hiện phép hợp (Union) bằng cách bỏ tất cả các màn chơi của cả hai người vào một \texttt{set}. Kiểm tra \texttt{set.size() == n}.
    \end{exampleblock}
\end{frame}

% Bài 7
\begin{frame}{7. Arrival of the General (144A)}
    \textbf{Link:} \href{https://codeforces.com/problemset/problem/144/A}{Codeforces 144A}
    \begin{block}{Tóm tắt đề bài}
        Tìm số bước hoán đổi ít nhất để đưa phần tử lớn nhất về đầu và phần tử nhỏ nhất về cuối.
    \end{block}
    \begin{exampleblock}{Chiến lược giải quyết}
        Luyện tập tìm vị trí Max/Min. \texttt{set} có thể dùng để lưu các giá trị duy nhất nếu bài toán yêu cầu mở rộng về sau.
    \end{exampleblock}
\end{frame}

% Bài 8
\begin{frame}{8. Gamer Hemose (1592A)}
    \textbf{Link:} \href{https://codeforces.com/problemset/problem/1592/A}{Codeforces 1592A}
    \begin{block}{Tóm tắt đề bài}
        Chọn 2 vũ khí mạnh nhất để tiêu diệt quái vật sao cho không dùng một vũ khí 2 lần liên tiếp.
    \end{block}
    \begin{exampleblock}{Chiến lược giải quyết}
        Dùng \texttt{set} (để tự động sắp xếp) hoặc \texttt{sort} để lấy ra 2 giá trị sát thương lớn nhất.
    \end{exampleblock}
\end{frame}

% Bài 9
\begin{frame}{9. Double Strings (1703D)}
    \textbf{Link:} \href{https://codeforces.com/problemset/problem/1703/D}{Codeforces 1703D}
    \begin{block}{Tóm tắt đề bài}
        Kiểm tra một chuỗi có bằng tổng (ghép) của 2 chuỗi khác có trong danh sách không.
    \end{block}
    \begin{exampleblock}{Chiến lược giải quyết}
        Bỏ tất cả các chuỗi vào \texttt{unordered\_set}. Với mỗi chuỗi, dùng vòng lặp cắt làm đôi tại mọi vị trí khả thi và dùng \texttt{count()} để kiểm tra sự tồn tại của cả hai nửa trong \texttt{set}.
    \end{exampleblock}
\end{frame}

% Bài 10
\begin{frame}{10. Favorite Sequence (1462A)}
    \textbf{Link:} \href{https://codeforces.com/problemset/problem/1462/A}{Codeforces 1462A}
    \begin{block}{Tóm tắt đề bài}
        Khôi phục dãy số gốc từ một dãy bị xáo trộn theo quy tắc lấy luân phiên từ hai đầu.
    \end{block}
    \begin{exampleblock}{Chiến lược giải quyết}
        Sử dụng cấu trúc dữ liệu mảng hoặc \texttt{deque}. Có thể dùng \texttt{set} để đánh dấu các phần tử đã xử lý nếu cần.
    \end{exampleblock}
\end{frame}

% Thêm các slide cho các bài từ 11-20 tương tự...
\begin{frame}{11. Two-gram (977B)}
    \textbf{Link:} \href{https://codeforces.com/problemset/problem/977/B}{Codeforces 977B}
    \begin{block}{Tóm tắt đề bài}
        Tìm cặp 2 ký tự (substring độ dài 2) xuất hiện nhiều nhất trong chuỗi.
    \end{block}
    \begin{exampleblock}{Chiến lược giải quyết}
        Dùng \texttt{map<string, int>} để đếm số lần xuất hiện của tất cả các cặp con độ dài 2.
    \end{exampleblock}
\end{frame}

\begin{frame}{12. Polycarp and Letters (864B)}
    \textbf{Link:} \href{https://codeforces.com/problemset/problem/864/B}{Codeforces 864B}
    \begin{block}{Tóm tắt đề bài}
        Tìm đoạn con dài nhất chỉ chứa các chữ cái thường khác nhau.
    \end{block}
    \begin{exampleblock}{Chiến lược giải quyết}
        Duyệt chuỗi: nếu gặp chữ hoa thì \texttt{clear()} \texttt{set}, nếu gặp chữ thường thì \texttt{insert()} vào \texttt{set}. Cập nhật kết quả cực đại từ \texttt{set.size()}.
    \end{exampleblock}
\end{frame}

\begin{frame}{16. Distinct Digits (1228A)}
    \textbf{Link:} \href{https://codeforces.com/problemset/problem/1228/A}{Codeforces 1228A}
    \begin{block}{Tóm tắt đề bài}
        Tìm một số trong đoạn $[l, r]$ mà tất cả các chữ số của nó đều khác nhau.
    \end{block}
    \begin{exampleblock}{Chiến lược giải quyết}
        Duyệt từng số, chuyển số thành \texttt{string}, bỏ từng ký tự số vào \texttt{set<char>}. Nếu \texttt{set.size() == string.length()} thì đó là số cần tìm.
    \end{exampleblock}
\end{frame}

\begin{frame}{Lời khuyên khi giải bài}
    \begin{itemize}
        \item \textbf{Tại sao lại có Map?} \texttt{set} để kiểm tra "có tồn tại hay không", còn \texttt{map} để biết "tồn tại bao nhiêu lần".
        \item \textbf{Khi nào dùng Unordered?} Với Elo 800-1000 ($n \le 10^5$), \texttt{std::set} vẫn chạy rất tốt. Hãy tập dùng \texttt{std::set} trước để làm quen với tính chất sắp xếp.
    \end{itemize}
\end{frame}

\begin{frame}[fragile]{Ví dụ: Boy or Girl (236A)}
\begin{lstlisting}[language=C++]
#include <iostream>
#include <set>
#include <string>

int main() {
    std::string s;
    std::cin >> s;
    std::set<char> distinct_chars;
    for(char c : s) {
        distinct_chars.insert(c);
    }
    
    if(distinct_chars.size() % 2 == 0) 
        std::cout << "CHAT WITH HER!";
    else 
        std::cout << "IGNORE HIM!";
    return 0;
}
\end{lstlisting}
\pause
\begin{block}{Kết quả}
    Chương trình sẽ đếm số ký tự duy nhất và đưa ra quyết định.
\end{block}
\end{frame}

\end{document}