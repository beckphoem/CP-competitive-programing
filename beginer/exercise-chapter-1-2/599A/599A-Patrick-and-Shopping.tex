\documentclass[10pt]{beamer}

% --- Cấu hình Tiếng Việt và Font ---
\usepackage[utf8]{inputenc}
\usepackage[T5]{fontenc}
\usepackage[vietnamese]{babel}
\usepackage{lmodern} % Font chữ sắc nét hơn

% --- Cấu hình Giao diện Beamer ---
\usetheme{Madrid}
\usecolortheme{whale}
\usefonttheme{professionalfonts}

% --- Cấu hình hiển thị Code ---
\usepackage{listings}
\usepackage{xcolor}

\definecolor{codegreen}{rgb}{0,0.6,0}
\definecolor{codegray}{rgb}{0.5,0.5,0.5}
\definecolor{codepurple}{rgb}{0.58,0,0.82}
\definecolor{backcolour}{rgb}{0.95,0.95,0.92}

\lstdefinestyle{mystyle}{
    backgroundcolor=\color{backcolour},   
    commentstyle=\color{codegreen},
    keywordstyle=\color{magenta},
    numberstyle=\tiny\color{codegray},
    stringstyle=\color{codepurple},
    basicstyle=\ttfamily\footnotesize,
    breakatwhitespace=false,         
    breaklines=true,                 
    captionpos=b,                    
    keepspaces=true,                 
    numbers=left,                    
    numbersep=5pt,                  
    showspaces=false,                
    showstringspaces=false,
    showtabs=false,                  
    tabsize=2
}
\lstset{style=mystyle}

% --- Metadata ---
\title[Codeforces 599A]{Giải thuật toán: Patrick and Shopping}
\subtitle{Tư duy thuật toán theo triết lý "Learning How to Learn"}
\author{Coach Tư Duy Thuật Toán}
\date{\today}

% --- Bắt đầu Slide ---
\begin{document}

% Slide Tiêu đề
\begin{frame}
  \titlepage
\end{frame}

% --- BƯỚC 1 ---
\begin{frame}{1. Phẫu thuật đề bài (Deconstruct)}
    \begin{block}{Góc nhìn hình học}
    Quên cốt truyện đi, hãy nhìn bài toán dưới dạng hình học:
    \begin{itemize}
        \item \textbf{3 Điểm:} Nhà (Xuất phát/Đích), Shop 1, Shop 2.
        \item \textbf{Các cạnh (Quãng đường):}
            \begin{itemize}
                \item $d_1$: Nhà $\leftrightarrow$ Shop 1
                \item $d_2$: Nhà $\leftrightarrow$ Shop 2
                \item $d_3$: Shop 1 $\leftrightarrow$ Shop 2
            \end{itemize}
    \end{itemize}
    \end{block}

    \begin{alertblock}{Nhiệm vụ}
    Xuất phát từ \textbf{Nhà} $\to$ Ghé thăm \textbf{cả 2 shop} $\to$ Quay về \textbf{Nhà}. \\
    \textbf{Mục tiêu:} Tìm tổng quãng đường nhỏ nhất ($\min$).
    \end{alertblock}
\end{frame}

% --- BƯỚC 2: CHUNK 1 ---
\begin{frame}{2. Vòng lặp tư duy - Chunk 1: Kịch bản di chuyển}
    \textbf{Logic (Ẩn dụ):} Vẽ đường nét liền đi qua 3 điểm.
    \vspace{0.5em}
    
    Có 3 kịch bản sơ khai:
    \begin{enumerate}
        \item \textbf{Đi vòng tròn:} 
        $$ \text{Nhà} \to \text{Shop 1} \to \text{Shop 2} \to \text{Nhà} $$
        
        \item \textbf{Đi kiểu "Con thoi" (Về nhà đổi hướng):}
        $$ \text{Nhà} \to \text{Shop 1} \to \text{Nhà} \dots \text{rồi} \dots \text{Nhà} \to \text{Shop 2} \to \text{Nhà} $$
        
        \item \textbf{Đi kiểu "Kẹp nách" (Đi ké):}
        $$ \text{Nhà} \to \text{Shop 1} \to \text{Shop 2} \to \text{Shop 1} \to \text{Nhà} $$
    \end{enumerate}

    \begin{alertblock}{Bẫy (Trap)}
    Đừng mặc định đi vòng tròn ($d_1+d_2+d_3$) là nhanh nhất. Nếu $d_3$ quá lớn, đi về nhà rồi sang shop kia còn lợi hơn.
    \end{alertblock}
\end{frame}

% --- THỬ THÁCH 1 ---
\begin{frame}{Thử thách tư duy 1 (Mental Check)}
    \textbf{Dữ liệu:} $d_1 = 10, d_2 = 20, d_3 = 100$ (Đường nối 2 shop siêu xa).
    
    \vspace{1em}
    \textbf{Tính toán các phương án:}
    \begin{itemize}
        \item \textbf{Cách A (Vòng tròn):} $d_1 + d_3 + d_2 = 10 + 100 + 20 = 130$
        \item \textbf{Cách B (Con thoi):} $2(d_1) + 2(d_2) = 20 + 40 = \textbf{60}$
        \item \textbf{Cách C (Đi ké):} $2(d_1) + 2(d_3) = 20 + 200 = 220$
    \end{itemize}
    
    \vspace{1em}
    \begin{exampleblock}{Kết luận}
    \textbf{Chọn Cách B (60).} \\
    Chứng minh: Đôi khi đường thẳng nối 2 điểm ($d_3$) là một cái "bẫy" chi phí cao.
    \end{exampleblock}
\end{frame}

% --- BƯỚC 2: CHUNK 2 ---
\begin{frame}{3. Vòng lặp tư duy - Chunk 2: Bức tranh toàn cảnh}
    Chúng ta đã bỏ sót một khả năng ở Chunk 1.
    
    Khi đi kiểu "Kẹp nách", tại sao lại chỉ ưu ái Shop 1?
    \begin{itemize}
        \item Thay vì: Nhà $\to$ Shop 1 $\to$ Shop 2 $\dots$
        \item Ta có thể: Nhà $\to$ Shop 2 $\to$ Shop 1 $\dots$
    \end{itemize}
    
    \textbf{Tổng kết 4 Chiến thuật di chuyển:}
    \begin{enumerate}
        \item \textbf{Vòng tròn:} $Cost = d_1 + d_2 + d_3$
        \item \textbf{Con thoi (Đi riêng):} $Cost = 2(d_1 + d_2)$
        \item \textbf{Ké qua Shop 1:} $Cost = 2(d_1 + d_3)$
        \item \textbf{Ké qua Shop 2:} $Cost = 2(d_2 + d_3)$
    \end{enumerate}
\end{frame}

% --- THỬ THÁCH 2 ---
\begin{frame}{Thử thách tư duy 2 (Mental Check)}
    \textbf{Dữ liệu:} $d_1 = 10, d_2 = 100$ (Xa tít), $d_3 = 5$ (Hai shop cạnh nhau).
    
    \vspace{1em}
    \textbf{Phân tích:}
    \begin{itemize}
        \item Nếu đi theo thói quen (Vòng tròn): $10 + 100 + 5 = 115$.
        \item Nếu dùng chiến thuật \textbf{Ké qua Shop 1}:
        $$ 2 \times (d_1 + d_3) = 2 \times (10 + 5) = \textbf{30} $$
    \end{itemize}
    
    \begin{exampleblock}{Kết luận}
    Đáp án là \textbf{30}. \\
    Con đường trực tiếp $d_2$ (100) giống như trạm thu phí giá cắt cổ. Đi vòng qua $d_1$ và $d_3$ rẻ hơn nhiều.
    \end{exampleblock}
\end{frame}

% --- CHỐT HẠ THUẬT TOÁN ---
\begin{frame}{4. Chốt hạ thuật toán (The Solution)}
    Không có một công thức duy nhất đúng cho mọi trường hợp.
    \vspace{0.5em}
    
    \textbf{Giải pháp:} Brute-force (Vét cạn).
    Tính tất cả 4 trường hợp và chọn Min.
    
    \begin{block}{Công thức tổng quát}
    Gọi $ans$ là kết quả cần tìm:
    $$
    ans = \min \begin{cases} 
    d_1 + d_2 + d_3 & (\text{Vòng tròn}) \\
    2(d_1 + d_2) & (\text{Con thoi}) \\
    2(d_1 + d_3) & (\text{Qua Shop 1}) \\
    2(d_2 + d_3) & (\text{Qua Shop 2})
    \end{cases}
    $$
    \end{block}

    \textbf{Thử thách cuối cùng (Mã giả):}
    \begin{itemize}
        \item Bước 1: Nhập $d_1, d_2, d_3$.
        \item Bước 2: Tính 4 biến $a, b, c, d$ theo công thức trên.
        \item Bước 3: Output $\to$ Dùng hàm \textbf{min(a, b, c, d)}.
    \end{itemize}
\end{frame}

% --- IMPLEMENTATION PYTHON ---
\begin{frame}[fragile]{Hiện thực hóa: Python Code}
Python hỗ trợ hàm \texttt{min()} với nhiều tham số rất tiện lợi.

\begin{lstlisting}[language=Python]
# Bước 1: Nhập dữ liệu
d1, d2, d3 = map(int, input().split())

# Bước 2 & 3: Tính toán và in ra min ngay lập tức
ans = min(
    d1 + d2 + d3,       # Cach 1: Vong tron
    2 * (d1 + d2),      # Cach 2: Di le tung cai
    2 * (d1 + d3),      # Cach 3: Ke qua shop 1
    2 * (d2 + d3)       # Cach 4: Ke qua shop 2
)

print(ans)
\end{lstlisting}
\end{frame}

% --- IMPLEMENTATION C++ ---
\begin{frame}[fragile]{Hiện thực hóa: C++ Code}
Với C++11 trở lên, dùng \texttt{min(\{...\})} để so sánh danh sách.

\begin{lstlisting}[language=C++]
#include <iostream>
#include <algorithm> // Thu vien chua ham min
using namespace std;

int main() {
    long long d1, d2, d3; 
    // Dung long long phong khi tong > 2 ty
    cin >> d1 >> d2 >> d3;

    long long opt1 = d1 + d2 + d3;
    long long opt2 = 2 * (d1 + d2);
    long long opt3 = 2 * (d1 + d3);
    long long opt4 = 2 * (d2 + d3);

    // Tim min trong danh sach khoi tao
    cout << min({opt1, opt2, opt3, opt4}); 
    
    return 0;
}
\end{lstlisting}
\end{frame}

% --- BÀI HỌC ---
\begin{frame}{Bài học rút ra (Key Takeaway)}
    \begin{enumerate}
        \item \textbf{Đừng tin vào trực giác hình học trên giấy:} 
        \begin{itemize}
            \item Toán học: $a+b > c$ (Bất đẳng thức tam giác).
            \item Thực tế (Giao thông/Đồ thị): Đường thẳng nối 2 điểm ($d_3$) có thể xa vô lý (tắc đường, đường xấu).
        \end{itemize}
        
        \vspace{1em}
        \item \textbf{Sức mạnh của Vét cạn (Brute-force):}
        \begin{itemize}
            \item Khi số lượng trường hợp ít (chỉ 4 cách), đừng cố tìm công thức toán học phức tạp để loại trừ.
            \item Hãy để máy tính tính hết và chọn cái tốt nhất.
        \end{itemize}
    \end{enumerate}
    
    \vspace{1em}
    \centerline{\textbf{Chúc mừng bạn đã chinh phục bài 599A!}}
\end{frame}

\end{document}