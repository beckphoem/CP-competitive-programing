\documentclass{beamer}

% Cấu hình ngôn ngữ và encoding
\usepackage[utf8]{inputenc}
\usepackage[T5]{fontenc} % Encoding cho tiếng Việt
\usepackage[vietnamese]{babel}

% Cấu hình giao diện (Theme)
\usetheme{Madrid}
\usecolortheme{beaver} % Tông màu đỏ/xám phù hợp với sự "Cảnh báo/Tư duy"

% Cấu hình hiển thị mã giả
\usepackage{listings}
\lstset{
  basicstyle=\ttfamily\small,
  breaklines=true,
  frame=single,
  backgroundcolor=\color{gray!10},
  keywordstyle=\color{blue}\bfseries,
  commentstyle=\color{green!60!black}
}

% Thông tin trang tiêu đề
\title[Codeforces 732A - Buy a Shovel]{Phân tích Tư duy Thuật toán: Codeforces 732A}
\subtitle{Triết lý "Learning How to Learn"}
\author{Coach Tư Duy Thuật Toán}
\date{\today}

\begin{document}

% Slide 1: Tiêu đề
\begin{frame}
  \titlepage
\end{frame}

% Slide 2: Giới thiệu
\begin{frame}{Lời mở đầu}
  Chào mừng bạn! Tôi đã nhận tín hiệu. Tôi là \textbf{Coach Tư Duy Thuật Toán} của bạn đây.

  Hôm nay chúng ta sẽ cùng "mổ xẻ" bài toán \textbf{Codeforces 732A - Buy a Shovel (Mua xẻng)}. 
  
  \vspace{0.5cm}
  \begin{alertblock}{Lưu ý}
    Đừng lo về code vội, hãy cất bàn phím đi và dùng "bộ não" trước đã.
  \end{alertblock}
\end{frame}

% Slide 3: Bước 1 - Briefing
\section{Bước 1: Tiếp nhận & Phẫu thuật}
\begin{frame}{Bước 1: Tiếp nhận \& Phẫu thuật (Briefing)}
  Đề bài kể về anh chàng Polycarp đi mua xẻng, nhưng chúng ta hãy gạt bỏ câu chuyện đó sang một bên. Đây là cốt lõi logic của vấn đề:

  \begin{itemize}
    \item \textbf{1. Dữ liệu đầu vào (Input):}
      \begin{itemize}
        \item $k$: Giá tiền của \textbf{một} chiếc xẻng.
        \item $r$: Mệnh giá của một đồng xu lẻ duy nhất mà bạn có (trong túi bạn có vô hạn tờ tiền mệnh giá 10, nhưng chỉ có đúng \textbf{một} đồng xu mệnh giá $r$).
      \end{itemize}
    \item \textbf{2. Mục tiêu (Goal):}
      \begin{itemize}
        \item Tìm số lượng xẻng \textbf{ít nhất} (gọi là $x$) mà bạn cần mua.
        \item \textbf{Điều kiện:} Bạn phải trả tiền \textbf{vừa đủ khít} (không thối lại tiền thừa).
      \end{itemize}
    \item \textbf{3. Công cụ thanh toán:}
      \begin{itemize}
        \item Bạn chỉ có thể trả bằng các tờ tiền \textbf{10 đồng} (số lượng vô hạn).
        \item Và (tùy chọn) dùng thêm đúng \textbf{một đồng xu $r$} nếu muốn.
      \end{itemize}
  \end{itemize}
\end{frame}

% Slide 4: Lộ trình tư duy
\begin{frame}{Lộ trình tư duy (Roadmap)}
  Chúng ta sẽ giải quyết bài toán này qua 3 mảnh ghép (Chunks):
  \begin{enumerate}
    \item \textbf{Chunk 1:} Phân tích "Cơ chế ví tiền" (Điều kiện chia hết).
    \item \textbf{Chunk 2:} Truy tìm "Chữ số tận cùng" (Last Digit Logic).
    \item \textbf{Chunk 3:} Chiến thuật thử sai (Simulation).
  \end{enumerate}
  
  \vspace{0.5cm}
  \textit{Bạn đã sẵn sàng bước vào Chunk 1: Phân tích cơ chế ví tiền chưa?}
\end{frame}

% Slide 5: Chunk 1 - Cơ chế ví tiền (Logic)
\section{Chunk 1: Cơ chế ví tiền}
\begin{frame}{Chunk 1: Cơ chế ví tiền (Điều kiện thanh toán)}
  Hãy tưởng tượng bạn đang đứng trước quầy thu ngân.
  Tổng số tiền bạn phải trả là:
  $$S = k \times x$$
  \textit{(Trong đó: $k$ là giá 1 cái xẻng, $x$ là số xẻng bạn mua)}.

  Trong túi bạn có:
  \begin{enumerate}
    \item Rất nhiều tờ \textbf{10 đồng}.
    \item Đúng \textbf{1 đồng xu lẻ} mệnh giá $r$.
  \end{enumerate}

  Để trả vừa đủ, tổng số tiền $S$ phải thỏa mãn 1 trong 2 trường hợp:
  \begin{itemize}
    \item \textbf{Trường hợp 1:} Dùng toàn tờ 10 đồng $\rightarrow S$ chia hết cho 10 (Tận cùng là \textbf{0}).
    \item \textbf{Trường hợp 2:} Dùng tờ 10 đồng + đồng xu $r$ $\rightarrow (S - r)$ chia hết cho 10 (Tận cùng là \textbf{$r$}).
  \end{itemize}

  \textbf{TÓM LẠI:} Chữ số tận cùng của $S$ bắt buộc phải là \textbf{0} hoặc \textbf{$r$}.
\end{frame}

% Slide 6: Chunk 1 - Bẫy & Check
\begin{frame}{Chunk 1: Cạm bẫy \& Thử thách tư duy}
  \begin{alertblock}{⚠️ Cạm bẫy (Trap)}
    Nhiều bạn nghĩ rằng bắt buộc phải dùng đồng xu $r$.
    \textbf{Sai!} Bạn có thể cất đồng xu đó đi nếu Tổng tiền $S$ đã tròn chục (tận cùng là 0).
  \end{alertblock}

  \begin{block}{❓ Thử thách tư duy (Mental Check)}
    Giả sử giá xẻng \textbf{$k = 12$} và đồng xu lẻ \textbf{$r = 8$}. Hãy kiểm tra:
    \begin{enumerate}
      \item Mua $x = 1 \rightarrow S = 12$ (Tận cùng 2). Có trả được không? \textbf{Không}.
      \item Mua $x = 4 \rightarrow S = 48$ (Tận cùng 8). Có trả được không? \textbf{Có} (trùng $r=8$).
      \item Mua $x = 5 \rightarrow S = 60$ (Tận cùng 0). Có trả được không? \textbf{Có} (tròn chục).
    \end{enumerate}
  \end{block}
  \textit{Bạn đã nắm vững "Điều kiện thắng"!}
\end{frame}

% Slide 7: Chunk 2 - Chiến thuật
\section{Chunk 2: Chiến thuật truy tìm}
\begin{frame}{Chunk 2: Chiến thuật truy tìm (The Loop)}
  Đề bài yêu cầu tìm số xẻng \textbf{ít nhất}. Cách đơn giản nhất là \textbf{"Thử từng cái một"}.
  
  Giống như bạn đi thử chìa khóa vào ổ vậy:
  \begin{enumerate}
    \item Thử mua 1 cái ($x=1$) $\rightarrow$ Tính tổng tiền $\rightarrow$ Kiểm tra đuôi.
    \begin{itemize}
        \item Nếu đúng: Dừng lại ngay! (Đây là đáp án nhỏ nhất).
        \item Nếu sai: Thử tiếp 2 cái ($x=2$).
    \end{itemize}
    \item Lặp lại cho đến khi tìm thấy.
  \end{enumerate}

  Để máy tính "nhìn" được chữ số tận cùng, dùng phép toán \textbf{Chia lấy dư cho 10} (Modulus \% 10).
  \begin{itemize}
      \item Ví dụ: $48 \% 10 = 8$
      \item Ví dụ: $60 \% 10 = 0$
  \end{itemize}
\end{frame}

% Slide 8: Chunk 2 - Bẫy & Check
\begin{frame}{Chunk 2: Cạm bẫy \& Thử thách tư duy}
  \begin{alertblock}{⚠️ Cạm bẫy (Trap)}
    \textit{"Nhỡ thử mãi không tìm được thì sao? Nhỡ nó chạy đến vô tận thì sao?"}
    \\ \textbf{Yên tâm:} Vì chỉ quan tâm chữ số tận cùng (0-9), quy luật sẽ lặp lại. Thực tế chỉ cần thử tối đa 10 lần.
  \end{alertblock}

  \begin{block}{❓ Thử thách tư duy (Mental Check)}
    Máy tính chạy bằng cơm. $k = 7$, $r = 3$. Thử dần $x$ tăng lên:
    \begin{itemize}
        \item $x=1 \rightarrow S=7$ (Sai: $7 \neq 0, 7 \neq 3$)
        \item $x=2 \rightarrow S=14$ (Sai)
        \item ... (Bạn đã tính ra) ...
        \item \textbf{$x=9 \rightarrow S=63$} (Tận cùng là 3 $\rightarrow$ Trùng $r=3$).
    \end{itemize}
  \end{block}
  \textbf{Bingo!} Chúng ta dừng lại và mua 9 cái.
\end{frame}

% Slide 9: Chunk 3 - Tổng kết
\section{Chunk 3: Tổng kết & Mã giả}
\begin{frame}[fragile]{Chunk 3: Tổng kết \& Mã giả (Wrap Up)}
  \textbf{Thuật toán (Algorithm):}
  \begin{enumerate}
      \item Nhập $k, r$.
      \item Vòng lặp đếm $i$ từ 1 tăng dần mãi.
      \item Mỗi bước tính `Tổng tiền = k * i`.
      \item Kiểm tra: Nếu (`% 10 == 0`) HOẶC (`% 10 == r`) $\rightarrow$ Chọn và Dừng.
  \end{enumerate}

  \textbf{Mã giả (Pseudocode):}
\begin{lstlisting}
Đọc k, r

Cho i chạy từ 1 đến vô tận:
    S = k * i
    duoi = S % 10  // Lấy chữ số tận cùng

    Nếu (duoi == 0) HOẶC (duoi == r):
        In ra i
        Dừng vòng lặp (Break)
\end{lstlisting}
\end{frame}

% Slide 10: Bài tập về đích
\begin{frame}{Bài tập về đích (Final Check)}
  Trước khi tôi đưa bạn xem code C++/Python mẫu, hãy xác nhận một điều nhỏ về cú pháp logic:

  \begin{block}{Câu hỏi}
    Trong lập trình, điều kiện \textbf{"Hoặc"} được viết là gì?
    \begin{enumerate}[A]
        \item \texttt{\&\&}
        \item \texttt{||}
    \end{enumerate}
  \end{block}

  \vspace{1cm}
  \centering \textbf{Chọn đáp án đúng để mở khóa Code mẫu nhé! 🔓}
\end{frame}

\end{document}