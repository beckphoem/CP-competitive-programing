\documentclass{beamer}
\usepackage[utf8]{inputenc}
\usepackage[vietnamese]{babel}
\usepackage{tcolorbox}
\usepackage{listings}
\usepackage{xcolor}
\usepackage{graphicx}

\usetheme{Madrid}
\usecolortheme{default}

% Colors and Style for code
\definecolor{codegreen}{rgb}{0,0.6,0}
\definecolor{codegray}{rgb}{0.5,0.5,0.5}
\definecolor{codepurple}{rgb}{0.58,0,0.82}
\definecolor{backcolour}{rgb}{0.95,0.95,0.92}

\lstdefinestyle{mystyle}{
    backgroundcolor=\color{backcolour},   
    commentstyle=\color{codegreen},
    keywordstyle=\color{magenta},
    numberstyle=\tiny\color{codegray},
    stringstyle=\color{codepurple},
    basicstyle=\ttfamily\footnotesize,
    breakatwhitespace=false,         
    breaklines=true,                 
    captionpos=b,                    
    keepspaces=true,                 
    numbers=left,                    
    numbersep=5pt,                  
    showspaces=false,                
    showstringspaces=false,
    showtabs=false,                  
    tabsize=2,
    escapechar=@
}

\lstset{style=mystyle}

\title[Domino piling]{Bài Tập Codeforces 50A: Domino piling \\ "Xếp Hình & Diện Tích"}
\subtitle{C++ Competitive Programming Series - Elo 800}
\author{Học Cùng C++}
\date{}

\begin{document}

\begin{frame}
  \titlepage
\end{frame}

% 1. Hình Dung
\begin{frame}{1. Hình Dung Bài Toán}
  \begin{block}{Đề bài}
      Bạn có bảng kích thước $M \times N$.
      Cần xếp tối đa các quân Domino $2 \times 1$ vào bảng.
      \begin{itemize}
          \item Không chồng đè.
          \item Không ra ngoài bảng.
          \item Có thể xoay ngang/dọc.
      \end{itemize}
  \end{block}

  \begin{alertblock}{Mục tiêu}
      Tìm số lượng quân Domino TỐI ĐA có thể xếp được.
  \end{alertblock}
\end{frame}

% 2. Tư Duy
\begin{frame}{2. Tư Duy: Diện Tích Là Chân Ái}
  Đừng cố gắng nghĩ cách xếp! Hãy nghĩ về \textbf{Diện Tích}.
  
  \begin{itemize}
      \item Diện tích 1 quân Domino = $2 \times 1 = 2$ ô vuông.
      \item Tổng diện tích bảng = $M \times N$ ô vuông.
  \end{itemize}

  Theo nguyên lý toán học:
  \[ \text{Số quân tối đa} = \frac{\text{Tổng diện tích bảng}}{\text{Diện tích 1 quân}} \]
\end{frame}

% 3. Chứng Minh
\begin{frame}{3. Có Xếp Kín Được (Gần) Hết Không?}
  Câu trả lời là: \textbf{LUÔN LUÔN CÓ THỂ!}
  
  \begin{columns}
      \column{0.5\textwidth}
      \begin{block}{Trường hợp Chẵn}
          Nếu $M \times N$ chẵn:
          Ta xếp kín mít 100\%.
          Không thừa ô nào.
      \end{block}

      \column{0.5\textwidth}
      \begin{block}{Trường hợp Lẻ}
          Nếu $M \times N$ lẻ:
          Ta xếp kín gần hết.
          Chỉ thừa đúng \textbf{1 ô duy nhất}.
      \end{block}
  \end{columns}

  \vspace{1em}
  Ví dụ: Bảng $3 \times 3 = 9$ ô. Xếp được 4 quân (8 ô). Dư 1 ô.
\end{frame}

% 4. Code & Mẹo
\begin{frame}[fragile]{4. Lời Giải & Mẹo C++}
  Trong C++, phép chia số nguyên \texttt{/} tự động "bỏ phần thập phân" (làm tròn xuống).
  
  \begin{itemize}
      \item $8 / 2 = 4$.
      \item $9 / 2 = 4$ (Máy tính bỏ phần .5).
  \end{itemize}
  
  $\rightarrow$ Điều này trùng khớp hoàn toàn với yêu cầu bài toán!

  \begin{block}{Code tham khảo}
\begin{lstlisting}[language=C++]
int m, n;
cin >> m >> n;

// @Tính diện tích và chia đôi@
int dienTich = m * n;
int soQuan = dienTich / 2;

cout << soQuan;
\end{lstlisting}
  \end{block}
\end{frame}

\end{document}
