\documentclass[10pt, aspectratio=169]{beamer}

% --- Cấu hình Tiếng Việt và Font (Khuyên dùng XeLaTeX) ---
\usepackage{fontspec}
\setsansfont{Arial} % Hoặc font hệ thống khác như Calibri, Roboto
\usepackage{polyglossia}
\setdefaultlanguage{vietnamese}

% --- Gói giao diện và màu sắc ---
\usetheme{Madrid}
\usecolortheme{beaver} % Màu đỏ/xám tạo cảm giác cảnh báo/tư duy
\usepackage{xcolor}
\usepackage{listings} % Để hiển thị code

% --- Cấu hình hiển thị Code C++ ---
\definecolor{codegreen}{rgb}{0,0.6,0}
\definecolor{codegray}{rgb}{0.5,0.5,0.5}
\definecolor{codepurple}{rgb}{0.58,0,0.82}
\definecolor{backcolour}{rgb}{0.95,0.95,0.92}

\lstset{
    language=C++,
    backgroundcolor=\color{backcolour},   
    commentstyle=\color{codegreen},
    keywordstyle=\color{magenta},
    numberstyle=\tiny\color{codegray},
    stringstyle=\color{codepurple},
    basicstyle=\ttfamily\footnotesize,
    breakatwhitespace=false,         
    breaklines=true,                 
    captionpos=b,                    
    keepspaces=true,                 
    numbers=left,                    
    numbersep=5pt,                  
    showspaces=false,                
    showstringspaces=false,
    showtabs=false,                  
    tabsize=2,
    extendedchars=true,
    inputencoding=utf8
}

% --- Thông tin bài giảng ---
\title[Codeforces 80A]{Codeforces 80A - Panoramix's Prediction}
\subtitle{Tư duy thuật toán theo triết lý "Learning How to Learn"}
\author{Coach Tư Duy Thuật Toán}
\date{\today}

\begin{document}

% --- Slide Tiêu đề ---
\begin{frame}
    \titlepage
\end{frame}

% --- Slide Mục lục ---
\begin{frame}{Lộ trình Tư duy}
    \tableofcontents
\end{frame}

% =========================================================================
\section{Khởi động & Phẫu thuật đề bài}
% =========================================================================

\begin{frame}{Chào mừng đến lớp huấn luyện tư duy!}
    \begin{block}{Lời nhắn từ Coach}
        Đừng vội lao vào viết hàm \texttt{isPrime} hay vòng lặp vội. Hãy để bàn phím sang một bên và khởi động não bộ trước.
    \end{block}

    \vspace{0.5cm}
    Chúng ta sẽ xử lý bài \textbf{Codeforces 80A - Panoramix's Prediction}.
\end{frame}

\begin{frame}{Bước 1: Tiếp nhận \& Phẫu thuật (Briefing)}
    Đề bài gốc kể về một câu chuyện tiên tri, nhưng hãy lột bỏ lớp vỏ bọc đó đi. Đây là "cốt lõi" trần trụi:

    \begin{itemize}
        \item \textbf{Input:} 2 số nguyên, gọi là $n$ và $m$. (Đề đảm bảo $n$ đã là số nguyên tố).
        \item \textbf{Yêu cầu:} Kiểm tra xem $m$ có phải là \textbf{số nguyên tố liền kề ngay sau} $n$ hay không.
        \item \textbf{Output:} In ra \texttt{YES} nếu đúng, \texttt{NO} nếu sai.
    \end{itemize}

    \begin{alertblock}{Lộ trình tư duy (3 Chunks)}
    \begin{enumerate}
        \item \textbf{Chunk 1:} Định nghĩa "Người hàng xóm tiếp theo".
        \item \textbf{Chunk 2:} Chiến thuật dò tìm (The Blind Walk).
        \item \textbf{Chunk 3:} Cú chốt hạ (Phán quyết).
    \end{enumerate}
    \end{alertblock}
\end{frame}

% =========================================================================
\section{Chunk 1: Định nghĩa logic}
% =========================================================================

\begin{frame}{Chunk 1: Định nghĩa "Người hàng xóm tiếp theo"}
    Hãy tưởng tượng tập hợp các số tự nhiên là một con đường dài.
    
    \begin{itemize}
        \item Các \textbf{Trạm dừng xe buýt} = \textbf{Số Nguyên Tố} (2, 3, 5, 7, 11...).
        \item Bãi cỏ ven đường = Hợp số (4, 6, 8, 9...).
    \end{itemize}

    \vspace{0.3cm}
    \textbf{Tình huống:} Chiếc xe buýt đang đỗ ở trạm $n$.
    \\ \textbf{Câu hỏi:} "Nếu xe lăn bánh tiếp, trạm dừng \textbf{đầu tiên} nó gặp có phải là $m$ không?"

    \begin{exampleblock}{Cạm bẫy (Traps) \emoji{warning}}
        \begin{itemize}
            \item \textbf{Sai lầm 1:} Chỉ kiểm tra xem $m$ có phải là số nguyên tố hay không. (Chưa đủ!)
            \item \textbf{Sai lầm 2:} Thấy $m > n$ và là số nguyên tố, vội kết luận YES. (Sai vì có thể bị "nhảy cóc").
        \end{itemize}
    \end{exampleblock}
\end{frame}

\begin{frame}{Thử thách tư duy (Mental Check)}
    Giả sử xe đang ở trạm $n = 7$.
    Hãy xác định kết quả \textbf{YES/NO} cho các trường hợp sau:

    \begin{enumerate}
        \item Case A: $m = 9$
        \item Case B: $m = 13$
        \item Case C: $m = 11$
    \end{enumerate}
    
    \vspace{0.5cm}
    \textit{(Hãy suy nghĩ trước khi sang slide tiếp theo...)}
\end{frame}

\begin{frame}{Giải đáp Thử thách}
    \begin{itemize}
        \item \textbf{1. Case A ($m=9$): NO}. (9 không phải trạm dừng - không phải SNT).
        \item \textbf{2. Case B ($m=13$): NO}. \textbf{Bẫy!} Dù 13 là SNT, nhưng xe buýt đã \textit{bỏ quên} trạm số 11. Đề bài yêu cầu "liền kề".
        \item \textbf{3. Case C ($m=11$): YES}. (Trạm dừng ngay tiếp theo).
    \end{itemize}
    
    \begin{block}{Kết luận Chunk 1}
        Phải tìm số nguyên tố \textbf{gần nhất} lớn hơn $n$, không được nhảy cóc.
    \end{block}
\end{frame}

% =========================================================================
\section{Chunk 2: Chiến thuật dò tìm}
% =========================================================================

\begin{frame}{Chunk 2: Chiến thuật dò tìm (The Blind Walk)}
    Máy tính không có bảng số nguyên tố trong đầu, nó phải "mò mẫm".
    
    \textbf{Hình ảnh ẩn dụ:} Bạn đứng ở $n$, bị bịt mắt, tay cầm máy dò SNT.
    Bạn nhích từng bước: $n+1, n+2, n+3 \dots$
    
    \begin{itemize}
        \item Máy kêu "Tít tít" (Là SNT) $\rightarrow$ \textbf{DỪNG LẠI NGAY!} (Đây là \texttt{True\_Next}).
        \item Máy "Im lặng" (Hợp số) $\rightarrow$ Bước tiếp.
    \end{itemize}

    \begin{block}{Kỹ thuật}
        Cần một vòng lặp chạy từ $i = n + 1$. Vòng lặp sẽ \texttt{break} ngay khi gặp SNT đầu tiên.
    \end{block}
\end{frame}

\begin{frame}[fragile]{Thử thách tư duy (Tracing)}
    Giả sử $n = 8$ (Test logic dò tìm).
    Liệt kê quá trình "bước đi" để tìm \texttt{True\_Next}:
    
    \vspace{0.3cm}
    \pause
    \begin{itemize}
        \item \textbf{Bước 1:} Kiểm tra 9 $\rightarrow$ Không phải SNT $\rightarrow$ Đi tiếp.
        \item \textbf{Bước 2:} Kiểm tra 10 $\rightarrow$ Không phải SNT $\rightarrow$ Đi tiếp.
        \item \textbf{Bước 3:} Kiểm tra 11 $\rightarrow$ LÀ SNT! $\rightarrow$ \textbf{DỪNG}.
    \end{itemize}
    
    \vspace{0.3cm}
    $\Rightarrow$ \texttt{True\_Next} của 8 là 11.
    
    \begin{lstlisting}[caption={Mô phỏng logic trong đầu}]
// Logic này sẽ được chuyển thành code
for (int i = n + 1; ; i++) {
    if (check_SNT(i)) {
        true_next = i;
        break; // Tim thay roi thi dung ngay
    }
}
    \end{lstlisting}
\end{frame}

% =========================================================================
\section{Chunk 3: Phán quyết}
% =========================================================================

\begin{frame}{Chunk 3: Phán quyết (Judgment Day)}
    Sau khi vòng lặp kết thúc, ta có \texttt{True\_Next}.
    Lúc này mới nhìn đến số $m$ của đề bài.
    
    \vspace{0.5cm}
    \textbf{So sánh thẻ căn cước:}
    \begin{itemize}
        \item Nếu \texttt{True\_Next} == $m$ $\rightarrow$ In \textbf{YES}.
        \item Nếu \texttt{True\_Next} != $m$ $\rightarrow$ In \textbf{NO}.
    \end{itemize}
    
    \textit{Lưu ý: Không cần kiểm tra tính nguyên tố của $m$ nữa, vì \texttt{True\_Next} chắc chắn đã là SNT rồi.}
\end{frame}

\begin{frame}{Tổng kết thuật toán (Blueprint)}
    \begin{enumerate}
        \item \textbf{Hàm phụ \texttt{check\_SNT(k)}:} Kiểm tra $k$ có phải SNT không.
        \item \textbf{Nhập} $n, m$.
        \item \textbf{Vòng lặp (The Blind Walk):}
        \begin{itemize}
            \item Chạy $i$ từ $n + 1$.
            \item Nếu \texttt{check\_SNT(i)} đúng $\rightarrow$ Dừng, lưu vào \texttt{True\_Next}.
            \item Nếu sai $\rightarrow$ Tăng $i$.
        \end{itemize}
        \item \textbf{So sánh:} \texttt{True\_Next} == $m$?
    \end{enumerate}
\end{frame}

% =========================================================================
\section{Thử thách cuối cùng (Edge Cases)}
% =========================================================================

\begin{frame}{Thử thách cuối cùng (Final Boss)}
    \textbf{Đề bài:} Giới hạn $n, m \le 50$.
    \\ \textbf{Tình huống:} $n = 47$ (47 là SNT).
    \\ Vòng lặp chạy: 48... 49... 50...
    
    \vspace{0.5cm}
    \textbf{Câu hỏi:}
    \begin{enumerate}
        \item \texttt{True\_Next} tìm được là số mấy?
        \item Nếu đề cho $m = 53$, kết quả in ra là gì?
    \end{enumerate}
\end{frame}

\begin{frame}{Giải mã sai lầm thường gặp}
    \begin{alertblock}{Bức tường vô hình}
        Nhiều bạn nghĩ in ra NO hoặc không in gì vì 53 vượt quá giới hạn 50 của đề bài.
    \end{alertblock}

    \pause
    \begin{block}{Sự thật (The Truth)}
        \textbf{Giới hạn chỉ áp dụng cho INPUT, không áp dụng cho LOGIC trung gian.}
        \begin{enumerate}
            \item Máy tính chạy từ 47 $\rightarrow$ tìm thấy 53 (SNT).
            \item So sánh: \texttt{True\_Next} (53) == $m$ (53).
            \item Kết quả: \textbf{YES}.
        \end{enumerate}
    \end{block}
    \textbf{Kết luận:} Thuật toán phải đúng về mặt toán học, bất kể giới hạn đầu vào.
\end{frame}

\begin{frame}{Bật đèn xanh (Green Light) \emoji{rocket}}
    Tư duy đã thông suốt! Bạn đã có đủ 3 mảnh ghép:
    
    \begin{enumerate}
        \item Hàm \texttt{check\_SNT}.
        \item Vòng lặp tìm \texttt{True\_Next} (từ $n+1$).
        \item So sánh kết quả.
    \end{enumerate}

    \vspace{1cm}
    \centering
    \Large \textbf{Hãy mở Editor và Code ngay bài Codeforces 80A!}
    
    \small \textit{(Nếu sai, hãy xem lại logic vòng lặp dò tìm)}
\end{frame}

\end{document}