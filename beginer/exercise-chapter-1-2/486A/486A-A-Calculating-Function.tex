\documentclass{beamer}
\usepackage[utf8]{inputenc}
\usepackage[vietnamese]{babel}
\usepackage{amsmath}
\usepackage{amssymb} % Thêm gói này để dùng các ký hiệu toán học
\usepackage{xcolor}

% Chọn theme
\usetheme{Madrid}
\usecolortheme{whale}

% Thông tin bài thuyết trình
\title[Codeforces 486A]{Giải Mã Tư Duy: Codeforces 486A}
\subtitle{Calculating Function - Learning How to Learn Edition}
\author{Coach Tư Duy Thuật Toán}
\date{\today}

\begin{document}

% Slide tiêu đề
\begin{frame}
    \titlepage
\end{frame}

% Mục lục
\begin{frame}{Lộ trình tư duy}
    \tableofcontents
\end{frame}

% --- BƯỚC 1 ---
\section{Bước 1: Phẫu thuật đề bài}
\begin{frame}{Bước 1: Phẫu thuật đề bài (Deconstruct)}
    \begin{block}{Công thức đề bài}
        $$f(n) = -1 + 2 - 3 + 4 - 5 + ... + (-1)^n n$$
    \end{block}

    \vspace{0.5cm}
    \textbf{[Dịch sang ngôn ngữ con người]}
    \begin{itemize}
        \item Số \textbf{LẺ} (1, 3, 5...): Mang dấu \textbf{TRỪ (-)}.
        \item Số \textbf{CHẴN} (2, 4, 6...): Mang dấu \textbf{CỘNG (+)}.
    \end{itemize}

    \vspace{0.5cm}
    \textbf{[Lộ trình tư duy]}
    \begin{enumerate}
        \item \textbf{Quan sát:} Tính tay thử vài số nhỏ để thấy "nhịp điệu".
        \item \textbf{Cạm bẫy:} Tại sao cách "trâu bò" (vòng lặp) lại thất bại?
        \item \textbf{Giải pháp:} Tìm công thức toán học $O(1)$.
    \end{enumerate}
\end{frame}

% --- CHUNK 1 ---
\section{Chunk 1: Quan sát quy luật}
\begin{frame}{Chunk 1: Quan sát quy luật (Pattern Recognition)}
    \textbf{Ẩn dụ:} Trò chơi "Kéo co logic".
    \begin{itemize}
        \item Số lẻ kéo lùi (Âm).
        \item Số chẵn đẩy tiến (Dương).
    \end{itemize}

    \vspace{0.5cm}

    \begin{alertblock}{Cạm bẫy (Trap)}
        Rất nhiều bạn nhìn thấy dấu $...$ là nghĩ ngay đến vòng lặp \texttt{for}. \\
        \textbf{Đừng làm thế!} Với $n = 10^{15}$, vòng lặp sẽ bị \textbf{Time Limit Exceeded}.
    \end{alertblock}

    \begin{exampleblock}{Thử thách tư duy (Mental Check)}
        Hãy tính nhẩm kết quả:
        \begin{itemize}
            \item $n = 4 \rightarrow -1 + 2 - 3 + 4 = \mathbf{2}$
            \item $n = 5 \rightarrow -1 + 2 - 3 + 4 - 5 = \mathbf{-3}$
            \item $n = 6 \rightarrow -1 + 2 - 3 + 4 - 5 + 6 = \mathbf{3}$
        \end{itemize}
    \end{exampleblock}
\end{frame}

% --- CHUNK 2 ---
\section{Chunk 2: Tìm công thức (Số Chẵn)}
\begin{frame}{Chunk 2: Tìm công thức Thần thánh (Trường hợp Chẵn)}
    \textbf{Quan sát kết quả với số CHẴN ($n$ is Even):}
    \begin{itemize}
        \item $n = 4 \rightarrow$ Kết quả là $2$
        \item $n = 6 \rightarrow$ Kết quả là $3$
    \end{itemize}

    \vspace{0.5cm}
    \textbf{Giải thích bằng ẩn dụ:}
    Gom thành từng cặp "đôi bạn cùng tiến":
    $$(-1 + 2) + (-3 + 4) + ...$$
    Mỗi cặp có giá trị là $1$. Với $n$ số, ta có $n/2$ cặp.

    \begin{block}{KEY: Công thức cho số Chẵn}
        $$\text{Result} = \frac{n}{2}$$
    \end{block}
    \textit{Ví dụ:} $n=100 \rightarrow 100/2 = 50$.
\end{frame}

% --- CHUNK 3 ---
\section{Chunk 3: Tìm công thức (Số Lẻ)}
\begin{frame}{Chunk 3: Chốt công thức (Trường hợp Lẻ)}
    \textbf{Quan sát kết quả với số LẺ ($n$ is Odd):}
    \begin{itemize}
        \item $n = 1 \rightarrow -1$
        \item $n = 3 \rightarrow -2$
        \item $n = 5 \rightarrow -3$
        \item $n = 99 \rightarrow -50$
    \end{itemize}

    \vspace{0.5cm}
    \textbf{Quy luật hình ảnh:}
    Kết quả giống phép chia đôi nhưng "làm tròn lên" và thêm dấu âm.

    \begin{block}{KEY: Công thức cho số Lẻ}
        $$\text{Result} = -\frac{n + 1}{2}$$
    \end{block}
    \textit{Kiểm chứng:} $n=5 \rightarrow -(5+1)/2 = -3$. (Đúng!)
\end{frame}

% --- TỔNG KẾT ---
\section{Tổng kết Thuật toán}
\begin{frame}{Tổng kết mảnh ghép (The Algorithm)}
    Chúng ta không cần vòng lặp. Chúng ta chỉ cần logic rẽ nhánh:

    \vspace{0.5cm}
    \begin{itemize}
        \item \textbf{Trường hợp 1:} Nếu $n$ Chẵn ($n \% 2 == 0$)
            $$\Rightarrow \text{In ra: } n / 2$$
        \item \textbf{Trường hợp 2:} Nếu $n$ Lẻ ($n \% 2 != 0$)
            $$\Rightarrow \text{In ra: } -(n + 1) / 2$$
    \end{itemize}

    \vspace{0.5cm}
    \centering
    \textbf{Độ phức tạp:} $O(1)$ (Nhanh nhất có thể).
\end{frame}

% --- CẢNH BÁO ---
\section{Cảnh báo quan trọng}
\begin{frame}{Cảnh báo cuối cùng (Edge Case \& Data Type)}
    \begin{alertblock}{Đề bài: $n \le 10^{15}$}
        Đây là con số "siêu to khổng lồ".
    \end{alertblock}

    \vspace{0.5cm}
    \textbf{Vấn đề:}
    \begin{itemize}
        \item Nếu dùng kiểu \texttt{int} (trong C++/Java): Chỉ chứa được khoảng $2 \times 10^9$.
        \item Kết quả: \textbf{Tràn số (Overflow)} $\rightarrow$ Sai kết quả.
    \end{itemize}

    \vspace{0.5cm}
    \textbf{Giải pháp:}
    \begin{itemize}
        \item[\checkmark] \textbf{C++:} Phải dùng \texttt{long long}.
        \item[\checkmark] \textbf{Java:} Phải dùng \texttt{long}.
        \item[\checkmark] \textbf{Python:} Tự động xử lý số lớn (An toàn).
    \end{itemize}
\end{frame}

% --- KẾT THÚC ---
\begin{frame}
    \centering
    \Huge
    \textbf{Bạn đã sẵn sàng Code chưa?} \\
    \vspace{1cm}
    \large
    Hãy cẩn thận với \texttt{long long}!
\end{frame}

\end{document}