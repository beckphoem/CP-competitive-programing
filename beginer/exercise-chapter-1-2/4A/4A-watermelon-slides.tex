\documentclass{beamer}
\usepackage[utf8]{inputenc}
\usepackage[vietnamese]{babel}
\usepackage{tcolorbox}
\usepackage{listings}
\usepackage{xcolor}
\usepackage{graphicx}

\usetheme{Madrid}
\usecolortheme{default}

% Colors for code blocks
\definecolor{codegreen}{rgb}{0,0.6,0}
\definecolor{codegray}{rgb}{0.5,0.5,0.5}
\definecolor{codepurple}{rgb}{0.58,0,0.82}
\definecolor{backcolour}{rgb}{0.95,0.95,0.92}

\lstdefinestyle{mystyle}{
    backgroundcolor=\color{backcolour},
    commentstyle=\color{codegreen},
    keywordstyle=\color{magenta},
    numberstyle=\tiny\color{codegray},
    stringstyle=\color{codepurple},
    basicstyle=\ttfamily\footnotesize,
    breakatwhitespace=false,
    breaklines=true,
    captionpos=b,
    keepspaces=true,
    numbers=left,
    numbersep=5pt,
    showspaces=false,
    showstringspaces=false,
    showtabs=false,
    tabsize=2,
    escapechar=@
}

\lstset{style=mystyle}

\title[Watermelon]{Bài Tập Codeforces 4A: Watermelon \\ "Chia Dưa Hấu"}
\subtitle{C++ Competitive Programming Series - Elo 800}
\author{Học Cùng C++}
\date{}

\begin{document}

\begin{frame}
  \titlepage
\end{frame}

% 1. Đề bài
\begin{frame}{1. Đề Bài}
  \begin{block}{Cốt truyện}
      Pete và Billy mua một quả dưa hấu nặng $w$ kg.
      Họ muốn chia quả dưa thành \textbf{2 phần} sao cho:
      \begin{itemize}
          \item Mỗi phần có khối lượng là số dương ($>0$).
          \item Khối lượng của \textbf{cả hai phần} đều phải là \textbf{số chẵn}.
      \end{itemize}
  \end{block}

  \begin{exampleblock}{Yêu cầu}
      Nhập $w$. In \texttt{YES} nếu chia được, \texttt{NO} nếu không.
  \end{exampleblock}
\end{frame}

% 2. Phân tích
\begin{frame}{2. Phân Tích Logic}
  Gọi hai phần là $a$ và $b$.
  Ta có: $a + b = w$.
  
  Yêu cầu: $a$ là số chẵn, $b$ là số chẵn.
  
  \begin{alertblock}{Tính chất toán học}
      Tổng của hai số chẵn luôn là một số chẵn.
      \[ \text{Chẵn} + \text{Chẵn} = \text{Chẵn} \]
  \end{alertblock}
  
  $\rightarrow$ Điều kiện cần đầu tiên: $w$ phải là \textbf{số chẵn}.
  Nếu $w$ là số lẻ (3, 5, 7...), chắc chắn in \texttt{NO}.
\end{frame}

% 3. Cạm bẫy
\begin{frame}{3. Cạm Bẫy (Edge Case)}
  Rất nhiều bạn chỉ kiểm tra: `if (w \% 2 == 0) cout << "YES";`
  
  \begin{tcolorbox}[colback=red!10,colframe=red!70!black,title=Trường hợp đặc biệt: $w = 2$]
      Số 2 là số chẵn.
      Nhưng ta chỉ có thể chia $2 = 1 + 1$.
      \begin{itemize}
          \item 1 là số lẻ.
          \item 1 là số lẻ.
      \end{itemize}
      $\rightarrow$ Không thỏa mãn yêu cầu "cả 2 phần đều chẵn".
  \end{tcolorbox}
  
  \textbf{Kết luận:} $w$ phải chẵn VÀ $w > 2$.
\end{frame}

% 4. Ví dụ
\begin{frame}{4. Kiểm Chứng Test Case}
  \begin{table}[]
      \begin{tabular}{|c|c|c|l|}
      \hline
      \textbf{w} & \textbf{Chẵn?} & \textbf{> 2?} & \textbf{Kết quả} \\ \hline
      8 & Có & Có & \textbf{YES} ($4+4$ hoặc $2+6$) \\ \hline
      5 & Không & Có & \textbf{NO} (Số lẻ) \\ \hline
      100 & Có & Có & \textbf{YES} ($2+98$) \\ \hline
      2 & Có & \textbf{Không} & \textbf{NO} ($1+1$, không phải chẵn) \\ \hline
      \end{tabular}
  \end{table}
\end{frame}

% 5. Code
\begin{frame}[fragile]{5. Lời Giải Tham Khảo}
\begin{lstlisting}[language=C++]
#include <iostream>
using namespace std;

int main() {
    int w;
    cin >> w;

    // Kiem tra w chan VA w lon hon 2
    if (w % 2 == 0 && w > 2) {
        cout << "YES";
    } else {
        cout << "NO";
    }

    return 0;
}
\end{lstlisting}
\end{frame}

\end{document}
