\documentclass{beamer}
\usepackage[utf8]{inputenc}
\usepackage[vietnamese]{babel}
\usepackage{tcolorbox}
\usepackage{listings}
\usepackage{xcolor}
\usepackage{graphicx}

\usetheme{Madrid}
\usecolortheme{default}

% Colors for code blocks
\definecolor{codegreen}{rgb}{0,0.6,0}
\definecolor{codegray}{rgb}{0.5,0.5,0.5}
\definecolor{codepurple}{rgb}{0.58,0,0.82}
\definecolor{backcolour}{rgb}{0.95,0.95,0.92}

\lstdefinestyle{mystyle}{
    backgroundcolor=\color{backcolour},   
    commentstyle=\color{codegreen},
    keywordstyle=\color{magenta},
    numberstyle=\tiny\color{codegray},
    stringstyle=\color{codepurple},
    basicstyle=\ttfamily\footnotesize,
    breakatwhitespace=false,         
    breaklines=true,                 
    captionpos=b,                    
    keepspaces=true,                 
    numbers=left,                    
    numbersep=5pt,                  
    showspaces=false,                
    showstringspaces=false,
    showtabs=false,                  
    tabsize=2,
    escapechar=@
}

\lstset{style=mystyle}

\title[Theatre Square]{Bài Tập Codeforces 1A: Theatre Square \\ "Lát Sân Quảng Trường"}
\subtitle{C++ Competitive Programming Series - Elo 800}
\author{Học Cùng C++}
\date{}

\begin{document}

\begin{frame}
  \titlepage
\end{frame}

% 1. Hình Dung
\begin{frame}{1. Hình Dung Bài Toán}
  \begin{block}{Đề bài}
      Bạn có sân kích thước $n \times m$. Cần lát kín bằng các viên gạch vuông $a \times a$.
      \begin{itemize}
          \item Không được cắt gạch.
          \item Được phép lát thừa ra ngoài.
          \item Gạch phải đặt thẳng hàng.
      \end{itemize}
  \end{block}

  \begin{alertblock}{Mục tiêu}
      Tìm số lượng viên gạch ÍT NHẤT cần dùng.
  \end{alertblock}
\end{frame}

% 2. Sai Lầm
\begin{frame}{2. Sai Lầm Thường Gặp: Tính Diện Tích}
  Nhiều bạn nghĩ:
  \[ \text{Số gạch} = \frac{\text{Diện tích sân}}{\text{Diện tích gạch}} = \frac{n \times m}{a \times a} \]
  
  \textbf{Tại sao sai?}
  \begin{itemize}
      \item Ví dụ: Sân $1 \times 5$, Gạch $2 \times 2$.
      \item Diện tích sân = 5. Diện tích gạch = 4.
      \item Phép chia: $5 / 4 = 1.25$.
      \item Thực tế: Chiều rộng 1 (cần 1 gạch 2m). Chiều dài 5 (cần 3 gạch 2m).
      \item Tổng cần: $1 \times 3 = 3$ viên. (Khác xa so với 1.25).
  \end{itemize}
\end{frame}

% 3. Tư Duy Đúng
\begin{frame}{3. Tư Duy Đúng: Chia Theo Cạnh}
  Chúng ta phải tính riêng cho từng chiều:
  \begin{enumerate}
      \item Chiều dài $n$ cần bao nhiêu viên gạch cạnh $a$?
      \item Chiều rộng $m$ cần bao nhiêu viên gạch cạnh $a$?
  \end{enumerate}

  Gọi số gạch dọc là $doc$, số gạch ngang là $ngang$.
  \[ \text{Tổng số gạch} = doc \times ngang \]

  \begin{block}{Vấn đề phép chia}
      Nếu $n = 6, a = 4$. $6 / 4 = 1$ (dư 2).
      Vì không được cắt gạch, phần dư 2 mét này \textbf{bắt buộc} phải dùng thêm 1 viên nguyên nữa.
      $\rightarrow$ Phải làm tròn lên (Ceiling).
  \end{block}
\end{frame}

% NEW SECTION: Bus Metaphor
\begin{frame}{4. Ẩn Dụ: Đi Xe Buýt}
  Hãy tưởng tượng bài toán Lát gạch giống hệt bài toán \textbf{Thuê Xe Buýt}:
  \begin{itemize}
      \item $n$ người cần đi, mỗi xe chở được $a$ người.
      \item Hỏi cần thuê ít nhất bao nhiêu xe? (Không bỏ lại ai).
  \end{itemize}

  \begin{columns}
      \column{0.5\textwidth}
      \begin{block}{Trường hợp 1: Chia hết}
          $n=10, a=5$.
          \begin{itemize}
              \item $10/5 = 2$ dư $0$.
              \item $\rightarrow$ Cần đúng 2 xe.
          \end{itemize}
      \end{block}

      \column{0.5\textwidth}
      \begin{block}{Trường hợp 2: Có dư}
          $n=13, a=5$.
          \begin{itemize}
              \item $13/5 = 2$ dư $3$.
              \item 3 người này cần thêm 1 xe nữa.
              \item $\rightarrow$ Cần $2 + 1 = 3$ xe.
          \end{itemize}
      \end{block}
  \end{columns}
\end{frame}

% NEW SECTION: Thinking with IF-ELSE
\begin{frame}[fragile]{5. Tư Duy Lập Trình: If-Else}
  Với máy tính, ta biểu diễn logic "Đi Xe Buýt" như sau:

\begin{lstlisting}[language=C++]
long long soGach;

if (n % a == 0) {
    // @Trường hợp chia hết: Vừa đủ@
    soGach = n / a;
} else {
    // @Trường hợp có dư: Phải thêm 1 viên@
    soGach = n / a + 1;
}
\end{lstlisting}

  $\rightarrow$ Cách này \textbf{dễ hiểu} và \textbf{an toàn} nhất cho người mới bắt đầu.
\end{frame}

% 4. Cong Thuc (Moved to Advanced)
\begin{frame}[fragile]{6. Cách Viết Tắt (Nâng Cao)}
  Dân "Chuyên Tin" thường gộp logic \texttt{if-else} trên thành 1 dòng duy nhất để code nhanh hơn:
  
  \begin{tcolorbox}[colback=yellow!10,colframe=orange!70!black,title=Công thức Ceiling]
      \[ \text{Kết quả} = (n + a - 1) / a \]
  \end{tcolorbox}

  \textbf{Tại sao nó đúng?}
  \begin{itemize}
      \item Logic: "Đổ thêm $a-1$ đơn vị vào để làm tràn ly".
      \item Nếu chia hết: Cộng $a-1$ không đủ tràn.
      \item Nếu dư: Cộng $a-1$ sẽ đẩy thương số lên +1.
  \end{itemize}
\end{frame}

% 5. Cạm Bẫy Data Type
\begin{frame}[fragile]{7. Cạm Bẫy Lớn Nhất: Tràn Số}
  \begin{itemize}
      \item Đề bài cho $n, m$ lên tới $10^9$.
      \item Nếu dùng \texttt{int}, tích $n \times m$ có thể lên tới $10^{18}$.
      \item \texttt{int} chỉ chứa được $\approx 2 \times 10^9$.
      \item $\rightarrow$ Bị \textbf{Tràn số (Overflow)} $\rightarrow$ Kết quả sai hoặc âm.
  \end{itemize}

  \begin{alertblock}{Giải pháp}
      Bắt buộc dùng \textbf{\texttt{long long}} trong C++.
  \end{alertblock}
\end{frame}

% 6. Code
\begin{frame}[fragile]{8. Lời Giải Tham Khảo}
\begin{lstlisting}[language=C++]
#include <iostream>
using namespace std;

int main() {
    // @1. Dùng long long để tránh tràn số@
    long long n, m, a;
    cin >> n >> m >> a;

    // @2. Tính số gạch cho chiều Dọc (n)@
    long long doc;
    if (n % a == 0) doc = n / a;
    else doc = n / a + 1;

    // @3. Tính số gạch cho chiều Ngang (m)@
    long long ngang;
    if (m % a == 0) ngang = m / a;
    else ngang = m / a + 1;

    // @3. Kết quả là tích của hai chiều@
    cout << doc * ngang;

    return 0;
}
\end{lstlisting}
\end{frame}

\end{document}
