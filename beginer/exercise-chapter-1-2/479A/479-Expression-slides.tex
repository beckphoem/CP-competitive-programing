\documentclass{beamer}
\usepackage[utf8]{inputenc}
\usepackage[vietnamese]{babel}
\usepackage{tcolorbox}
\usepackage{listings}
\usepackage{xcolor}
\usepackage{graphicx}
\usepackage{amsmath}

\usetheme{Madrid}
\usecolortheme{default}

% Colors for code blocks
\definecolor{codegreen}{rgb}{0,0.6,0}
\definecolor{codegray}{rgb}{0.5,0.5,0.5}
\definecolor{codepurple}{rgb}{0.58,0,0.82}
\definecolor{backcolour}{rgb}{0.95,0.95,0.92}

\lstdefinestyle{mystyle}{
    backgroundcolor=\color{backcolour},
    commentstyle=\color{codegreen},
    keywordstyle=\color{magenta},
    numberstyle=\tiny\color{codegray},
    stringstyle=\color{codepurple},
    basicstyle=\ttfamily\footnotesize,
    breakatwhitespace=false,
    breaklines=true,
    captionpos=b,
    keepspaces=true,
    numbers=left,
    numbersep=5pt,
    showspaces=false,
    showstringspaces=false,
    showtabs=false,
    tabsize=2,
    escapechar=@
}

\lstset{style=mystyle}

\title[Expression]{Bài Tập Codeforces 479A: Expression \\ "Biểu Thức Lớn Nhất"}
\subtitle{C++ Competitive Programming Series - Elo 1000}
\author{Học Cùng C++}
\date{}

\begin{document}

\begin{frame}
  \titlepage
\end{frame}

% 1. Đề bài
\begin{frame}{1. Đề Bài}
  \begin{block}{Cốt truyện}
      Bạn được cung cấp 3 số nguyên dương $a, b, c$.
      \begin{itemize}
          \item Ba số này giữ nguyên vị trí, không được đổi chỗ.
          \item Bạn được phép chèn các dấu cộng ($+$), nhân ($*$) và ngoặc đơn $()$.
      \end{itemize}
  \end{block}

  \begin{exampleblock}{Yêu cầu}
      Tìm cách đặt dấu sao cho giá trị biểu thức thu được là \textbf{LỚN NHẤT}.
  \end{exampleblock}
\end{frame}

% 2. Phân tích
\begin{frame}{2. Phân Tích Logic}
  Có 4 kịch bản chính để tạo ra số lớn nhất từ 3 số $a, b, c$:
  
  \begin{itemize}
      \item \textbf{Cộng tất cả}: $a + b + c$ (Tốt khi có nhiều số 1)
      \item \textbf{Nhân tất cả}: $a * b * c$ (Tốt khi các số đều lớn)
      \item \textbf{Gom nhóm trái}: $(a + b) * c$ (Ưu tiên cộng số nhỏ trước rồi nhân)
      \item \textbf{Gom nhóm phải}: $a * (b + c)$
  \end{itemize}

  \begin{alertblock}{Lưu ý quan trọng}
      Ta không cần xét $a + b * c$ hay $a * b + c$ vì chúng luôn nhỏ hơn hoặc bằng các trường hợp có ngoặc.
  \end{alertblock}
\end{frame}

% 3. Chiến thuật
\begin{frame}{3. Chiến Thuật "Quăng Lưới" (Brute Force)}
  Thay vì xét quá nhiều trường hợp `if-else` phức tạp (như kiểm tra số 1 nằm ở đâu), ta có thể áp dụng chiến thuật đơn giản:
  
  \begin{enumerate}
      \item Máy tính tính toán rất nhanh.
      \item Số lượng công thức cần thử rất ít (chỉ có 4 công thức).
  \end{enumerate}
  
  $\rightarrow$ \textbf{Giải pháp:} Tính cả 4 giá trị và lấy giá trị lớn nhất (Max).
  
  \[ \text{Result} = \max(ans_1, ans_2, ans_3, ans_4) \]
\end{frame}

% 4. Ví dụ
\begin{frame}{4. Kiểm Chứng Test Case}
  \begin{table}[]
      \begin{tabular}{|c|l|c|}
      \hline
      \textbf{Input} & \textbf{Các trường hợp} & \textbf{Max} \\ \hline
      \texttt{1 2 3} & 
      \parbox{5cm}{
      $1+2+3=6$ \\ 
      $1*2*3=6$ \\ 
      $(1+2)*3=9$ \checkmark \\ 
      $1*(2+3)=5$
      } & \textbf{9} \\ \hline
      \texttt{2 10 3} & 
      \parbox{5cm}{
      $2+10+3=15$ \\ 
      $2*10*3=60$ \checkmark \\ 
      $(2+10)*3=36$ \\ 
      $2*(10+3)=26$
      } & \textbf{60} \\ \hline
      \texttt{1 1 10} & $(1+1)*10=20$ & \textbf{20} \\ \hline
      \end{tabular}
  \end{table}
\end{frame}

% 5. Code
\begin{frame}[fragile]{5. Lời Giải Tham Khảo}
\begin{lstlisting}[language=C++]
#include <iostream>
#include <algorithm> // De dung ham max
using namespace std;

int main() {
    int a, b, c;
    cin >> a >> b >> c;

    int ans1 = a + b + c;
    int ans2 = a * b * c;
    int ans3 = (a + b) * c;
    int ans4 = a * (b + c);

    // Tim so lon nhat trong 4 so
    int result = max({ans1, ans2, ans3, ans4});
    
    cout << result;
    return 0;
}
\end{lstlisting}
\begin{itemize}
    \item Lưu ý: Hàm \texttt{max} với danh sách khởi tạo \texttt{\{...\}} có trong C++11 trở lên.
\end{itemize}
\end{frame}

\end{document}
