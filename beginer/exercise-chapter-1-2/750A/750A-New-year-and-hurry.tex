\documentclass{beamer}

% --- CẤU HÌNH TIẾNG VIỆT (QUAN TRỌNG) ---
\usepackage[utf8]{inputenc}
\usepackage[T5]{fontenc} % Bắt buộc phải là T5 cho tiếng Việt
\usepackage[vietnam]{babel}
\usepackage{lmodern}     % Dùng font Latin Modern để không bị lỗi font

% --- CẤU HÌNH THEME ---
\usetheme{Madrid}
\usecolortheme{beaver} 
\setbeamertemplate{navigation symbols}{}

% --- CẤU HÌNH CODE BLOCK ---
\usepackage{listings}
\usepackage{xcolor}

\definecolor{codegreen}{rgb}{0,0.6,0}
\definecolor{codegray}{rgb}{0.5,0.5,0.5}
\definecolor{codepurple}{rgb}{0.58,0,0.82}
\definecolor{backcolour}{rgb}{0.95,0.95,0.92}

\lstdefinestyle{mystyle}{
    backgroundcolor=\color{backcolour},   
    commentstyle=\color{codegreen},
    keywordstyle=\color{magenta},
    numberstyle=\tiny\color{codegray},
    stringstyle=\color{codepurple},
    basicstyle=\ttfamily\footnotesize,
    breakatwhitespace=false,         
    breaklines=true,                 
    captionpos=b,                    
    keepspaces=true,                 
    numbers=left,                    
    numbersep=5pt,                  
    showspaces=false,                
    showstringspaces=false,
    showtabs=false,                  
    tabsize=2,
    language=C++
}
\lstset{style=mystyle}

% --- THÔNG TIN SLIDE ---
\title[750A - New Year and Hurry]{Phân tích Tư duy Thuật toán\\Codeforces 750A - New Year and Hurry}
\subtitle{Learning How to Learn Edition}
\author{Coach Tư duy Thuật toán}
\date{\today}

\begin{document}

% ---------------------------------------------------------
% SLIDE TITLE
% ---------------------------------------------------------
\begin{frame}
    \titlepage
\end{frame}

% ---------------------------------------------------------
% SLIDE MỤC LỤC
% ---------------------------------------------------------
\begin{frame}{Mục lục}
    \tableofcontents
\end{frame}

% ---------------------------------------------------------
% PHẦN 1: GIỚI THIỆU
% ---------------------------------------------------------
\section{Giới thiệu chung}
\begin{frame}{Giới thiệu bài toán}
    Chào mừng bạn đến với thử thách chạy đua với thời gian mang tên \textbf{750A - New Year and Hurry}.
    
    \vspace{0.5cm}
    Đây là một bài toán kinh điển về tư duy:
    \begin{itemize}
        \item \textbf{Mô phỏng (Simulation)}
        \item \textbf{Toán học cơ bản}
    \end{itemize}
    
    \vspace{0.5cm}
    \textit{Hãy cùng tôi phẫu thuật nó!}
\end{frame}

% ---------------------------------------------------------
% PHẦN 2: PHẪU THUẬT ĐỀ BÀI
% ---------------------------------------------------------
\section{Bước 1: Phẫu thuật đề bài}
\begin{frame}{Bước 1: Phẫu thuật đề bài (Deconstruct)}
    Hãy quên anh chàng Limak đi. Chúng ta hãy nhìn bài toán dưới dạng \textbf{Quản lý ngân sách}.

    \begin{enumerate}
        \item \textbf{Tài nguyên tổng:} Thời gian từ 20:00 đến 24:00 (nửa đêm).
        \item \textbf{Chi phí cố định:} Thời gian $k$ phút để đi từ nhà đến bữa tiệc. (Phải dành ra khoản này trước tiên).
        \item \textbf{Chi phí biến đổi:}
            \begin{itemize}
                \item Bài toán 1 tốn: $5 \times 1 = 5$ phút.
                \item Bài toán 2 tốn: $5 \times 2 = 10$ phút.
                \item Bài toán $i$ tốn: $5 \times i$ phút.
            \end{itemize}
        \item \textbf{Mục tiêu:} Giải được \textbf{nhiều bài nhất có thể} (số $n$ lớn nhất) mà không bị "âm" thời gian.
    \end{enumerate}
\end{frame}

\begin{frame}{Lộ trình tư duy}
    \begin{block}{Các mảnh ghép cần giải quyết (Micro-Chunks)}
        \begin{itemize}
            \item \textbf{Chunk 1:} Tính "Túi tiền thời gian" thực tế (Còn lại bao nhiêu phút để làm bài?).
            \item \textbf{Chunk 2:} Cơ chế "Cộng dồn" (Chi phí tăng dần như thế nào?).
            \item \textbf{Chunk 3:} Tìm điểm dừng (Khi nào thì hết tiền?).
        \end{itemize}
    \end{block}
    \vspace{0.5cm}
    \centerline{\textit{Bạn đã sẵn sàng chưa? Chúng ta vào miếng ghép đầu tiên nhé.}}
\end{frame}

% ---------------------------------------------------------
% PHẦN 3: CHUNK 1
% ---------------------------------------------------------
\section{Chunk 1: Tính "Túi tiền thời gian"}
\begin{frame}{Chunk 1: Tính "Túi tiền thời gian" (Time Budget)}
    \begin{block}{Logic (Explain)}
        Trước khi biết mua được bao nhiêu món hàng, bạn phải biết trong túi mình còn chính xác bao nhiêu tiền.
        \begin{itemize}
            \item Khung thời gian: 20:00 đến 24:00.
            \item Không được dùng hết, phải chừa lại $k$ phút đi đường.
        \end{itemize}
    \end{block}

    \begin{exampleblock}{Ẩn dụ}
        Hãy tưởng tượng bạn có một bình xăng. Dung tích bình là khoảng thời gian từ 8h tối đến 12h đêm. Nhưng để lái xe đến đích, bạn mất $k$ lít xăng. Vậy bạn còn bao nhiêu lít xăng để chạy loanh quanh (giải bài)?
    \end{exampleblock}

    \begin{alertblock}{Bẫy (Trap)}
        Nhiều bạn quên đổi đơn vị! Đề bài cho $k$ là \textbf{phút}, nhưng khung giờ là \textbf{tiếng}. Phải quy đổi về cùng một đơn vị (phút) trước khi trừ.
    \end{alertblock}
\end{frame}

\begin{frame}{Thử thách tư duy (Mental Check 1)}
    \textbf{Giả sử:} Đề bài cho $k = 10$ (mất 10 phút đi đường).
    
    \vspace{0.2cm}
    \textbf{Hỏi:} Bạn còn lại tối đa bao nhiêu phút dành \textbf{riêng cho việc giải bài}?
    \begin{itemize}
        \item[A.] 3 giờ 50 phút.
        \item[B.] 230 phút.
        \item[C.] 240 phút.
        \item[D.] 190 phút.
    \end{itemize}

    \pause
    \vspace{0.5cm}
    \textbf{Đáp án: B (230 phút)}
    \begin{itemize}
        \item Tổng thời gian: 4 tiếng $= 4 \times 60 = 240$ phút.
        \item Trừ đi đường: $240 - 10 = 230$ phút.
    \end{itemize}
    \textit{Đây chính là "Vốn" (Budget) của bạn.}
\end{frame}

% ---------------------------------------------------------
% PHẦN 4: CHUNK 2
% ---------------------------------------------------------
\section{Chunk 2: Cơ chế "Cộng dồn"}
\begin{frame}{Chunk 2: Cơ chế "Cộng dồn" (Cumulative Cost)}
    \begin{block}{Logic (Explain)}
        Các bài toán không có giá bằng nhau. Bài sau đắt hơn bài trước.
        \begin{itemize}
            \item Bài 1 giá: $5 \times 1 = 5$ phút.
            \item Bài 2 giá: $5 \times 2 = 10$ phút.
            \item Bài 3 giá: $5 \times 3 = 15$ phút.
        \end{itemize}
    \end{block}

    \begin{exampleblock}{Ẩn dụ: Xếp gạch}
        Để xây xong hàng 2, bạn không chỉ cần 10 viên, mà tổng cộng bạn đã dùng: $5$ (hàng 1) + $10$ (hàng 2) = \textbf{15 viên}.
    \end{exampleblock}

    \begin{alertblock}{Bẫy (Trap)}
        Rất nhiều bạn nhầm lẫn: "Giải bài thứ 3 tốn 15 phút, vậy nếu có 20 phút thì giải được bài 3". \\
        \textbf{Sai!} Vì để giải được bài 3, bạn \textbf{phải giải xong} bài 1 và bài 2 trước đã.
    \end{alertblock}
\end{frame}

\begin{frame}{Thử thách tư duy (Mental Check 2)}
    Vẫn với túi tiền \textbf{230 phút}. Nếu bạn muốn giải \textbf{3 bài đầu tiên}.

    \begin{enumerate}
        \item Bạn tốn tổng cộng bao nhiêu phút?
        \item Sau khi giải xong 3 bài, "Vốn" của bạn còn dư bao nhiêu?
    \end{enumerate}

    \pause
    \vspace{0.5cm}
    \textbf{Đáp án:}
    \begin{itemize}
        \item Tổng chi phí: $5 + 10 + 15 = 30$ phút.
        \item Thời gian còn lại: $230 - 30 = 200$ phút.
    \end{itemize}
    \textit{Chuẩn không cần chỉnh!}
\end{frame}

% ---------------------------------------------------------
% PHẦN 5: CHUNK 3
% ---------------------------------------------------------
\section{Chunk 3: Điểm dừng}
\begin{frame}{Chunk 3: Điểm dừng (The Limit)}
    \begin{block}{Logic (Explain)}
        Máy tính sẽ làm việc như một đứa trẻ đi siêu thị. Nhặt lần lượt từng món (Bài 1, Bài 2...) bỏ vào giỏ cho đến khi xảy ra 1 trong 2 trường hợp:
        \begin{enumerate}
            \item \textbf{Hết tiền:} Tổng thời gian vượt quá Budget.
            \item \textbf{Hết hàng:} Đã giải hết tất cả $n$ bài.
        \end{enumerate}
    \end{block}

    \begin{alertblock}{Bẫy (Trap)}
        Sai lầm "chí mạng": Quên kiểm tra số lượng bài $n$.
        Nếu Budget siêu to nhưng đề chỉ có 3 bài ($n=3$), đáp án tối đa chỉ là 3.
    \end{alertblock}
\end{frame}

\begin{frame}{Thử thách tư duy (Mental Check 3)}
    Chạy thử "bộ não máy tính" với trường hợp:
    \begin{itemize}
        \item Đề thi có: \textbf{$n = 4$ bài}.
        \item Thời gian đi lại: \textbf{$k = 222$ phút}.
    \end{itemize}

    \vspace{0.2cm}
    \textbf{Câu hỏi:}
    \begin{enumerate}
        \item "Túi tiền thời gian" còn lại bao nhiêu phút?
        \item Giải được tối đa bao nhiêu bài? (Chứng minh bằng trừ dần).
    \end{enumerate}

    \pause
    \vspace{0.3cm}
    \textbf{Đáp án:}
    \begin{itemize}
        \item Vốn: $240 - 222 = 18$ phút.
        \item Mua bài 1 (5p) + bài 2 (10p) = 15p.
        \item Dư $18 - 15 = 3$ phút $\rightarrow$ Không đủ mua bài 3 (15p).
        \item \textbf{Kết quả:} 2 bài.
    \end{itemize}
\end{frame}

% ---------------------------------------------------------
% PHẦN 6: TỔNG KẾT
% ---------------------------------------------------------
\section{Tổng kết thuật toán}
\begin{frame}{Tổng kết thuật toán (The Strategy)}
    Viết chương trình C++ thực hiện quy trình:

    \begin{enumerate}
        \item \textbf{Bước 1:} Lấy dữ liệu $n$ (số bài thi) và $k$ (thời gian đi lại).
        \item \textbf{Bước 2:} Tính "Vốn thời gian" còn lại: `time_left = 240 - k`.
        \item \textbf{Bước 3:} Tạo một cái rổ đếm số bài làm được: `count = 0`.
        \item \textbf{Bước 4:} Chạy vòng lặp thử làm từng bài (từ $i = 1$ đến $n$):
            \begin{itemize}
                \item Tính giá bài $i$: `cost = 5 * i`.
                \item \textbf{Kiểm tra:} Nếu `time_left >= cost` (Đủ tiền):
                \begin{itemize}
                    \item Trừ tiền: `time_left = time_left - cost`.
                    \item Bỏ vào rổ: `count++`.
                \end{itemize}
                \item Nếu không đủ tiền: \textbf{DỪNG NGAY} (`break`).
            \end{itemize}
        \item \textbf{Bước 5:} In ra cái rổ `count`.
    \end{enumerate}
\end{frame}

\section{Mã giả cụ thể}
\begin{frame}[fragile]{Mã giả cụ thể (Code Implementation)}
\begin{lstlisting}
#include <iostream>
using namespace std;

int main() {
    int n, k;
    cin >> n >> k; // 1. Nhập dữ liệu

    int time_left = 240 - k; // 2. Tính vốn (Chunk 1)
    int count = 0;

    // 3. Vòng lặp tư duy (Chunk 2 & 3)
    for (int i = 1; i <= n; i++) {
        int cost = 5 * i; // Giá bài i
        if (time_left >= cost) {
            time_left = time_left - cost; // Trừ tiền
            count++; // Tăng đếm
        } else {
            break; // Hết tiền -> Dừng
        }
    }
    cout << count << endl; // 4. In kết quả
    return 0;
}
\end{lstlisting}
\end{frame}

% ---------------------------------------------------------
% PHẦN 7: LỜI KHUYÊN
% ---------------------------------------------------------
\section{Lời khuyên cuối cùng}
\begin{frame}{Lời khuyên cuối cùng của Coach}
    \begin{exampleblock}{Bản chất bài toán}
        Bài toán \textbf{750A} thực chất chỉ là một bài toán \textbf{đi chợ}:
        "Có 240 nghìn, đi xe ôm mất k nghìn, còn lại bao nhiêu tiền để mua táo? Táo càng mua càng đắt, mua được bao nhiêu quả thì hết tiền?"
    \end{exampleblock}

    \vspace{0.5cm}
    \textbf{Bài tập về nhà:}
    \begin{itemize}
        \item Thử sức bài: \textbf{Codeforces 546A - Soldier and Bananas}.
        \item Chủ đề: Tư duy tích lũy (tương tự).
    \end{itemize}
    
    \vspace{0.5cm}
    \centerline{\textbf{Chúc bạn code vui và "Accepted"!}}
\end{frame}

\end{document}