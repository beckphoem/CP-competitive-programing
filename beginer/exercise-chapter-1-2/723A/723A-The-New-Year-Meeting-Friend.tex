\documentclass{beamer}
\usepackage[utf8]{inputenc}
\usepackage[vietnamese]{babel}
\usepackage{amsmath}
\usepackage{xcolor}

% Chọn theme
\usetheme{Madrid}
\usecolortheme{whale}

% Cấu hình thông tin bài giảng
\title[Codeforces 723A]{Tư duy thuật toán: Codeforces 723A}
\subtitle{The New Year: Meeting Friends}
\author{Coach Tư Duy Thuật Toán}
\date{\today}

\begin{document}

% Slide tiêu đề
\begin{frame}
    \titlepage
\end{frame}

% Mục lục
\begin{frame}{Mục lục}
    \tableofcontents
\end{frame}

% ---------------------------------------------------------
\section{Bước 1: Tiếp nhận và Phẫu thuật (Briefing)}
% ---------------------------------------------------------

\begin{frame}{Chào mừng}
    \textbf{Chào mừng bạn đến với phòng tập tư duy thuật toán!}
    
    Hôm nay chúng ta sẽ khởi động với một bài toán thú vị về hình học trên một đường thẳng.
    
    \vspace{1em}
    \textbf{Bước đầu tiên: Tiếp nhận \& Phẫu thuật bài toán.}
\end{frame}

\begin{frame}{1. Dịch đề sang "Ngôn ngữ con người"}
    Bỏ qua cốt truyện rườm rà, đề bài cốt lõi như sau:
    \begin{itemize}
        \item Có \textbf{3 người bạn} đứng ở 3 vị trí khác nhau trên trục số ($x_1, x_2, x_3$).
        \item Họ muốn hẹn gặp nhau tại một điểm $P$ trên con đường này.
        \item \textbf{Mục tiêu:} Tìm $P$ sao cho \textbf{tổng quãng đường} cả 3 người phải đi bộ là \textbf{nhỏ nhất}.
        \item \textbf{Output:} In ra tổng quãng đường nhỏ nhất đó.
    \end{itemize}
\end{frame}

\begin{frame}{2. Ẩn dụ hình ảnh (Metaphor)}
    Hãy tưởng tượng 3 ngôi nhà nằm trên một con phố dài:
    
    \begin{center}
        (A) ....... (B) ............... (C)
    \end{center}
    
    Họ cần chọn một vị trí để tụ tập. Vị trí đó có thể là:
    \begin{itemize}
        \item Nhà của A.
        \item Nhà của B.
        \item Nhà của C.
        \item Hoặc một quán trà đá bất kỳ nằm đâu đó trên phố.
    \end{itemize}
\end{frame}

\begin{frame}{3. Lộ trình tư duy (Roadmap)}
    Để giải bài này, không code vội, cần đi qua các mảnh ghép logic:
    
    \begin{enumerate}
        \item \textbf{Chunk 1:} Sắp xếp trật tự thế giới (Xác định vị trí tương đối).
        \item \textbf{Chunk 2:} Thử nghiệm điểm hẹn (Chiến thuật "Kéo co").
        \item \textbf{Chunk 3:} Chốt công thức tính toán.
    \end{enumerate}
    
    \begin{block}{Trạng thái}
        Bạn đã sẵn sàng "mổ xẻ" vấn đề đầu tiên chưa?
        \textbf{Mục tiêu:} Tìm điểm gặp nhau để tổng đi bộ là ít nhất.
    \end{block}
\end{frame}

% ---------------------------------------------------------
\section{Chunk 1: Sắp xếp trật tự thế giới}
% ---------------------------------------------------------

\begin{frame}{1. Logic (Tư duy)}
    Đề bài cho 3 số ngẫu nhiên (Ví dụ: 7 1 4). Trên trục số, các vị trí luôn nằm theo thứ tự từ bé đến lớn.
    
    Để giải quyết bài toán "khoảng cách", não bộ cần \textbf{"Xếp hàng"} lại chúng:
    \begin{itemize}
        \item Ai đứng ngoài cùng bên trái? $\rightarrow$ Giá trị \textbf{Min}.
        \item Ai đứng ngoài cùng bên phải? $\rightarrow$ Giá trị \textbf{Max}.
        \item Ai đứng kẹp ở giữa? $\rightarrow$ Giá trị \textbf{Mid}.
    \end{itemize}
\end{frame}

\begin{frame}{2. Bẫy tư duy (Trap)}
    Rất nhiều bạn mới học thường vội vàng lấy luôn số đầu tiên làm mốc.
    
    \begin{alertblock}{Sai lầm thường gặp}
        Ví dụ: Input 10 30 20.
        Nếu vẽ sơ đồ $10 \rightarrow 30 \rightarrow 20$ là sai thực tế.
    \end{alertblock}
    
    \begin{block}{Thực tế}
        Nó phải là: $10 \rightarrow 20 \rightarrow 30$.
    \end{block}
\end{frame}

\begin{frame}{Thử thách tư duy (Mental Check)}
    \textbf{Đề bài:} Giả sử input là 6 15 2. Hãy sắp xếp lại trong đầu.
    
    \vspace{1em}
    \textbf{Đáp án của bạn:}
    \begin{enumerate}
        \item Nhà bên trái nhất (Min): 2
        \item Nhà ở giữa (Mid): 6
        \item Nhà bên phải nhất (Max): 15
    \end{enumerate}
    
    \vspace{1em}
    \textbf{Nhận xét:} Chính xác! Hình ảnh rõ ràng: \\
    \textbf{Nhà Ông Bé (2) ———— Nhà Ông Giữa (6) ———— Nhà Ông Lớn (15)}
\end{frame}

% ---------------------------------------------------------
\section{Chunk 2: Chiến thuật Kéo co}
% ---------------------------------------------------------

\begin{frame}{1. Logic (Tư duy)}
    Hãy tưởng tượng con đường nối từ \textbf{Ông Bé (2)} đến \textbf{Ông Lớn (15)} là một sợi dây.
    
    \begin{itemize}
        \item Hẹn ở bất kỳ đâu trong đoạn [2, 15]: Tổng quãng đường của Ông Bé + Ông Lớn luôn bằng độ dài sợi dây ($15 - 2 = 13$).
        \item \textbf{Vấn đề nằm ở Ông Giữa (6)!}
        \begin{itemize}
            \item Nếu điểm hẹn lệch khỏi nhà Ông Giữa $\rightarrow$ Ông ấy phải đi bộ $\rightarrow$ Tổng tăng thêm.
            \item Nếu điểm hẹn \textbf{tại ngay nhà Ông Giữa} $\rightarrow$ Ông ấy không đi bước nào (Distance = 0) $\rightarrow$ Tiết kiệm nhất!
        \end{itemize}
    \end{itemize}
\end{frame}

\begin{frame}{2. Kết luận \& Bẫy tư duy}
    \textbf{Kết luận logic:}
    \begin{enumerate}
        \item Ông Bé đi đến nhà Ông Giữa.
        \item Ông Lớn đi đến nhà Ông Giữa.
        \item Ông Giữa... ngồi yên uống trà.
    \end{enumerate}
    
    \vspace{1em}
    \begin{alertblock}{Bẫy tư duy (Trap)}
        Không dùng "Trung bình cộng" (Average).
        Trong bài toán này, \textbf{"Trung vị" (Median)} mới là vua. Đừng phức tạp hóa bằng phép chia!
    \end{alertblock}
\end{frame}

\begin{frame}{Thử thách tư duy (Mental Check)}
    Bộ số: \textbf{2, 6, 15}. So sánh 2 phương án:
    
    \begin{columns}
        \column{0.5\textwidth}
        \begin{block}{Phương án A (Tại 6)}
            \begin{itemize}
                \item Bé (2): 4 bước
                \item Lớn (15): 9 bước
                \item Giữa (6): 0 bước
                \item \textbf{Tổng = 13}
            \end{itemize}
        \end{block}
        
        \column{0.5\textwidth}
        \begin{block}{Phương án B (Tại 7)}
            \begin{itemize}
                \item Bé (2): 5 bước
                \item Lớn (15): 8 bước
                \item Giữa (6): 1 bước
                \item \textbf{Tổng = 14}
            \end{itemize}
        \end{block}
    \end{columns}
    
    \vspace{1em}
    \textbf{Kết luận:} $13 < 14$. Phương án gặp tại nhà ông Giữa (Trung vị) là tối ưu nhất.
\end{frame}

% ---------------------------------------------------------
\section{Chunk 3: Chốt công thức Ma thuật}
% ---------------------------------------------------------

\begin{frame}{1. Logic (Tư duy)}
    Chúng ta đã đồng ý:
    \begin{itemize}
        \item Điểm hẹn = Giá trị ở giữa ($Mid$).
        \item Tổng quãng đường = (Khoảng cách Min $\to$ Mid) + (Khoảng cách Max $\to$ Mid).
    \end{itemize}
    
    Công thức:
    $$Distance = (Mid - Min) + (Max - Mid)$$
    $$Distance = Mid - Min + Max - Mid$$
    \begin{alertblock}{Công thức Ma Thuật}
        $$Distance = Max - Min$$
    \end{alertblock}
    \textbf{Điều kỳ diệu:} Vị trí $Mid$ bị triệt tiêu! Kết quả chỉ là khoảng cách giữa ông lớn nhất và bé nhất.
\end{frame}

\begin{frame}{2. Thuật toán cuối cùng \& Trường hợp biên}
    \textbf{Quy trình 3 bước:}
    \begin{enumerate}
        \item Nhập 3 số $a, b, c$.
        \item Tìm $Max$ và $Min$.
        \item Kết quả = $Max - Min$.
    \end{enumerate}
    
    \vspace{1em}
    \textbf{Trường hợp biên (Edge Case):} Input 10 10 10.
    \begin{itemize}
        \item Thực tế: Ở chung chỗ, không cần đi đâu.
        \item Công thức: $10 - 10 = 0$.
    \end{itemize}
    $\rightarrow$ Công thức đúng cho cả trường hợp trùng nhau!
\end{frame}

% ---------------------------------------------------------
\section{Tổng kết chiến thuật (Wrap Up)}
% ---------------------------------------------------------

\begin{frame}[fragile]{Tổng kết \& Mã giả}
    \begin{block}{Chiến thuật}
        \begin{enumerate}
            \item \textbf{Sắp xếp:} Nhìn 3 số dạng `Min -> Mid -> Max`.
            \item \textbf{Chọn điểm:} Điểm hẹn tối ưu là `Mid`.
            \item \textbf{Công thức:} Tổng = `Max - Min`.
        \end{enumerate}
    \end{block}
    
    \begin{exampleblock}{Mã giả (Pseudocode)}
\begin{verbatim}
BƯỚC 1: Đọc 3 số nguyên a, b, c.
BƯỚC 2: Tìm Max trong 3 số.
        Tìm Min trong 3 số.
BƯỚC 3: Kết quả = Max - Min.
BƯỚC 4: In Kết quả.
\end{verbatim}
    \end{exampleblock}
\end{frame}

\begin{frame}{Nhiệm vụ cuối cùng: Implementation}
    Bạn định dùng ngôn ngữ nào (C++, Python, Java...)?
    
    \begin{itemize}
        \item \textbf{Cách 1 (Dùng hàm có sẵn):}
        Sử dụng `max(a,b,c)` và `min(a,b,c)`. Đây là cách nhanh nhất.
        
        \item \textbf{Cách 2 (Sắp xếp):}
        Cho 3 số vào mảng, dùng `sort()`, sau đó lấy `số_cuối - số_đầu`.
    \end{itemize}
    
    \vspace{1em}
    \textbf{Action:} Hãy viết code theo ngôn ngữ sở trường của bạn!
\end{frame}

\end{document}