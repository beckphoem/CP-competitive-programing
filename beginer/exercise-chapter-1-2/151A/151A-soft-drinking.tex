\documentclass[10pt, aspectratio=169]{beamer}

% Cấu hình ngôn ngữ và font chữ
\usepackage[utf8]{inputenc}
\usepackage[T5]{fontenc}
\usepackage[vietnamese]{babel}

% Giao diện (Theme)
\usetheme{Madrid}
\usecolortheme{whale}

% Cấu hình màu sắc và font
\setbeamerfont{title}{size=\huge, series=\bfseries}
\setbeamerfont{frametitle}{size=\Large, series=\bfseries}

% Thông tin bài giảng
\title[Codeforces 151A]{Giải Mã Tư Duy Thuật Toán}
\subtitle{Bài toán: Codeforces 151A - Soft Drinking}
\author[Coach Tư Duy]{Coach Tư Duy Thuật Toán (LHTL Edition)}
\date{\today}

\begin{document}

% --- Slide Tiêu đề ---
\begin{frame}
    \titlepage
\end{frame}

% --- Slide Mục lục ---
\begin{frame}{Nội dung bài học}
    \tableofcontents
\end{frame}

% =============================================================================
% MỞ ĐẦU
% =============================================================================
\section{Giới thiệu bài toán}
\begin{frame}{Codeforces 151A - Soft Drinking}
    \begin{block}{Chào mừng bạn!}
        Chúng ta sẽ cùng nhau "mổ xẻ" bài \textbf{Codeforces 151A - Soft Drinking}.
    \end{block}
    
    \vspace{0.5cm}
    Đây là một bài toán kinh điển về:
    \begin{itemize}
        \item \textbf{Quản lý tài nguyên}.
        \item \textbf{Tìm điểm thắt nút (Bottleneck)}.
    \end{itemize}
    
    \vspace{0.5cm}
    \textit{Đừng để đống biến số $n, k, l, c, d, p, nl, np$ làm bạn hoa mắt. Chúng ta sẽ dọn dẹp nó ngay bây giờ.}
\end{frame}

% =============================================================================
% PHẦN 1: PHẪU THUẬT
% =============================================================================
\section{Phẫu thuật đề bài (Deconstruct)}
\begin{frame}{1. 🔪 Phẫu thuật đề bài (Deconstruct)}
    Hãy quên chuyện "uống nước ngọt" đi. Hãy tưởng tượng bạn là một \textbf{Bar trưởng} đang pha chế các \textbf{Combo đồ uống}.
    
    \begin{alertblock}{Quy tắc pha 1 Combo}
        Bạn \textbf{BẮT BUỘC} phải có đủ 3 thành phần cùng lúc:
        \begin{enumerate}
            \item \textbf{Nước:} Một lượng ml nhất định.
            \item \textbf{Chanh:} Một lát chanh.
            \item \textbf{Muối:} Một lượng gam muối.
        \end{enumerate}
        \textit{Nếu thiếu bất kỳ cái nào $\rightarrow$ Không thể tạo ra Combo.}
    \end{alertblock}
\end{frame}

\begin{frame}{Dữ liệu đầu vào (Kho hàng)}
    Dữ liệu đầu vào (Input) thực chất là kho hàng của bạn:
    
    \begin{itemize}
        \item \textbf{Nhân lực:} $n$ (số người bạn).
        \item \textbf{Kho Nước:} $k$ chai, mỗi chai $l$ ml.
        \item \textbf{Kho Chanh:} $c$ quả, mỗi quả cắt được $d$ lát.
        \item \textbf{Kho Muối:} $p$ gam.
        \item \textbf{Công thức pha 1 Combo:} Cần $nl$ ml nước và $np$ gam muối (và luôn luôn là 1 lát chanh).
    \end{itemize}
\end{frame}

% =============================================================================
% PHẦN 2: LỘ TRÌNH
% =============================================================================
\section{Lộ trình tư duy}
\begin{frame}{2. 🗺️ Lộ trình tư duy}
    Chúng ta sẽ đi qua 3 mảnh ghép (Chunks):
    
    \vspace{0.5cm}
    \begin{enumerate}
        \item \textbf{Chunk 1: Tổng kiểm kê kho hàng} \\
        (Quy đổi mọi thứ về đơn vị nhỏ nhất).
        
        \item \textbf{Chunk 2: Quy tắc "Chiếc thùng gỗ" (Bottleneck)} \\
        (Tìm xem nguyên liệu nào sẽ hết trước).
        
        \item \textbf{Chunk 3: Chia phần công bằng} \\
        (Tính ra kết quả cuối cùng cho mỗi người).
    \end{enumerate}
\end{frame}

% =============================================================================
% PHẦN 3: CHUNK 1
% =============================================================================
\section{Chunk 1: Tổng kiểm kê}
\begin{frame}{🚀 Chunk 1: Tổng kiểm kê kho hàng}
    Trước khi pha chế, bạn phải biết trong kho mình có \textbf{tổng cộng} bao nhiêu nguyên liệu rời.
    
    \begin{exampleblock}{Quy đổi đơn vị}
        \begin{itemize}
            \item \textbf{Nước:} Bạn có $k$ chai, mỗi chai $l$ ml.
            \item \textbf{Chanh:} Bạn có $c$ quả, mỗi quả cắt được $d$ lát.
            \item \textbf{Muối:} Đã có sẵn $p$ gam (không cần tính).
        \end{itemize}
    \end{exampleblock}
    
    \textbf{Nhiệm vụ:} Tính tổng số lượng thực tế có thể dùng được.
\end{frame}

\begin{frame}{❓ Thử thách tư duy (Mental Check)}
    \textbf{Giả sử kho hàng có số liệu sau:}
    \begin{itemize}
        \item $k = 3$ (chai), $l = 100$ (ml/chai).
        \item $c = 5$ (quả chanh), $d = 2$ (lát/quả).
        \item $p = 50$ (gam muối).
    \end{itemize}
    
    \vspace{0.3cm}
    \begin{columns}
        \column{0.5\textwidth}
        \textbf{Kết quả kiểm kê:}
        \begin{enumerate}
            \item Nước: $3 \times 100 = 300$ ml.
            \item Chanh: $5 \times 2 = 10$ lát.
            \item Muối: $50$ gam.
        \end{enumerate}
        
        \column{0.5\textwidth}
        \begin{alertblock}{Lưu ý quan trọng}
            Ở bước này, muối là $p$, không phải $p/np$. \\
            Phép chia thuộc về bước "Tính khả năng phục vụ" sau này.
        \end{alertblock}
    \end{columns}
\end{frame}

% =============================================================================
% PHẦN 4: CHUNK 2
% =============================================================================
\section{Chunk 2: Quy tắc Chiếc thùng gỗ}
\begin{frame}{🚀 Chunk 2: Quy tắc "Chiếc thùng gỗ" (Bottleneck)}
    Bây giờ bạn đã có tổng nguyên liệu. Chúng ta cần xem từng loại nguyên liệu \textbf{độc lập} có thể tạo ra tối đa bao nhiêu Combo.
    
    \vspace{0.3cm}
    \textbf{Công thức pha 1 Combo cần:} $nl$ ml nước, $1$ lát chanh, $np$ gam muối.
    
    \begin{block}{Nguyên lý Chiếc thùng gỗ}
        Bạn chỉ có thể tạo ra số Combo bằng với nguyên liệu \textbf{ít nhất/hạn hẹp nhất}.
        \\ \textit{(Giống như thùng gỗ chỉ đựng được nước đến chiều cao của thanh gỗ thấp nhất).}
    \end{block}
\end{frame}

\begin{frame}{❓ Thử thách tư duy (Quan trọng)}
    \textbf{Dữ liệu:}
    \begin{itemize}
        \item Kho: Nước \textbf{300}, Chanh \textbf{10}, Muối \textbf{50}.
        \item Công thức cần: \textbf{5} ml nước, \textbf{1} lát chanh, \textbf{2} gam muối.
    \end{itemize}
    
    \vspace{0.3cm}
    \textbf{Tính toán:}
    \begin{enumerate}
        \item Nước: $300 / 5 = \mathbf{60}$ ly.
        \item Chanh: $10 / 1 = \mathbf{10}$ ly.
        \item Muối: $50 / 2 = \mathbf{25}$ ly.
    \end{enumerate}
    
    \vspace{0.3cm}
    \textbf{Quy luật cốt lõi:} Dù nước đủ 60 ly, muối đủ 25 ly, nhưng làm đến ly thứ 10 thì chanh hết sạch!
    
    $\rightarrow$ Số Combo tối đa = \textbf{min(60, 10, 25) = 10 Combo}.
\end{frame}

% =============================================================================
% PHẦN 5: CHUNK 3
% =============================================================================
\section{Chunk 3: Chia phần công bằng}
\begin{frame}{🚀 Chunk 3: Chia phần công bằng (Final Step)}
    Chúng ta đã có \textbf{10 Combo} trên bàn.
    \\ Đề bài hỏi: \textbf{"Mỗi người bạn nhận được bao nhiêu ly?"}
    
    \vspace{0.5cm}
    \textbf{Dữ kiện cuối cùng:} Số người bạn $n$.
    
    \begin{block}{Quy tắc chia}
        Phải chia đều. Nếu chia không hết thì bỏ phần dư (chia lấy phần nguyên), vì không thể uống "nửa ly" trong bài toán này.
    \end{block}
\end{frame}

\begin{frame}{❓ Thử thách về đích}
    Giả sử có tổng \textbf{10 Combo}.
    
    \vspace{0.5cm}
    \begin{columns}
        \column{0.33\textwidth}
        \textbf{TH1: $n=3$ người}
        $$10 / 3 = 3$$
        (Dư 1, bỏ qua)
        
        \column{0.33\textwidth}
        \textbf{TH2: $n=4$ người}
        $$10 / 4 = 2$$
        (Dư 2, bỏ qua)
        
        \column{0.33\textwidth}
        \textbf{TH3: $n=11$ người}
        $$10 / 11 = 0$$
        (Không đủ chia)
    \end{columns}
    
    \vspace{0.5cm}
    \centering
    \textit{Số nguyên chia nhau sẽ mất phần thập phân (trong C++/Python/Java).}
\end{frame}

% =============================================================================
% TỔNG KẾT
% =============================================================================
\section{Tổng kết chiến thuật}
\begin{frame}{📝 Tổng kết chiến thuật (Wrap-up)}
    Đây là \textbf{Bản thiết kế thuật toán} cho bài 151A:
    
    \begin{enumerate}
        \item \textbf{Bước 1: Quy đổi tổng lực}
        \begin{itemize}
            \item $Total\_Water = k \times l$
            \item $Total\_Lime = c \times d$
            \item $Total\_Salt = p$
        \end{itemize}
        
        \item \textbf{Bước 2: Tìm giới hạn (Bottleneck)}
        \begin{itemize}
            \item $Toast\_Water = Total\_Water / nl$
            \item $Toast\_Lime = Total\_Lime / 1$
            \item $Toast\_Salt = Total\_Salt / np$
            \item $\Rightarrow Max\_Toast = \min(Toast\_Water, Toast\_Lime, Toast\_Salt)$
        \end{itemize}
        
        \item \textbf{Bước 3: Chia phần}
        \begin{itemize}
            \item $Result = Max\_Toast / n$
        \end{itemize}
    \end{enumerate}
\end{frame}

% =============================================================================
% NHIỆM VỤ
% =============================================================================
\section{Giao nhiệm vụ}
\begin{frame}[fragile]{💻 Giao nhiệm vụ cuối}
    Bây giờ là lúc bạn chuyển tư duy này thành code.
    
    \begin{exampleblock}{Gợi ý khi code}
        Hàm tìm giá trị nhỏ nhất:
        \begin{itemize}
            \item \textbf{C++:}
            \begin{verbatim}
min({a, b, c}) 
// Hoặc
min(a, min(b, c))
            \end{verbatim}
            \item \textbf{Python:}
            \begin{verbatim}
min(a, b, c)
            \end{verbatim}
        \end{itemize}
    \end{exampleblock}
    
    \vspace{0.3cm}
    Hãy thử viết code và nộp bài (submit). Nếu gặp lỗi, hãy quay lại đây để "debug" tư duy!
\end{frame}

\begin{frame}
    \centering
    \Huge \textbf{Chúc bạn thành công!} \\
    \large Hẹn gặp lại ở bài toán tiếp theo.
\end{frame}

\end{document}