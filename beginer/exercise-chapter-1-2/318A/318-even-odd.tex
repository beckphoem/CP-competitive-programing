\documentclass{beamer}

% --- CẤU HÌNH TIẾNG VIỆT & FONT ---
\usepackage[utf8]{inputenc}
\usepackage[T5]{fontenc}      % Bắt buộc cho tiếng Việt đúng chuẩn
\usepackage[vietnamese]{babel}
\usepackage{lmodern}          % Bộ font hỗ trợ tiếng Việt tốt hơn font mặc định

% --- CẤU HÌNH TOÁN HỌC ---
\usepackage{amsmath}
\usepackage{amssymb}
\usepackage{xcolor}

% --- CẤU HÌNH THEME ---
\usetheme{Madrid}
\usecolortheme{whale}

% Sửa lỗi hiển thị bookmark PDF tiếng Việt
\hypersetup{unicode=true} 

% Thông tin bài thuyết trình
\title[Codeforces 318A - Even Odds]{Codeforces 318A - Even Odds}
\subtitle{Rèn luyện tư duy toán học (Math vs Brute Force)}
\author{Coach Tư Duy Thuật Toán}
\date{\today}

% Cấu hình hiển thị code
\usepackage{listings}
\lstset{basicstyle=\ttfamily, breaklines=true}

\begin{document}

% --- Slide Tiêu đề ---
\begin{frame}
    \titlepage
\end{frame}

% --- Slide Giới thiệu ---
\begin{frame}{Lời chào}
    \begin{block}{Coach Tư Duy Thuật Toán}
        Chào mừng bạn. Rất tốt, bạn đã chọn bài \textbf{Codeforces 318A - Even Odds}. 
        
        Đây là một bài toán kinh điển để rèn luyện tư duy toán học (Math) thay vì dùng vòng lặp (Brute Force).
        
        Chúng ta sẽ không vội vàng viết code. Hãy cùng nhau ``mổ xẻ'' nó ngay bây giờ.
    \end{block}
\end{frame}

% --- MỤC LỤC ---
\begin{frame}{Nội dung chính}
    \tableofcontents
\end{frame}

% =========================================================================
% SECTION 1: BRIEFING
% =========================================================================
\section{Bước 1: Tiếp nhận \& Phẫu thuật (Briefing)}

\begin{frame}{Bước 1: Tiếp nhận \& Phẫu thuật (Briefing)}
    \textbf{Tên bài toán:} Even Odds (Chẵn Lẻ).

    \begin{block}{Đề bài (Ngôn ngữ con người)}
        Người ta viết các số nguyên từ $1$ đến $n$ lên bảng. Nhưng họ không viết theo thứ tự bình thường ($1, 2, 3 \dots$). Họ viết theo quy tắc \textbf{``Lẻ trước, Chẵn sau''}:
        \begin{enumerate}
            \item Đầu tiên là tất cả các số lẻ (tăng dần).
            \item Sau đó mới đến tất cả các số chẵn (tăng dần).
        \end{enumerate}
        \textbf{Nhiệm vụ:} Tìm xem con số nằm ở vị trí thứ $k$ trong dãy số mới này là số mấy?
    \end{block}
\end{frame}

\begin{frame}{Lộ trình tư duy}
    Để giải quyết bài này, chúng ta cần đi qua 2 mảnh ghép logic (Micro-Chunks):

    \begin{itemize}
        \item \textbf{Chunk 1: Tìm ``Biên giới''} \\
        (Có bao nhiêu số lẻ? Khi nào thì bước sang vùng số chẵn?).
        \vspace{0.5cm}
        \item \textbf{Chunk 2: Công thức truy xuất} \\
        (Nếu ở vùng Lẻ thì tính sao? Nếu ở vùng Chẵn thì tính sao?).
    \end{itemize}
    
    \vspace{0.5cm}
    \textit{Bạn đã sẵn sàng bước vào mảnh ghép đầu tiên chưa?}
\end{frame}

% =========================================================================
% SECTION 2: CHUNK 1 - BIÊN GIỚI
% =========================================================================
\section{Chunk 1: Xác định ``Biên giới''}

\begin{frame}{1. Logic (Explain)}
    Hãy tưởng tượng dãy số từ $1$ đến $n$ giống như một lớp học có $n$ học sinh.
    Thầy giáo chia lớp thành 2 nhóm:
    
    \begin{itemize}
        \item \textbf{Nhóm 1 (Nhóm Lẻ):} Gồm các bạn mang số báo danh 1, 3, 5... xếp hàng trước.
        \item \textbf{Nhóm 2 (Nhóm Chẵn):} Gồm các bạn mang số báo danh 2, 4, 6... xếp hàng nối đuôi phía sau.
    \end{itemize}
    
    \vspace{0.5cm}
    \textbf{Vấn đề cốt lõi đầu tiên:} Để biết vị trí thứ $k$ nằm ở Nhóm 1 hay Nhóm 2, ta phải biết \textbf{Nhóm 1 có tất cả bao nhiêu thành viên}.
    
    Ta gọi số lượng thành viên nhóm lẻ là $P$ (Partition point).
\end{frame}

\begin{frame}{2. Bẫy (Trap) \alert{\textbf{WARN}}}
    \begin{alertblock}{Sai lầm thường gặp}
        Rất nhiều bạn mặc định chia đôi: $P = n / 2$.
    \end{alertblock}

    \begin{itemize}
        \item Với $n = 10$, số lẻ là $\{1, 3, 5, 7, 9\} \rightarrow$ có 5 số. ($10/2 = 5 \rightarrow$ Đúng).
        \item Nhưng với $n = 7$, số lẻ là $\{1, 3, 5, 7\} \rightarrow$ có 4 số.
        \item Nếu dùng phép chia số nguyên trong máy tính: $7 / 2 = 3 \rightarrow$ \textbf{SAI} (Thiếu mất 1 người).
    \end{itemize}
\end{frame}

\begin{frame}{3. Thử thách (Challenge) - Phần 1}
    \begin{exampleblock}{Câu hỏi}
        Giúp tôi xác định số lượng phần tử lẻ ($P$) trong các trường hợp sau:
        \begin{enumerate}
            \item Nếu $n = 4$ (Dãy: 1, 3, 2, 4).
            \item Nếu $n = 5$ (Dãy: 1, 3, 5, 2, 4).
        \end{enumerate}
    \end{exampleblock}
    
    \textbf{Câu hỏi chốt:} Công thức toán học nào đúng để tính số lượng số lẻ $P$ cho mọi $n$?
    \begin{itemize}
        \item A. $n / 2$
        \item B. $(n - 1) / 2$
        \item C. $(n + 1) / 2$ (Lấy phần nguyên)
    \end{itemize}
\end{frame}

\begin{frame}{3. Thử thách (Challenge) - Đáp án}
    \textbf{Đáp án:}
    
    \pause % Hiệu ứng: Bấm chuột mới hiện đáp án
    
    Chính xác! 🎯 \textbf{Đáp án C} là chuẩn nhất.

    Trong lập trình C++ (và nhiều ngôn ngữ khác), phép chia số nguyên \texttt{(int / int)} sẽ tự động làm tròn xuống (ví dụ $3.5 \rightarrow 3$).
    
    Vì vậy, công thức:
    \[ P = (n + 1) / 2 \]
    chính là mẹo toán học để ``làm tròn lên'', đảm bảo tính đúng số lượng cho cả trường hợp $n$ lẻ và $n$ chẵn.

    \vspace{0.5cm}
    \textit{Tuyệt vời. Bạn đã nắm được biến quan trọng nhất: \textbf{Biên giới $P$}.}
\end{frame}

% =========================================================================
% SECTION 3: CHUNK 2 - THE MAGIC FORMULA
% =========================================================================
\section{Chunk 2: Định vị \& Truy xuất}

\begin{frame}{1. Logic (Explain)}
    Bây giờ chúng ta cầm trong tay tấm vé số thứ tự $k$. Chúng ta cần biết mình phải đi vào \textbf{Cửa số 1 (Nhóm Lẻ)} hay \textbf{Cửa số 2 (Nhóm Chẵn)}.
    
    \textbf{Quy trình:} So sánh vị trí $k$ với biên giới $P = (n+1)/2$.
\end{frame}

\begin{frame}{Trường hợp 1: Nhóm Lẻ ($k \le P$)}
    Nếu $k \le P$ (Vị trí nằm trong nửa đầu):
    
    \begin{itemize}
        \item Dãy số là: 1, 3, 5, 7...
        \item Vị trí thứ 1 $\rightarrow$ giá trị 1 ($2 \times 1 - 1$)
        \item Vị trí thứ 2 $\rightarrow$ giá trị 3 ($2 \times 2 - 1$)
    \end{itemize}
    
    \begin{block}{Công thức tổng quát}
        \[ \text{Value} = 2 \times k - 1 \]
    \end{block}
\end{frame}

\begin{frame}{Trường hợp 2: Nhóm Chẵn ($k > P$)}
    Nếu $k > P$ (Vị trí nằm ở nửa sau):
    
    \begin{itemize}
        \item Lúc này, ta đã bước qua hết $P$ số lẻ.
        \item Vị trí \textbf{thực sự} của ta trong nhóm chẵn là: $new\_k = k - P$.
        \item Dãy số chẵn: 2, 4, 6, 8...
        \item Vị trí thứ 1 trong nhóm chẵn $\rightarrow$ giá trị 2 ($2 \times 1$)
    \end{itemize}
    
    \begin{block}{Công thức tổng quát}
        \[ \text{Value} = 2 \times (k - P) \]
    \end{block}
\end{frame}

\begin{frame}{2. Bẫy (Trap) \alert{\textbf{TLE}}}
    \begin{alertblock}{Sai lầm chết người}
        Ở bài này, giới hạn của $n$ và $k$ lên tới $10^{12}$ (1000 tỷ).
        \begin{itemize}
            \item Dùng vòng lặp \texttt{for (int i = 1...)} để đếm.
            \item \textbf{Hậu quả:} Máy tính chạy quá 1 giây $\rightarrow$ \textbf{Time Limit Exceeded (TLE)}.
        \end{itemize}
    \end{alertblock}

    \textbf{Giải pháp:} Bắt buộc dùng công thức $O(1)$ như trên, không được dùng vòng lặp. Và kiểu dữ liệu phải là \texttt{long long}.
\end{frame}

\begin{frame}{3. Thử thách (Challenge) - Áp dụng}
    Hãy áp dụng 2 công thức trên để giải quyết tình huống sau.
    
    \textbf{Dữ liệu:} Cho $n = 10, k = ?$. (Ta đã biết $P = 5$).

    \begin{exampleblock}{Câu hỏi}
        \begin{enumerate}
            \item Nếu $k = 3$: Kết quả là bao nhiêu?
            \item Nếu $k = 7$: Kết quả là bao nhiêu? (Gợi ý: $7 > 5$, hãy tìm vị trí tương đối trước).
        \end{enumerate}
    \end{exampleblock}
    
    \pause % Hiệu ứng ẩn đáp án
    
    \textbf{Kết quả:}
    \begin{itemize}
        \item \textbf{Logic 1:} $k=3 \le 5$ (Vùng Lẻ) $\rightarrow 2 \times 3 - 1 = \mathbf{5}$.
        \item \textbf{Logic 2:} $k=7 > 5$ (Vùng Chẵn) $\rightarrow$ Bỏ qua 5 số lẻ, tìm số chẵn thứ $(7-5)=2$. $\rightarrow 2 \times 2 = \mathbf{4}$.
    \end{itemize}
    \textit{Hoàn hảo!}
\end{frame}

% =========================================================================
% SECTION 4: CHUNK 3 - DATA TYPE TRAP
% =========================================================================
\section{Chunk 3: Bẫy dữ liệu (Data Type Trap)}

\begin{frame}{Chunk 3: Bẫy dữ liệu (Data Type Trap)}
    Đây là nơi 50\% người mới (Newbies) bị ``ăn hành'' (Wrong Answer hoặc Runtime Error) dù thuật toán đúng.

    \textbf{Dữ kiện:}
    \begin{itemize}
        \item Đề bài cho $n$ và $k$ có thể lên tới $10^{12}$ (1000 tỷ).
        \item Kiểu \texttt{int} trong C++ chỉ chứa được tối đa khoảng $2 \times 10^9$ (2 tỷ).
    \end{itemize}

    \begin{alertblock}{Kết luận}
        Nếu bạn khai báo \texttt{int n, k;}, biến của bạn sẽ bị tràn số (Overflow) giống như đổ 10 lít nước vào cái chai 2 lít vậy. Kết quả sẽ ra sai lệch hoàn toàn.
    \end{alertblock}

    \textbf{Giải pháp:} Bắt buộc dùng \texttt{long long} (chứa được tới $9 \times 10^{18}$).
\end{frame}

% =========================================================================
% SECTION 5: TỔNG KẾT & NEXT STEP
% =========================================================================
\section{Tổng kết thuật toán (The Blueprint)}

\begin{frame}[fragile]{Tổng kết thuật toán (The Blueprint)}
    Đây là bản thiết kế để bạn chuyển ngữ sang C++:

    \begin{enumerate}
        \item \textbf{Input:} Nhập $n, k$ (Nhớ dùng \texttt{long long}).
        \item \textbf{Bước 1 (Tính P):} 
        \begin{verbatim}
long long part = (n + 1) / 2;
        \end{verbatim}
        \item \textbf{Bước 2 (Kiểm tra vị trí):}
        \begin{itemize}
            \item \textbf{Nếu} $k \le part$: In ra $2 \times k - 1$.
            \item \textbf{Ngược lại (Else)}: In ra $2 \times (k - part)$.
        \end{itemize}
        \item \textbf{Output:} Kết quả.
    \end{enumerate}
\end{frame}

\begin{frame}{Next Step}
    \huge \centering \textbf{Bạn đã có đủ mọi ``nguyên liệu''!}
    
    \normalsize
    \vspace{1cm}
    \begin{enumerate}
        \item Công thức tính biên giới.
        \item Công thức tính giá trị từng vùng.
        \item Loại dữ liệu cần dùng.
    \end{enumerate}
    
    \vspace{1cm}
    \textit{Bạn có muốn tự tay viết đoạn code này để tôi kiểm tra (Review), hay bạn muốn tôi cung cấp Code mẫu chuẩn (Best Practice) ngay bây giờ?}
\end{frame}

\begin{frame}
    \centering
    \Huge \textbf{Cảm ơn bạn đã theo dõi!}
\end{frame}

\end{document}