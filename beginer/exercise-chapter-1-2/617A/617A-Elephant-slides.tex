\documentclass{beamer}
\usepackage[utf8]{inputenc}
\usepackage[vietnamese]{babel}
\usepackage{tcolorbox}
\usepackage{listings}
\usepackage{xcolor}
\usepackage{graphicx}

\usetheme{Madrid}
\usecolortheme{default}

% Colors and Style for code
\definecolor{codegreen}{rgb}{0,0.6,0}
\definecolor{codegray}{rgb}{0.5,0.5,0.5}
\definecolor{codepurple}{rgb}{0.58,0,0.82}
\definecolor{backcolour}{rgb}{0.95,0.95,0.92}

\lstdefinestyle{mystyle}{
    backgroundcolor=\color{backcolour},   
    commentstyle=\color{codegreen},
    keywordstyle=\color{magenta},
    numberstyle=\tiny\color{codegray},
    stringstyle=\color{codepurple},
    basicstyle=\ttfamily\footnotesize,
    breakatwhitespace=false,         
    breaklines=true,                 
    captionpos=b,                    
    keepspaces=true,                 
    numbers=left,                    
    numbersep=5pt,                  
    showspaces=false,                
    showstringspaces=false,
    showtabs=false,                  
    tabsize=2,
    escapechar=@
}

\lstset{style=mystyle}

\title[Elephant]{Bài Tập Codeforces 617A: Elephant \\ "Chiến Thuật Tham Lam"}
\subtitle{C++ Competitive Programming Series - Elo 800}
\author{Học Cùng C++}
\date{}

\begin{document}

\begin{frame}
  \titlepage
\end{frame}

% 1. Hình Dung
\begin{frame}{1. Hình Dung Bài Toán}
  \begin{block}{Đề bài}
      Chú Voi con đang ở vị trí 0. Nhà ở vị trí $x$.
      Voi có thể bước các bước dài: 1, 2, 3, 4, hoặc 5 mét.
  \end{block}

  \begin{alertblock}{Mục tiêu}
      Về nhà với số bước đi \textbf{ÍT NHẤT}.
  \end{alertblock}
\end{frame}

% 2. Chiến Thuật
\begin{frame}{2. Chiến Thuật: Tham Lam (Greedy)}
  Để về nhanh nhất, ta phải đi những bước \textbf{dài nhất có thể}.
  
  \begin{itemize}
      \item Luôn ưu tiên đi bước dài nhất: \textbf{5m}.
  \end{itemize}

  \begin{exampleblock}{Ví dụ: Nhà ở vị trí 12}
      \begin{enumerate}
          \item Bước 1: Đi 5m (Đến vị trí 5). Còn xa 7m.
          \item Bước 2: Đi 5m (Đến vị trí 10). Còn xa 2m.
          \item Bước 3: Đi nốt 2m (Dùng bước 2).
          \item $\rightarrow$ Tổng: 3 bước.
      \end{enumerate}
  \end{exampleblock}
\end{frame}

% 3. Phân Tích Logic
\begin{frame}{3. Phân Tích Logic Chia Dư}
  Bài toán quy về việc: Số $x$ chứa được bao nhiêu số 5?
  
  \begin{columns}
      \column{0.5\textwidth}
      \begin{block}{Trường hợp 1: Chia hết}
          $x = 10$.
          $10 = 5 + 5$.
          $\rightarrow$ Số bước = $10 / 5 = 2$.
      \end{block}

      \column{0.5\textwidth}
      \begin{block}{Trường hợp 2: Có dư}
          $x = 12$.
          $12 = 5 + 5 + \textbf{2}$.
          Phần dư (dù là 1, 2, 3, hay 4) cũng chỉ cần thêm \textbf{1 bước}.
          $\rightarrow$ Số bước = $2 + 1 = 3$.
      \end{block}
  \end{columns}
\end{frame}

% 4. Code Mẫu
\begin{frame}[fragile]{4. Lời Giải Tham Khảo}
  Dùng cấu trúc \texttt{if-else} để xử lý phần dư.

\begin{lstlisting}[language=C++]
int x;
cin >> x;

if (x % 5 == 0) {
    // @Trường hợp chia hết: Vừa đủ các bước 5m@
    cout << x / 5;
} else {
    // @Trường hợp có dư: Cần thêm 1 bước cuối@
    cout << (x / 5) + 1;
}
\end{lstlisting}
  
  \textit{Mẹo nâng cao: Có thể dùng công thức toán học làm tròn trần \texttt{(x + 4) / 5} để viết ngắn hơn (như bài 1A Theatre Square).}
\end{frame}

\end{document}
