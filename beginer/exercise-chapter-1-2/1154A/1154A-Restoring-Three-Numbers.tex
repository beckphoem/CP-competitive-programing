\documentclass[10pt]{beamer}

% --- Cấu hình Tiếng Việt và Giao diện ---
\usepackage[utf8]{inputenc}
\usepackage[vietnamese]{babel}
\usepackage{amsmath}
\usepackage{xcolor}

% Chọn Theme (Giao diện)
\usetheme{Madrid} % Theme cơ bản, trang trọng
\usecolortheme{dolphin} % Tông màu xanh dương dễ chịu

% Tùy chỉnh màu sắc cho giống "Coach"
\definecolor{CoachGreen}{RGB}{0, 100, 0}
\definecolor{TrapRed}{RGB}{180, 0, 0}

% Thông tin trang bìa
\title[Codeforces 1154A]{Giải Mã Tư Duy Thuật Toán}
\subtitle{Bài toán: Restoring Three Numbers (1154A)}
\author{Coach Tư Duy (LHTL Edition)}
\date{\today}

\begin{document}

% --- Slide 1: Trang bìa ---
\begin{frame}
    \titlepage
\end{frame}

% --- Slide 2: Mục lục lộ trình ---
\begin{frame}{Lộ trình Tư duy}
    \tableofcontents
\end{frame}

% =================================================================
\section{Bước 1: Tiếp nhận \& Phẫu thuật}
% =================================================================

\begin{frame}{Bước 1: Tiếp nhận \& Phẫu thuật (Briefing)}
    \begin{block}{1. Bản chất đề bài (Ngôn ngữ con người)}
        \begin{itemize}
            \item Có 3 số bí mật $a, b, c$ (đều $> 0$).
            \item Input cho 4 con số đã bị xáo trộn:
            \begin{itemize}
                \item $a + b$
                \item $a + c$
                \item $b + c$
                \item $a + b + c$
            \end{itemize}
            \item \textbf{Nhiệm vụ:} Từ 4 số lộn xộn này, tìm lại $a, b, c$.
        \end{itemize}
    \end{block}

    \begin{exampleblock}{2. Lộ trình tư duy (Roadmap)}
        Chúng ta sẽ đi qua 2 mảnh ghép (Chunks):
        \begin{enumerate}
            \item \textbf{Chunk 1:} Truy tìm "Trùm cuối" (Số lớn nhất là gì?).
            \item \textbf{Chunk 2:} Công thức ngược (Tìm lại từng thành phần).
        \end{enumerate}
    \end{exampleblock}
\end{frame}

% =================================================================
\section{Bước 2: Vòng lặp Tư duy (Chunk 1)}
% =================================================================

\begin{frame}{Chunk 1: Truy tìm "Trùm cuối"}
    \begin{block}{Logic (Ẩn dụ hóa)}
        \begin{itemize}
            \item Tưởng tượng $a, b, c$ là cân nặng 3 quả tạ.
            \item Người ta cân các cặp đôi và cân cả 3 quả cùng lúc.
            \item Vì $a, b, c > 0$ $\rightarrow$ Lần cân \textbf{cả 3 quả} chắc chắn nặng nhất.
        \end{itemize}
    \end{block}

    \vspace{0.5cm}
    \centering
    \textbf{\textcolor{blue}{QUY LUẬT:}} \\
    Trong 4 số đầu vào, số có giá trị \textbf{LỚN NHẤT (MAX)} chính là tổng:
    \[ S = a + b + c \]
\end{frame}

\begin{frame}{Chunk 1: Bẫy \& Thử thách}
    \begin{alertblock}{Bẫy (Trap) \textbf{⚠️}}
        Đừng nghĩ số đầu tiên hay số cuối cùng là Max. \\
        Đề bài nói 4 số bị \textbf{xáo trộn ngẫu nhiên}. Số Max có thể nằm bất cứ đâu.
    \end{alertblock}

    \begin{exampleblock}{Thử thách tư duy (Mental Check) \textbf{❓}}
        \textbf{Input:} \texttt{3  6  5  4}
        
        \textbf{Phân tích:}
        \begin{itemize}
            \item Số lớn nhất là \textbf{6}.
            \item Vậy: $a + b + c = 6$.
            \item Ba số còn lại (\texttt{3, 5, 4}) là tổng của các cặp ($a+b, b+c, c+a$).
        \end{itemize}
    \end{exampleblock}
\end{frame}

% =================================================================
\section{Bước 2: Vòng lặp Tư duy (Chunk 2)}
% =================================================================

\begin{frame}{Chunk 2: Tháo gỡ (Tìm lại a, b, c)}
    \begin{block}{Logic (Phép toán trừ) \textbf{➖}}
        Chúng ta có:
        \begin{itemize}
            \item Tổng lớn (Max): $a + b + c$
            \item Số nhỏ (Ví dụ): $a + b$
        \end{itemize}
        
        \textbf{Công thức tìm số còn thiếu:}
        \[ (a + b + c) - (a + b) = c \]
    \end{block}

    \begin{itemize}
        \item \textbf{Nôm na:} Lấy cái \textbf{Tổng lớn nhất} trừ đi \textbf{từng số nhỏ}, phần dư ra chính là $a, b, c$.
    \end{itemize}
\end{frame}

\begin{frame}{Chunk 2: Kiểm chứng thực tế}
    \begin{exampleblock}{Ví dụ: Input \texttt{3 6 5 4}}
        \begin{enumerate}
            \item Tìm Max = \textbf{6}.
            \item Ba số còn lại là: \texttt{3, 5, 4}.
            \item Áp dụng công thức trừ:
            \begin{itemize}
                \item $6 - 3 = \mathbf{3}$
                \item $6 - 5 = \mathbf{1}$
                \item $6 - 4 = \mathbf{2}$
            \end{itemize}
        \end{enumerate}
    \end{exampleblock}
    
    \begin{block}{Kết quả}
        Ba số cần tìm là: \textbf{3, 1, 2}. \\
        (Kiểm tra lại: $3+1+2 = 6$, khớp với Max).
    \end{block}
\end{frame}

% =================================================================
\section{Bước 3: Tổng kết chiến thuật}
% =================================================================

\begin{frame}{Bước 3: Tổng kết chiến thuật (Wrap Up)}
    Để máy tính giải quyết bài này gọn gàng nhất (tránh \texttt{if/else} lằng nhằng), ta dùng chiến thuật:
    
    \begin{center}
        \LARGE \textbf{"XẾP HÀNG" (SORTING)}
    \end{center}

    \vspace{0.5cm}
    
    \begin{enumerate}
        \item Nhận 4 số vào Mảng (Array).
        \item \textbf{Sắp xếp tăng dần:} Số bé đứng trước, số lớn đứng sau.
        \item Số ở vị trí cuối cùng (index 3) chắc chắn là \textbf{MAX}.
        \item Ba số đầu (index 0, 1, 2) là các cặp tổng.
    \end{enumerate}
\end{frame}

\begin{frame}[fragile]{Bản đồ Code (Blueprint)}
    \begin{block}{Chiến thuật lập trình}
        \begin{enumerate}
            \item \textbf{INPUT:} Nhập mảng \texttt{arr}.
            \item \textbf{SORT:} Sắp xếp \texttt{arr} tăng dần.
            \item \textbf{OUTPUT:} In ra kết quả phép trừ:
            \begin{itemize}
                \item $\texttt{arr}[3] - \texttt{arr}[0]$
                \item $\texttt{arr}[3] - \texttt{arr}[1]$
                \item $\texttt{arr}[3] - \texttt{arr}[2]$
            \end{itemize}
        \end{enumerate}
    \end{block}

    \begin{alertblock}{Lưu ý cho Coder}
        \begin{itemize}
            \item \textbf{Python:} Dùng \texttt{list.sort()}
            \item \textbf{C++:} Dùng \texttt{std::sort(a, a+4)}
        \end{itemize}
    \end{alertblock}
\end{frame}

% --- Slide Cuối ---
\begin{frame}{Nhiệm vụ cuối cùng}
    \centering
    \Huge
    \textbf{Code thôi!} \\
    \vspace{1cm}
    \normalsize
    Hãy mở IDE lên và hiện thực hóa chiến thuật này.\\
    Chúc bạn \textbf{Accepted (AC)} ngay lần nộp đầu tiên! 🚀
\end{frame}

\end{document}