\documentclass[10pt]{beamer}

% Cấu hình ngôn ngữ và encoding
\usepackage[utf8]{inputenc}
\usepackage[vietnamese]{babel}
\usepackage{amsmath}
\usepackage{listings}
\usepackage{xcolor}

% Chọn theme
\usetheme{Madrid}
\usecolortheme{beaver}

% Cấu hình hiển thị code
\definecolor{codegreen}{rgb}{0,0.6,0}
\definecolor{codegray}{rgb}{0.5,0.5,0.5}
\definecolor{codepurple}{rgb}{0.58,0,0.82}
\definecolor{backcolour}{rgb}{0.95,0.95,0.92}

\lstdefinestyle{mystyle}{
    backgroundcolor=\color{backcolour},   
    commentstyle=\color{codegreen},
    keywordstyle=\color{magenta},
    numberstyle=\tiny\color{codegray},
    stringstyle=\color{codepurple},
    basicstyle=\ttfamily\footnotesize,
    breakatwhitespace=false,         
    breaklines=true,                 
    captionpos=b,                    
    keepspaces=true,                 
    numbers=left,                    
    numbersep=5pt,                  
    showspaces=false,                
    showstringspaces=false,
    showtabs=false,                  
    tabsize=2
}
\lstset{style=mystyle}

% Thông tin bài giảng
\title[Codeforces 581A]{Codeforces 581A - Vasya the Hipster}
\subtitle{Tư duy thuật toán theo phương pháp "Learning How to Learn"}
\author{Coach Tư Duy Thuật Toán}
\date{\today}

\begin{document}

% --- Slide Tiêu đề ---
\begin{frame}
    \titlepage
\end{frame}

% --- Slide Giới thiệu ---
\begin{frame}{Giới thiệu: Coach LHTL}
    \begin{block}{Chào mừng!}
        Chào mừng bạn đến với lớp học tư duy thuật toán! \\
        Tôi đã kích hoạt chế độ \textbf{Coach LHTL (Learning How to Learn)}.
    \end{block}

    \vspace{0.5cm}
    
    Hôm nay chúng ta sẽ xử lý bài toán: \\
    \textbf{\Large Codeforces 581A - Vasya the Hipster}
    
    \vspace{0.5cm}
    
    \begin{alertblock}{Lưu ý quan trọng}
        Đừng vội code! Hãy cất bàn phím đi. Chúng ta sẽ "mổ xẻ" bài toán này trước.
    \end{alertblock}
\end{frame}

% --- Phần 1: Tiếp nhận & Phẫu thuật ---
\section{1. Tiếp nhận \& Phẫu thuật}
\begin{frame}{1. Tiếp nhận \& Phẫu thuật (Briefing)}
    \begin{block}{Tóm tắt đề bài (Ngôn ngữ con người)}
        Bạn có 2 đống tất (vớ):
        \begin{itemize}
            \item Đống màu Đỏ có số lượng là $a$.
            \item Đống màu Xanh có số lượng là $b$.
        \end{itemize}
    \end{block}

    \begin{exampleblock}{Nhiệm vụ: Tính 2 con số}
        \begin{enumerate}
            \item \textbf{Số ngày "sành điệu":} Mỗi ngày mang 1 chiếc Đỏ + 1 chiếc Xanh (cho đến khi không thể mang kiểu này được nữa).
            \item \textbf{Số ngày "thường thường":} Sau khi hết ngày sành điệu, bạn lấy số tất còn thừa (chắc chắn chỉ còn 1 màu) để mang cặp (2 chiếc cùng màu).
        \end{enumerate}
    \end{exampleblock}

    \textbf{Lộ trình tư duy:}
    \begin{itemize}
        \item \textbf{Chunk 1:} Tìm số lượng cặp "Sành điệu" (Khác màu).
        \item \textbf{Chunk 2:} Xử lý đống tất thừa để tính số ngày "Thường thường" (Cùng màu).
    \end{itemize}
\end{frame}

% --- Phần 2: Chunk 1 - Logic & Thử thách ---
\section{2. Chunk 1: Cặp đôi hoàn hảo}
\begin{frame}{2. Vòng lặp tư duy - Chunk 1: Cặp đôi hoàn hảo}
    \begin{block}{Logic (Ẩn dụ hóa)}
        Hãy tưởng tượng đây là một sàn khiêu vũ.
        \begin{itemize}
            \item Phe Đỏ có $a$ người.
            \item Phe Xanh có $b$ người.
            \item Để nhảy điệu "Sành điệu", bắt buộc phải ghép \textbf{1 người Đỏ} với \textbf{1 người Xanh}.
        \end{itemize}
        Cuộc vui sẽ dừng lại ngay khi \textbf{một trong hai phe hết người}.
    \end{block}

    \begin{alertblock}{Bẫy tư duy (Trap)}
        Nhiều bạn nghĩ phức tạp hóa vấn đề bằng vòng lặp \texttt{while}. Nhưng thực tế đây là một phép so sánh đơn giản.
    \end{alertblock}

    \begin{exampleblock}{Thử thách tư duy (Mental Check)}
        Giả sử: Đỏ ($a$) = \textbf{7} chiếc, Xanh ($b$) = \textbf{4} chiếc. \\
        \textbf{Câu hỏi:} Bạn sẽ có bao nhiêu ngày "Sành điệu"? Tại sao?
    \end{exampleblock}
\end{frame}

% --- Phần 2: Chunk 1 - Giải đáp ---
\begin{frame}{2. Giải đáp Chunk 1}
    \begin{center}
        \Huge \textbf{Đáp án: 4}
    \end{center}
    
    \vspace{0.5cm}

    \begin{block}{Phân tích}
        \textbf{Chính xác!} Bạn đã nắm được quy luật cốt lõi: \\
        \emph{"Quyết định bởi kẻ yếu thế hơn"}.
    \end{block}

    \vspace{0.5cm}

    Trong lập trình, chúng ta gọi đây là tìm giá trị nhỏ nhất: 
    \[ \min(a, b) \]
    
    Chúng ta đã giải quyết xong con số đầu tiên (Output 1). Giờ hãy xử lý phần còn lại.
\end{frame}

% --- Phần 3: Chunk 2 - Logic & Thử thách ---
\section{3. Chunk 2: Vét sạch kho}
\begin{frame}{3. Vòng lặp tư duy - Chunk 2: Vét sạch kho (Leftovers)}
    \begin{block}{Logic (Tư duy trừu tượng)}
        Sau khi "bữa tiệc" sành điệu kết thúc (4 cặp đã rời đi):
        \begin{itemize}
            \item Phe Xanh: Hết sạch (vì ít hơn).
            \item Phe Đỏ: Vẫn còn dư.
        \end{itemize}
        Số lượng tất còn dư chính là: \textbf{Hiệu số giữa hai đống} ($|a - b|$).
    \end{block}

    \begin{block}{Luật chơi "Thường thường"}
        Bạn chỉ còn lại một đống tất cùng màu. Cứ \textbf{2 chiếc} ghép lại thành 1 đôi.
    \end{block}

    \begin{alertblock}{Bẫy tư duy (Trap)}
        Nếu còn dư 1 chiếc lẻ loi thì sao? \\
        $\rightarrow$ Chiếc lẻ đó sẽ bị bỏ đi. Đây là phép \textbf{chia lấy phần nguyên} (Integer Division).
    \end{alertblock}
\end{frame}

\begin{frame}{3. Thử thách Chunk 2}
    \begin{exampleblock}{Thử thách tư duy (Mental Check)}
        Vẫn với ví dụ cũ: Đỏ ($a$) = \textbf{7}, Xanh ($b$) = \textbf{4}.
        \begin{enumerate}
            \item Còn dư lại bao nhiêu chiếc tất (sau khi đã lấy 4 cặp sành điệu)?
            \item Từ số tất dư đó, bạn ghép được thêm \textbf{bao nhiêu cặp} cùng màu nữa?
        \end{enumerate}
    \end{exampleblock}
\end{frame}

% --- Phần 3: Chunk 2 - Giải đáp ---
\begin{frame}{3. Giải đáp Chunk 2}
    \begin{center}
        \Large \textbf{1. Dư: 3 chiếc} \\
        \Large \textbf{2. Ghép được: 1 cặp}
    \end{center}

    \vspace{0.5cm}

    \begin{block}{Chuẩn không cần chỉnh!}
        \begin{itemize}
            \item Dư 3 chiếc $\rightarrow$ Ghép được 1 đôi $\rightarrow$ Còn 1 chiếc lẻ bị "ra rìa" (bỏ đi).
            \item Trong lập trình: Phép tính $3 \div 2 = 1$ (bỏ phần dư) chính là bản chất của phép chia số nguyên.
        \end{itemize}
    \end{block}
\end{frame}

% --- Phần 4: Tổng kết ---
\section{4. Tổng kết & Chốt thuật toán}
\begin{frame}{4. Tổng kết \& Chốt thuật toán (The Grand Finale)}
    \textbf{Sơ đồ logic (Algorithm Flow):}
    
    \begin{enumerate}
        \item \textbf{Input:} Nhập $a$ (Đỏ) và $b$ (Xanh).
        \item \textbf{Bước 1 (Tính ngày Sành điệu):}
            \begin{itemize}
                \item Tìm số nhỏ hơn trong hai số.
                \item Công thức: \texttt{sanh\_dieu = min(a, b)}
            \end{itemize}
        \item \textbf{Bước 2 (Tính tất thừa):}
            \begin{itemize}
                \item Tìm chênh lệch giữa hai đống.
                \item Công thức: \texttt{phan\_thua = abs(a - b)}
            \end{itemize}
        \item \textbf{Bước 3 (Tính ngày Thường thường):}
            \begin{itemize}
                \item Lấy phần thừa chia đôi.
                \item Công thức: \texttt{thuong\_thuong = phan\_thua / 2}
            \end{itemize}
        \item \textbf{Output:} In ra \texttt{sanh\_dieu} và \texttt{thuong\_thuong}.
    \end{enumerate}
\end{frame}

% --- Phần 4: Mã giả ---
\begin{frame}[fragile]{4. Mã giả (Pseudocode) - C++ Style}
    \begin{lstlisting}[language=C++]
// Nhap a, b
cin >> a >> b;

// Tinh toan
// 1. So ngay sanh dieu la min cua a va b
int days_fashion = min(a, b);

// 2. So tat thua la tri tuyet doi hieu a va b
int remaining_socks = abs(a - b); 
// Luu y: Can thu vien <cmath> hoac tu viet if/else

// 3. So ngay thuong thuong la phan nguyen cua phep chia 2
int days_boring = remaining_socks / 2;

// Ket qua
cout << days_fashion << " " << days_boring;
    \end{lstlisting}

    \vspace{0.5cm}
    \begin{center}
        \textbf{🚀 BƯỚC CUỐI CÙNG:} \\
        Hãy thử viết code hoàn chỉnh và nộp thử (Submit)! \\
        Nếu gặp lỗi, hãy gọi: \textbf{"Coach, bài tiếp theo!"} 👊
    \end{center}
\end{frame}

\end{document}