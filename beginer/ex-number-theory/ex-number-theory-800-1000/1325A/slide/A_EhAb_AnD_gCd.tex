\documentclass{beamer}
\usepackage[utf8]{inputenc}
\usepackage[T5]{fontenc} % Bắt buộc để hiển thị tiếng Việt
\usepackage[vietnamese]{babel}
\usepackage{tcolorbox}
\usepackage{listings}
\usepackage{xcolor}
\usepackage{booktabs}
\usetheme{Madrid}

\definecolor{codegreen}{rgb}{0,0.6,0}
\definecolor{codegray}{rgb}{0.5,0.5,0.5}
\definecolor{codepurple}{rgb}{0.58,0,0.82}
\definecolor{backcolour}{rgb}{0.95,0.95,0.92}

\lstdefinestyle{mystyle}{
    backgroundcolor=\color{backcolour},   
    commentstyle=\color{codegreen},
    keywordstyle=\color{magenta},
    numberstyle=\tiny\color{codegray},
    stringstyle=\color{codepurple},
    basicstyle=\ttfamily\scriptsize,
    breakatwhitespace=false,         
    breaklines=true,                 
    captionpos=b,                    
    keepspaces=true,                 
    numbers=left,                    
    numbersep=4pt,                  
    showspaces=false,                
    showstringspaces=false,
    showtabs=false,                  
    tabsize=2,
    escapechar=@
}

\lstset{style=mystyle}

\title[EhAb AnD gCd]{Huấn luyện viên Tư duy Thuật toán}
\subtitle{Giải mã bài toán Codeforces 1325A - EhAb AnD gCd}
\author{Slide Learning CPP}
\date{\today}

\begin{document}

\begin{frame}
  \titlepage
\end{frame}

\begin{frame}{1. Phẫu thuật đề bài (Deconstruct)}
  \begin{block}{Yêu cầu bài toán}
    Cho một số nguyên dương $x$. Tìm hai số nguyên dương $a$ và $b$ sao cho:
    $$GCD(a, b) + LCM(a, b) = x$$
  \end{block}

  \begin{exampleblock}{Giải thích ẩn dụ}
    \begin{itemize}
      \item \textbf{GCD (Ước chung lớn nhất):} Độ dài của cái "thước đo" dài nhất có thể đo vừa khít cả hai sợi dây $a$ và $b$.
      \item \textbf{LCM (Bội chung nhỏ nhất):} Quãng đường ngắn nhất để hai vận động viên $a$ và $b$ gặp nhau tại điểm xuất phát.
    \end{itemize}
  \end{exampleblock}
\end{frame}

\begin{frame}{2. Lộ trình tư duy (Roadmap)}
  Chúng ta sẽ đi qua 2 mảnh ghép (Chunks):
  \vspace{0.5cm}
  \begin{itemize}
    \item \textbf{Chunk 1:} Khám phá mối quan hệ giữa các số đặc biệt (Sức mạnh của số 1).
    \item \textbf{Chunk 2:} Tìm ra "công thức vạn năng" để giải quyết mọi bộ dữ liệu.
  \end{itemize}
\end{frame}

\begin{frame}{3. Mảnh ghép 1: Sức mạnh của số 1}
  Trong thế giới của $GCD$ và $LCM$, số 1 rất dễ đoán:
  \begin{itemize}
    \item $GCD(1, k) = 1$ (Số 1 là ước của mọi số).
    \item $LCM(1, k) = k$ (Mọi số đều là bội của 1).
  \end{itemize}

  \begin{alertblock}{Thử thách tư duy}
    Giả sử chọn $a = 1$ và $b = 5$. Hãy tính:
    \begin{enumerate}
      \item $GCD(1, 5) = ?$ \pause \textbf{1}
      \item $LCM(1, 5) = ?$ \pause \textbf{5}
      \item Tổng $GCD + LCM = ?$ \pause \textbf{6}
    \end{enumerate}
    \pause
    \textit{Nhận xét: Tổng đúng bằng $1 + 5 = 6$.}
  \end{alertblock}
\end{frame}

\begin{frame}{4. Tổng quát hóa bài toán}
  Nếu ta luôn chọn cố định số đầu tiên là $a = 1$:
  \begin{itemize}
    \item Ta có phương trình: $GCD(1, b) + LCM(1, b) = x$
    \item Tương đương: $1 + b = x$
  \end{itemize}
  
  \pause
  \begin{block}{Chìa khóa vạn năng}
    Để $1 + b = x$, ta chỉ cần chọn:
    $$\mathbf{b = x - 1}$$
  \end{block}
  
  \pause
  \begin{exampleblock}{Ví dụ}
    Với $x = 100$, chọn $a = 1$ và $b = 99$.
    \begin{itemize}
        \item $GCD(1, 99) = 1$
        \item $LCM(1, 99) = 99$
        \item $1 + 99 = 100$ (Thỏa mãn!)
    \end{itemize}
  \end{exampleblock}
\end{frame}

\begin{frame}[fragile]{5. Mã giả (Pseudocode)}
  Chiến thuật: Luôn in ra hai số $1$ và $x-1$.

\begin{lstlisting}[language=Python, caption=Mã giả giải thuật]
Nhap vao so luong bo test t
Lap t lan:
    Nhap vao so x
    In ra: 1 va (x - 1)
\end{lstlisting}

  \begin{alertblock}{Kiểm tra bẫy logic (Edge Case)}
    Nếu $x = 2$:
    \begin{itemize}
      \item Thuật toán in ra: \textbf{1 1}
      \item $GCD(1, 1) + LCM(1, 1) = 1 + 1 = 2$.
      \item Thỏa mãn điều kiện nguyên dương $a, b \ge 1$.
    \end{itemize}
  \end{alertblock}
\end{frame}

\begin{frame}{6. Tổng kết}
  \begin{block}{Tính chất bất biến}
    $$GCD(1, x-1) + LCM(1, x-1) = x$$
    Luôn đúng với mọi số nguyên dương $x \ge 2$.
  \end{block}

  \vspace{0.5cm}
  \textbf{Bạn muốn tiếp tục với thử thách nào?}
  \begin{enumerate}
    \item Luyện tập thêm bài toán số học tương tự.
    \item Chuyển sang chủ đề Mảng (Array) hoặc Chuỗi (String).
    \item Xem mã nguồn C++ hoàn chỉnh.
  \end{enumerate}
\end{frame}

\end{document}