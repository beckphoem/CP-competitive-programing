\documentclass{beamer}
\usepackage[utf8]{inputenc}
\usepackage[T5]{fontenc} % Bắt buộc để hiển thị tiếng Việt
\usepackage[vietnamese]{babel}
\usepackage{tcolorbox}
\usepackage{listings}
\usepackage{xcolor}
\usepackage{booktabs}
\usetheme{Madrid}

\definecolor{codegreen}{rgb}{0,0.6,0}
\definecolor{codegray}{rgb}{0.5,0.5,0.5}
\definecolor{codepurple}{rgb}{0.58,0,0.82}
\definecolor{backcolour}{rgb}{0.95,0.95,0.92}

\lstdefinestyle{mystyle}{
    backgroundcolor=\color{backcolour},   
    commentstyle=\color{codegreen},
    keywordstyle=\color{magenta},
    numberstyle=\tiny\color{codegray},
    stringstyle=\color{codepurple},
    basicstyle=\ttfamily\scriptsize,
    breakatwhitespace=false,         
    breaklines=true,                 
    captionpos=b,                    
    keepspaces=true,                 
    numbers=left,                    
    numbersep=4pt,                  
    showspaces=false,                
    showstringspaces=false,
    showtabs=false,                  
    tabsize=2,
    escapechar=@
}

\lstset{style=mystyle}

\title[Codeforces 1328A]{Huấn luyện viên Tư duy Thuật toán}
\subtitle{Phân tích bài 1328A - Divisibility Problem}
\author{Slide Learning CPP}
\date{\today}

\begin{document}

\begin{frame}
  \titlepage
\end{frame}

\begin{frame}{Chào bạn!}
  \begin{block}{Sứ mệnh}
    Chúng ta sẽ cùng nhau "mổ xẻ" các bài toán theo phong cách \textbf{Micro-Chunks}, tập trung hoàn toàn vào chiến thuật và logic thay vì chỉ nhìn vào code.
  \end{block}
\end{frame}

% --- Slide 1 ---
\begin{frame}{Bước 1: Tiếp nhận \& Phẫu thuật (Briefing)}
  \begin{itemize}
    \item \textbf{Đề bài:} Tìm số bước tối thiểu để biến $a$ thành một số chia hết cho $b$.
    \item \textbf{Quy tắc:} Trong mỗi bước, chỉ được phép tăng $a$ lên 1 đơn vị ($a = a + 1$).
    \item \textbf{Mục tiêu:} Tìm số bước tăng ít nhất sao cho $a \pmod b = 0$.
  \end{itemize}

  \begin{exampleblock}{Lộ trình tư duy}
    \begin{enumerate}
      \item \textbf{Chunk 1:} Bản chất phép chia hết và "khoảng cách".
      \item \textbf{Chunk 2:} Xử lý trường hợp đặc biệt.
      \item \textbf{Chunk 3:} Tối ưu hóa - Tại sao không dùng vòng lặp?
    \end{enumerate}
  \end{exampleblock}
\end{frame}

% --- Slide 2 ---
\begin{frame}{Chunk 1: Khoảng cách tới "vạch đích"}
  Hãy tưởng tượng số $b$ giống như \textbf{độ dài của một bước chân}. Bạn đang ở vị trí $a$ và muốn nhảy đến một vị trí là bội số của $b$.

  \begin{alertblock}{Thử thách tư duy}
    Nếu $a = 13$ và $b = 4$:
    \begin{itemize}
        \item Các mốc chia hết cho 4 gần đó là: 4, 8, 12, 16, 20...
        \item Vì bạn chỉ có thể \textbf{tăng} $a$, mốc tiếp theo là bao nhiêu?
    \end{itemize}
  \end{alertblock}

  \pause
  \textbf{Đáp án:} Mốc tiếp theo là \textbf{16}. Cần $16 - 13 = 3$ bước.
\end{frame}

% --- Slide 3 ---
\begin{frame}{Chunk 2: Công thức "Nhảy cóc"}
  Trong lập trình thi đấu, nếu $a$ và $b$ rất lớn ($10^9$), dùng vòng lặp \texttt{while (a \% b != 0) a++;} sẽ bị \textbf{TLE} (Time Limit Exceeded).

  \begin{block}{Tổng quát hóa}
    Khi chia $a$ cho $b$:
    \begin{itemize}
      \item Phần nguyên: $a / b$.
      \item Số dư: $r = a \pmod b$.
    \end{itemize}
  \end{block}

  \begin{exampleblock}{Gợi ý}
    Nếu có 13 cái kẹo ($a=13$), túi mỗi túi 4 cái ($b=4$), bạn dư 1 cái ($r=1$). Cần thêm bao nhiêu cái để đủ 1 túi nữa?
  \end{exampleblock}

  \pause
  \textbf{Công thức:} $b - (a \pmod b)$.
\end{frame}

% --- Slide 4 ---
\begin{frame}{Chunk 3: Xử lý "Bẫy" logic}
  \begin{alertblock}{Vấn đề}
    Nếu $a$ đã chia hết cho $b$ ngay từ đầu (VD: $a=8, b=4$)?
  \end{alertblock}

  \begin{itemize}
    \item Số dư $a \pmod b = 0$.
    \item Áp dụng công thức $b - (a \pmod b) \Rightarrow 4 - 0 = 4$.
    \item \textbf{Thực tế:} Cần 0 bước.
  \end{itemize}

  \pause
  \begin{block}{Giải pháp dùng IF}
    \texttt{if (a \% b == 0) return 0;}\\
    \texttt{else return b - (a \% b);}
  \end{block}
\end{frame}

% --- Slide 5 ---
\begin{frame}{Chunk 4: Công thức "Một dòng"}
  \begin{exampleblock}{Mẹo toán học}
    Để gộp cả hai trường hợp mà không cần \texttt{if}:\\
    \centering \Large \texttt{(b - (a \% b)) \% b}
  \end{exampleblock}

  \pause
  \textbf{Kiểm chứng:}
  \begin{enumerate}
    \item $a=13, b=4$: $(4 - (13 \% 4)) \% 4 = (4 - 1) \% 4 = 3 \% 4 = \mathbf{3}$.
    \item $a=8, b=4$: $(4 - (8 \% 4)) \% 4 = (4 - 0) \% 4 = 4 \% 4 = \mathbf{0}$.
  \end{enumerate}
\end{frame}

% --- Slide 6 ---
\begin{frame}[fragile]{Bước cuối: Chốt thuật toán}
  \begin{block}{Mã giả (Pseudo-code)}
\begin{lstlisting}[language=C++]
@\textbf{Nhap}@ t (so luong test case)
@\textbf{Lap}@ t lan:
    @\textbf{Nhap}@ a, b
    @\textbf{Ket qua}@ = (b - (a % b)) % b
    @\textbf{In ra}@ @\textbf{Ket qua}@
\end{lstlisting}
  \end{block}

  \begin{alertblock}{Lưu ý Edge Case}
    \begin{itemize}
      \item $a, b \le 10^9$: Sử dụng kiểu dữ liệu \texttt{int} là đủ.
      \item Luôn chú ý trường hợp $a < b$: Công thức vẫn hoạt động đúng.
    \end{itemize}
  \end{alertblock}
  
  \begin{center}
    \textbf{Bạn đã sẵn sàng để tự viết Code chưa? 🚀}
  \end{center}
\end{frame}

\end{document}