\documentclass{beamer}
\usepackage[utf8]{inputenc}
\usepackage[T5]{fontenc} % Bắt buộc để hiển thị tiếng Việt
\usepackage[vietnamese]{babel}
\usepackage{tcolorbox}
\usepackage{listings}
\usepackage{xcolor}
\usepackage{booktabs}
\usetheme{Madrid}

\definecolor{codegreen}{rgb}{0,0.6,0}
\definecolor{codegray}{rgb}{0.5,0.5,0.5}
\definecolor{codepurple}{rgb}{0.58,0,0.82}
\definecolor{backcolour}{rgb}{0.95,0.95,0.92}

\lstdefinestyle{mystyle}{
    backgroundcolor=\color{backcolour},   
    commentstyle=\color{codegreen},
    keywordstyle=\color{magenta},
    numberstyle=\tiny\color{codegray},
    stringstyle=\color{codepurple},
    basicstyle=\ttfamily\scriptsize,
    breakatwhitespace=false,         
    breaklines=true,                 
    captionpos=b,                    
    keepspaces=true,                 
    numbers=left,                    
    numbersep=4pt,                  
    showspaces=false,                
    showstringspaces=false,
    showtabs=false,                  
    tabsize=2,
    escapechar=@
}

\lstset{style=mystyle}

\title{Codeforces 1370A - Maximum GCD}
\subtitle{Tư duy thuật toán qua Micro-chunking}
\author{Slide Learning CPP}
\date{\today}

\begin{document}

\begin{frame}
    \titlepage
\end{frame}

\begin{frame}{Phẫu thuật đề bài (Deconstruct)}
    \begin{block}{Cốt lõi vấn đề}
        \begin{itemize}
            \item \textbf{Dữ liệu cho:} Một số nguyên dương $n$.
            \item \textbf{Nhiệm vụ:} Tìm hai số nguyên $a$ và $b$ sao cho:
            \begin{enumerate}
                \item $1 \le a < b \le n$.
                \item $GCD(a, b)$ là lớn nhất có thể.
            \end{enumerate}
            \item \textbf{Kết quả:} Giá trị GCD lớn nhất đó.
        \end{itemize}
    \end{block}

    \begin{itemize}
        \item Hiểu bản chất GCD qua hình ảnh thực tế.
        \item Tìm mối quan hệ giữa "khoảng cách" và "ước chung".
        \item Xác định quy luật tối ưu.
    \end{itemize}
\end{frame}

\begin{frame}{Chunk 1: Bản chất của GCD và sự "cộng hưởng"}
    \begin{block}{Ẩn dụ "Thước đo"}
        Nếu $GCD(a, b) = g$, nghĩa là cả $a$ và $b$ đều được ghép từ những "viên gạch" độ dài $g$.
        \begin{itemize}
            \item $a = x \cdot g$
            \item $b = y \cdot g$
        \end{itemize}
        Vì $a < b$, nên $b$ phải nhiều hơn $a$ ít nhất một "viên gạch" $g$. Tức là khoảng cách giữa $a$ và $b$ ít nhất phải là $g$.
    \end{block}

    \begin{alertblock}{Bẫy logic}
        Cố gắng chọn hai số thật lớn gần $n$ nhưng không có chung ước lớn. Ví dụ: $n=100$, chọn 99 và 100 thì $GCD(99, 100) = 1$.
    \end{alertblock}
\end{frame}

\begin{frame}{Chunk 1: Thử thách tư duy}
    \begin{exampleblock}{Thử thách}
        Giả sử $n=10$. Bạn muốn tìm một "viên gạch" $g$ lớn nhất sao cho có thể xếp được ít nhất hai chồng gạch ($a$ và $b$) mà cả hai không quá $n$.
    \end{exampleblock}

    \begin{itemize}
        \item Nếu chọn $g = 5$: Ta có $a = 5$ (1 viên), $b = 10$ (2 viên). Thỏa mãn $\le 10$.
        \item Nếu chọn $g = 6$: Ta có $a = 6$, số tiếp theo phải là $12$. (Vượt quá $n=10$).
    \end{itemize}
    
    \pause
    \textbf{Kết luận:} $g$ lớn nhất sao cho $2 \cdot g \le n$.
\end{frame}

\begin{frame}{Chunk 2: Tìm công thức tổng quát}
    \begin{itemize}
        \item Nếu $n=5$: Các cặp có thể là (1,2), (2,4)... GCD lớn nhất là $5/2 = 2$.
        \item Nếu $n=11$: Cặp tối ưu là (5, 10). GCD lớn nhất là $11/2 = 5$.
    \end{itemize}

    \begin{block}{Quy luật toán học}
        Kết quả là $\lfloor n / 2 \rfloor$ (Phần nguyên của $n$ chia 2).
    \end{block}

    \pause
    \begin{exampleblock}{Tại sao con số này là "vô đối"?}
        Nếu chọn $g > n/2$, thì số bội tiếp theo là $2g$ chắc chắn sẽ $> n$. Bạn không thể tìm được số thứ hai trong phạm vi cho phép.
    \end{exampleblock}
\end{frame}

\begin{frame}{Chunk 3: Bẫy về số lượng Test Case}
    \begin{alertblock}{Lưu ý về hiệu năng}
        Bài toán có $T$ bộ dữ liệu. Nếu dùng vòng lặp thử từng số từ $n$ về 1, độ phức tạp sẽ là $O(T \cdot n)$, dễ dẫn đến \textbf{Time Limit Exceeded}.
    \end{alertblock}

    \begin{block}{Giải pháp tối ưu}
        Sử dụng công thức $\lfloor n / 2 \rfloor$, độ phức tạp chỉ là $O(1)$ cho mỗi test case. Cực kỳ nhanh!
    \end{block}
\end{frame}

\begin{frame}[fragile]{Tổng kết thuật toán (Pseudocode)}
    \begin{lstlisting}[language=Python, caption=Mô phỏng thuật toán]
@Nhập số lượng test case@ t
@Lặp@ t @lần:@
    @Nhập số nguyên@ n
    @Kết quả =@ n / 2 @(chia lấy nguyên)@
    @In kết quả ra màn hình@
    \end{lstlisting}

    \begin{exampleblock}{Trường hợp biên: $n=1$}
        $\lfloor 1/2 \rfloor = 0$. Thực tế đề bài thường cho $n \ge 2$ để đảm bảo tồn tại cặp $(a, b)$.
    \end{exampleblock}
\end{frame}

\begin{frame}{Bước cuối: Hiện thực hóa ý tưởng}
    \begin{block}{Lưu ý trong C++}
        Toán tử \texttt{/} giữa hai số nguyên sẽ tự động thực hiện phép chia lấy phần nguyên.
        \begin{itemize}
            \item \texttt{10 / 2 = 5}
            \item \texttt{11 / 2 = 5}
        \end{itemize}
    \end{block}

    \begin{center}
        \huge \textbf{Bạn đã sẵn sàng lập trình chưa?}
    \end{center}
\end{frame}

\end{document}