\documentclass{beamer}
\usepackage[utf8]{inputenc}
\usepackage[T5]{fontenc} % Bắt buộc để hiển thị tiếng Việt
\usepackage[vietnamese]{babel}
\usepackage{tcolorbox}
\usepackage{listings}
\usepackage{xcolor}
\usepackage{booktabs}
\usetheme{Madrid}

\definecolor{codegreen}{rgb}{0,0.6,0}
\definecolor{codegray}{rgb}{0.5,0.5,0.5}
\definecolor{codepurple}{rgb}{0.58,0,0.82}
\definecolor{backcolour}{rgb}{0.95,0.95,0.92}

\lstdefinestyle{mystyle}{
    backgroundcolor=\color{backcolour},   
    commentstyle=\color{codegreen},
    keywordstyle=\color{magenta},
    numberstyle=\tiny\color{codegray},
    stringstyle=\color{codepurple},
    basicstyle=\ttfamily\scriptsize,
    breakatwhitespace=false,         
    breaklines=true,                 
    captionpos=b,                    
    keepspaces=true,                 
    numbers=left,                    
    numbersep=4pt,                  
    showspaces=false,                
    showstringspaces=false,
    showtabs=false,                  
    tabsize=2,
    escapechar=@
}

\lstset{style=mystyle}

\title{Chiến lược Phát triển Tư duy Số học}
\subtitle{Hệ thống Bài tập Codeforces Rating 800-1000}
\author{Slide Learning C++}
\date{\today}

\begin{document}

\begin{frame}
    \titlepage
\end{frame}

\begin{frame}{Giới thiệu chung}
    \begin{block}{Tầm quan trọng}
        Nắm vững lý thuyết số cơ bản là cột mốc quan trọng đối với lập trình viên thi đấu.
    \end{block}
    \begin{itemize}
        \item Rating 800-1000 yêu cầu phân tích toán học sơ cấp.
        \item Nhận diện quy luật và tối ưu hóa thuật toán.
        \item Thay thế tư duy lập trình thuần túy bằng tư duy tối ưu hóa.
    \end{itemize}
\end{frame}

\begin{frame}{1. 1328A - Divisibility Problem (Rating 800)}
    \textbf{Link:} \url{https://codeforces.com/problemset/problem/1328/A}
    \begin{block}{Đề bài}
        Cho hai số nguyên dương $a$ và $b$. Tìm số bước tối thiểu để $a$ trở thành một số chia hết cho $b$, mỗi bước bạn có thể tăng $a$ thêm 1 đơn vị.
    \end{block}
    \pause
    \begin{exampleblock}{Hướng giải quyết}
        Sử dụng phép chia dư để tránh TLE:
        \begin{itemize}
            \item Nếu $a \pmod b == 0$, kết quả là 0.
            \item Ngược lại, số bước là $b - (a \pmod b)$.
        \end{itemize}
    \end{exampleblock}
\end{frame}

\begin{frame}{2. 1370A - Maximum GCD (Rating 800)}
    \textbf{Link:} \url{https://codeforces.com/problemset/problem/1370/A}
    \begin{block}{Đề bài}
        Cho số nguyên $n$. Tìm giá trị lớn nhất của $GCD(a, b)$ với $1 \le a < b \le n$.
    \end{block}
    \pause
    \begin{exampleblock}{Hướng giải quyết}
        Để $GCD(a, b) = g$, thì $a$ và $b$ phải là bội của $g$.
        \begin{itemize}
            \item Cặp bội nhỏ nhất là $g$ và $2g$.
            \item Điều kiện $2g \le n \Rightarrow g \le n/2$.
            \item Đáp án tối ưu: $\lfloor n/2 \rfloor$.
        \end{itemize}
    \end{exampleblock}
\end{frame}

\begin{frame}{3. 1325A - EhAb AnD gCd (Rating 800)}
    \textbf{Link:} \url{https://codeforces.com/problemset/problem/1325/A}
    \begin{block}{Đề bài}
        Cho số nguyên dương $x$. Tìm bất kỳ cặp số $(a, b)$ sao cho $GCD(a, b) + LCM(a, b) = x$.
    \end{block}
    \pause
    \begin{exampleblock}{Hướng giải quyết}
        Sử dụng tính chất của số 1:
        \begin{itemize}
            \item $GCD(1, x-1) = 1$.
            \item $LCM(1, x-1) = x-1$.
            \item Với $x$, ta luôn có cặp $(1, x-1)$ thỏa mãn phương trình $1 + (x-1) = x$.
        \end{itemize}
    \end{exampleblock}
\end{frame}

\begin{frame}{4. 1374A - Required Remainder (Rating 800)}
    \textbf{Link:} \url{https://codeforces.com/problemset/problem/1374/A}
    \begin{block}{Đề bài}
        Cho $x, y, n$. Tìm số $k$ lớn nhất sao cho $0 \le k \le n$ và $k \pmod x = y$.
    \end{block}
    \pause
    \begin{exampleblock}{Hướng giải quyết}
        Sử dụng công thức trực tiếp:
        \begin{itemize}
            \item $k = \lfloor (n-y)/x \rfloor \cdot x + y$.
            \item Công thức này đảm bảo tìm được bội số của $x$ phù hợp nhất mà khi cộng thêm $y$ vẫn không vượt quá $n$.
        \end{itemize}
    \end{exampleblock}
\end{frame}

\begin{frame}{5. 1633A - Div. 7 (Rating 800)}
    \textbf{Link:} \url{https://codeforces.com/problemset/problem/1633/A}
    \begin{block}{Đề bài}
        Thay đổi ít nhất các chữ số của $n$ để số mới chia hết cho 7 và không có số 0 ở đầu.
    \end{block}
    \pause
    \begin{exampleblock}{Hướng giải quyết}
        \begin{itemize}
            \item Kiểm tra nếu $n \pmod 7 == 0$.
            \item Nếu chưa, thử thay đổi chữ số hàng đơn vị của $n$ từ $0$ đến $9$.
            \item Trong dải 10 số liên tiếp, luôn có ít nhất một số chia hết cho 7.
        \end{itemize}
    \end{exampleblock}
\end{frame}

\begin{frame}{6. 1088A - Ehab and another construction problem (Rating 800)}
    \textbf{Link:} \url{https://codeforces.com/problemset/problem/1088/A}
    \begin{block}{Đề bài}
        Tìm $a, b$ sao cho $1 \le a, b \le x$, $a$ chia hết cho $b$, $a \cdot b > x$, và $a/b < x$.
    \end{block}
    \pause
    \begin{exampleblock}{Hướng giải quyết}
        \begin{itemize}
            \item Với $x=1$, không có đáp án (in -1).
            \item Với $x > 1$, có thể chọn đơn giản $a = x$ (nếu $x$ chẵn) và $b = 2$ hoặc chọn $a = x, b = x$.
            \item Một lựa chọn an toàn khác là cặp $(x, x)$ cho hầu hết các trường hợp $x > 1$.
        \end{itemize}
    \end{exampleblock}
\end{frame}

\begin{frame}{7. 1343A - Candies (Rating 900)}
    \textbf{Link:} \url{https://codeforces.com/problemset/problem/1343/A}
    \begin{block}{Đề bài}
        Tổng kẹo mua trong $k$ ngày ($k > 1$) là $n$. Ngày 1 mua $x$, ngày sau gấp đôi ngày trước. Tìm $x$ nguyên dương.
    \end{block}
    \pause
    \begin{exampleblock}{Hướng giải quyết}
        Tổng kẹo: $x(2^0 + 2^1 + \dots + 2^{k-1}) = x(2^k - 1) = n$.
        \begin{itemize}
            \item Duyệt $k$ từ $2$ đến $30$.
            \item Nếu $n \pmod{(2^k - 1)} == 0$ thì $x = n/(2^k - 1)$ là đáp án.
        \end{itemize}
    \end{exampleblock}
\end{frame}

\begin{frame}{8. 1475B - New Year's Number (Rating 900)}
    \textbf{Link:} \url{https://codeforces.com/problemset/problem/1475/B}
    \begin{block}{Đề bài}
        Kiểm tra $n$ có thể biểu diễn thành tổng của các số 2020 và 2021 không.
    \end{block}
    \pause
    \begin{exampleblock}{Hướng giải quyết}
        Giải phương trình $2020x + 2021y = n$.
        \begin{itemize}
            \item Đặt $q = n / 2020$ và $r = n \pmod{2020}$.
            \item Ta có $n = 2020q + r$.
            \item Để biểu diễn dưới dạng $2020x + 2021y$, ta cần $y = r$ và $x = q - r$.
            \item Điều kiện: $r \le q$.
        \end{itemize}
    \end{exampleblock}
\end{frame}

\begin{frame}{9. 1474B - Different Divisors (Rating 1000)}
    \textbf{Link:} \url{https://codeforces.com/problemset/problem/1474/B}
    \begin{block}{Đề bài}
        Tìm số $n$ nhỏ nhất có ít nhất 4 ước sao cho khoảng cách giữa 2 ước bất kỳ ít nhất là $d$.
    \end{block}
    \pause
    \begin{exampleblock}{Hướng giải quyết}
        Ước đầu tiên là 1.
        \begin{itemize}
            \item Ước thứ hai $p$ là số nguyên tố nhỏ nhất $\ge 1 + d$.
            \item Ước thứ ba $q$ là số nguyên tố nhỏ nhất $\ge p + d$.
            \item Số $n$ nhỏ nhất sẽ là $p \cdot q$.
        \end{itemize}
    \end{exampleblock}
\end{frame}

\begin{frame}{10. 1203C - Common Divisors (Rating 1000)}
    \textbf{Link:} \url{https://codeforces.com/problemset/problem/1203/C}
    \begin{block}{Đề bài}
        Tìm số lượng ước chung của tất cả các phần tử trong mảng $a$.
    \end{block}
    \pause
    \begin{exampleblock}{Hướng giải quyết}
        \begin{itemize}
            \item Tìm $G = GCD(a_1, a_2, \dots, a_n)$.
            \item Số lượng ước chung của mảng chính là số lượng ước của $G$.
            \item Đếm ước của $G$ bằng cách duyệt từ $1$ đến $\sqrt{G}$.
        \end{itemize}
    \end{exampleblock}
\end{frame}

\begin{frame}[fragile]{Lộ trình và Kỹ thuật Tối ưu}
    \begin{alertblock}{Lưu ý kỹ thuật}
        \begin{itemize}
            \item \textbf{Kiểu dữ liệu:} Với $n \le 10^{12}$, bắt buộc dùng \texttt{long long}.
            \item \textbf{Độ phức tạp:} Ưu tiên $O(1)$ hoặc $O(\log n)$. Đếm ước dùng $O(\sqrt{G})$.
        \end{itemize}
    \end{alertblock}
    
    \begin{block}{Lời khuyên luyện tập}
        \begin{itemize}
            \item Giải 15-20 bài/tuần để nhạy bén với quy luật.
            \item Ghi chú lại các lỗi tràn số hoặc sai số \texttt{double}.
        \end{itemize}
    \end{block}
\end{frame}

\end{document}