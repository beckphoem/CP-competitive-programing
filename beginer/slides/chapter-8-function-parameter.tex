\documentclass{beamer}
\usepackage[utf8]{inputenc}
\usepackage[T5]{fontenc} % Bắt buộc để hiển thị tiếng Việt
\usepackage[vietnamese]{babel}
\usepackage{tcolorbox}
\usepackage{listings}
\usepackage{xcolor}
\usepackage{booktabs}
\usetheme{Madrid}

% Định nghĩa màu sắc cho code
\definecolor{codegreen}{rgb}{0,0.6,0}
\definecolor{codegray}{rgb}{0.5,0.5,0.5}
\definecolor{codepurple}{rgb}{0.58,0,0.82}
\definecolor{backcolour}{rgb}{0.95,0.95,0.92}

% Cấu hình hiển thị code C++
\lstdefinestyle{mystyle}{
    backgroundcolor=\color{backcolour},   
    commentstyle=\color{codegreen},
    keywordstyle=\color{magenta},
    numberstyle=\tiny\color{codegray},
    stringstyle=\color{codepurple},
    basicstyle=\ttfamily\scriptsize,
    breakatwhitespace=false,         
    breaklines=true,                 
    captionpos=b,                    
    keepspaces=true,                 
    numbers=left,                    
    numbersep=4pt,                  
    showspaces=false,                
    showstringspaces=false,
    showtabs=false,                  
    tabsize=2,
    language=C++,
    escapechar=@
}

\lstset{style=mystyle}

% Thông tin bài giảng
\title{C++ Function Parameters}
\subtitle{Biến mã tĩnh lặng trở nên linh hoạt}
\author{Hành trình chinh phục C++}
\date{\today}

\begin{document}

% Slide Tiêu đề
\begin{frame}
    \titlepage
\end{frame}

% Slide Mở đầu: Bức tranh toàn cảnh
\begin{frame}{Bức Tranh Toàn Cảnh (The Big Picture)}
    \begin{block}{Ẩn dụ: Chiếc máy xay sinh tố}
        Hãy tưởng tượng một hàm (Function) giống như một \textbf{Chiếc máy xay sinh tố}.
        \begin{itemize}
            \item Nếu không có tham số: Máy bị "hàn chết", chỉ xay đúng một quả chuối có sẵn.
            \item \textbf{Function Parameters}: Chính là cái \textbf{phiễu (khe nạp)} phía trên.
        \end{itemize}
    \end{block}
    
    \vspace{0.5cm}
    Nhờ nó, bạn có thể thả dâu tây, xoài, táo vào. Cùng một chiếc máy, nhưng nguyên liệu khác nhau sẽ tạo ra kết quả khác nhau.
\end{frame}

% Slide Lộ trình
\begin{frame}{Lộ Trình Khám Phá (Roadmap)}
    Chúng ta sẽ đi qua 6 chặng để nắm vững từng viên gạch:
    \begin{enumerate}
        \item \textbf{Parameters và Arguments:} Phân biệt "Cái Phiễu" và "Trái Cây".
        \item \textbf{Default Parameters:} Cài đặt mặc định (Khi lười chọn món).
        \item \textbf{Multiple Parameters:} Công thức pha chế hỗn hợp.
        \item \textbf{Return Values:} Shipper giao hàng tận tay.
        \item \textbf{Pass by Reference:} Sửa trực tiếp vào bản gốc (Quan trọng).
        \item \textbf{Passing Arrays:} Xử lý cả một thùng hàng.
    \end{enumerate}
\end{frame}

% --- CHƯƠNG 1 ---
\section{Chương 1: Parameters và Arguments}

\begin{frame}{Chương 1: Parameters và Arguments}
    \textbf{Tư duy cốt lõi:}
    \begin{itemize}
        \item \textbf{Parameter (Tham số):} Biến được định nghĩa lúc \textbf{thiết kế} hàm (Cái nhãn "Bỏ trái cây vào đây").
        \item \textbf{Argument (Đối số):} Dữ liệu thực tế truyền vào lúc \textbf{sử dụng} hàm (Quả táo thật).
    \end{itemize}
\end{frame}

\begin{frame}[fragile]{Syntax: Parameters và Arguments}
\begin{lstlisting}
#include <iostream>
#include <string>
using namespace std;

// CACH THIET KE:
// "tenNguoi" o day chinh la PARAMETER (Cai khuon)
void chaoHoi(string tenNguoi) {
  cout << "Xin chao " << tenNguoi << "!\n";
}

int main() {
  // CACH SU DUNG:
  // "Nam" va "Lan" o day chinh la ARGUMENT (Du lieu that)
  chaoHoi("Nam");
  chaoHoi("Lan");
  
  return 0;
}
\end{lstlisting}
\end{frame}

\begin{frame}[fragile]{Kiểm tra sự hiểu biết (Active Recall)}
    Cho đoạn code sau:
\begin{lstlisting}
void tinhTong(int soA) {
  cout << soA + 10;
}
int main() {
  int x = 5;
  tinhTong(x);
  return 0;
}
\end{lstlisting}
    
    Trong dòng \texttt{tinhTong(x);}:
    \begin{itemize}
        \item \texttt{soA} được gọi là gì?
        \item \texttt{x} được gọi là gì?
    \end{itemize}

    \pause
    \begin{alertblock}{Đáp án}
        \textbf{soA} là Parameter (Cái khuôn). \\
        \textbf{x} là Argument (Nguyên liệu thực tế).
    \end{alertblock}
\end{frame}

% --- CHƯƠNG 2 ---
\section{Chương 2: Default Parameters}

\begin{frame}{Chương 2: Default Parameters (Tham số mặc định)}
    \begin{block}{Tình huống quán cà phê}
        \begin{itemize}
            \item Khách gọi "Cho ly cà phê": Nhân viên làm theo \textbf{công thức chuẩn} (Mặc định).
            \item Khách gọi "Cà phê 100\% đường": Nhân viên \textbf{ghi đè} công thức chuẩn.
        \end{itemize}
    \end{block}
    
    Trong C++, \textbf{Default Parameter} dùng giá trị có sẵn nếu người dùng không truyền tham số.
\end{frame}

\begin{frame}[fragile]{Syntax: Default Parameters}
    Bí mật nằm ở dấu bằng \texttt{=} khi khai báo hàm.

\begin{lstlisting}
// Gan gia tri mac dinh "Viet Nam" ngay tai day
void xuatXu(string quocGia = "Viet Nam") {
  cout << "Toi den tu " << quocGia << "\n";
}

int main() {
  // Truong hop 1: CO truyen tham so -> Ghi de
  xuatXu("Nhat Ban"); // In ra: Toi den tu Nhat Ban
  
  // Truong hop 2: KHONG truyen tham so -> Dung mac dinh
  xuatXu();           // In ra: Toi den tu Viet Nam
  return 0;
}
\end{lstlisting}
\end{frame}

\begin{frame}[fragile]{Kiểm tra sự hiểu biết}
    Đoạn code sau in ra màn hình những gì?
\begin{lstlisting}
void thoiTiet(int nhietDo = 25) {
  cout << "Nhiet do la: " << nhietDo << " do C\n";
}
int main() {
  thoiTiet(30);
  thoiTiet(); 
}
\end{lstlisting}

    \pause
    \begin{alertblock}{Đáp án}
        1. Nhiet do la: 30 do C \\
        2. Nhiet do la: 25 do C
    \end{alertblock}
\end{frame}

% --- CHƯƠNG 3 ---
\section{Chương 3: Multiple Parameters}

\begin{frame}{Chương 3: Multiple Parameters (Nhiều tham số)}
    Để pha chế phức tạp, hàm cần nhiều nguyên liệu: Loại trái cây, Lượng đường, Đá...
    
    \begin{alertblock}{Quy tắc vàng}
        \begin{itemize}
            \item Các tham số ngăn cách bởi dấu phẩy \texttt{,}.
            \item \textbf{Thứ tự là mệnh lệnh!} Phải truyền đúng thứ tự định nghĩa.
        \end{itemize}
    \end{alertblock}
\end{frame}

\begin{frame}[fragile]{Syntax: Multiple Parameters}
\begin{lstlisting}
// Ham can 2 nguyen lieu theo thu tu: Chu truoc, So sau
void thongTinNhanVien(string ten, int tuoi) {
  cout << ten << " nam nay " << tuoi << " tuoi.\n";
}

int main() {
  // ĐÚNG:
  thongTinNhanVien("Tung", 25);
  
  // SAI (Lỗi ngay lập tức):
  // thongTinNhanVien(25, "Tung"); 
  
  return 0;
}
\end{lstlisting}
\end{frame}

% --- CHƯƠNG 4 ---
\section{Chương 4: Return Values}

\begin{frame}{Chương 4: Return Values (Giá trị trả về)}
    \begin{columns}
        \column{0.5\textwidth}
        \begin{block}{Hàm Void (Hư vô)}
            Giống như hét lên một câu rồi thôi.
            Không thể "cầm nắm" kết quả để dùng tiếp.
        \end{block}
        
        \column{0.5\textwidth}
        \begin{block}{Hàm Return (Shipper)}
            Giống như thợ làm bánh.
            Họ làm xong và \textbf{trả lại (return)} cái bánh cho bạn để bạn mang đi đâu tùy ý.
        \end{block}
    \end{columns}
\end{frame}

\begin{frame}[fragile]{Syntax: Return Values}
    Thay \texttt{void} bằng kiểu dữ liệu (\texttt{int}, \texttt{string}...) và dùng lệnh \texttt{return}.

\begin{lstlisting}
int tinhTong(int x, int y) {
  return x + y; // Tra ve ket qua, khong in ra
}

int main() {
  // Cat ket qua vao bien
  int ketQua = tinhTong(5, 3); 
  cout << "Tong: " << ketQua;
  
  // Hoac dung truc tiep trong phep tinh khac
  cout << "Tong + 10: " << (tinhTong(5, 3) + 10);
  return 0;
}
\end{lstlisting}
\end{frame}

\begin{frame}[fragile]{Kiểm tra sự hiểu biết}
    Điền vào chỗ trống để tính tổng diện tích 2 hình vuông:
\begin{lstlisting}
int dienTichVuong(int canh) { return canh * canh; }

int main() {
  int hinhA = dienTichVuong(5); // 25
  int hinhB = dienTichVuong(3); // 9
  
  int tongDienTich = ...? 
  cout << tongDienTich;
}
\end{lstlisting}

    \pause
    \begin{alertblock}{Đáp án}
        Code: \texttt{int tongDienTich = hinhA + hinhB;} \\
        Kết quả in ra: \textbf{34}
    \end{alertblock}
\end{frame}

% --- CHƯƠNG 5 ---
\section{Chương 5: Pass by Reference}

\begin{frame}{Chương 5: Pass by Reference (Truyền tham chiếu)}
    Đây là phần quan trọng nhất!
    
    \begin{itemize}
        \item \textbf{Pass by Value (Mặc định):} Gửi file đính kèm email. Bạn sửa bản copy, bản gốc máy tôi vẫn y nguyên.
        \item \textbf{Pass by Reference (Tham chiếu):} Gửi \textbf{link Google Docs}. Bạn sửa trên link, bản gốc của tôi thay đổi ngay lập tức.
    \end{itemize}
    
    \textbf{Dấu hiệu nhận biết:} Ký tự \texttt{\&} khi khai báo tham số.
\end{frame}

\begin{frame}[fragile]{Syntax: Pass by Reference (\&)}
    Ví dụ kinh điển: Hoán đổi (Swap).

\begin{lstlisting}
// int &x: "Toi muon lay duong link toi bien x"
void hoanDoi(int &x, int &y) {
  int tam = x;
  x = y;
  y = tam;
}

int main() {
  int a = 10, b = 20;
  hoanDoi(a, b);
  cout << a << " - " << b; // In ra: 20 - 10
}
\end{lstlisting}
    \vspace{0.2cm}
    \textit{Nếu không có dấu \&, a và b vẫn sẽ là 10 - 20.}
\end{frame}

\begin{frame}[fragile]{Cạm bẫy (Kiểm tra sự hiểu biết)}
    Hãy cẩn thận với đoạn code sau!
\begin{lstlisting}
void tangGiaTri(int &a, int b) { // Chi a co dau &, b thi khong
  a = a + 1;
  b = b + 1;
}

int main() {
  int x = 5;
  int y = 5;
  tangGiaTri(x, y);
  cout << "x=" << x << ", y=" << y;
}
\end{lstlisting}

    \pause
    \begin{alertblock}{Đáp án}
        \textbf{x = 6} (Do có \texttt{\&} nên bản gốc bị sửa). \\
        \textbf{y = 5} (Do không có \texttt{\&}, chỉ là bản copy bị sửa, bản gốc y nguyên).
    \end{alertblock}
\end{frame}

% --- CHƯƠNG 6 ---
\section{Chương 6: Passing Arrays}

\begin{frame}{Chương 6: Passing Arrays (Truyền mảng)}
    \begin{block}{Đặc biệt}
        Mảng (Array) \textbf{MẶC ĐỊNH} luôn luôn là tham chiếu (Google Docs).
        Không cần dấu \texttt{\&}.
    \end{block}
    
    Khi đưa một mảng vào hàm, mọi thay đổi bên trong hàm đều tác động trực tiếp lên mảng gốc.
\end{frame}

\begin{frame}[fragile]{Syntax: Passing Arrays}
\begin{lstlisting}
// mangSo[]: Nhan vao mot mảng
void inMang(int mangSo[], int kichThuoc) {
  for (int i = 0; i < kichThuoc; i++) {
    // Thu sua doi gia tri truc tiep
    mangSo[i] = mangSo[i] * 2; 
    cout << mangSo[i] << " ";
  }
}

int main() {
  int soYeuThich[3] = {1, 2, 3};
  inMang(soYeuThich, 3); // In ra: 2 4 6
  
  // Mang goc da bi thay doi vinh vien!
  cout << "\nSo dau tien: " << soYeuThich[0]; // In ra: 2
}
\end{lstlisting}
\end{frame}

% --- TỔNG KẾT ---
\section{Tổng kết & Thử thách}

\begin{frame}{Tổng kết hành trình}
    \begin{enumerate}
        \item \textbf{Parameters vs Arguments:} Cái khuôn và nguyên liệu.
        \item \textbf{Default Parameters:} Chế độ cho người "lười".
        \item \textbf{Multiple Parameters:} Đúng thứ tự là chân ái.
        \item \textbf{Return Values:} Shipper giao hàng.
        \item \textbf{Pass by Reference (\&):} Sửa bản gốc (Google Docs).
        \item \textbf{Passing Arrays:} Luôn sửa bản gốc (Mặc định).
    \end{enumerate}
\end{frame}

\begin{frame}[fragile]{Thử thách Final Boss}
    \textbf{Đề bài:} Viết hàm \texttt{doiChoVaNhanDoi}:
    \begin{itemize}
        \item Nhận vào 2 số nguyên.
        \item Hoán đổi vị trí của chúng.
        \item SAU ĐÓ nhân đôi giá trị cả hai.
        \item Ví dụ: Vào \texttt{a=1, b=2} $\rightarrow$ Ra \texttt{a=4, b=2}.
    \end{itemize}

    \pause
    \begin{exampleblock}{Lời giải (Model Code)}
\begin{lstlisting}
void doiChoVaNhanDoi(int &a, int &b) {
  // 1. Hoan doi (Swap)
  int temp = a;
  a = b;
  b = temp;
  
  // 2. Nhan doi (Double)
  a *= 2; // a = a * 2
  b *= 2; // b = b * 2
}
\end{lstlisting}
    \end{exampleblock}
\end{frame}

\begin{frame}{Bước tiếp theo}
    Bạn đã hoàn thành xuất sắc chủ đề Function Parameters!
    
    \vspace{1cm}
    
    \begin{exampleblock}{Next Step}
        Chủ đề tiếp theo: \textbf{Function Overloading} (Nạp chồng hàm) - Nghệ thuật dùng 1 cái tên cho nhiều hàm khác nhau.
    \end{exampleblock}
    
    \vspace{0.5cm}
    \centering \textit{Chúc bạn học tốt!}
\end{frame}

\end{document}