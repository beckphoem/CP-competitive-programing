\documentclass{beamer}
\usepackage[utf8]{inputenc}
\usepackage[vietnamese]{babel}
\usepackage{tcolorbox}
\usepackage{listings}
\usepackage{xcolor}

\usetheme{Madrid}
\usecolortheme{default}

% Colors for code blocks
\definecolor{codegreen}{rgb}{0,0.6,0}
\definecolor{codegray}{rgb}{0.5,0.5,0.5}
\definecolor{codepurple}{rgb}{0.58,0,0.82}
\definecolor{backcolour}{rgb}{0.95,0.95,0.92}

\lstdefinestyle{mystyle}{
    backgroundcolor=\color{backcolour},   
    commentstyle=\color{codegreen},
    keywordstyle=\color{magenta},
    numberstyle=\tiny\color{codegray},
    stringstyle=\color{codepurple},
    basicstyle=\ttfamily\footnotesize,
    breakatwhitespace=false,         
    breaklines=true,                 
    captionpos=b,                    
    keepspaces=true,                 
    numbers=left,                    
    numbersep=5pt,                  
    showspaces=false,                
    showstringspaces=false,
    showtabs=false,                  
    tabsize=2,
    escapechar=@
}

\lstset{style=mystyle}

\title[Nhập Xuất Cơ Bản]{Bài 1: Lời Chào Đầu Tiên \\ Giao Tiếp Với "Người Bạn" Máy Tính}
\subtitle{C++ Competitive Programming Series}
\author{Học Cùng C++}
\date{}

\begin{document}

\begin{frame}
  \titlepage
\end{frame}

\begin{frame}{Chào Mừng Bạn!}
  \begin{itemize}
    \item \textbf{Máy tính là ai?}
    \begin{itemize}
        \item Một "Người Bạn Nhỏ" siêu tốc độ.
        \item Cực kỳ ngây thơ và cần sự chỉ dẫn chính xác.
    \end{itemize}
    \item \textbf{Mục tiêu hôm nay:} Học câu thần chú đầu tiên để "người bạn" này cất tiếng nói.
  \end{itemize}
  
  \begin{alertblock}{Tưởng tượng}
  Viết code giống như viết một \textbf{công thức nấu ăn} hoặc một \textbf{kịch bản phim} cho máy tính thực hiện.
  \end{alertblock}
\end{frame}

\begin{frame}[fragile]{Bức Tranh Toàn Cảnh}
  Đây là một chương trình C++ hoàn chỉnh:
  
  \begin{lstlisting}[language=C++]
#include<bits/stdc++.h>
using namespace std;

int main(){
    cout << "Hello World!" << endl;
}
  \end{lstlisting}
  
  Chúng ta sẽ cùng "mổ xẻ" từng bộ phận ngay sau đây!
\end{frame}

% Section 1: Include
\begin{frame}[fragile]{1. Hộp Dụng Cụ Vạn Năng (\texttt{\#include})}
  \begin{lstlisting}[language=C++]
#include<bits/stdc++.h>
  \end{lstlisting}

  \begin{block}{Ẩn dụ: Hộp Dụng Cụ}
    Khi sửa xe hay nấu ăn, bạn cần mang đồ nghề ra trước tiên.
  \end{block}

  \begin{itemize}
    \item \texttt{\#include}: Lệnh bảo máy tính "Mang hộp dụng cụ ra đây!"
    \item \texttt{bits/stdc++.h}: Chiếc "Siêu Hộp Dụng Cụ" chứa mọi thứ (kìm, búa, gia vị...) cần thiết cho lập trình thi đấu.
  \end{itemize}
  
  \textit{$\rightarrow$ Không có nó, máy tính sẽ không biết làm gì cả!}
\end{frame}

% Section 2: Namespace
\begin{frame}[fragile]{2. Tên Thân Mật (\texttt{using namespace std;})}
  \begin{lstlisting}[language=C++]
using namespace std;
  \end{lstlisting}

  \begin{block}{Ẩn dụ: Gọi Nickname}
    Thay vì gọi tên đầy đủ dài dòng: "Nguyễn Trần Lê Văn Tèo", chúng ta chỉ gọi là "Tèo".
  \end{block}

  \begin{itemize}
    \item Các lệnh chuẩn thường có họ là \texttt{std::} (ví dụ \texttt{std::cout}).
    \item Dòng này giúp bỏ qua chữ \texttt{std::} mỗi khi dùng lệnh.
    \item Giúp code gọn gàng và đỡ "mỏi miệng".
  \end{itemize}
\end{frame}

% Section 3: Main
\begin{frame}[fragile]{3. Nút Start (\texttt{main})}
  \begin{lstlisting}[language=C++]
int main(){
    // @Mọi thứ diễn ra ở đây@
}
  \end{lstlisting}

  \begin{block}{Ẩn dụ: Cánh Cửa Chính / Nút Start}
    Đây là nơi bắt đầu mọi hành động.
  \end{block}

  \begin{itemize}
    \item Máy tính luôn tìm hàm \texttt{main} để bắt đầu chạy.
    \item \texttt{\{ \}}: Vạch xuất phát (\texttt{\{}) và vạch đích (\texttt{\}}).
    \item Robot máy tính chỉ chạy trong khu vực này.
  \end{itemize}
\end{frame}

% Section 4: Cout
\begin{frame}[fragile]{4. Chiếc Loa Phát Thanh (\texttt{cout})}
  \begin{lstlisting}[language=C++]
cout << "Hello World!" << endl;
  \end{lstlisting}

  \begin{description}
    \item[\texttt{cout} (xi-ao):] Cái \textbf{LOA} hoặc \textbf{MIỆNG} của máy tính. (Computer OUT).
    \item[\texttt{<<}:] Cái \textbf{PHỄU}, hứng chữ đổ vào miệng máy tính.
    \item[\texttt{"..."}:] Lời thoại nguyên văn. Máy tính in y chang những gì trong ngoặc kép.
    \item[\texttt{endl}:] Phím \textbf{ENTER} (xuống dòng).
  \end{description}
\end{frame}

\begin{frame}[fragile]{Tổng Kết}
  Hãy cùng ghép lại bức tranh:

  \begin{enumerate}
    \item \textbf{\texttt{\#include}}: Lấy \textbf{Hộp đồ nghề} ra.
    \item \textbf{\texttt{using namespace}}: Gọi \textbf{Tên thân mật}.
    \item \textbf{\texttt{main()}}: Bấm \textbf{Nút Start}, vào vùng làm việc \texttt{\{ \}}.
    \item \textbf{\texttt{cout <<}}: Dùng \textbf{Loa} nói lời thoại ra màn hình.
  \end{enumerate}

  \begin{center}
    \Large \textbf{Bạn đã sẵn sàng để viết dòng code đầu tiên chưa?}
  \end{center}
\end{frame}

\end{document}
