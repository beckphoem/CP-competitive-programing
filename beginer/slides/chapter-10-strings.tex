\documentclass{beamer}
\usepackage[utf8]{inputenc}
\usepackage[T5]{fontenc} % Bắt buộc để hiển thị tiếng Việt
\usepackage[vietnamese]{babel}
\usepackage{tcolorbox}
\usepackage{listings}
\usepackage{xcolor}
\usepackage{booktabs}
\usetheme{Madrid}

\definecolor{codegreen}{rgb}{0,0.6,0}
\definecolor{codegray}{rgb}{0.5,0.5,0.5}
\definecolor{codepurple}{rgb}{0.58,0,0.82}
\definecolor{backcolour}{rgb}{0.95,0.95,0.92}

\lstdefinestyle{mystyle}{
    backgroundcolor=\color{backcolour},   
    commentstyle=\color{codegreen},
    keywordstyle=\color{magenta},
    numberstyle=\tiny\color{codegray},
    stringstyle=\color{codepurple},
    basicstyle=\ttfamily\scriptsize,
    breakatwhitespace=false,         
    breaklines=true,                 
    captionpos=b,                    
    keepspaces=true,                 
    numbers=left,                    
    numbersep=4pt,                  
    showspaces=false,                
    showstringspaces=false,
    showtabs=false,                  
    tabsize=2,
    escapechar=@
}

\lstset{style=mystyle}

% Thông tin bài giảng
\title[C++ Strings]{C++ Strings: Phép thuật của Ngôn ngữ}
\subtitle{Hành trình qua 5 Trạm dừng chân}
\author{Slide Learning C++}
\date{\today}

\begin{document}

% Slide Tiêu đề
\begin{frame}
    \titlepage
\end{frame}

% Slide Giới thiệu
\begin{frame}{Lời chào từ thế giới Lập trình}
    \begin{block}{String là gì?}
        Nếu máy tính chỉ biết đến những con số 0 và 1 khô khan (`int`, `float`), thì \textbf{String} (Chuỗi ký tự) chính là "phép thuật" giúp máy tính hiểu và nói được ngôn ngữ con người.
    \end{block}

    \begin{itemize}
        \item Hãy tưởng tượng \texttt{string} giống như một \textbf{xâu chuỗi hạt}.
        \item Mỗi chữ cái (`char`) là một hạt cườm lấp lánh trên đó.
    \end{itemize}
\end{frame}

% Slide Lộ trình
\begin{frame}{Bản đồ kho báu String (Lộ trình học tập)}
    Chúng ta sẽ đi qua 5 trạm dừng chân thú vị:
    \vspace{0.5cm}
    \begin{enumerate}
        \item \textbf{Trạm 1:} Tạo tác Xâu Chuỗi (C++ Strings Basic)
        \item \textbf{Trạm 2:} Phép thuật Nối Dây (Concatenation)
        \item \textbf{Trạm 3:} Thước đo thần kỳ (Length)
        \item \textbf{Trạm 4:} Kỹ thuật "Gắp Hạt" \& "Đổi Màu" (Access \& Change)
        \item \textbf{Trạm 5:} Chiếc Hộp Thần Kỳ (User Input)
    \end{enumerate}
\end{frame}

% --- TRẠM 1 ---
\section{Trạm 1: Basic}

\begin{frame}{Trạm 1: Tạo tác Xâu Chuỗi}
    \begin{block}{Bức tranh toàn cảnh: Hộp Đựng Chữ}
        Bộ nhớ máy tính giống như một cái tủ đựng đồ:
        \begin{itemize}
            \item \texttt{int}: Ngăn đựng số (5, 10, 100).
            \item \texttt{string}: Ngăn đựng văn bản (Tên, câu chuyện, lời bài hát).
        \end{itemize}
    \end{block}

    \begin{alertblock}{Hộp công cụ bí mật}
        Để dùng string, bắt buộc phải khai báo thư viện:
        \texttt{\#include <string>}
    \end{alertblock}
\end{frame}

\begin{frame}[fragile]{Cú pháp \& Ví dụ}
    \begin{exampleblock}{Quy tắc vàng}
        Giá trị của chuỗi phải luôn nằm trong \textbf{dấu ngoặc kép} (\texttt{" "}).
        \begin{itemize}
            \item Đúng: \texttt{"Xin chao"}
            \item Sai: \texttt{'Xin chao'} (Dấu nháy đơn chỉ dành cho \texttt{char}).
        \end{itemize}
    \end{exampleblock}

    \begin{lstlisting}[language=C++]
#include <iostream>
#include <string> // Hop cong cu can thiet
using namespace std;

int main() {
  // Tao mot bien string
  string loiChao = "Xin chao! To la lap trinh vien.";
  
  // In chuoi ra man hinh
  cout << loiChao;
  return 0;
}
    \end{lstlisting}
\end{frame}

\begin{frame}[fragile]{Kiểm tra nhanh (Check-up 1)}
    \textbf{Câu hỏi:} Nếu viết code thế này thì máy tính có hiểu không? Tại sao?

    \begin{lstlisting}[language=C++]
string tenCuaBan = 'Nam'; 
    \end{lstlisting}

    \pause
    \vspace{0.5cm}
    \begin{alertblock}{Đáp án: Báo lỗi!}
        Máy tính sẽ "dỗi" ngay lập tức.
        \begin{itemize}
            \item Dấu nháy đơn \texttt{' '} chỉ dành cho một hạt cườm (\texttt{char}).
            \item Dấu nháy kép \texttt{" "} mới dành cho cả xâu chuỗi (\texttt{string}).
        \end{itemize}
    \end{alertblock}
\end{frame}

% --- TRẠM 2 ---
\section{Trạm 2: Concatenation}

\begin{frame}{Trạm 2: Phép thuật Nối Dây}
    Hãy tưởng tượng mỗi biến \texttt{string} là một \textbf{toa tàu}.
    
    \begin{block}{Dấu Cộng (+) thần thánh}
        Trong thế giới String, dấu \texttt{+} không phải là cộng giá trị toán học, mà là \textbf{dán dính} các toa tàu lại.
    \end{block}
    
    Ví dụ:
    \texttt{"Nguyen"} + \texttt{"An"} = \texttt{"NguyenAn"}
    
    \vspace{0.3cm}
    \textbf{Lưu ý:} Máy tính không tự thêm dấu cách. Ta phải tự chèn một "toa tàu rỗng" vào giữa.
\end{frame}

\begin{frame}[fragile]{Ví dụ Nối chuỗi}
    \begin{lstlisting}[language=C++]
#include <iostream>
#include <string>
using namespace std;

int main() {
  string ho = "Nguyen";
  string ten = "An";
  
  // Cach 1: Dinh chum (Xau)
  // string hoVaTen = ho + ten; -> "NguyenAn"
  
  // Cach 2: Co them dau cach " " (Dep)
  string hoVaTen = ho + " " + ten;
  
  cout << hoVaTen; // In ra: Nguyen An
  return 0;
}
    \end{lstlisting}
    
    \begin{block}{Góc chuyên gia}
        Có thể dùng hàm \texttt{append()} để nối đuôi, nhưng dùng \texttt{+} sẽ tự nhiên hơn.
    \end{block}
\end{frame}

\begin{frame}[fragile]{Kiểm tra nhanh (Check-up 2)}
    \textbf{Câu hỏi:} Kết quả của đoạn code sau là gì?
    
    \begin{lstlisting}[language=C++]
string x = "10";
string y = "20";
string z = x + y;
cout << z;
    \end{lstlisting}

    A. 30 \\
    B. 1020 \\
    C. Báo lỗi

    \pause
    \vspace{0.3cm}
    \begin{exampleblock}{Đáp án: B (1020)}
        Vì \texttt{x} và \texttt{y} nằm trong dấu ngoặc kép, máy tính coi chúng là \textbf{văn bản}, không phải con số. Nó sẽ dán số "10" bên cạnh số "20".
    \end{exampleblock}
\end{frame}

% --- TRẠM 3 ---
\section{Trạm 3: Length}

\begin{frame}[fragile]{Trạm 3: Thước đo thần kỳ}
    Để đếm số lượng ký tự trong chuỗi (ví dụ kiểm tra độ dài mật khẩu), ta dùng hàm \texttt{.length()}.
    
    \begin{lstlisting}[language=C++]
string bangChuCai = "ABCDEFGHIJKLMNOPQRSTUVWXYZ";
cout << bangChuCai.length(); 
// Ket qua: 26
    \end{lstlisting}
    
    \begin{alertblock}{Bí mật của "Khoảng trắng"}
        Máy tính đếm tất cả mọi thứ trong dấu ngoặc kép, bao gồm cả \textbf{dấu cách (space)}.
        
        Ví dụ: \texttt{"Xin chao"} có độ dài là \textbf{8} (chứ không phải 7).
    \end{alertblock}
    
    \textit{Lưu ý: \texttt{.size()} và \texttt{.length()} trong C++ String tương đương nhau.}
\end{frame}

\begin{frame}[fragile]{Kiểm tra nhanh (Check-up 3)}
    \textbf{Câu hỏi:} Đoạn code sau in ra số mấy?
    
    \begin{lstlisting}[language=C++]
string biMat = "A B C";
cout << biMat.length();
    \end{lstlisting}

    \pause
    \vspace{0.5cm}
    \begin{exampleblock}{Đáp án: 5}
        Hãy đếm kỹ:
        \begin{enumerate}
            \item A
            \item (dấu cách)
            \item B
            \item (dấu cách)
            \item C
        \end{enumerate}
        Tổng cộng là 5 ký tự.
    \end{exampleblock}
\end{frame}

% --- TRẠM 4 ---
\section{Trạm 4: Access \& Change}

\begin{frame}{Trạm 4: Kỹ thuật "Gắp Hạt" \& "Đổi Màu"}
    Biến \texttt{string} giống như dãy tủ khóa. Để lấy đồ, bạn cần biết số thứ tự (index).
    
    \begin{alertblock}{Quy tắc Số 0}
        Máy tính bắt đầu đếm từ số \textbf{0}, không phải số 1.
    \end{alertblock}
    
    Ví dụ chuỗi \textbf{"Hello"}:
    
    \begin{table}[]
        \centering
        \begin{tabular}{|c|c|c|c|c|}
        \hline
        H & e & l & l & o \\ \hline
        \textbf{0} & \textbf{1} & \textbf{2} & \textbf{3} & \textbf{4} \\ \hline
        \end{tabular}
    \end{table}
\end{frame}

\begin{frame}[fragile]{Thay đổi ký tự}
    Bạn có thể thay đổi nội dung từng "ngăn tủ".
    
    \begin{lstlisting}[language=C++]
string monAn = "Hello";

// 1. Truy cap (Lay ra)
cout << monAn[0]; // In ra: H

// 2. Thay doi (Doi mau)
monAn[0] = 'J'; 

cout << monAn; // In ra: Jello
    \end{lstlisting}
    
    \begin{alertblock}{Cảnh báo quan trọng}
        Khi làm việc với từng ký tự đơn lẻ (\texttt{monAn[0]}), bắt buộc dùng \textbf{dấu nháy đơn} \texttt{' '}.
        \begin{itemize}
            \item Đúng: \texttt{monAn[0] = 'J';}
            \item Sai: \texttt{monAn[0] = "J";}
        \end{itemize}
    \end{alertblock}
\end{frame}

\begin{frame}[fragile]{Kiểm tra nhanh (Check-up 4)}
    \textbf{Câu hỏi:} Chữ cái nào sẽ hiện lên màn hình?
    
    \begin{lstlisting}[language=C++]
string ten = "VIETNAM";
cout << ten[4];
    \end{lstlisting}

    A. T \hspace{1cm} B. N \hspace{1cm} C. A

    \pause
    \vspace{0.5cm}
    \begin{exampleblock}{Đáp án: B (Chữ N)}
        Đếm từ 0:
        V(0) - I(1) - E(2) - T(3) - \textbf{N(4)}.
    \end{exampleblock}
\end{frame}

% --- TRẠM 5 ---
\section{Trạm 5: User Input}

\begin{frame}[fragile]{Trạm 5: Chiếc Hộp Thần Kỳ}
    \begin{block}{Vấn đề của \texttt{cin}}
        \texttt{cin} giống như người đưa thư lười biếng. Nó dừng lại ngay khi gặp dấu cách.
        Nếu nhập "Tran Hung", \texttt{cin} chỉ lấy được "Tran".
    \end{block}
    
    \begin{exampleblock}{Giải pháp: \texttt{getline()}}
        Dùng "máy hút bụi" \texttt{getline} để lấy cả dòng.
        Cú pháp: \texttt{getline(cin, TenBien);}
    \end{exampleblock}

    \begin{lstlisting}[language=C++]
string hoVaTen;
cout << "Nhap ten day du: ";
// cin >> hoVaTen; -> Loi neu co dau cach
getline(cin, hoVaTen); // -> Lay tron ven ca dong
cout << "Xin chao: " << hoVaTen;
    \end{lstlisting}
\end{frame}

% --- TỔNG KẾT & BÀI TẬP ---
\section{Tổng kết}

\begin{frame}{Tổng kết hành trình}
    Chúng ta đã thu được các chiến lợi phẩm:
    \begin{enumerate}
        \item \textbf{Khởi tạo:} \texttt{\#include <string>} và dùng \texttt{" "}.
        \item \textbf{Nối chuỗi:} Dùng dấu \texttt{+} để dính các toa tàu.
        \item \textbf{Độ dài:} Dùng \texttt{.length()} (đếm cả dấu cách).
        \item \textbf{Truy cập:} Dùng \texttt{[ ]} và đếm từ \textbf{0}.
        \item \textbf{Nhập liệu:} Dùng \texttt{getline(cin, bien)} để nhập cả câu.
    \end{enumerate}
\end{frame}

\begin{frame}{Bài tập về nhà (Challenge)}
    \begin{block}{Đề bài}
        Viết chương trình thực hiện 3 việc:
        \begin{enumerate}
            \item Nhập vào \textbf{tên đầy đủ} (Ví dụ: "Le Loi").
            \item In ra \textbf{chữ cái đầu tiên}.
            \item In ra \textbf{độ dài} của tên.
        \end{enumerate}
    \end{block}
\end{frame}

\begin{frame}[fragile]{Lời giải tham khảo}
    Đây là đoạn code giải quyết thử thách trên:

    \begin{lstlisting}[language=C++]
#include <iostream>
#include <string>
using namespace std;

int main() {
    string name;
    
    cout << "Nhap ten cua ban: ";
    // Dung getline de nhap ten co dau cach
    getline(cin, name);
    
    cout << "Chu cai dau tien: " << name[0] << endl;
    
    // Dung .size() hoac .length() deu duoc
    cout << "Do dai ten ban: " << name.size() << endl;

    return 0;
}
    \end{lstlisting}
\end{frame}

\begin{frame}{Thử thách nâng cao (Bonus Stage)}
    \begin{exampleblock}{Đề bài khó}
        Viết chương trình nhập tên, in ra chữ cái \textbf{CUỐI CÙNG} trong tên.
    \end{exampleblock}

    \textbf{Gợi ý tư duy:}
    \begin{itemize}
        \item Tên "NAM" (độ dài 3), chữ cuối 'M' ở vị trí số 2.
        \item Tên "TUAN" (độ dài 4), chữ cuối 'N' ở vị trí số 3.
        \item Quy luật: \texttt{Vị trí cuối = Độ dài - 1}
    \end{itemize}
    
    \vspace{0.5cm}
    \centering
    \textit{Chúc các bạn thành công trên con đường lập trình C++!}
\end{frame}

\end{document}