\documentclass{beamer}
\usepackage[utf8]{inputenc}
\usepackage[T5]{fontenc} % Bat buoc de hien thi tieng Viet
\usepackage[vietnamese]{babel}
\usepackage{tcolorbox}
\usepackage{listings}
\usepackage{xcolor}
\usepackage{booktabs}
\usetheme{Madrid}

\definecolor{codegreen}{rgb}{0,0.6,0}
\definecolor{codegray}{rgb}{0.5,0.5,0.5}
\definecolor{codepurple}{rgb}{0.58,0,0.82}
\definecolor{backcolour}{rgb}{0.95,0.95,0.92}

\lstdefinestyle{mystyle}{
    backgroundcolor=\color{backcolour},   
    commentstyle=\color{codegreen},
    keywordstyle=\color{magenta},
    numberstyle=\tiny\color{codegray},
    stringstyle=\color{codepurple},
    basicstyle=\ttfamily\scriptsize,
    breakatwhitespace=false,         
    breaklines=true,                 
    captionpos=b,                    
    keepspaces=true,                 
    numbers=left,                    
    numbersep=4pt,                  
    showspaces=false,                
    showstringspaces=false,
    showtabs=false,                  
    tabsize=2,
    escapechar=@
}

\lstset{style=mystyle}

\title{Khám phá std::map trong C++}
\author{Đọc Sách - Người đồng hành tri thức}
\date{2026}

\begin{document}

\frame{\titlepage}

\begin{frame}{Lộ trình Khám phá "Chiếc túi Thần kỳ" Map}
    \begin{enumerate}
        \item \textbf{Chương 1:} Chiếc hộp vạn năng là gì? (Key và Value)
        \item \textbf{Chương 2:} Cách tạo ra chiếc hộp (Khai báo và Insertion)
        \item \textbf{Chương 3:} Tìm đồ trong bóng tối (Truy xuất C++17/20)
        \item \textbf{Chương 4:} Kiểm kê kho hàng (Structured Bindings)
        \item \textbf{Chương 5:} Sức mạnh tiềm ẩn (Tại sao Map lại nhanh?)
    \end{enumerate}
\end{frame}

\begin{frame}{Chương 1: Chiếc hộp vạn năng là gì?}
    \begin{block}{Sự khác biệt giữa Vector và Map}
        \begin{itemize}
            \item \textbf{Vector/Array:} Ngăn tủ đánh số $0, 1, 2...$ Phải nhớ số thứ tự.
            \item \textbf{Map:} Dán nhãn trực tiếp lên cửa tủ. Gọi tên nhãn, cửa tự mở.
        \end{itemize}
    \end{block}

    \begin{exampleblock}{Thành phần cốt lõi}
        \begin{itemize}
            \item \textbf{Key (Khóa):} Nhãn dán duy nhất (ví dụ: "TEO").
            \item \textbf{Value (Giá trị):} Món đồ bên trong (ví dụ: điểm 10).
        \end{itemize}
    \end{exampleblock}
    \textit{Ẩn dụ: Map giống như một cuốn từ điển. Từ vựng là Key, nghĩa của từ là Value.}
\end{frame}

\begin{frame}[fragile]{Chương 2: Cách tạo ra chiếc hộp}
    \begin{block}{Khai báo}
        \texttt{map<Kiểu\_Key, Kiểu\_Value> tên\_biến;}
    \end{block}

    \begin{lstlisting}[language=C++]
#include <iostream>
#include <map>
#include <string>
using namespace std;

int main() {
    map<string, long long> danh_ba;

    // Cach 1: Gan truc tiep
    danh_ba["Nguyen Van A"] = 9876543210;

    // Cach 2: Dung insert (C++17)
    danh_ba.insert({"Le Van C", 555666777});

    // Cach 3: insert_or_assign (C++20)
    danh_ba.insert_or_assign("Nguyen Van A", 111222333);
    return 0;
}
    \end{lstlisting}
\end{frame}

\begin{frame}{Cơ chế tự sắp xếp của Map}
    \begin{block}{Quản gia ngăn nắp}
        Mỗi khi bạn thêm một phần tử, \texttt{std::map} tự động sắp xếp các ngăn tủ theo \textbf{thứ tự từ điển (A-Z)} của các Key.
    \end{block}

    \begin{exampleblock}{Câu hỏi tương tác}
        Giả sử tôi có đoạn code sau:
        \texttt{kho\_banh["Banh Mi"] = 10;}
        \texttt{kho\_banh["Banh Mi"] = 5;}
        Theo bạn, trong \texttt{kho\_banh} có bao nhiêu ngăn tủ tên là "Banh Mi"?
        \pause
        \textbf{Đáp án:} Chỉ có 1 ngăn duy nhất, giá trị mới (5) sẽ đè lên giá trị cũ (10).
    \end{exampleblock}
\end{frame}

\begin{frame}[fragile]{Chương 3: Tìm đồ trong bóng tối}
    \begin{block}{Kiểm tra tồn tại (C++20)}
        \begin{lstlisting}[language=C++]
if (kho_banh.contains("Banh Mi")) {
    cout << "Co banh mi!";
}
        \end{lstlisting}
    \end{block}

    \begin{block}{Tìm và lấy dữ liệu (C++17)}
        \begin{lstlisting}[language=C++]
if (auto it = kho_banh.find("Banh Mi"); it != kho_banh.end()) {
    cout << "So luong: " << it->second;
}
        \end{lstlisting}
    \end{block}

    \begin{alertblock}{Cảnh báo nguy hiểm}
        Đừng dùng \texttt{[]} để kiểm tra sự tồn tại. Nó sẽ tự động tạo một ngăn tủ trống nếu Key chưa có!
    \end{alertblock}
\end{frame}

\begin{frame}[fragile]{Chương 4: Kiểm kê kho hàng}
    \begin{block}{Structured Bindings (C++17)}
        Sử dụng "chiếc kéo thần kỳ" để tách đôi Key và Value ngay trong vòng lặp.
    \end{block}

    \begin{lstlisting}[language=C++]
map<string, int> tui_do = {{"Kiem", 1}, {"Mau", 10}};

// Dung & de chay nhanh, const de bao ve du lieu
for (auto const& [ten, so_luong] : tui_do) {
    cout << "Ten: " << ten << " | SL: " << so_luong << endl;
}
    \end{lstlisting}

    \begin{exampleblock}{Lợi ích}
        \begin{itemize}
            \item Trực quan: Nhìn thấy \texttt{ten} và \texttt{so\_luong} thay vì \texttt{first}, \texttt{second}.
            \item Ngăn nắp: Tự động in ra theo thứ tự A-Z của Key.
        \end{itemize}
    \end{exampleblock}
\end{frame}

\begin{frame}[fragile]{Chương 4.5: Iterator - Chiếc gậy chỉ đường}
    \begin{block}{Duyệt thủ công bằng Iterator}
        \begin{lstlisting}[language=C++]
map<string, int>::iterator it;
for (it = tui_do.begin(); it != tui_do.end(); ++it) {
    cout << it->first << " : " << it->second << endl;
}
        \end{lstlisting}
    \end{block}

    \begin{exampleblock}{Sức mạnh của auto}
        Thay vì viết \texttt{map<string, int>::iterator}, hãy dùng \texttt{auto}:
        \begin{lstlisting}[language=C++]
for (auto it = tui_do.begin(); it != tui_do.end(); ++it) { ... }
        \end{lstlisting}
    \end{exampleblock}
\end{frame}

\begin{frame}{Chương 5: Sức mạnh tiềm ẩn}
    \begin{block}{Tại sao Map lại nhanh?}
        Map không tìm kiếm tuần tự. Nó sử dụng cấu trúc \textbf{Cây tìm kiếm nhị phân (BST)}. 
    \end{block}
    
    \begin{itemize}
        \item \textbf{Độ phức tạp:} $O(\log n)$.
        \item \textbf{Ví dụ:} Với 1 triệu món đồ, Map chỉ cần tối đa khoảng 20 bước thử để tìm ra kết quả.
    \end{itemize}

    \begin{exampleblock}{Khi nào nên dùng Map?}
        \begin{itemize}
            \item Tra cứu dữ liệu (Từ điển, danh bạ).
            \item Đếm tần suất xuất hiện.
            \item Khi Key không phải là số nguyên (string, char, struct...).
        \end{itemize}
    \end{exampleblock}
\end{frame}

\begin{frame}[fragile]{Trận chiến cuối cùng: Quản lý Kho Vũ Khí}
    \begin{exampleblock}{Thử thách cho bạn}
        Viết chương trình:
        \begin{enumerate}
            \item Tạo \texttt{map<string, int> vukhi}.
            \item Thêm "Kiem" (50), "Cung" (40).
            \item Dùng \texttt{auto \&} để giảm sát thương tất cả đi 10.
            \item Kiểm tra "Riu" bằng \texttt{contains}.
        \end{enumerate}
    \end{exampleblock}

    \pause
    \begin{lstlisting}[language=C++]
for (auto & [ten, sat_thuong] : vukhi) {
    sat_thuong -= 10; // Su dung & de sua du lieu that
}

if (!vukhi.contains("Riu")) {
    vukhi["Riu"] = 60;
}
    \end{lstlisting}
\end{frame}

\begin{frame}{Tổng kết công thức "Vàng"}
    \begin{block}{Ghi nhớ cụm từ: "Auto - Xem - Thật"}
        \texttt{for (auto const \& [k, v] : my\_map)}
    \end{block}
    \begin{itemize}
        \item \textbf{Auto:} Tự nhận diện kiểu dữ liệu.
        \item \textbf{Xem (const):} Bảo vệ dữ liệu không bị sửa nhầm.
        \item \textbf{Thật (\&):} Truy cập trực tiếp, không tốn công copy (Tăng tốc độ).
    \end{itemize}
\end{frame}

\end{document}