\documentclass{beamer}
\usepackage[utf8]{inputenc}
\usepackage[T5]{fontenc} % Bắt buộc để hiển thị tiếng Việt
\usepackage[vietnamese]{babel}
\usepackage{tcolorbox}
\usepackage{listings}
\usepackage{xcolor}
\usepackage{booktabs}
\usetheme{Madrid}

\definecolor{codegreen}{rgb}{0,0.6,0}
\definecolor{codegray}{rgb}{0.5,0.5,0.5}
\definecolor{codepurple}{rgb}{0.58,0,0.82}
\definecolor{backcolour}{rgb}{0.95,0.95,0.92}

\lstdefinestyle{mystyle}{
    backgroundcolor=\color{backcolour},   
    commentstyle=\color{codegreen},
    keywordstyle=\color{magenta},
    numberstyle=\tiny\color{codegray},
    stringstyle=\color{codepurple},
    basicstyle=\ttfamily\scriptsize,
    breakatwhitespace=false,         
    breaklines=true,                 
    captionpos=b,                    
    keepspaces=true,                 
    numbers=left,                    
    numbersep=4pt,                  
    showspaces=false,                
    showstringspaces=false,
    showtabs=false,                  
    tabsize=2,
    escapechar=@
}

\lstset{style=mystyle}

\title{Chương 18: Set (Tập hợp) trong C++}
\subtitle{Chinh phục tri thức cùng Learning How to Learn}
\author{Slide Learning C++}
\date{2026}

\begin{document}

\begin{frame}
    \titlepage
\end{frame}

\begin{frame}{Lời chào dẫn nhập}
    \begin{block}{Người đồng hành}
        Chào bạn! Tôi là người đồng hành cùng bạn trên con đường chinh phục tri thức. Chúng ta sẽ cùng khám phá công cụ thông minh mang tên \textbf{Set (Tập hợp)}.
    \end{block}
    
    \begin{exampleblock}{Triết lý học tập}
        Chúng ta sẽ không học vẹt code. Chúng ta xây dựng các \textbf{mô hình tâm trí (mental models)} để thực sự nhìn thấy cách dữ liệu vận hành.
    \end{exampleblock}
\end{frame}

\begin{frame}{Bản đồ hành trình}
    \begin{enumerate}
        \item \textbf{Chương 1:} Set là gì? – Chiếc túi thần kỳ của sự độc nhất.
        \item \textbf{Chương 2:} Khai báo và Khởi tạo – Chuẩn bị túi.
        \item \textbf{Chương 3:} Thêm và Xóa – Quy tắc gác cửa.
        \item \textbf{Chương 4:} Tìm kiếm (C++20) – Kỹ năng thám tử.
        \item \textbf{Chương 5:} Duyệt Set – Cuộc diễu hành các con số.
        \item \textbf{Chương 6:} Iterator – Ngón tay chỉ đường.
    \end{enumerate}
\end{frame}

\begin{frame}{Chương 1: Set là gì?}
    Hãy tưởng tượng một chiếc túi thần kỳ có hai khả năng:
    \begin{itemize}
        \item \textbf{Chống trùng lặp:} Bạn bỏ 2 viên bi giống hệt nhau, túi sẽ tự động nhổ viên thứ hai ra. Chỉ giữ lại duy nhất một bản sao.
        \item \textbf{Tự động sắp xếp:} Bỏ số 5, 1, 9 vào túi, khi mở ra chúng sẽ đứng theo hàng: 1, 5, 9.
    \end{itemize}
    
    \begin{alertblock}{Bức tranh toàn cảnh}
        Trong Set, mỗi thành viên là \textbf{duy nhất} và luôn được \textbf{sắp xếp} ngay ngắn từ nhỏ đến lớn.
    \end{alertblock}
\end{frame}

\begin{frame}{Kiểm tra tư duy 1}
    \textbf{Câu hỏi:} Nếu tôi có một Set $\{2, 4, 8\}$ và thêm số 4 vào một lần nữa. Kết quả sẽ là gì?
    \begin{itemize}
        \item A. $\{2, 4, 4, 8\}$
        \item B. $\{2, 4, 8\}$
        \item C. $\{4, 2, 8\}$
    \end{itemize}
    \pause
    \begin{block}{Đáp án: B}
        Chính xác! Vì quy luật gác cửa, Set từ chối nhận thêm số 4 đã tồn tại.
    \end{block}
\end{frame}

\begin{frame}[fragile]{Chương 2: Khai báo và Khởi tạo}
    Để dùng Set, bạn cần \texttt{\#include <set>}.
    
    \begin{lstlisting}[language=C++]
#include <iostream>
#include <set> 

using namespace std;

int main() {
    // Cach 1: Khai bao mot chiec tui rong
    set<int> tui_so_nguyen;

    // Cach 2: (C++17) Khoi tao co san phan tu
    set tui_than_ky = {1, 5, 2, 5, 1}; 
    // Ket qua thuc te: {1, 2, 5}

    return 0;
}
    \end{lstlisting}
\end{frame}

\begin{frame}{Kiểm tra tư duy 2}
    \textbf{Câu hỏi:} Với dòng code \texttt{set danh\_sach = \{10, 20, 10, 30\};}
    \begin{enumerate}
        \item Kiểu dữ liệu máy tính tự hiểu là gì?
        \item Các phần tử thực tế trong túi là gì?
    \end{enumerate}
    \pause
    \begin{exampleblock}{Giải đáp}
        1. Kiểu \texttt{int}. \\
        2. Kết quả: $\{10, 20, 30\}$.
    \end{exampleblock}
\end{frame}

\begin{frame}{Chương 3: Thêm và Xóa}
    \begin{block}{Lệnh .insert() - Thêm vào}
        Người gác cổng kiểm tra: Nếu đã có thì từ chối, nếu chưa có thì cho vào và \textbf{xếp đúng vị trí thứ tự}.
    \end{block}
    
    \begin{block}{Lệnh .erase() - Đuổi ra}
        Tìm đúng giá trị và mời ra khỏi hàng. Hàng ngũ tự động khép kín lại.
    \end{block}
    
    \begin{exampleblock}{Tốc độ}
        Thêm/Xóa trong Set rất nhanh nhờ cấu trúc \textbf{Cây (Tree)} bên dưới.
    \end{exampleblock}
\end{frame}

\begin{frame}[fragile]{Thực hành Thêm và Xóa}
\begin{lstlisting}[language=C++]
#include <iostream>
#include <set>
using namespace std;

int main() {
    set<int> tap_hop;
    tap_hop.insert(40);
    tap_hop.insert(10);
    tap_hop.insert(20);
    tap_hop.insert(10); // Bi tu choi

    // XOA PHAN TU
    tap_hop.erase(20); 

    // Ket qua con lai: {10, 40}
    return 0;
}
\end{lstlisting}
\end{frame}

\begin{frame}{Chương 4: Tìm kiếm (C++20)}
        \begin{block}{Ẩn dụ Từ điển}
        Tìm chữ "M" không lật từng trang. Bạn mở đôi sách, thấy "K", biết "M" ở nửa sau. Cách "chặt đôi" này giúp Set tìm kiếm cực nhanh.
    \end{block}
    
    \begin{exampleblock}{Lệnh .contains() mới}
        Từ C++20, thay vì dùng các lệnh phức tạp, ta chỉ cần: \\
        \texttt{if (tui\_do\_choi.contains(30)) \{ ... \}}
    \end{exampleblock}
\end{frame}

\begin{frame}[fragile]{Code Tìm kiếm hiện đại}
\begin{lstlisting}[language=C++]
#include <iostream>
#include <set>
using namespace std;

int main() {
    set tui_do_choi = {10, 20, 30, 40, 50};

    if (tui_do_choi.contains(30)) {
        cout << "Tim thay so 30!";
    } else {
        cout << "Khong co 30.";
    }
    return 0;
}
\end{lstlisting}
\end{frame}

\begin{frame}{Chương 5: Duyệt Set}
    Duyệt (Iteration) là đi qua từng phần tử để in ra hoặc tính toán.
    
    \begin{alertblock}{Lưu ý quan trọng}
        Set không dùng chỉ số \texttt{[i]} như Vector. Vì nó không lưu kiểu ngăn kéo đánh số mà lưu theo kiểu \textbf{Cây}. Bạn phải đi từ gốc đến ngọn.
    \end{alertblock}
    
    \begin{block}{Cách hiện đại (For-each)}
        \texttt{for (const auto\& ten : biet\_doi) \{ ... \}}
    \end{block}
\end{frame}

\begin{frame}[fragile]{Thực hành Duyệt Set}
\begin{lstlisting}[language=C++]
#include <iostream>
#include <set>
#include <string>
using namespace std;

int main() {
    set<string> biet_doi = {"Ironman", "Thor", "Hulk", "Thor"};

    for (const auto& ten : biet_doi) {
        cout << "- " << ten << endl;
    }
    return 0;
}
// Ket qua: Hulk, Ironman, Thor (Sap xep chu cai)
\end{lstlisting}
\end{frame}

\begin{frame}{Chương 6: Iterator - Ngón tay chỉ đường}
    \begin{itemize}
        \item \texttt{s.begin()}: Ngón tay chỉ vào \textbf{người đầu tiên}.
        \item \texttt{s.end()}: Ngón tay chỉ vào \textbf{vị trí sau người cuối cùng}.
        \item \texttt{++it}: Dịch ngón tay sang người kế tiếp.
        \item \texttt{*it}: Lấy giá trị tại nơi ngón tay đang chỉ.
    \end{itemize}
    
    \pause
    \begin{alertblock}{Nguy hiểm: Vòng lặp vô tận}
        Nếu dùng \texttt{while} mà quên \texttt{++it}, ngón tay sẽ đứng yên một chỗ mãi mãi!
    \end{alertblock}
\end{frame}

\begin{frame}[fragile]{Nhiệm vụ: Chiếc máy lọc thông minh}
    \begin{lstlisting}[language=C++]
#include <iostream>
#include <vector>
#include <set>
using namespace std;

int main() {
    // 1. Danh sach loi trung lap
    vector<int> diem_loi = {8, 5, 9, 8, 7, 5, 10, 7};

    // 2. Do vao Set de loc
    set<int> diem_sach(diem_loi.begin(), diem_loi.end());

    // 3. In ket qua: 5 7 8 9 10
    for (int x : diem_sach) cout << x << " ";
    
    return 0;
}
\end{lstlisting}
\end{frame}

\begin{frame}{Tổng kết và Bài tập tốt nghiệp}
    \begin{block}{Tóm tắt}
        1. Duy nhất và Tự sắp xếp. \\
        2. \texttt{.insert()}, \texttt{.erase()}. \\
        3. \texttt{.contains()} (C++20). \\
        4. Duyệt bằng For-each hoặc Iterator.
    \end{block}
    
    \begin{exampleblock}{Câu hỏi tốt nghiệp}
        Nếu bạn có danh sách điểm $\{8, 5, 9, 8, 7, 5\}$, kết quả sau khi qua Set là gì?
        \pause \\
        \textbf{Đáp án:} $\{5, 7, 8, 9\}$.
    \end{exampleblock}
\end{frame}

\end{document}