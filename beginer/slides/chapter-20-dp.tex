\documentclass{beamer}
\usepackage[utf8]{inputenc}
\usepackage[T5]{fontenc} % Bắt buộc để hiển thị tiếng Việt
\usepackage[vietnamese]{babel}
\usepackage{tcolorbox}
\usepackage{listings}
\usepackage{xcolor}
\usepackage{booktabs}
\usetheme{Madrid}

\definecolor{codegreen}{rgb}{0,0.6,0}
\definecolor{codegray}{rgb}{0.5,0.5,0.5}
\definecolor{codepurple}{rgb}{0.58,0,0.82}
\definecolor{backcolour}{rgb}{0.95,0.95,0.92}

\lstdefinestyle{mystyle}{
    backgroundcolor=\color{backcolour},   
    commentstyle=\color{codegreen},
    keywordstyle=\color{magenta},
    numberstyle=\tiny\color{codegray},
    stringstyle=\color{codepurple},
    basicstyle=\ttfamily\scriptsize,
    breakatwhitespace=false,         
    breaklines=true,                 
    captionpos=b,                    
    keepspaces=true,                 
    numbers=left,                    
    numbersep=4pt,                  
    showspaces=false,                
    showstringspaces=false,
    showtabs=false,                  
    tabsize=2,
    escapechar=@
}

\lstset{style=mystyle}

\title{Lập trình C++: Quy hoạch động (Dynamic Programming)}
\subtitle{Chiến thuật "Chia để trị và Ghi nhớ"}
\author{Slide Learning CPP}
\date{\today}

\begin{document}

\begin{frame}
    \titlepage
\end{frame}

\begin{frame}{Lộ trình khám phá}
    \begin{block}{4 Trạm dừng chân quan trọng}
        \begin{itemize}
            \item \textbf{Chương 1:} Khái niệm "Ghi nhớ để không lãng phí".
            \item \textbf{Chương 2:} Chiếc túi chứa kiến thức (Vector 1D).
            \item \textbf{Chương 3:} Bản đồ tọa độ (Vector 2D).
            \item \textbf{Chương 4:} Thực hành giải bài toán kinh điển.
        \end{itemize}
    \end{block}
    
    \begin{exampleblock}{Tư duy cốt lõi}
        Thay vì mỗi ngày đều đo đạc lại từ đầu, bạn chỉ cần nhìn vào bản vẽ đã lưu lại để tiếp tục xây cao hơn.
    \end{exampleblock}
\end{frame}

\begin{frame}{Chương 1: Ghi nhớ để không lãng phí}
    \begin{block}{Phép toán đơn giản}
        Tính: $1+1+1+1+1$. Kết quả = 5. \\
        Thêm một số "+1" vào cuối hàng. Kết quả = 6.
    \end{block}
    
    \begin{itemize}
        \item \textbf{Tại sao nhanh?}: Bạn không đếm lại từ đầu mà sử dụng kết quả 5 đã nhớ.
        \item \textbf{Ẩn dụ "Hố cát"}: Mỗi khi giải xong bài toán nhỏ, hãy đào một cái hố (ô nhớ) và đặt kết quả vào đó.
    \end{itemize}
\end{frame}

\begin{frame}[fragile]{Chương 1: Công cụ "Đào hố" trong C++}
    Để lưu trữ trong C++, chúng ta sử dụng \texttt{vector}.
    
    \begin{itemize}
        \item \textbf{Vector 1D}: Dãy các hộp xếp hàng ngang.
        \item \textbf{Vector 2D}: Tủ nhiều ngăn kéo (hàng và cột).
    \end{itemize}

\begin{lstlisting}[language=C++]
// Khai bao vector 1 chieu co 10 phan tu, ban dau deu bang 0
vector<int> f(10, 0);

// Khai bao vector 2 chieu (10 dong, 10 cot), ban dau bang 0
vector<vector<int>> dp(10, vector<int>(10, 0));
\end{lstlisting}
\end{frame}

\begin{frame}{Chương 2: Chiếc túi kiến thức (Vector 1D)}
    \begin{block}{Ẩn dụ: Chiếc thắt lưng thợ sửa chữa}
        \texttt{vector<int> f(n)} là thắt lưng có $n$ chiếc túi đánh số từ 0 đến $n-1$.
    \end{block}
    
    \begin{alertblock}{2 Bước vận hành DP}
        \begin{enumerate}
            \item \textbf{Khởi tạo (Base case)}: Đặt những giá trị cơ bản đầu tiên vào túi.
            \item \textbf{Công thức truy hồi (State transition)}: Cách dùng các túi cũ để tính túi mới.
        \end{enumerate}
    \end{alertblock}
\end{frame}

\begin{frame}[fragile]{Chương 2: Ví dụ dãy Fibonacci}
    Công thức: $f[i] = f[i-1] + f[i-2]$
    
\begin{lstlisting}[language=C++]
int n = 10;
vector<int> f(n + 1);

// Buoc 1: Khoi tao
f[0] = 0;
f[1] = 1;

// Buoc 2: Truy hoi
for (int i = 2; i <= n; i++) {
    f[i] = f[i-1] + f[i-2];
}
\end{lstlisting}
\end{frame}

\begin{frame}{Chương 3: Bản đồ tọa độ (Vector 2D)}
    \begin{block}{Ẩn dụ: Tòa nhà nhiều tầng}
        \texttt{vector<vector<int>> dp(hang, vector<int>(cot))}
        \begin{itemize}
            \item \texttt{dp[i]}: Tầng thứ $i$ của tòa nhà.
            \item \texttt{dp[i][j]}: Căn phòng số $j$ tại tầng thứ $i$.
        \end{itemize}
    \end{block}
    
    \begin{exampleblock}{Tư duy lưới}
        Để tính giá trị \texttt{dp[i][j]}, ta có thể nhìn vào phòng bên cạnh \texttt{dp[i][j-1]} hoặc phòng ở tầng trên \texttt{dp[i-1][j]}.
    \end{exampleblock}
\end{frame}

\begin{frame}{Thử thách tư duy}
    \begin{block}{Câu hỏi}
        Nếu bạn chỉ được phép \textbf{đi sang phải} hoặc \textbf{đi xuống dưới} trên một lưới ô vuông. Để biết số cách đi đến ô \texttt{dp[i][j]}, bạn cần cộng số cách từ những ô nào?
    \end{block}
    
    \pause
    
    \begin{itemize}
        \item \textbf{Đáp án}: Ô bên trái \texttt{dp[i][j-1]} và ô phía trên \texttt{dp[i-1][j]}.
    \end{itemize}
\end{frame}

\begin{frame}[fragile]{Chương 4: Bài toán Con đường kiến đi}
    \begin{exampleblock}{Đề bài}
        Đếm số cách đi từ (0,0) đến (M,N) nếu chỉ được đi sang phải hoặc xuống dưới.
    \end{exampleblock}

\begin{lstlisting}[language=C++]
// Khoi tao o dau tien
dp[0][0] = 1;

for (int i = 0; i < n; i++) {
    for (int j = 0; j < m; j++) {
        if (i == 0 && j == 0) continue;
        
        int tu_phia_tren = (i > 0) ? dp[i-1][j] : 0;
        int tu_ben_trai = (j > 0) ? dp[i][j-1] : 0;
        
        dp[i][j] = tu_phia_tren + tu_ben_trai;
    }
}
\end{lstlisting}
\end{frame}

\begin{frame}{Tổng kết}
    \begin{block}{Ghi nhớ}
        Quy hoạch động giống như việc **xây cầu**. Bạn không thể xây nhịp thứ 10 nếu chưa xây nhịp thứ 9.
    \end{block}
    
    \begin{itemize}
        \item Lưu trữ kết quả bài toán nhỏ.
        \item Tái sử dụng để giải bài toán lớn hơn.
        \item Tiết kiệm thời gian tính toán.
    \end{itemize}
    
    \begin{center}
        \textbf{Chúc các bạn chinh phục thành công DP!}
    \end{center}
\end{frame}

\end{document}