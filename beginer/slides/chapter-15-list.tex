\documentclass{beamer}
\usepackage[utf8]{inputenc}
\usepackage[T5]{fontenc} % Bắt buộc để hiển thị tiếng Việt
\usepackage[vietnamese]{babel}
\usepackage{tcolorbox}
\usepackage{listings}
\usepackage{xcolor}
\usepackage{booktabs}
\usetheme{Madrid}

\definecolor{codegreen}{rgb}{0,0.6,0}
\definecolor{codegray}{rgb}{0.5,0.5,0.5}
\definecolor{codepurple}{rgb}{0.58,0,0.82}
\definecolor{backcolour}{rgb}{0.95,0.95,0.92}

\lstdefinestyle{mystyle}{
    backgroundcolor=\color{backcolour},   
    commentstyle=\color{codegreen},
    keywordstyle=\color{magenta},
    numberstyle=\tiny\color{codegray},
    stringstyle=\color{codepurple},
    basicstyle=\ttfamily\scriptsize,
    breakatwhitespace=false,         
    breaklines=true,                 
    captionpos=b,                    
    keepspaces=true,                 
    numbers=left,                    
    numbersep=4pt,                  
    showspaces=false,                
    showstringspaces=false,
    showtabs=false,                  
    tabsize=2,
    escapechar=@
}

\lstset{style=mystyle}

% Thông tin slide
\title[Làm chủ C++ List]{Lộ Trình Khám Phá: Làm Chủ C++ List}
\subtitle{Từ Kệ Sách đến Đoàn Tàu Hỏa}
\author{Người Đồng Hành Learning How to Learn}
\date{\today}

\begin{document}

% Slide Tiêu đề
\begin{frame}
    \titlepage
\end{frame}

% Slide Giới thiệu & Lộ trình
\begin{frame}{Lộ Trình Khám Phá}
    Chào bạn! Hôm nay chúng ta sẽ học cách làm "trưởng tàu" với \texttt{std::list}.
    
    \begin{block}{Lộ Trình: Làm Chủ C++ List}
        \begin{itemize}
            \item \textbf{Chương 1: Cuộc Đối Đầu Giữa Đoàn Tàu và Kệ Sách} \\
            Hiểu bản chất \texttt{std::list} vs \texttt{std::vector}.
            
            \item \textbf{Chương 2: Người Dẫn Đường Tự Động} \\
            Làm quen với \texttt{iterator} và \texttt{auto}.
            
            \item \textbf{Chương 3: Hành Trình Kiểm Tra Từng Toa Tàu} \\
            Duyệt danh sách bằng \texttt{while} và \texttt{for}.
            
            \item \textbf{Chương 4: Những Câu Thần Chú Của Trưởng Tàu} \\
            Các thuật toán: \texttt{sort}, \texttt{merge}, \texttt{reverse},...
        \end{itemize}
    \end{block}
\end{frame}

% --- CHƯƠNG 1 ---
\begin{frame}{Chương 1: Cuộc Đối Đầu Giữa Đoàn Tàu và Kệ Sách}
    Hãy tưởng tượng bộ nhớ máy tính là một sân chơi.
    
    \begin{block}{1. Vector: Chiếc Kệ Sách Gỗ Cố Định}
        \begin{itemize}
            \item \textbf{Đặc điểm:} Các ô sách dính liền thành một khối.
            \item \textbf{Điểm mạnh:} Truy cập ngẫu nhiên siêu nhanh (Lấy ngay ô số 5).
            \item \textbf{Điểm yếu:} Chèn vào giữa rất mệt (Phải đẩy toàn bộ sách phía sau lùi lại).
        \end{itemize}
    \end{block}

    \begin{exampleblock}{2. List: Đoàn Tàu Hỏa Nối Đuôi Nhau}
        Trong C++, nó là \textit{Doubly Linked List}.
        \begin{itemize}
            \item \textbf{Toa tàu (Node):} Chứa dữ liệu.
            \item \textbf{Móc nối (Pointer):} Các toa nối với nhau bằng móc xích.
        \end{itemize}
    \end{exampleblock}
\end{frame}

\begin{frame}{Tại sao "Đoàn Tàu" (List) thắng "Kệ Sách" (Vector)?}
    \begin{block}{Sự linh hoạt}
        Khi chèn một toa mới vào giữa:
        \begin{enumerate}
            \item Tháo móc xích giữa hai toa cũ.
            \item Móc toa mới vào giữa.
            \item Xong! Không cần di chuyển các toa khác.
        \end{enumerate}
    \end{block}

    \begin{alertblock}{Cái giá phải trả}
        \begin{itemize}
            \item \textbf{Vector:} Nhanh khi tìm, chậm khi sửa đổi.
            \item \textbf{List:} Chậm khi tìm (phải đi bộ từ đầu tàu), siêu nhanh khi sửa đổi.
        \end{itemize}
    \end{alertblock}
\end{frame}

\begin{frame}{Câu Hỏi Kiểm Tra Tư Duy}
    \begin{exampleblock}{Bài toán Playlist Nhạc}
        Bạn viết ứng dụng nghe nhạc. Người dùng hay kéo bài hát từ cuối lên đầu, hoặc chèn bài mới vào giữa.
        Bạn chọn \textbf{Vector} hay \textbf{List}?
    \end{exampleblock}

    \pause
    \vspace{1cm}
    
    \textbf{Đáp án:} \texttt{std::list} (Đoàn tàu).
    
    \textit{Giải thích:} Việc đổi thứ tự bài hát chỉ đơn giản là tháo móc toa này gắn vào toa kia. Không cần bê vác nặng nhọc như vector.
\end{frame}

% --- CHƯƠNG 2 ---
\begin{frame}[fragile]{Chương 2: Người Dẫn Đường Tự Động}
    Với List, bạn không thể gọi "Toa số 5". Bạn cần một \textbf{Iterator} (Người Soát Vé).

    \begin{itemize}
        \item Bắt đầu tại \texttt{begin()}.
        \item Bước sang toa kế tiếp bằng \texttt{++}.
        \item Soi đèn lấy dữ liệu bằng \texttt{*iterator}.
        \item Dừng lại tại \texttt{end()} (khoảng không sau toa cuối).
    \end{itemize}

    \begin{block}{Sức mạnh của \texttt{auto}}
        Thay vì viết tên kiểu dài dòng, hãy dùng thẻ tên vạn năng.
    \end{block}

\begin{lstlisting}[language=C++]
// Cach cu: Dai dong
std::list<int>::iterator nguoi_soat_ve = my_list.begin();

// Cach moi: Ngan gon, sanh dieu
auto nguoi_soat_ve = my_list.begin();
\end{lstlisting}
\end{frame}

\begin{frame}[fragile]{Ví dụ Minh Họa: Iterator}
\begin{lstlisting}[language=C++]
#include <iostream>
#include <list>
using namespace std;

int main() {
    // 1. Tao doan tau (List)
    list<int> doan_tau = {10, 20, 30};

    // 2. Thue nguoi soat ve (Iterator) dung auto
    // Ong ay dang dung o toa dau tien (so 10)
    auto nguoi_soat_ve = doan_tau.begin();

    // 3. Soi den pin xem trong toa co gi (* de lay gia tri)
    // In ra: Toa dau tien cho: 10
    cout << "Toa dau tien cho: " << *nguoi_soat_ve << endl; 

    return 0;
}
\end{lstlisting}
\end{frame}

\begin{frame}{Câu Hỏi Kiểm Tra Tư Duy}
    \begin{exampleblock}{Câu hỏi}
        Người Soát Vé đang đứng yên ở đầu tàu. Để ông ấy đi từ toa này sang toa kế tiếp, chúng ta dùng phép toán nào?
    \end{exampleblock}

    \pause
    \vspace{1cm}
    
    \textbf{Đáp án:} Phép toán \texttt{++} (Increment).
    
    Bạn ra lệnh \texttt{it++}, Người Soát Vé sẽ bước sang toa kế tiếp ngay lập tức.
\end{frame}

% --- CHƯƠNG 3 ---
\begin{frame}[fragile]{Chương 3: Hành Trình Kiểm Tra Từng Toa Tàu}
    \textbf{Cách 1: Thủ Công (While)} - Hiểu rõ từng bước chân.

\begin{lstlisting}[language=C++]
list<int> diem_so = {8, 9, 10};
auto it = diem_so.begin(); // Dung o dau tau

while (it != diem_so.end()) { // Chung nao chua rot khoi tau
    cout << *it << " ";       // 1. Soi den pin
    it++;                     // 2. Buoc sang toa ke tiep
}
\end{lstlisting}

    \vspace{0.5cm}
    \textbf{Cách 2: Tự Động (Range-based for loop)} - Băng chuyền siêu tốc.

\begin{lstlisting}[language=C++]
// Voi moi 'x' nam TRONG 'diem_so'
for (auto x : diem_so) {
    cout << x << " ";
}
\end{lstlisting}
\end{frame}

\begin{frame}[fragile]{Nguy Hiểm: Xóa Toa Tàu (Erase)}
    \begin{alertblock}{CẢNH BÁO CRASH}
        Nếu bạn xóa toa tàu hiện tại (\texttt{erase(it)}) nhưng vòng lặp \texttt{for} vẫn tự động \texttt{it++}, chương trình sẽ sập vì iterator bị treo lơ lửng.
    \end{alertblock}

    \textbf{Kỹ thuật "Nhảy tàu" (Safe Erase):}
    Hàm \texttt{erase()} trả về địa chỉ toa kế tiếp. Hãy bám lấy nó!

\begin{lstlisting}[language=C++]
list<int> doan_tau = {1, 5, 9, 5, 10};

// Phan 'buoc di' trong for de TRONG
for (auto it = doan_tau.begin(); it != doan_tau.end(); /* TRONG */ ) {
    
    if (*it == 5) {
        // Ky thuat "Nhay tau":
        // Xoa toa hien tai va nhay sang toa ke tiep
        it = doan_tau.erase(it); 
    } 
    else {
        // Neu khong xoa, moi di bo
        it++; 
    }
}
\end{lstlisting}
\end{frame}

\begin{frame}{Câu Hỏi Kiểm Tra "Trưởng Tàu"}
    \begin{exampleblock}{Tình huống}
        Danh sách: \texttt{\{2, 4, 6\}}.
        Bạn xóa số \textbf{6} (toa cuối) bằng lệnh \texttt{it = erase(it)}.
        Lúc này ông Soát Vé sẽ đứng ở đâu?
    \end{exampleblock}

    \pause
    \vspace{1cm}
    
    \textbf{Đáp án:} Ông ấy đứng ở \texttt{end()}.
    
    Hàm \texttt{erase} trả về \texttt{end()} khi xóa phần tử cuối cùng. Vòng lặp kiểm tra thấy \texttt{it == end()} nên dừng lại an toàn.
\end{frame}

% --- CHƯƠNG 4 ---
\begin{frame}[fragile]{Chương 4: Những Câu Thần Chú Của Trưởng Tàu}
    \begin{alertblock}{Cái Bẫy Sắp Xếp (Sorting Trap)}
        \texttt{std::sort(list.begin(), list.end())} $\rightarrow$ \textbf{LỖI!}
        Lý do: List không hỗ trợ truy cập ngẫu nhiên.
    \end{alertblock}

    \begin{block}{Giải pháp: Dùng hàm thành viên}
        \texttt{my\_list.sort();}
    \end{block}

    \textbf{Bộ Ba Phép Thuật:}
    \begin{itemize}
        \item \textbf{Sắp xếp:} \texttt{my\_list.sort();} (Bé $\rightarrow$ Lớn).
        \item \textbf{Đảo ngược:} \texttt{my\_list.reverse();} (Đầu $\leftrightarrow$ Đuôi).
        \item \textbf{Lọc trùng:} \texttt{my\_list.unique();} (Chỉ lọc 2 toa \textbf{đứng cạnh nhau}).
    \end{itemize}
\end{frame}

\begin{frame}[fragile]{Ví dụ Tổng Hợp Sức Mạnh}
\begin{lstlisting}[language=C++]
#include <iostream>
#include <list>
using namespace std;

int main() {
    list<int> tau = {4, 2, 2, 5, 1, 5};

    // 1. Sap xep truoc
    tau.sort(); 
    // Tau thanh: {1, 2, 2, 4, 5, 5}

    // 2. Loai bo toa trung nhau (khi da xep ke nhau)
    tau.unique();
    // Tau thanh: {1, 2, 4, 5}

    // 3. Dao nguoc lai
    tau.reverse();
    // Tau thanh: {5, 4, 2, 1}

    return 0;
}
\end{lstlisting}
\end{frame}

\begin{frame}{Câu Hỏi Tốt Nghiệp (Final Boss)}
    \begin{exampleblock}{Thử thách Unique}
        Danh sách: \texttt{list<int> my\_list = \{1, 2, 1, 2\};}
        Chạy ngay: \texttt{my\_list.unique();} (KHÔNG sort trước).
        Kết quả là gì?
        \begin{enumerate}[A.]
            \item \texttt{\{1, 2\}}
            \item \texttt{\{1, 2, 1, 2\}}
        \end{enumerate}
    \end{exampleblock}

    \pause
    \vspace{0.5cm}
    
    \textbf{Đáp án: B. \{1, 2, 1, 2\}}
    
    \textit{Lý do:} \texttt{unique()} bị "cận thị", chỉ lọc được các phần tử đứng sát cạnh nhau.
    \textbf{Quy tắc vàng:} Luôn \texttt{sort} trước khi \texttt{unique}.
\end{frame}

% --- Bonus: Greater ---
\begin{frame}[fragile]{Nâng Cao: Sắp Xếp Giảm Dần}
    Muốn xếp từ Lớn về Bé, ta dùng \texttt{sort(greater...)}.

    \begin{alertblock}{Lưu ý về cú pháp}
        Viết \texttt{greater<auto>()} $\rightarrow$ \textbf{SAI Cú pháp}.
    \end{alertblock}

    \begin{exampleblock}{Giải pháp: Transparent Operator (C++14)}
        Dùng \texttt{greater<>()} (Bỏ trống ngoặc nhọn). Đây là chiếc găng tay "Free Size".
    \end{exampleblock}

\begin{lstlisting}[language=C++]
#include <functional> // Bat buoc

list<int> diem = {5, 1, 9, 3};

// Cach cu (Phai chi dinh kieu):
// diem.sort(greater<int>()); 

// CACH MOI (Khuyen dung, ngan gon):
diem.sort(greater<>()); 

// Ket qua: 9, 5, 3, 1
\end{lstlisting}
\end{frame}

\begin{frame}{Tổng Kết Hành Trình}
    Chúc mừng bạn đã hoàn thành khóa học cấp tốc về \texttt{std::list}!

    \begin{block}{Tóm tắt kho báu}
        \begin{enumerate}
            \item \textbf{Bản chất:} Đoàn tàu móc xích. Thêm/Xóa nhanh, Tìm kiếm chậm.
            \item \textbf{Duyệt:} Dùng \texttt{for(auto x : list)} hoặc Iterator.
            \item \textbf{Thao tác:} Nhớ kỹ thuật "Nhảy tàu" \texttt{it = erase(it)} để xóa an toàn.
            \item \textbf{Thuật toán:} Dùng đồ "chính chủ": \texttt{sort()}, \texttt{reverse()}, \texttt{unique()}.
        \end{enumerate}
    \end{block}

    \centering
    \vspace{1cm}
    \Large \textbf{Bạn đã sẵn sàng cho bài tập thực hành chưa?}
\end{frame}

\end{document}