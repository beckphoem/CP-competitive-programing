\documentclass{beamer}
\usepackage[utf8]{inputenc}
\usepackage[T5]{fontenc} % Bắt buộc để hiển thị tiếng Việt
\usepackage[vietnamese]{babel}
\usepackage{tcolorbox}
\usepackage{listings}
\usepackage{xcolor}
\usepackage{booktabs}
\usetheme{Madrid}

% Định nghĩa màu sắc cho code
\definecolor{codegreen}{rgb}{0,0.6,0}
\definecolor{codegray}{rgb}{0.5,0.5,0.5}
\definecolor{codepurple}{rgb}{0.58,0,0.82}
\definecolor{backcolour}{rgb}{0.95,0.95,0.92}

% Cấu hình hiển thị code
\lstdefinestyle{mystyle}{
    backgroundcolor=\color{backcolour},   
    commentstyle=\color{codegreen},
    keywordstyle=\color{magenta},
    numberstyle=\tiny\color{codegray},
    stringstyle=\color{codepurple},
    basicstyle=\ttfamily\scriptsize,
    breakatwhitespace=false,         
    breaklines=true,                 
    captionpos=b,                    
    keepspaces=true,                 
    numbers=left,                    
    numbersep=4pt,                  
    showspaces=false,                
    showstringspaces=false,
    showtabs=false,                  
    tabsize=2,
    escapechar=@
}

\lstset{style=mystyle}

% Thông tin slide
\title{Chinh phục C++ Vector}
\subtitle{Hành trình khám phá chiếc balô thần kỳ}
\author{Slide Learning C++}
\date{\today}

\begin{document}

% --- Slide Tiêu đề ---
\begin{frame}
    \titlepage
\end{frame}

% --- Slide Giới thiệu Lộ trình ---
\begin{frame}{Lộ trình: Chiếc Balô Thần Kỳ}
    Chào em! Hãy tưởng tượng chúng ta đang chuẩn bị cho một chuyến đi dã ngoại thú vị. Để hiểu về Vector, chúng ta sẽ đi qua 4 trạm dừng chân chính:

    \begin{block}{Các trạm dừng chân}
    \begin{enumerate}
        \item \textbf{Trạm 1: Vector là gì?} \\ Tại sao Mảng (Array) giống như cái khay trứng cố định, còn Vector lại là "Chiếc túi thần kỳ"?
        \item \textbf{Trạm 2: Sắm túi và Xếp đồ (Create \& Add)} \\ Cách khai báo và nhét từng món đồ vào đáy túi (\texttt{push\_back}).
        \item \textbf{Trạm 3: Lục lọi và Tráo đổi (Access \& Change)} \\ Cách lấy đúng món đồ mình cần hoặc đổi món này lấy món kia.
        \item \textbf{Trạm 4: Kiểm kê và Soi độ rộng (Loop \& Size)} \\ Cách rà soát lại toàn bộ đồ đạc và xem túi nặng bao nhiêu.
    \end{enumerate}
    \end{block}
\end{frame}

% --- TRẠM 1: VECTOR LÀ GÌ ---
\begin{frame}{Trạm 1: Vector là gì? - Tạm biệt "Khay trứng"}
    
    \begin{columns}
        \column{0.5\textwidth}
        \begin{alertblock}{Mảng (Array) - Khay trứng}
            \begin{itemize}
                \item Kích thước cố định.
                \item 10 lỗ chỉ đựng được 10 quả.
                \item Quả thứ 11? Không vừa!
                \item Chỉ có 2 quả? Lãng phí chỗ.
            \end{itemize}
        \end{alertblock}

        \column{0.5\textwidth}
        \begin{exampleblock}{Vector - Túi thần kỳ}
            \begin{itemize}
                \item Co giãn linh hoạt (Mảng động).
                \item Lúc đầu xẹp lép.
                \item Nhét đồ vào $\to$ Phồng lên.
                \item Lấy đồ ra $\to$ Co lại gọn gàng.
            \end{itemize}
        \end{exampleblock}
    \end{columns}

    \vspace{0.5cm}
    \textbf{Chốt lại:} Vector chính là một \textbf{Mảng động (Dynamic Array)}.
\end{frame}

% --- Slide Code Include ---
\begin{frame}[fragile]{Khởi động Chiếc Balô (Cú pháp)}
    Để sử dụng được chiếc túi thần kỳ này, chúng ta phải "gọi" nó ra từ thư viện C++.
    
    \begin{lstlisting}[language=C++, caption={Khai báo thư viện}]
#include <iostream>
#include <vector>  // <--- Day la lenh goi thu vien Vector

using namespace std;

int main() {
  // Chung ta se tao chiec tui o day
  return 0;
}
    \end{lstlisting}
    
    \textbf{Lưu ý:} Dòng \texttt{\#include <vector>} giống như việc hô to: \textit{"Doremon ơi, cho tớ mượn cái túi thần kỳ!"}.
\end{frame}

% --- Checkpoint 1 ---
\begin{frame}{Kiểm tra nhanh (Checkpoint)}
    \begin{block}{Câu hỏi}
        Nếu em muốn lập trình một danh sách \textbf{các bạn đăng ký đi tham quan}, nhưng em \textbf{không biết trước} sẽ có bao nhiêu bạn tham gia (có thể 5 bạn, có thể 50 bạn).
        
        Em nên dùng \textbf{Mảng (Array)} hay \textbf{Vector}? Tại sao?
    \end{block}

    \pause
    \vspace{1cm}
    
    \begin{exampleblock}{Đáp án}
        \textbf{Vector!} \\
        Vì số lượng người tham gia thay đổi liên tục, Vector có thể co giãn để chứa vừa đủ số người, không lo thiếu chỗ hay thừa chỗ như Mảng.
    \end{exampleblock}
\end{frame}

% --- TRẠM 2: SẮM TÚI VÀ XẾP ĐỒ ---
\begin{frame}[fragile]{Trạm 2: Sắm túi (Khai báo)}
    Em phải nói rõ cho máy tính biết túi đựng cái gì (Số nguyên hay Chữ).

    \begin{block}{Cú pháp}
        \texttt{vector<Kiểu dữ liệu> Tên\_túi;}
    \end{block}

    \textbf{Ví dụ:}
    \begin{itemize}
        \item \texttt{vector<string> xe\_hoi;} $\to$ Túi tên \texttt{xe\_hoi}, đựng chữ.
        \item \texttt{vector<int> diem\_so;} $\to$ Túi tên \texttt{diem\_so}, đựng số.
    \end{itemize}
\end{frame}

\begin{frame}[fragile]{Nhét đồ vào túi (push\_back)}
    Câu thần chú: \textbf{\texttt{.push\_back()}}.
    
    Hãy hình dung hành động \textbf{xếp hàng}:
    \begin{itemize}
        \item Người vào sau phải đứng sau lưng người trước ("Back").
    \end{itemize}

    \begin{exampleblock}{Ví dụ minh họa với túi \texttt{xe\_hoi}}
        \begin{enumerate}
            \item \texttt{xe\_hoi.push\_back("Vinfast");} \\ $\to$ Túi: [Vinfast]
            \item \texttt{xe\_hoi.push\_back("Toyota");} \\ $\to$ Túi: [Vinfast, Toyota]
        \end{enumerate}
    \end{exampleblock}
\end{frame}

\begin{frame}[fragile]{Code thực tế: Mua túi và Xếp xe}
    \begin{lstlisting}[language=C++]
#include <iostream>
#include <vector> // Nho goi thu vien

using namespace std;

int main() {
  // 1. Mua cai tui ten la "xe_hoi" chi dung chuoi ky tu
  vector<string> xe_hoi;

  // 2. Nhet chiec xe dau tien vao
  xe_hoi.push_back("Vinfast");

  // 3. Nhet tiep chiec xe thu hai
  xe_hoi.push_back("BMW");

  // 4. Nhet chiec xe thu ba
  xe_hoi.push_back("Ford");
  
  // Luc nay trong tui dang la: Vinfast, BMW, Ford
  return 0;
}
    \end{lstlisting}
\end{frame}

% --- Checkpoint 2 ---
\begin{frame}{Kiểm tra nhanh (Checkpoint)}
    \begin{block}{Tình huống xếp hàng trà sữa}
        \begin{enumerate}
            \item Bạn \textbf{Nam} đứng vào hàng đầu tiên.
            \item Bạn \textbf{Lan} dùng lệnh \texttt{.push\_back()}.
            \item Bạn \textbf{Hùng} dùng lệnh \texttt{.push\_back()}.
        \end{enumerate}
        \textbf{Hỏi:} Ai đứng \textbf{đầu hàng} và ai đứng \textbf{cuối hàng}?
    \end{block}

    \pause
    \vspace{0.5cm}

    \begin{exampleblock}{Đáp án}
        \begin{itemize}
            \item Đầu hàng (vị trí 0): \textbf{Nam}.
            \item Cuối hàng: \textbf{Hùng}.
        \end{itemize}
        Nguyên tắc: "Vào sau đứng sau".
    \end{exampleblock}
\end{frame}

% --- TRẠM 3: LỤC LỌI VÀ TRÁO ĐỔI ---
\begin{frame}{Trạm 3: Lục lọi và Tráo đổi}
    Túi đang có: \textbf{Vinfast, BMW, Ford}.
    
    \begin{alertblock}{Quy tắc vàng: Máy tính đếm từ 0!}
        Vector giống như dãy tủ gửi đồ có đánh số (Index):
        \begin{itemize}
            \item Ô đầu tiên: \textbf{Số 0}.
            \item Ô thứ hai: \textbf{Số 1}.
            \item Ô thứ ba: \textbf{Số 2}.
        \end{itemize}
    \end{alertblock}

    \begin{itemize}
        \item \texttt{xe\_hoi[0]} $\to$ Vinfast
        \item \texttt{xe\_hoi[1]} $\to$ BMW
        \item \texttt{xe\_hoi[2]} $\to$ Ford
    \end{itemize}
\end{frame}

\begin{frame}[fragile]{Cách lấy đồ và Đổi đồ}
    \textbf{1. Lấy đồ ra (Access)}
    \begin{itemize}
        \item Dùng ngoặc vuông: \texttt{cout << xe\_hoi[0];}
        \item Dùng lệnh \texttt{.at()}: \texttt{cout << xe\_hoi.at(0);}
    \end{itemize}

    \vspace{0.5cm}

    \textbf{2. Tráo đổi đồ (Change)}
    \begin{itemize}
        \item Chỉ đúng cái tủ và gán giá trị mới.
        \item \texttt{xe\_hoi[0] = "Tesla";}
        \item \textbf{Kết quả:} Túi trở thành [Tesla, BMW, Ford].
    \end{itemize}
\end{frame}

\begin{frame}[fragile]{Code thực tế: Truy cập và Thay đổi}
    \begin{lstlisting}[language=C++]
#include <iostream>
#include <vector>
#include <string>
using namespace std;

int main() {
  vector<string> xe_hoi;
  xe_hoi.push_back("Vinfast"); // Vi tri 0
  xe_hoi.push_back("BMW");     // Vi tri 1
  xe_hoi.push_back("Ford");    // Vi tri 2

  // 1. Lay do ra xem: In xe o vi tri 1 (chiec thu 2)
  cout << "Xe o vi tri 1 la: " << xe_hoi[1] << "\n"; 

  // 2. Trao doi do: Doi xe vi tri 0 thanh Opel
  xe_hoi[0] = "Opel";

  // In lai de kiem tra
  cout << "Xe o vi tri 0 bay gio la: " << xe_hoi[0] << "\n";
  return 0;
}
    \end{lstlisting}
\end{frame}

% --- Checkpoint 3 ---
\begin{frame}{Kiểm tra nhanh (Checkpoint)}
    \begin{block}{Câu hỏi mẹo}
        Túi có 3 món: \texttt{["Opel", "BMW", "Ford"]}. \\
        Nếu em viết lệnh: \texttt{cout << xe\_hoi[3];} chuyện gì sẽ xảy ra?
        
        A. In ra "Ford". \\
        B. In ra ô trống. \\
        C. Báo lỗi hoặc in ra linh tinh.
    \end{block}

    \pause
    \vspace{0.5cm}

    \begin{alertblock}{Đáp án: C}
        Vì đếm từ 0, nên 3 món đồ nằm ở tủ số \textbf{0, 1, 2}. \\
        Tủ số \textbf{3} là "tủ ma" (chưa được tạo), chọc vào sẽ bị lỗi!
    \end{alertblock}
\end{frame}

% --- TRẠM 4: KIỂM KÊ VÀ SOI ĐỘ RỘNG ---
\begin{frame}[fragile]{Trạm 4: Kiểm kê và Soi độ rộng}
    \textbf{1. Soi độ rộng (Size) - Túi nặng bao nhiêu?}
    \begin{itemize}
        \item Lệnh: \texttt{.size()}
        \item Ví dụ: \texttt{cout << xe\_hoi.size();} $\to$ Ra số 3.
    \end{itemize}

    \vspace{0.3cm}
    
    \textbf{2. Kiểm kê hàng loạt (Loop)}
    
    \textit{Cách 1: Truyền thống (Dùng chỉ số i)}
    \begin{lstlisting}[language=C++, basicstyle=\ttfamily\tiny]
for (int i = 0; i < xe_hoi.size(); i++) {
    cout << xe_hoi[i] << "\n";
}
    \end{lstlisting}

    \textit{Cách 2: Hiện đại (For-each) - "Quét mã vạch"}
    \begin{lstlisting}[language=C++, basicstyle=\ttfamily\tiny]
for (string xe : xe_hoi) {
    cout << xe << "\n";
}
    \end{lstlisting}
\end{frame}

\begin{frame}[fragile]{Code thực tế: Vòng lặp}
    \begin{lstlisting}[language=C++]
// ... (Da khai bao vector va push_back)

  // 1. In ra kich thuoc tui
  cout << "Tong so xe la: " << xe_hoi.size() << "\n";

  // 2. Diem danh kieu Truyen thong (Dung i)
  cout << "--- Cach 1: Dung i ---\n";
  for (int i = 0; i < xe_hoi.size(); i++) {
    cout << xe_hoi[i] << "\n";
  }

  // 3. Diem danh kieu Hien dai (For-each) -> Khuyen khich!
  cout << "--- Cach 2: Quet tung cai ---\n";
  for (string x : xe_hoi) {
    cout << x << "\n";
  }
    \end{lstlisting}
\end{frame}

% --- FINAL EXAM ---
\begin{frame}{Kiểm tra tốt nghiệp (Final Exam)}
    \begin{block}{Câu hỏi cuối cùng}
        Giả sử túi \texttt{xe\_hoi} có \textbf{5} chiếc xe. \\
        Anh dùng vòng lặp: \texttt{for (int i = 0; i < xe\_hoi.size(); i++)}
        
        \begin{enumerate}
            \item Biến \texttt{i} sẽ bắt đầu từ số mấy?
            \item Biến \texttt{i} sẽ kết thúc ở số mấy (số lớn nhất để lấy xe)?
        \end{enumerate}
    \end{block}

    \pause
    \vspace{0.5cm}

    \begin{exampleblock}{Đáp án}
        \begin{enumerate}
            \item Bắt đầu từ \textbf{0}.
            \item Kết thúc ở \textbf{4}.
        \end{enumerate}
        (Vì kích thước là 5, các chỉ số hợp lệ là 0, 1, 2, 3, 4).
    \end{exampleblock}
\end{frame}

% --- TỔNG KẾT ---
\begin{frame}{Tổng kết hành trình: Bùa hộ mệnh}
    \begin{table}
        \centering
        \begin{tabular}{l l l}
            \toprule
            \textbf{Hành động} & \textbf{Code} & \textbf{Ví dụ đời thường} \\
            \midrule
            Mua túi & \texttt{vector<string> t;} & Mua balô rỗng. \\
            Nhét đồ & \texttt{t.push\_back("A");} & Nhét táo vào đáy. \\
            Lấy đồ & \texttt{t[0]} / \texttt{t.at(0)} & Lấy món đầu tiên. \\
            Đổi đồ & \texttt{t[0] = "B";} & Tráo táo thành lê. \\
            Đếm đồ & \texttt{t.size()} & Xem túi nặng bao nhiêu. \\
            Kiểm kê & \texttt{for (string x : t)} & Quét từng món. \\
            \bottomrule
        \end{tabular}
    \end{table}

    \begin{block}{Bước tiếp theo}
        Em đã tốt nghiệp khóa "Vector Cấp Tốc"! Em có muốn thử sức với bài tập thực hành nhỏ để test trình độ coding không?
    \end{block}
\end{frame}

\end{document}