\documentclass{beamer}
\usepackage[utf8]{inputenc}
\usepackage[T5]{fontenc} % Bắt buộc để hiển thị tiếng Việt
\usepackage[vietnamese]{babel}
\usepackage{tcolorbox}
\usepackage{listings}
\usepackage{xcolor}
\usepackage{booktabs}
\usetheme{Madrid}

\definecolor{codegreen}{rgb}{0,0.6,0}
\definecolor{codegray}{rgb}{0.5,0.5,0.5}
\definecolor{codepurple}{rgb}{0.58,0,0.82}
\definecolor{backcolour}{rgb}{0.95,0.95,0.92}

\lstdefinestyle{mystyle}{
    backgroundcolor=\color{backcolour},   
    commentstyle=\color{codegreen},
    keywordstyle=\color{magenta},
    numberstyle=\tiny\color{codegray},
    stringstyle=\color{codepurple},
    basicstyle=\ttfamily\scriptsize,
    breakatwhitespace=false,         
    breaklines=true,                 
    captionpos=b,                    
    keepspaces=true,                 
    numbers=left,                    
    numbersep=4pt,                  
    showspaces=false,                
    showstringspaces=false,
    showtabs=false,                  
    tabsize=2,
    escapechar=@
}

\lstset{style=mystyle}

\title{Hành trình chinh phục C++: Queue và Deque}
\author{Slide Learning C++}
\date{2026}

\begin{document}

\begin{frame}
    \titlepage
\end{frame}

\begin{frame}{Lời chào}
    \begin{block}{Chào bạn!}
        Tôi là người đồng hành của bạn trong hành trình chinh phục tri thức. Chúng ta sẽ cùng nhau khám phá thế giới lập trình C++, cụ thể là về \textbf{Queue (Hàng đợi)}.
    \end{block}

\end{frame}

\section{1. Bức tranh toàn cảnh}
\begin{frame}{1. BỨC TRANH TOÀN CẢNH (THE BIG PICTURE)}
    Hãy tưởng tượng bạn đang đứng trong một hàng đợi mua trà sữa:
    \begin{itemize}
        \item Người nào đến trước thì được mua trước và rời đi trước.
        \item Người nào đến sau thì phải đứng vào cuối hàng và đợi đến lượt.
    \end{itemize}
    \begin{exampleblock}{Định nghĩa}
        Trong lập trình, Queue là một cấu trúc dữ liệu hoạt động theo nguyên tắc \textbf{FIFO (First In, First Out - Vào trước, Ra trước)}.
    \end{exampleblock}
    \begin{alertblock}{Lưu ý}
        Trong bài học này, chúng ta sẽ sử dụng \texttt{std::deque} để tận dụng sự linh hoạt và các công cụ hiện đại như \texttt{iterator}.
    \end{alertblock}
\end{frame}

\section{2. Lộ trình khám phá}
\begin{frame}{2. LỘ TRÌNH KHÁM PHÁ (THE MAP)}
    \begin{enumerate}
        \item \textbf{Chương 1:} Khởi tạo Hàng Đợi.
        \item \textbf{Chương 2:} Duyệt bằng \texttt{while}.
        \item \textbf{Chương 3:} Vòng lặp \texttt{for} bán tự động.
        \item \textbf{Chương 4:} \texttt{auto} Iterator.
        \item \textbf{Chương 5:} Thư viện \texttt{<algorithm>}.
    \end{enumerate}
    \begin{block}{Sẵn sàng chưa?}
        Chúng ta sẽ bắt đầu ngay với Chương 1!
    \end{block}
\end{frame}

\section{Chương 1: Khởi tạo}
\begin{frame}{CHƯƠNG 1: KHỞI TẠO HÀNG ĐỢI}
    \begin{itemize}
        \item \textbf{\texttt{std::queue}:} Giống ống nghiệm hẹp, chỉ thấy đầu và cuối.
        \item \textbf{\texttt{std::deque}:} Giống dãy hành lang kính, có thể nhìn thấy mọi vị trí.
    \end{itemize}
    \begin{block}{Tại sao chọn \texttt{deque}?}
        Vì nó cho phép sử dụng \texttt{iterator} và \texttt{algorithms} một cách linh hoạt nhất.
    \end{block}
\end{frame}

\begin{frame}[fragile]{Chương 1: Code mẫu}
\begin{lstlisting}[language=C++]
#include <iostream>
#include <deque> 
using namespace std;

int main() {
    // 1. Khoi tao hang doi rong
    deque<int> hang_doi;

    // 2. Them phan tu vao cuoi (Push Back)
    hang_doi.push_back(10); // Nguoi so 10 den truoc
    hang_doi.push_back(20);
    hang_doi.push_back(30);

    // 3. Xem nguoi dung dau (Front)
    cout << "Nguoi dung dau: " << hang_doi.front() << endl; 

    return 0;
}
\end{lstlisting}
\end{frame}

\begin{frame}{Kiểm tra nhanh (Check-point)}
    \begin{alertblock}{Câu hỏi}
        Nếu thực hiện lệnh:
        \begin{enumerate}
            \item \texttt{push\_back(5)}
            \item \texttt{push\_back(15)}
            \item \texttt{push\_back(25)}
        \end{enumerate}
        Lệnh \texttt{hang\_doi.front()} sẽ trả về kết quả bao nhiêu? Tại sao?
    \end{alertblock}
    \pause
    \begin{exampleblock}{Đáp án}
        Con số \textbf{5} vì nó là người xếp hàng sớm nhất (FIFO).
    \end{exampleblock}
\end{frame}

\section{Chương 2: Duyệt bằng while}
\begin{frame}{CHƯƠNG 2: DUYỆT BẰNG WHILE}
    \begin{itemize}
        \item \textbf{Quy trình:} Kiểm tra còn khách không $\rightarrow$ Phục vụ người đầu $\rightarrow$ Người đó rời đi.
        \item \textbf{Hành động:} Sử dụng \texttt{front()} và \texttt{pop\_front()}.
    \end{itemize}
    \begin{alertblock}{Cảnh báo}
        Duyệt bằng \texttt{while} kết hợp \texttt{pop\_front()} sẽ làm hàng đợi \textbf{trống rỗng} sau khi kết thúc.
    \end{alertblock}
\end{frame}

\begin{frame}[fragile]{Chương 2: Code mẫu}
\begin{lstlisting}[language=C++]
#include <iostream>
#include <deque>
using namespace std;

int main() {
    deque<int> hang_doi = {5, 15, 25};
    while (!hang_doi.empty()) {
        int nguoi_dang_cho = hang_doi.front();
        cout << "Dang phuc vu: " << nguoi_dang_cho << endl;
        hang_doi.pop_front(); // Xoa khoi dau hang
    }
    return 0;
}
\end{lstlisting}
\end{frame}

\begin{frame}{Kiểm tra sự hiểu biết}
    \begin{alertblock}{Câu hỏi}
        Nếu bạn quên dòng \texttt{hang\_doi.pop\_front();} trong vòng lặp \texttt{while}, chuyện gì sẽ xảy ra?
    \end{alertblock}
    \pause
    \begin{exampleblock}{Đáp án}
        Lỗi \textbf{vòng lặp vô tận (infinite loop)}. Chương trình sẽ bị treo vì hàng đợi không bao giờ rỗng.
    \end{exampleblock}
\end{frame}

\section{Chương 3: Range-based for}
\begin{frame}{CHƯƠNG 3: VÒNG LẶP FOR BÁN TỰ ĐỘNG}
    \begin{itemize}
        \item \textbf{Phép ẩn dụ:} Người quản lý đi điểm danh bằng mắt (Clipboard).
        \item \textbf{Đặc điểm:} Xem dữ liệu mà không làm mất dữ liệu.
        \item \textbf{Từ khóa \texttt{auto}:} Trợ lý thông minh tự nhận diện kiểu dữ liệu.
    \end{itemize}
\end{frame}

\begin{frame}[fragile]{Chương 3: Code mẫu}
\begin{lstlisting}[language=C++]
#include <iostream>
#include <deque>
#include <string>
using namespace std;

int main() {
    deque<string> hang_cho = {"An", "Binh", "Chi"};
    
    for (auto nguoi : hang_cho) {
        cout << "- " << nguoi << endl;
    }
    
    cout << "So nguoi van con: " << hang_cho.size() << endl;
    return 0;
}
\end{lstlisting}
\end{frame}

\section{Chương 4: Iterator}
\begin{frame}{CHƯƠNG 4: AUTO ITERATOR - CHIẾC KÍNH HIỂN VI}
    \begin{itemize}
        \item \textbf{Iterator:} Giống như một con trỏ Laser.
        \item \texttt{begin()}: Trỏ vào người đầu tiên.
        \item \texttt{end()}: Trỏ vào vị trí \textbf{sau} người cuối cùng.
        \item \texttt{*it}: "Soi" giá trị tại vị trí laser đang trỏ.
    \end{itemize}
\end{frame}

\begin{frame}[fragile]{Chương 4: Code mẫu}
\begin{lstlisting}[language=C++]
    // Duyet bang Iterator
    for (auto it = hang_cho.begin(); it != hang_cho.end(); ++it) {
        cout << "Vi tri nay la: " << *it << endl;
    }
\end{lstlisting}
\end{frame}

\begin{frame}{Tư duy lập trình}
    \begin{alertblock}{Câu hỏi}
        Nếu muốn thay đổi tên tất cả khách hàng thành "Khách hàng thân thiết" tại chỗ, dùng \texttt{while} hay \texttt{iterator} hiệu quả hơn?
    \end{alertblock}
    \pause
    \begin{exampleblock}{Đáp án}
        Dùng \textbf{Iterator} hoặc \textbf{Range-based for (by reference)}. Nó cho phép sửa trực tiếp mà không cần xóa rồi thêm lại.
    \end{exampleblock}
\end{frame}

\section{Chương 5: Algorithm}
\begin{frame}{CHƯƠNG 5: THƯ VIỆN <ALGORITHM>}
    \begin{itemize}
        \item \textbf{\texttt{sort}:} Sắp xếp hàng ngũ.
        \item \textbf{\texttt{reverse}:} Đảo ngược vị trí 180 độ.
        \item \textbf{\texttt{find}:} Tìm kiếm vị trí một người cụ thể.
    \end{itemize}
    \begin{block}{Lưu ý}
        Các hàm này hoạt động dựa trên \texttt{iterator} (\texttt{begin} và \texttt{end}).
    \end{block}
\end{frame}

\begin{frame}[fragile]{Chương 5: Code mẫu}
\begin{lstlisting}[language=C++]
#include <algorithm>
#include <deque>
#include <iostream>
using namespace std;

int main() {
    deque<int> diem = {8, 2, 9};
    sort(diem.begin(), diem.end()); // {2, 8, 9}
    reverse(diem.begin(), diem.end()); // {9, 8, 2}
    
    auto it = find(diem.begin(), diem.end(), 8);
    if (it != diem.end()) cout << "Tim thay 8!";
    return 0;
}
\end{lstlisting}
\end{frame}

\section{Tổng kết}
\begin{frame}{BÀI KIỂM TRA CUỐI KHÓA}
    \begin{alertblock}{Thử thách cuối cùng}
        Giả sử bạn có hàng đợi \texttt{deque<int> q = \{1, 2, 3, 4, 5\};}.\\
        Nếu thực hiện lệnh: \texttt{reverse(q.begin(), q.end());}\\
        Sau đó gọi lệnh: \texttt{cout << q.front();}\\
        Con số nào sẽ hiện ra?
    \end{alertblock}
    \pause
    \begin{exampleblock}{Đáp án}
        Con số \textbf{5}. (Hàng bị đảo ngược thành \{5, 4, 3, 2, 1\}).
    \end{exampleblock}
\end{frame}

\begin{frame}
    \begin{center}
        \Huge \textbf{Chúc mừng bạn đã tốt nghiệp khóa học!}
    \end{center}
\end{frame}

\end{document}