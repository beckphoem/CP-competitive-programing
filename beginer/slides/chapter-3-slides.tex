\documentclass{beamer}
\usepackage[utf8]{inputenc}
\usepackage[vietnamese]{babel}
\usepackage{tcolorbox}
\usepackage{listings}
\usepackage{xcolor}

\usetheme{Madrid}
\usecolortheme{default}

% Colors for code blocks
\definecolor{codegreen}{rgb}{0,0.6,0}
\definecolor{codegray}{rgb}{0.5,0.5,0.5}
\definecolor{codepurple}{rgb}{0.58,0,0.82}
\definecolor{backcolour}{rgb}{0.95,0.95,0.92}

\lstdefinestyle{mystyle}{
    backgroundcolor=\color{backcolour},   
    commentstyle=\color{codegreen},
    keywordstyle=\color{magenta},
    numberstyle=\tiny\color{codegray},
    stringstyle=\color{codepurple},
    basicstyle=\ttfamily\footnotesize,
    breakatwhitespace=false,         
    breaklines=true,                 
    captionpos=b,                    
    keepspaces=true,                 
    numbers=left,                    
    numbersep=5pt,                  
    showspaces=false,                
    showstringspaces=false,
    showtabs=false,                  
    tabsize=2,
    escapechar=@
}

\lstset{style=mystyle}

\title[Cấu Trúc Rẽ Nhánh]{Bài 3: Cấu Trúc Rẽ Nhánh \\ "Những Ngã Rẽ Cuộc Đời"}
\subtitle{C++ Competitive Programming Series}
\author{Học Cùng C++}
\date{}

\begin{document}

\begin{frame}
  \titlepage
\end{frame}

% 1. Comparison Operators
\begin{frame}{1. Toán Tử So Sánh - "Cán Cân Công Lý"}
  \begin{block}{Nguyên lý}
      Đặt 2 giá trị lên bàn cân. Kết quả chỉ có thể là \textbf{ĐÚNG (True)} hoặc \textbf{SAI (False)}.
  \end{block}

  \begin{center}
  \begin{tabular}{|c|c|c|}
      \hline
      \textbf{Toán tử} & \textbf{Ý nghĩa} & \textbf{Ví dụ} (a=5, b=3) \\
      \hline
      \texttt{==} & Bằng nhau (2 dấu bằng) & \texttt{a == b} (False) \\
      \texttt{!=} & Khác nhau & \texttt{a != b} (True) \\
      \texttt{>} & Lớn hơn & \texttt{a > b} (True) \\
      \texttt{<} & Nhỏ hơn & \texttt{a < b} (False) \\
      \hline
  \end{tabular}
  \end{center}

  \begin{alertblock}{Cảnh báo}
      Đừng nhầm lẫn \texttt{=} (Gán) và \texttt{==} (So sánh)!
  \end{alertblock}
\end{frame}

% 2. If statement
\begin{frame}[fragile]{2. Câu Lệnh \texttt{if} - "Người Gác Cổng"}
  
  \begin{columns}
      \column{0.6\textwidth}
      \begin{lstlisting}[language=C++]
int diem = 8;

if (diem >= 5) {
    // @Chỉ mở cửa nếu Đúng@
    cout << "Chuc mung!"; 
}
      \end{lstlisting}
      
      \column{0.4\textwidth}
      \begin{block}{Cơ chế hoạt động}
          \begin{itemize}
              \item Điều kiện \textbf{ĐÚNG} $\rightarrow$ Mở cửa (Thực hiện lệnh).
              \item Điều kiện \textbf{SAI} $\rightarrow$ Đuổi đi (Bỏ qua).
          \end{itemize}
      \end{block}
  \end{columns}
\end{frame}

% 3. If-Else
\begin{frame}[fragile]{3. Câu Lệnh \texttt{if - else} - "Ngã Ba Đường"}
  
  \textit{"Nếu Mưa thì Ở nhà, Ngược lại thì Đi chơi."} (Chọn 1 trong 2).

  \begin{lstlisting}[language=C++]
int n = 7;

if (n % 2 == 0) {
    cout << "So Chan"; // @Đường 1: Nếu Đúng@
} else {
    cout << "So Le";   // @Đường 2: Nếu Sai@
}
  \end{lstlisting}
  
  $\rightarrow$ Chắc chắn \textbf{một trong hai} hành động sẽ xảy ra.
\end{frame}

% 4. Else-If
\begin{frame}[fragile]{4. Cấu Trúc \texttt{if - else if} - "Chiếc Nón Phân Loại"}
  
  Xếp loại học sinh vào các Nhà (Giỏi, Khá, TB...):

  \begin{lstlisting}[language=C++]
double diem = 7.5;

if (diem >= 8.0) {
    cout << "Gioi";
} else if (diem >= 6.5) {
    cout << "Kha"; // @7.5 thỏa mãn ở đây -> XONG!@
} else if (diem >= 5.0) {
    cout << "Trung Binh"; // @Không bao giờ chạy xuống đây@
} else {
    cout << "Yeu";
}
  \end{lstlisting}

  \begin{block}{Lưu ý}
      Chỉ xét từ trên xuống dưới. Gặp cái nào đúng đầu tiên thì thực hiện và \textbf{KẾT THÚC} luôn.
  \end{block}
\end{frame}

% 5. Logical Operators
\begin{frame}[fragile]{5. Toán Tử Logic - "Bộ Đôi Vệ Sĩ"}
  
  \begin{columns}
      \column{0.5\textwidth}
      \begin{block}{\texttt{\&\&} (VÀ) - Ông Bố Khó Tính}
          Phải thỏa mãn \textbf{TẤT CẢ}.
          \begin{itemize}
              \item Vé \textbf{VÀ} Khẩu trang $\rightarrow$ Vào.
          \end{itemize}
      \end{block}
      \begin{lstlisting}[language=C++]
if (n >= 1 && n <= 10)
// @Trong khoảng [1, 10]@
      \end{lstlisting}

      \column{0.5\textwidth}
      \begin{block}{\texttt{||} (HOẶC) - Bà Mẹ Dễ Tính}
          Chỉ cần \textbf{MỘT} cái đúng.
          \begin{itemize}
              \item Vé VIP \textbf{HOẶC} Quen biết $\rightarrow$ Vào.
          \end{itemize}
      \end{block}
      \begin{lstlisting}[language=C++]
if (n % 3 == 0 || n % 5 == 0)
// @Chia hết cho 3 hoặc 5@
      \end{lstlisting}
  \end{columns}
\end{frame}

% 6. Challenge
\begin{frame}{Thử Thách: Kiểm Tra Năm Nhuận}
  Viết chương trình nhập vào một năm và kiểm tra xem đó có phải \textbf{Năm Nhuận} không.
  
  \vspace{0.5cm}
  \textbf{Gợi ý điều kiện:}
  \begin{enumerate}
      \item Chia hết cho 400.
      \item \textbf{HOẶC}
      \item (Chia hết cho 4 \textbf{VÀ} KHÔNG chia hết cho 100).
  \end{enumerate}

  \vspace{0.5cm}
  \texttt{if ( (n \% 400 == 0) || (n \% 4 == 0 \&\& n \% 100 != 0) )}
\end{frame}

\end{document}
