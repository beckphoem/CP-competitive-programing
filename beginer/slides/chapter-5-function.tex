\documentclass{beamer}
\usepackage[utf8]{inputenc}
\usepackage[T5]{fontenc} % Bắt buộc để hiển thị tiếng Việt
\usepackage[vietnamese]{babel}
\usepackage{tcolorbox}
\usepackage{listings}
\usepackage{xcolor}
\usepackage{booktabs}
\usetheme{Madrid}

% Cấu hình màu sắc cho code
\definecolor{codegreen}{rgb}{0,0.6,0}
\definecolor{codegray}{rgb}{0.5,0.5,0.5}
\definecolor{codepurple}{rgb}{0.58,0,0.82}
\definecolor{backcolour}{rgb}{0.95,0.95,0.92}

% Cấu hình style cho listings
\lstdefinestyle{mystyle}{
    backgroundcolor=\color{backcolour},   
    commentstyle=\color{codegreen},
    keywordstyle=\color{magenta},
    numberstyle=\tiny\color{codegray},
    stringstyle=\color{codepurple},
    basicstyle=\ttfamily\scriptsize,
    breakatwhitespace=false,         
    breaklines=true,                 
    captionpos=b,                    
    keepspaces=true,                 
    numbers=left,                    
    numbersep=4pt,                  
    showspaces=false,                
    showstringspaces=false,
    showtabs=false,                  
    tabsize=2,
    escapechar=@,
    language=C++
}

\lstset{style=mystyle}

% Thông tin bài giảng
\title[C++ Functions]{Functions (Hàm): Phép Thuật và Những Câu Thần Chú}
\subtitle{Chương 5: Khám phá quyền năng của C++}
\author{Lớp học lập trình đặc biệt}
\date{\today}

\begin{document}

% Slide tiêu đề
\frame{\titlepage}

% Slide giới thiệu
\begin{frame}{Lời chào mừng}
    Chào mừng em gia nhập lớp học lập trình đặc biệt này!
    
    \vspace{0.5cm}
    Chúng ta sẽ cùng khám phá một trong những "quyền năng" mạnh mẽ nhất của thế giới C++: \textbf{Functions (Hàm)}.
    
    \begin{block}{Định nghĩa theo phong cách Phép thuật}
        Nếu lập trình là việc em ra lệnh cho máy tính, thì \textbf{Function (Hàm)} chính là cách em dạy máy tính những "tuyệt chiêu" đặc biệt để dùng đi dùng lại mà không cần dạy lại từ đầu.
    \end{block}
\end{frame}

% Slide Lộ trình
\begin{frame}{Bản đồ kho báu: Chinh phục C++ Functions}
    \begin{block}{Lộ trình Khám phá}
        \begin{enumerate}
            \item \textbf{Chương 1: Cuốn Sách Phép Thuật (Create a Function)}
            \begin{itemize}
                \item Cách tạo ra một "câu thần chú" và gói ghém hành động.
            \end{itemize}
            
            \item \textbf{Chương 2: Hô Vang Câu Thần Chú (Call a Function)}
            \begin{itemize}
                \item Cách kích hoạt sức mạnh để máy tính thực thi.
            \end{itemize}
            
            \item \textbf{Chương 3: Sức Mạnh Của Sự Tái Sử Dụng}
            \begin{itemize}
                \item Tại sao viết code một lần nhưng dùng được cả ngàn lần?
            \end{itemize}
        \end{enumerate}
    \end{block}
\end{frame}

% --- CHƯƠNG 1 ---
\section{Chương 1: Tạo Hàm}

\begin{frame}{Chương 1: Cuốn Sách Phép Thuật}
    \begin{block}{Tại sao phải tạo Hàm?}
        Hãy tưởng tượng em chỉ huy Robot pha mì tôm. Quy trình gồm 5 bước:
        \begin{enumerate}
            \item Lấy bát.
            \item Xé gói mì.
            \item Đổ nước sôi.
            \item Đậy nắp.
            \item Chờ 3 phút.
        \end{enumerate}
    \end{block}
    
    \textbf{Vấn đề:} Pha cho 40 người = Viết lại 5 dòng x 40 lần = 200 dòng lệnh (Chán ngắt!).
    
    \textbf{Giải pháp:} Viết 5 bước vào một cái hộp, dán nhãn \texttt{PhaMi}. Em đã tạo ra một \textbf{"Khối kiến thức" (Chunk)}.
\end{frame}

\begin{frame}[fragile]{Giải phẫu một câu thần chú}
    Trong C++, cú pháp để "dạy" máy tính kỹ năng mới:

\begin{lstlisting}
void tenHam() {
  // Nhung hanh dong cu the nam o day
}
\end{lstlisting}

    \begin{itemize}
        \item \textbf{\texttt{void} (Hư không):} Nghĩa là "trống rỗng". Hàm thực hiện xong là nghỉ, không cần trả lại kết quả (giá trị) cho chương trình chính.
        \item \textbf{\texttt{tenHam} (Tên hàm):} Nhãn dán trên hộp (Ví dụ: \texttt{PhaMi}, \texttt{XinChao}). \textit{Lưu ý: Viết dính liền, không dấu cách.}
        \item \textbf{\texttt{()}:} Nơi chứa nguyên liệu đầu vào (Hiện tại đang để trống).
        \item \textbf{\texttt{\{ ... \}} (Căn Phòng Bí Mật):} Nơi chứa mã lệnh tuyệt chiêu.
    \end{itemize}
\end{frame}

\begin{frame}[fragile]{Ví dụ thực tế}
    Tạo một câu thần chú tên là \texttt{myFunction} để in ra dòng chữ:
    
\begin{lstlisting}
void myFunction() {
  cout << "I just got executed!";
}
\end{lstlisting}

    \pause
    \begin{alertblock}{KHOAN ĐÃ! MỘT SỰ THẬT BẤT NGỜ}
        Nếu em chỉ viết đoạn code trên và bấm "Chạy" (Run), \textbf{sẽ KHÔNG CÓ GÌ XẢY RA cả!}
        
        \vspace{0.2cm}
        \textit{Lý do:} Em mới chỉ \textbf{VIẾT} câu thần chú vào sách. Em \textbf{CHƯA ĐỌC} nó lên! 
        (Súng đã chế tạo nhưng chưa ai bóp cò).
    \end{alertblock}
\end{frame}

\begin{frame}[fragile]{Kiểm tra nhanh (Check-point)}
    \begin{exampleblock}{Câu hỏi}
        Trong đoạn code dưới đây, đâu là "Căn phòng bí mật" chứa hành động, và đâu là "Cái nhãn tên"?
    \end{exampleblock}

\begin{lstlisting}
void chemHoaQua() {
  cout << "Chem dua hau!";
  cout << "Chem dua!";
}
\end{lstlisting}

    \pause
    \begin{block}{Đáp án}
        \begin{itemize}
            \item \textbf{Tên hàm (Nhãn):} \texttt{chemHoaQua}
            \item \textbf{Căn phòng bí mật:} Cặp ngoặc nhọn \texttt{\{ \}} và nội dung bên trong.
        \end{itemize}
        Cặp \texttt{()} giống như công tắc để bật cái tên đó lên.
    \end{block}
\end{frame}

% --- CHƯƠNG 2 ---
\section{Chương 2: Gọi Hàm}

\begin{frame}[fragile]{Chương 2: Hô Vang Câu Thần Chú}
    \begin{block}{Sân Khấu Chính: \texttt{main()}}
        \begin{itemize}
            \item \textbf{Hàm \texttt{main()}}: Sân Khấu Trung Tâm. Chương trình luôn bắt đầu từ đây.
            \item \textbf{Hàm của em (\texttt{myFunction})}: Ca sĩ đứng trong cánh gà.
        \end{itemize}
    \end{block}

    Để kích hoạt, em phải viết tên hàm kèm dấu \texttt{();} bên trong \texttt{main()}. Đây là \textbf{Calling a function}.

\begin{lstlisting}
// 1. CANH GA (Dinh nghia)
void xinChao() {
  cout << "Xin chao cac ban!";
}

// 2. SAN KHAU CHINH (Chay)
int main() {
  xinChao(); // <--- HO THAN CHU!
  return 0;
}
\end{lstlisting}
\end{frame}

\begin{frame}[fragile]{Sức mạnh của việc gọi nhiều lần}
    Trong \texttt{main()}, em có thể hô câu thần chú bao nhiêu lần tùy thích (giống như nút Replay).

\begin{lstlisting}
int main() {
  xinChao();
  xinChao();
  xinChao();
  return 0;
}
\end{lstlisting}
    
    Thay vì viết lại lệnh \texttt{cout} 3 lần, em chỉ cần gọi tên hàm 3 lần.
\end{frame}

\begin{frame}[fragile]{Thử thách tư duy (Mind Challenge)}
    \begin{exampleblock}{Tình huống}
        Màn hình sẽ in ra chính xác những gì? Hãy đóng vai robot và chạy từng dòng lệnh.
    \end{exampleblock}

\begin{lstlisting}
void voTay() {
  cout << "Bop! ";
}

int main() {
  cout << "Bat dau: ";
  voTay();
  voTay();
  cout << "Het bai.";
  return 0;
}
\end{lstlisting}

    \pause
    \begin{block}{Kết quả}
        \textbf{"Bắt đầu: Bốp! Bốp! Hết bài."}
        \begin{enumerate}
            \item In "Bắt đầu: "
            \item Nhảy vào \texttt{voTay()} lần 1 $\rightarrow$ in "Bốp! "
            \item Nhảy vào \texttt{voTay()} lần 2 $\rightarrow$ in "Bốp! "
            \item Quay về in "Hết bài."
        \end{enumerate}
    \end{block}
\end{frame}

% --- CHƯƠNG 3 ---
\section{Chương 3: Tại sao dùng Hàm?}

\begin{frame}{Chương 3: Sức Mạnh Của Sự Tái Sử Dụng}
    \begin{quote}
        "Tôi chọn người lười để làm việc khó, vì họ sẽ tìm ra cách dễ nhất để làm nó." - Bill Gates
    \end{quote}
    
    Hàm là công cụ giúp lập trình viên "lười một cách vĩ đại".
    
    \begin{block}{Năng lực 1: Bảo Trì (Sửa 1 được 100)}
        Ví dụ: Chuỗi 100 cửa hàng trà sữa. Đổi công thức từ "trân châu đen" sang "trân châu trắng".
        \begin{itemize}
            \item \textbf{Không dùng hàm:} Phải sửa code ở 100 nơi (Copy/Paste).
            \item \textbf{Dùng hàm:} Chỉ sửa trong hàm \texttt{PhaTraSua} tại trụ sở chính. 100 cửa hàng tự động cập nhật.
        \end{itemize}
    \end{block}
\end{frame}

\begin{frame}[fragile]{Năng lực 2: Sự Gọn Gàng (Chia để trị)}
    Thay vì viết 1000 dòng code lộn xộn trong \texttt{main()}, chúng ta chia nhỏ vấn đề:

\begin{lstlisting}
int main() {
   xayMongNha();   // Ham xay mong
   dungCotNha();   // Ham dung cot
   lopMaiNha();    // Ham lop mai
   sonTuong();     // Ham son tuong
   return 0;
}
\end{lstlisting}

    Nhìn vào đây, ai cũng hiểu quy trình xây nhà mà không cần biết chi tiết trát vôi vữa thế nào (Tư duy trừu tượng hóa).
\end{frame}

\begin{frame}{Tổng kết hành trình}
    Chúc mừng em đã hoàn thành khóa học cấp tốc về Functions!
    
    \begin{enumerate}
        \item \textbf{Tạo Hàm (\texttt{void tenHam() \{...\}}):} Gói ghém hành động vào hộp bí mật.
        \item \textbf{Gọi Hàm (\texttt{tenHam();}):} Hô thần chú để kích hoạt từ \texttt{main()}.
        \item \textbf{Tại sao dùng Hàm:} Tái sử dụng code, dễ sửa lỗi (Sửa 1 được 100), code gọn gàng.
    \end{enumerate}
\end{frame}

\begin{frame}{Bài tập cuối khóa (Final Boss)}
    \begin{exampleblock}{Tình huống Game Bắn Súng}
        \begin{itemize}
            \item Em có hàm \texttt{banSung()} chứa 5 dòng lệnh tính toán đạn bay.
            \item Nhân vật bắn 500 lần trong game.
            \item Đột nhiên phát hiện công thức sai.
        \end{itemize}
        \textbf{Câu hỏi:} 
        \begin{enumerate}
            \item Nếu KHÔNG dùng hàm (Copy-Paste): Phải sửa bao nhiêu dòng code?
            \item Nếu DÙNG hàm: Phải sửa bao nhiêu chỗ?
        \end{enumerate}
    \end{exampleblock}

    \pause
    \begin{alertblock}{Đáp án}
        \begin{enumerate}
            \item \textbf{Không dùng hàm:} Phải sửa 500 chỗ khác nhau (tương đương tìm và sửa 2500 dòng lệnh). Rất dễ sót lỗi!
            \item \textbf{Dùng hàm:} Chỉ cần sửa đúng \textbf{1 chỗ} duy nhất (trong định nghĩa hàm \texttt{banSung}).
        \end{enumerate}
    \end{alertblock}
\end{frame}

\end{document}