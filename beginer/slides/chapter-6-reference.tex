\documentclass{beamer}
\usepackage[utf8]{inputenc}
\usepackage[T5]{fontenc} % Bắt buộc để hiển thị tiếng Việt
\usepackage[vietnamese]{babel}
\usepackage{tcolorbox}
\usepackage{listings}
\usepackage{xcolor}
\usepackage{booktabs}
\usetheme{Madrid}

% Cấu hình màu sắc cho code
\definecolor{codegreen}{rgb}{0,0.6,0}
\definecolor{codegray}{rgb}{0.5,0.5,0.5}
\definecolor{codepurple}{rgb}{0.58,0,0.82}
\definecolor{backcolour}{rgb}{0.95,0.95,0.92}

\lstdefinestyle{mystyle}{
    backgroundcolor=\color{backcolour},   
    commentstyle=\color{codegreen},
    keywordstyle=\color{magenta},
    numberstyle=\tiny\color{codegray},
    stringstyle=\color{codepurple},
    basicstyle=\ttfamily\scriptsize,
    breakatwhitespace=false,         
    breaklines=true,                 
    captionpos=b,                    
    keepspaces=true,                 
    numbers=left,                    
    numbersep=4pt,                  
    showspaces=false,                
    showstringspaces=false,
    showtabs=false,                  
    tabsize=2,
    escapechar=@
}

\lstset{style=mystyle}

% Thông tin bài giảng
\title{C++ References (Tham chiếu)}
\subtitle{Chương 6: Bí mật về những "Biệt danh"}
\author{Slide Learning CPP}
\date{\today}

\begin{document}

% 1. Slide Tiêu đề
\begin{frame}
    \titlepage
\end{frame}

% 2. Giới thiệu mở đầu
\begin{frame}{Khởi động: Bức tranh toàn cảnh}
    Chào mừng bạn đến với thế giới của References!
    
    \begin{exampleblock}{Ví dụ: Nguyễn Văn An và Tí}
        Hãy tưởng tượng bạn có một người bạn tên là \textbf{"Nguyễn Văn An"}. 
        Ở nhà, bố mẹ gọi bạn ấy là \textbf{"Tí"}.
        
        \begin{itemize}
            \item Dù gọi là "An" hay "Tí", thì \textbf{người đó vẫn là một}.
            \item Nếu "Tí" bị dính mực, áo của "An" cũng bị dính mực.
        \end{itemize}
    \end{exampleblock}

    \vspace{0.5cm}
    Trong C++, \textbf{Biến chính} là tên khai sinh, còn \textbf{Reference} chính là biệt danh.
\end{frame}

% 3. Lộ trình
\begin{frame}{Lộ trình khám phá}
    Chúng ta sẽ đi qua 3 phần chính:
    \begin{enumerate}
        \item \textbf{Chương 1: Tạo ra "Biệt danh" (Creating References)}
        \item \textbf{Chương 2: Phép thuật "Tuy hai mà một" (How it works)}
        \item \textbf{Chương 3: Phân biệt "Biệt danh" và "Địa chỉ nhà" (Reference vs. Memory Address)}
    \end{enumerate}
\end{frame}

% --- CHƯƠNG 1 ---

\begin{frame}{Chương 1: Tạo ra "Biệt danh"}
    \begin{block}{Ẩn dụ: Nhãn dán trên chiếc hộp}
        \begin{enumerate}
            \item Bạn có một chiếc hộp đựng \textbf{Pizza} dán nhãn \texttt{mon\_an}.
            \item Bạn dán thêm nhãn thứ hai lên \textbf{chính hộp đó} tên là \texttt{mon\_phu}.
        \end{enumerate}
        $\rightarrow$ Lấy đồ từ hộp nào cũng ra cùng một chiếc Pizza.
    \end{block}

    \vspace{0.5cm}
    \textbf{Cú pháp C++:} Dùng ký hiệu \textbf{\&} (dấu và).
    
    $$ \texttt{kieu\_du\_lieu \&ten\_biet\_danh = ten\_bien\_goc;} $$
\end{frame}

\begin{frame}[fragile]{Code ví dụ: Tạo Reference}
    Hãy xem cách máy tính xử lý "biệt danh":

\begin{lstlisting}[language=C++]
#include <iostream>
#include <string>
using namespace std;

int main() {
  // 1. Tao bien goc (Cai nhan thu nhat)
  string mon_an = "Pizza";

  // 2. Tao bien tham chieu (Cai nhan thu hai)
  // Chu y dau & o day nhe!
  string &bua_trua = mon_an;

  cout << "Ten bien goc: " << mon_an << "\n";
  cout << "Ten biet danh: " << bua_trua << "\n";

  return 0;
}
\end{lstlisting}

    \textbf{Kết quả:} Cả hai đều in ra "Pizza".
\end{frame}

\begin{frame}{Góc kiểm tra nhanh}
    \begin{alertblock}{Câu đố}
        Nếu trong đoạn code trước, mình thay đổi giá trị của \texttt{mon\_an} thành \textbf{"Banh Mi"}.
        
        Lúc đó in \texttt{bua\_trua} ra, nó sẽ là "Pizza" (cũ) hay "Banh Mi" (mới)?
    \end{alertblock}

    \pause
    \vspace{1cm}
    
    \begin{exampleblock}{Đáp án: Banh Mi!}
        Chính xác! Vì chúng cùng dán trên một chiếc hộp, nên thay ruột hộp thì nhãn nào cũng "đọc" ra món mới cả.
    \end{exampleblock}
\end{frame}

% --- CHƯƠNG 2 ---

\begin{frame}{Chương 2: Phép thuật "Tuy hai mà một"}
    \begin{block}{Ẩn dụ: Chiếc TV và hai cái điều khiển}
        Hãy tưởng tượng biến trong C++ như một chiếc TV.
        \begin{itemize}
            \item \textbf{Biến gốc (\texttt{mon\_an}):} Điều khiển màu Đỏ.
            \item \textbf{Tham chiếu (\texttt{bua\_trua}):} Điều khiển màu Xanh.
        \end{itemize}
        Cả hai cùng kết nối tới \textbf{MỘT} chiếc TV.
    \end{block}
    
    Nếu bố dùng điều khiển Xanh bật "Bún Chả", bạn nhìn màn hình cũng sẽ thấy "Bún Chả".
\end{frame}

\begin{frame}[fragile]{Code thử nghiệm: Thay đổi giá trị}
    Sức mạnh của Reference: Can thiệp trực tiếp vào biến gốc.

\begin{lstlisting}[language=C++]
#include <iostream>
#include <string>
using namespace std;

int main() {
  string mon_an = "Pizza";      // Man hinh dang chieu Pizza
  string &bua_trua = mon_an;    // Ket noi them dieu khien 2

  // Thay doi gia tri thong qua "Biet danh"
  bua_trua = "Bun Cha";

  // Kiem tra lai bien goc
  cout << "Mon an bay gio la: " << mon_an << "\n";
  
  return 0;
}
\end{lstlisting}

    \begin{block}{Kết quả}
    Mon an bay gio la: Bun Cha
    \end{block}
\end{frame}

\begin{frame}{Trạm dừng chân suy ngẫm}
    \begin{alertblock}{Câu hỏi quan trọng}
        Trong ví dụ trên, biến \texttt{mon\_an} và biến \texttt{bua\_trua} có phải là 2 biến khác nhau chiếm 2 chỗ trong bộ nhớ không? Hay chỉ 1 chỗ?
    \end{alertblock}

    \pause
    \vspace{0.5cm}

    \begin{exampleblock}{Đáp án: Chỉ 1 chỗ!}
        \begin{itemize}
            \item Chúng dùng chung vùng nhớ.
            \item \textbf{Lợi ích:} Giúp chương trình chạy nhanh hơn vì không phải copy dữ liệu.
        \end{itemize}
    \end{exampleblock}
\end{frame}

% --- CHƯƠNG 3 ---

\begin{frame}{Chương 3: Phân biệt "Biệt danh" và "Địa chỉ"}
    Ký hiệu \textbf{\&} đóng 2 vai khác nhau tùy ngữ cảnh:

    \begin{columns}
        \column{0.5\textwidth}
        \begin{block}{Vai 1: Reference}
            \begin{itemize}
                \item Xuất hiện khi \textbf{khai báo biến mới}.
                \item Cú pháp: \texttt{string \&x = ...}
                \item Ý nghĩa: Tạo biệt danh.
                \item \textit{Ví dụ: Tấm biển tên trước cổng.}
            \end{itemize}
        \end{block}

        \column{0.5\textwidth}
        \begin{block}{Vai 2: Address Operator}
            \begin{itemize}
                \item Đứng trước biến \textbf{đã tồn tại}.
                \item Cú pháp: \texttt{\&x}
                \item Ý nghĩa: Lấy địa chỉ bộ nhớ.
                \item \textit{Ví dụ: Số nhà (123 đường ABC).}
            \end{itemize}
        \end{block}
    \end{columns}
\end{frame}

\begin{frame}[fragile]{Chứng minh: Hai tên gọi, Một địa chỉ}
\begin{lstlisting}[language=C++]
int main() {
  string mon_an = "Pizza";
  string &bua_trua = mon_an; // Dung & de tao biet danh

  // 1. In ra gia tri
  cout << "Gia tri: " << mon_an << "\n";
  
  // 2. In ra dia chi (Dung & truoc ten bien)
  cout << "Dia chi mon_an:   " << &mon_an << "\n";
  cout << "Dia chi bua_trua: " << &bua_trua << "\n";

  return 0;
}
\end{lstlisting}
    
    \begin{block}{Kết quả (Minh họa)}
    Gia tri: Pizza\\
    Dia chi mon\_an:   0x6dfed4\\
    Dia chi bua\_trua: 0x6dfed4
    \end{block}
    \textit{Địa chỉ giống hệt nhau $\rightarrow$ Cùng một vị trí bộ nhớ.}
\end{frame}

% --- TỔNG KẾT & BÀI TẬP ---

\begin{frame}{Tổng kết hành trình}
    \begin{enumerate}
        \item \textbf{Reference là gì?} Là "biệt danh" cho một biến đã có.
        \item \textbf{Đặc điểm:} Không tạo bản copy, dùng chung vùng nhớ. Thay đổi một, ảnh hưởng tất cả.
        \item \textbf{Phân biệt ký hiệu \&:}
        \begin{itemize}
            \item \texttt{int \&x}: Khai báo tham chiếu.
            \item \texttt{\&x}: Lấy địa chỉ bộ nhớ.
        \end{itemize}
    \end{enumerate}
\end{frame}

\begin{frame}[fragile]{Thử thách Code nhỏ}
    \begin{alertblock}{Đề bài}
        Viết đoạn code ngắn:
        \begin{enumerate}
            \item Tạo biến \texttt{diem\_so} bằng 10.
            \item Tạo tham chiếu \texttt{ket\_qua} trỏ tới \texttt{diem\_so}.
            \item Thay đổi \texttt{ket\_qua} thành 100.
            \item In ra \texttt{diem\_so}.
        \end{enumerate}
    \end{alertblock}

    \pause
    \begin{exampleblock}{Đáp án (Chuẩn ngữ pháp)}
\begin{lstlisting}[language=C++]
int diem_so = 10;
int &ket_qua = diem_so;
ket_qua = 100;

// Ket qua in ra se la 100
cout << diem_so; 
\end{lstlisting}
    \end{exampleblock}
\end{frame}

\begin{frame}{Bước tiếp theo: Trùm cuối xuất hiện}
    Bạn đã làm chủ được References (Biệt danh).
    
    \vspace{1cm}
    
    \begin{block}{Next Step: POINTERS (CON TRỎ)}
        Nếu Reference là "Biệt danh", thì Pointers là làm việc trực tiếp với "Địa chỉ nhà" (\texttt{0x...}) mà chúng ta vừa thấy.
        
        \vspace{0.2cm}
        \textit{Bạn đã sẵn sàng để gặp "Trùm cuối" chưa?}
    \end{block}
\end{frame}

\end{document}