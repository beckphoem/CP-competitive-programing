\documentclass{beamer}
\usepackage[utf8]{inputenc}
\usepackage[T5]{fontenc} % Bắt buộc để hiển thị tiếng Việt
\usepackage[vietnamese]{babel}
\usepackage{tcolorbox}
\usepackage{listings}
\usepackage{xcolor}
\usepackage{booktabs}
\usetheme{Madrid}

% Định nghĩa màu sắc cho code
\definecolor{codegreen}{rgb}{0,0.6,0}
\definecolor{codegray}{rgb}{0.5,0.5,0.5}
\definecolor{codepurple}{rgb}{0.58,0,0.82}
\definecolor{backcolour}{rgb}{0.95,0.95,0.92}

% Cấu hình hiển thị code
\lstdefinestyle{mystyle}{
    backgroundcolor=\color{backcolour},   
    commentstyle=\color{codegreen},
    keywordstyle=\color{magenta},
    numberstyle=\tiny\color{codegray},
    stringstyle=\color{codepurple},
    basicstyle=\ttfamily\scriptsize,
    breakatwhitespace=false,         
    breaklines=true,                 
    captionpos=b,                    
    keepspaces=true,                 
    numbers=left,                    
    numbersep=4pt,                  
    showspaces=false,                
    showstringspaces=false,
    showtabs=false,                  
    tabsize=2,
    escapechar=@
}

\lstset{style=mystyle}

% Thông tin bài giảng
\title{Chinh Phục C++ For Loop}
\subtitle{Xây dựng Cỗ máy Tự động hóa}
\author{Slide Learning CPP}
\date{\today}

\begin{document}

% Slide Tiêu đề
\begin{frame}
    \titlepage
\end{frame}

% Slide 1: Giới thiệu lộ trình
\begin{frame}{Lộ trình bài học}
    Chào bạn! Hôm nay chúng ta sẽ biến khái niệm For Loop thành một trò chơi lắp ráp tư duy thú vị: \textbf{"Cỗ máy tự động hóa"}.

    \begin{block}{3 Bước chinh phục}
    \begin{enumerate}
        \item \textbf{Bức tranh toàn cảnh:} Tại sao lại cần For Loop? (Phép ẩn dụ về Robot chép phạt).
        \item \textbf{Giải phẫu cỗ máy:} 3 công tắc bí mật vận hành vòng lặp.
        \item \textbf{Thực chiến:} Code ví dụ và các bài tập kiểm tra tư duy.
    \end{enumerate}
    \end{block}
\end{frame}

% Slide 2: Bức tranh toàn cảnh
\begin{frame}{Phần 1: Bức tranh toàn cảnh (The Big Picture)}
    \begin{alertblock}{Vấn đề: Chép phạt thủ công}
    Hãy tưởng tượng bạn phải chép 100 dòng chữ: "Em hứa sẽ làm bài tập đầy đủ".
    \begin{itemize}
        \item \textbf{Cách thủ công:} Viết từng dòng một $\rightarrow$ Mỏi tay, chán ngắt. Trong code, tương đương việc copy-paste lệnh \texttt{cout} 100 lần.
    \end{itemize}
    \end{alertblock}

    \begin{exampleblock}{Giải pháp: Robot For Loop}
    Bạn chế tạo một con Robot và ra lệnh:
    \textit{"Hãy bắt đầu từ dòng 0. Hễ số dòng còn nhỏ hơn 100 thì viết tiếp. Mỗi lần xong một dòng thì tự đếm thêm 1 số."}
    \end{exampleblock}
    
    $\rightarrow$ Vòng lặp \texttt{for} giúp thực hiện công việc lặp lại hàng nghìn lần chỉ với vài dòng code.
\end{frame}

% Slide 3: Cú pháp (Syntax)
\begin{frame}[fragile]{Phần 2: Giải phẫu cỗ máy (Syntax)}
    Cú pháp chuẩn của vòng lặp \texttt{for} như một bảng điều khiển với 3 công tắc:

\begin{lstlisting}[language=C++]
for (statement 1; statement 2; statement 3) {
  // code block to be executed
}
\end{lstlisting}

    \vspace{0.5cm}
    Chúng ta sẽ đi sâu vào từng "công tắc" (Statements).
\end{frame}

% Slide 4: Chi tiết 3 Statements
\begin{frame}{Chi tiết 3 công tắc điều khiển}
    \begin{itemize}
        \item \textbf{Statement 1: Vạch Xuất Phát (Khởi tạo)}
        \begin{itemize}
            \item Nhiệm vụ: Thiết lập biến đếm ban đầu.
            \item Ví dụ: \texttt{int i = 0;} (Bắt đầu từ 0).
        \end{itemize}
        
        \vspace{0.3cm}
        \item \textbf{Statement 2: Trọng Tài (Điều kiện)}
        \begin{itemize}
            \item Nhiệm vụ: Kiểm tra xem có được chạy tiếp không.
            \item Ví dụ: \texttt{i < 5;} (Nếu đúng thì chạy tiếp, sai thì dừng).
        \end{itemize}
        
        \vspace{0.3cm}
        \item \textbf{Statement 3: Bước Nhảy (Tăng/Giảm)}
        \begin{itemize}
            \item Nhiệm vụ: Thay đổi biến đếm sau mỗi lần lặp để tránh vòng lặp vô tận.
            \item Ví dụ: \texttt{i++} (Tăng i lên 1 đơn vị).
        \end{itemize}
    \end{itemize}
\end{frame}

% Slide 5: Thực chiến Code
\begin{frame}[fragile]{Phần 3: Thực chiến (Code Example)}
    \textbf{Nhiệm vụ:} In ra các số từ 0 đến 4.

\begin{lstlisting}[language=C++]
#include <iostream>
using namespace std;

int main() {
  // Bat dau: i = 0; Dieu kien: i < 5; Buoc nhay: i++
  for (int i = 0; i < 5; i++) {
    cout << "Gia tri cua i la: " << i << "\n";
  }
  return 0;
}
\end{lstlisting}

    \begin{block}{Kết quả màn hình}
\begin{verbatim}
Gia tri cua i la: 0
...
Gia tri cua i la: 4
\end{verbatim}
    \end{block}
\end{frame}

% Slide 6: Phân tích dòng chảy
\begin{frame}{Phân tích dòng chảy (Flow)}
    Quá trình máy tính thực hiện:
    \begin{enumerate}
        \item \textbf{Khởi tạo:} Máy tạo ra $i = 0$.
        \item \textbf{Kiểm tra:} 0 có nhỏ hơn 5 không? $\rightarrow$ \textbf{CÓ}.
        \item \textbf{Thực thi:} In ra "Gia tri cua i la: 0".
        \item \textbf{Bước nhảy:} Tăng $i$ lên 1 (lúc này $i=1$).
        \item \textbf{Lặp lại:} Quay lại bước kiểm tra...
        \item ... Lặp liên tục đến khi $i = 5$.
        \item \textbf{Dừng:} 5 có nhỏ hơn 5 không? $\rightarrow$ \textbf{KHÔNG}. Thoát vòng lặp.
    \end{enumerate}
\end{frame}

% Slide 7: Quiz 1
\begin{frame}[fragile]{Kiểm tra sự hiểu biết}
    \begin{alertblock}{Câu hỏi}
        Trong đoạn code trước, nếu thay đổi \textbf{Statement 3} từ \texttt{i++} thành \texttt{i = i + 2}, chuyện gì sẽ xảy ra?
    \end{alertblock}

    \begin{itemize}
        \item[A.] Nó vẫn in từ 0 đến 4 bình thường.
        \item[B.] Nó sẽ in ra các số: 0, 2, 4.
        \item[C.] Nó sẽ chạy mãi mãi không dừng.
    \end{itemize}

    \pause
    \vspace{0.5cm}
    \begin{exampleblock}{Đáp án: B}
        Chính xác! Robot bây giờ sẽ "nhảy cóc" 2 bước (bỏ qua số lẻ).
        \begin{itemize}
            \item Vạch xuất phát: $i=0$
            \item Đích đến: $i < 5$
            \item Di chuyển: Nhảy +2 (0 $\rightarrow$ 2 $\rightarrow$ 4).
        \end{itemize}
    \end{exampleblock}
\end{frame}

% Slide 8: Thử thách đếm ngược
\begin{frame}[fragile]{Thử thách: Đảo ngược thời gian}
    \textbf{Nhiệm vụ:} Lập trình tên lửa đếm ngược từ 10 về 0.
    
    Hãy điền vào chỗ trống:
\begin{lstlisting}[language=C++]
for (int i = ???; i >= 0; ???) {
    cout << i << "\n";
}
\end{lstlisting}

    \pause
    \begin{block}{Lời giải}
\begin{lstlisting}[language=C++]
for (int i = 10; i >= 0; i--) {
    cout << i << "\n";
}
\end{lstlisting}
    \end{block}
    \begin{itemize}
        \item Xuất phát: 10.
        \item Điều kiện: Lớn hơn hoặc bằng 0 ($>=0$).
        \item Bước nhảy: Giảm đi 1 (\texttt{i--}).
    \end{itemize}
\end{frame}

% Slide 9: Nested Loops Concept
\begin{frame}{Vòng lặp lồng nhau (Nested Loops)}
    \textbf{Phép ẩn dụ: Chiếc Đồng Hồ}
    \begin{itemize}
        \item \textbf{Vòng lặp ngoài (Outer Loop):} Kim Giờ.
        \item \textbf{Vòng lặp trong (Inner Loop):} Kim Phút.
    \end{itemize}

    \begin{block}{Quy luật bất biến}
        "Chỉ khi nào kim phút chạy đủ một vòng (hết nhiệm vụ của vòng trong), thì kim giờ mới được phép nhích lên 1 nấc (bước tiếp theo của vòng ngoài)."
    \end{block}
\end{frame}

% Slide 10: Nested Loops Code
\begin{frame}[fragile]{Code mô phỏng đồng hồ}
\begin{lstlisting}[language=C++]
// VONG LAP NGOAI (Kim Gio): i tu 1 den 2
for (int i = 1; i <= 2; ++i) {
    cout << "Gio thu: " << i << "\n"; 

    // VONG LAP TRONG (Kim Phut): j tu 1 den 3
    // Chay het tu 1-3 moi lan 'i' xuat hien
    for (int j = 1; j <= 3; ++j) {
        cout << "   Phut thu: " << j << "\n";
    }
}
\end{lstlisting}

    \begin{block}{Kết quả}
    Gio thu: 1 $\rightarrow$ Phut thu: 1, 2, 3 \\
    Gio thu: 2 $\rightarrow$ Phut thu: 1, 2, 3
    \end{block}
\end{frame}

% Slide 11: Quiz Nested Loops
\begin{frame}{Thử thách tư duy (Mental Gym)}
    \begin{alertblock}{Câu hỏi}
        Nếu vòng ngoài chạy từ 1 đến \textbf{3}, vòng trong chạy từ 1 đến \textbf{5}.
        Dòng chữ "Phut thu: ..." sẽ được in ra tổng cộng bao nhiêu lần?
    \end{alertblock}

    \begin{itemize}
        \item[A.] 8 lần (lấy $3 + 5$)
        \item[B.] 15 lần (lấy $3 \times 5$)
        \item[C.] 5 lần (chỉ tính vòng trong)
    \end{itemize}

    \pause
    \vspace{0.5cm}
    \begin{exampleblock}{Đáp án: B (15 lần)}
        Đây chính là phép nhân trong lập trình:
        $$ 3 \text{ (vòng ngoài)} \times 5 \text{ (vòng trong)} = 15 \text{ lần} $$
    \end{exampleblock}
\end{frame}

% Slide 12: Foreach Loop
\begin{frame}[fragile]{Mảnh ghép cuối: Foreach Loop}
    \textbf{Ẩn dụ: Băng chuyền hành lý}
    Thay vì chạy đi tìm đồ vật (dùng index $i$), bạn đứng yên và băng chuyền (Array) đưa từng món đồ đến tay bạn.

    \begin{block}{Cú pháp}
\begin{lstlisting}[language=C++]
for (type variableName : arrayName) {
  // code block
}
\end{lstlisting}
    \end{block}
    \begin{itemize}
        \item Không cần biến đếm $i$.
        \item Không cần điều kiện dừng phức tạp.
        \item Đọc là: "Với mỗi \texttt{variableName} nằm trong \texttt{arrayName}".
    \end{itemize}
\end{frame}

% Slide 13: Foreach Example
\begin{frame}[fragile]{Ví dụ Foreach}
\begin{lstlisting}[language=C++]
int main() {
  // Bang chuyen chua 4 con so
  int soMayMan[] = {10, 20, 30, 40};

  // Voi moi "so" nam trong "soMayMan"
  for (int so : soMayMan) {
    cout << so << "\n";
  }
  return 0;
}
\end{lstlisting}
    
    Code gọn gàng, tự động chạy từ đầu đến cuối mảng.
\end{frame}

% Slide 14: Bài tập tốt nghiệp
\begin{frame}[fragile]{Bài tập tốt nghiệp For Loop}
    Cho mảng: \texttt{int diemSo[] = \{5, 8, 10\};}
    
    Đoạn code sau thực hiện:
\begin{lstlisting}[language=C++]
for (int d : diemSo) {
  cout << d + 1 << "\n";
}
\end{lstlisting}

    \begin{alertblock}{Câu hỏi: Biến \texttt{d} đại diện cho cái gì?}
        \begin{itemize}
            \item[A.] \textbf{Vị trí} của điểm số (0, 1, 2).
            \item[B.] \textbf{Giá trị} thực sự của điểm số (5, 8, 10).
            \item[C.] Tổng số lượng điểm số (3).
        \end{itemize}
    \end{alertblock}

    \pause
    \begin{exampleblock}{Đáp án: B (Giá trị)}
        Foreach lấy trực tiếp giá trị (Value). Kết quả in ra sẽ là: 6, 9, 11.
    \end{exampleblock}
\end{frame}

% Slide 15: Tổng kết
\begin{frame}{Tổng kết hành trình}
    Chúng ta đã đi qua 3 trạm dừng chân:
    
    \begin{enumerate}
        \item \textbf{For Loop cơ bản:} Robot chép phạt (Lặp theo số lần).
        \item \textbf{Nested Loop:} Chiếc đồng hồ (Vòng trong xong, vòng ngoài mới bước).
        \item \textbf{Foreach Loop:} Băng chuyền hành lý (Duyệt qua danh sách, không cần đếm).
    \end{enumerate}

    \vspace{1cm}
    \begin{block}{Bước tiếp theo: Quyền năng kiểm soát}
        Làm sao để phanh gấp hoặc bỏ qua chướng ngại vật?
        Hẹn gặp lại ở bài học về \textbf{Break} và \textbf{Continue}!
    \end{block}
\end{frame}

\end{document}