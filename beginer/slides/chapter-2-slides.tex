\documentclass{beamer}
\usepackage[utf8]{inputenc}
\usepackage[vietnamese]{babel}
\usepackage{tcolorbox}
\usepackage{listings}
\usepackage{xcolor}

\usetheme{Madrid}
\usecolortheme{default}

% Colors for code blocks
\definecolor{codegreen}{rgb}{0,0.6,0}
\definecolor{codegray}{rgb}{0.5,0.5,0.5}
\definecolor{codepurple}{rgb}{0.58,0,0.82}
\definecolor{backcolour}{rgb}{0.95,0.95,0.92}

\lstdefinestyle{mystyle}{
    backgroundcolor=\color{backcolour},   
    commentstyle=\color{codegreen},
    keywordstyle=\color{magenta},
    numberstyle=\tiny\color{codegray},
    stringstyle=\color{codepurple},
    basicstyle=\ttfamily\footnotesize,
    breakatwhitespace=false,         
    breaklines=true,                 
    captionpos=b,                    
    keepspaces=true,                 
    numbers=left,                    
    numbersep=5pt,                  
    showspaces=false,                
    showstringspaces=false,
    showtabs=false,                  
    tabsize=2,
    escapechar=@
}

\lstset{style=mystyle}

\title[Biến và Toán Tử]{Bài 2: Biến và Toán Tử \\ Những Chiếc Hộp Thần Kỳ}
\subtitle{C++ Competitive Programming Series}
\author{Học Cùng C++}
\date{}

\begin{document}

\begin{frame}
  \titlepage
\end{frame}

% 1. Variable Definition
\begin{frame}{1. Biến (Variable) Là Gì?}
  \begin{block}{Ẩn dụ: Chiếc Hộp Thần Kỳ}
    Hãy tưởng tượng bộ nhớ máy tính là một \textbf{Kho Hàng Khổng Lồ}.
    \begin{itemize}
        \item \textbf{Biến} là những chiếc hộp để bạn cất giữ đồ đạc (thông tin).
        \item Mỗi hộp chỉ chứa được \textbf{một giá trị} tại một thời điểm.
    \end{itemize}
  \end{block}

  \begin{center}
      \textit{Bỏ cái mới vào $\rightarrow$ Cái cũ biến mất!}
  \end{center}

  \begin{exampleblock}{Ví dụ}
    \texttt{int a = 5;} \\
    $\rightarrow$ Tạo hộp tên "a", bỏ số 5 vào.
  \end{exampleblock}
\end{frame}

% 2. Naming Rules
\begin{frame}{2. Quy Tắc Đặt Tên (Nhãn Dán)}
  Giống như đặt tên cho \textbf{Thú Cưng}, phải tuân thủ luật lệ:

  \begin{enumerate}
      \item \textbf{Không dấu cách:} Phải viết liền. 
      \begin{itemize}
          \item \textcolor{red}{Sai:} \texttt{con meo}
          \item \textcolor{blue}{Đúng:} \texttt{conMeo}, \texttt{con\_meo}
      \end{itemize}
      \item \textbf{Không bắt đầu bằng số:}
      \begin{itemize}
          \item \textcolor{red}{Sai:} \texttt{1conMeo}
          \item \textcolor{blue}{Đúng:} \texttt{conMeo1}
      \end{itemize}
      \item \textbf{Không dùng ký tự lạ:} Chỉ dùng chữ, số, `\_`.
      \item \textbf{Phân biệt hoa thường:} \texttt{Meo} $\neq$ \texttt{meo}.
  \end{enumerate}
\end{frame}

% 3. Data Types (Int/Long Long)
\begin{frame}[fragile]{3. Kiểu Dữ Liệu: Hình Dáng Chiếc Hộp (1)}
  Không thể nhét con voi vào hộp diêm!

  \begin{columns}
      \column{0.5\textwidth}
      \begin{block}{\texttt{int} - Hộp Vuông Nhỏ}
          \begin{itemize}
              \item \textbf{Đựng:} Số nguyên (không dấu chấm).
              \item \textbf{Ví dụ:} 1, 100, -5.
              \item \textbf{Sức chứa:} $\approx$ 2 tỷ.
          \end{itemize}
      \end{block}

      \column{0.5\textwidth}
      \begin{block}{\texttt{long long} - Container}
          \begin{itemize}
              \item \textbf{Đựng:} Số nguyên SIÊU TO.
              \item \textbf{Dùng khi:} \texttt{int} không chứa nổi (tỷ phú, thiên văn).
              \item \textbf{Ví dụ:} \texttt{9000000000LL}
          \end{itemize}
      \end{block}
  \end{columns}
\end{frame}

% 4. Data Types (Float/Double)
\begin{frame}{3. Kiểu Dữ Liệu: Hình Dáng Chiếc Hộp (2)}
  \begin{block}{Số Thực - Những Chai Nước}
      Dùng để đựng số có phần thập phân (3.14, 8.5).
  \end{block}

  \begin{description}
      \item[\texttt{float}] Chai nhỏ, độ chính xác bình thường.
      \item[\texttt{double}] Chai lớn, có "kính lúp" soi kỹ hơn. Độ chính xác cao gấp đôi.
  \end{description}

  \begin{alertblock}{Lời khuyên}
      Luôn ưu tiên dùng \textbf{\texttt{double}} cho tính toán số thực để tránh sai số!
  \end{alertblock}
\end{frame}

% 5. Data Types (Char/Bool)
\begin{frame}{3. Kiểu Dữ Liệu: Hình Dáng Chiếc Hộp (3)}
  \begin{columns}
      \column{0.5\textwidth}
      \begin{block}{\texttt{char} - Hộp Đựng Nhẫn}
          \begin{itemize}
              \item Đựng đúng \textbf{1 ký tự}.
              \item Phải để trong dấu nháy đơn \texttt{' '}.
              \item Ví dụ: \texttt{'A'}, \texttt{'x'}.
          \end{itemize}
      \end{block}

      \column{0.5\textwidth}
      \begin{block}{\texttt{bool} - Công Tắc Đèn}
          \begin{itemize}
              \item Chỉ có 2 trạng thái.
              \item \texttt{true} (Đúng/Bật).
              \item \texttt{false} (Sai/Tắt).
          \end{itemize}
      \end{block}
  \end{columns}
\end{frame}

% 6. Operators
\begin{frame}{4. Toán Tử - Cỗ Máy Chế Biến}
  \begin{itemize}
      \item \textbf{Cơ bản:} \texttt{+} (Cộng), \texttt{-} (Trừ), \texttt{*} (Nhân).
      \item \textbf{Phép Chia \texttt{/}:}
      \begin{itemize}
          \item Nguyên chia Nguyên = Nguyên (Làm tròn xuống!).
          \item Ví dụ: $5 / 2 = 2$.
          \item Muốn ra thập phân: $5.0 / 2 = 2.5$.
      \end{itemize}
      \item \textbf{Chiếc Đồng Hồ \texttt{\%}:}
      \begin{itemize}
          \item Phép chia lấy DƯ.
          \item $5 \% 2 = 1$ (5 chia 2 dư 1).
          \item Dùng để kiểm tra chẵn lẻ, tính chu kỳ.
      \end{itemize}
  \end{itemize}
\end{frame}

% NEW SECTION: Type Casting
\begin{frame}[fragile]{5. Phép "Biến Hình" (Ép Kiểu)}
  \begin{block}{Ẩn dụ: Sang Chiết Chất Lỏng}
    Đổ nước từ chai vào hộp, hoặc cắt gọt trái cây cho vừa hộp.
  \end{block}

  \begin{columns}
    \column{0.5\textwidth}
    \textbf{a. Tự Động (Implicit)} \\
    \textit{"Đổ từ cốc nhỏ sang xô lớn"}
    \begin{lstlisting}[language=C++]
int a = 5;
double b = a; 
// @b là 5.0 (An toàn)@
    \end{lstlisting}

    \column{0.5\textwidth}
    \textbf{b. Ép Buộc (Explicit)} \\
    \textit{"Cắt gọt cho vừa hộp"}
    \begin{lstlisting}[language=C++]
double pi = 3.14;
int n = (int)pi; 
// @n là 3 (Mất phần đuôi)@
    \end{lstlisting}
  \end{columns}
\end{frame}

% NEW SECTION: Char Arithmetic
\begin{frame}[fragile]{6. Cộng Trừ Ký Tự (Char Math)}
  \begin{block}{Ẩn dụ: Cầu Thang Chữ Cái}
    Mỗi ký tự đứng trên một bậc thang (ASCII). 
    \begin{itemize}
        \item 'A' đứng ở bậc 65.
        \item 'B' đứng ở bậc 66.
    \end{itemize}
  \end{block}

  \begin{lstlisting}[language=C++]
char c = 'A';
c = c + 1; // @Bước lên 1 bậc@
cout << c; // @In ra: B@
  \end{lstlisting}

  \begin{alertblock}{Mẹo nhỏ}
    Khoảng cách giữa chữ hoa và thường là 32 bậc.
    \texttt{'a' - 32 = 'A'}
  \end{alertblock}
\end{frame}

% NEW SECTION: Typeid
\begin{frame}[fragile]{7. Kính Chiếu Yêu (typeid)}
  \begin{itemize}
      \item Dùng để xem "nguyên hình" của một biến.
      \item Giống như máy quét mã vạch.
  \end{itemize}

  \begin{lstlisting}[language=C++]
#include <typeinfo> // @Cần thư viện này@

int x = 10;
cout << typeid(x).name(); 
// @In ra ký hiệu kiểu (ví dụ "i")@
  \end{lstlisting}
\end{frame}

% 7. Example Code
\begin{frame}[fragile]{Ví Dụ Tổng Hợp}
Chương trình tính Chu vi và Diện tích Hình chữ nhật:

\begin{lstlisting}[language=C++]
#include<bits/stdc++.h>
using namespace std;

int main(){ 
    // @1. Khai báo hộp (biến)@
    int chieuDai = 10;
    int chieuRong = 5;

    // @2. Chế biến (Toán tử)@
    int chuVi = (chieuDai + chieuRong) * 2;
    int dienTich = chieuDai * chieuRong;

    // @3. Xuất kho (In ra màn hình)@
    cout << "Chu vi la: " << chuVi << endl;
    cout << "Dien tich la: " << dienTich << endl;
}
\end{lstlisting}
\end{frame}

% 8. Challenge
\begin{frame}{Thử Thách}
  \begin{center}
      \LARGE
      Hãy viết chương trình tính tuổi của bạn vào năm \textbf{2050}!
  \end{center}
  
  \vspace{1cm}
  
  \textbf{Gợi ý:}
  \begin{enumerate}
      \item Tạo biến \texttt{namSinh}.
      \item Dùng phép toán trừ: $2050 - \texttt{namSinh}$.
      \item In kết quả ra màn hình.
  \end{enumerate}
\end{frame}

\end{document}
