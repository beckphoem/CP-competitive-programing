\documentclass{beamer}
\usepackage[utf8]{inputenc}
\usepackage[T5]{fontenc} % Bắt buộc để hiển thị tiếng Việt
\usepackage[vietnamese]{babel}
\usepackage{tcolorbox}
\usepackage{listings}
\usepackage{xcolor}
\usepackage{booktabs}
\usetheme{Madrid}

\definecolor{codegreen}{rgb}{0,0.6,0}
\definecolor{codegray}{rgb}{0.5,0.5,0.5}
\definecolor{codepurple}{rgb}{0.58,0,0.82}
\definecolor{backcolour}{rgb}{0.95,0.95,0.92}

\lstdefinestyle{mystyle}{
    backgroundcolor=\color{backcolour},   
    commentstyle=\color{codegreen},
    keywordstyle=\color{magenta},
    numberstyle=\tiny\color{codegray},
    stringstyle=\color{codepurple},
    basicstyle=\ttfamily\scriptsize,
    breakatwhitespace=false,         
    breaklines=true,                 
    captionpos=b,                    
    keepspaces=true,                 
    numbers=left,                    
    numbersep=4pt,                  
    showspaces=false,                
    showstringspaces=false,
    showtabs=false,                  
    tabsize=2,
    escapechar=@
}

\lstset{style=mystyle, language=C++}

\title{Chinh Phục C++ Iterator}
\subtitle{Hành trình của "Người Soát Vé"}
\author{Slide Learning C++}
\date{\today}

\begin{document}

\begin{frame}
    \titlepage
\end{frame}

\begin{frame}{1. Bức tranh toàn cảnh: Iterator là gì?}
    \begin{block}{Phép ẩn dụ: Đoàn tàu hỏa}
        Hãy tưởng tượng \texttt{vector} trong C++ là một đoàn tàu chở đầy đồ chơi.
        Để kiểm tra từng món đồ, ta cần một \textbf{Người soát vé}.
    \end{block}

    \begin{itemize}
        \item Người soát vé không phải là đoàn tàu.
        \item Người soát vé không phải là món đồ chơi.
        \item Người soát vé biết cách: \textbf{đứng tại một toa}, \textbf{xem bên trong}, và \textbf{bước tiếp}.
    \end{itemize}

    \begin{exampleblock}{Định nghĩa}
        \textbf{Iterator} chính là "Người soát vé" giúp bạn duyệt qua danh sách mà không cần nhớ số thứ tự (index).
    \end{exampleblock}
\end{frame}

\begin{frame}{2. Giải mã Siêu năng lực (4 Lệnh cơ bản)}
    Để làm việc, "Người soát vé" cần 4 lệnh cơ bản:

    \begin{enumerate}
        \item \textbf{\texttt{begin()}}: Vạch xuất phát. Nhảy dù xuống \textbf{toa đầu tiên}.
        \item \textbf{\texttt{end()}}: Biển báo "Vực thẳm".
            \begin{alertblock}{Lưu ý quan trọng}
                \texttt{end()} \textbf{KHÔNG PHẢI} là toa cuối cùng. Nó là khoảng không \textbf{ngay sau} toa cuối cùng.
            \end{alertblock}
        \item \textbf{\texttt{*}} (Dereference): Đôi mắt thần. "Mở cửa toa" để xem dữ liệu bên trong (ví dụ: \texttt{*it}).
        \item \textbf{\texttt{++}} (Cộng cộng): Bước chân. Đi sang toa kế tiếp.
    \end{enumerate}
\end{frame}

\begin{frame}[fragile]{3. Code mẫu: Iterator truyền thống}
    Dưới đây là cách "Người soát vé" đi bộ từ đầu đến cuối tàu.

\begin{lstlisting}
#include <iostream>
#include <vector>
#include <string>
using namespace std;

int main() {
  // 1. Tao doan tau (vector)
  vector<string> doan_tau = {"Pizza", "Ga ran", "Tra sua"};

  // 2. Tao nguoi soat ve (iterator)
  vector<string>::iterator nguoi_soat_ve;

  // 3. Bat dau hanh trinh
  for (nguoi_soat_ve = doan_tau.begin(); 
       nguoi_soat_ve != doan_tau.end(); 
       nguoi_soat_ve++) {
    
    // Dung mat than (*) de xem mon an
    cout << "Mon an: " << *nguoi_soat_ve << "\n";
  }
  return 0;
}
\end{lstlisting}
\end{frame}

\begin{frame}{Thử thách tư duy}
    \begin{block}{Câu hỏi kiểm tra}
        Nếu "Người soát vé" (Iterator) đang đứng ở vị trí \texttt{end()}, chú ấy có thể dùng "mắt thần" (\texttt{*}) để lấy dữ liệu được không? Tại sao?
    \end{block}

    \pause

    \begin{alertblock}{Câu trả lời}
        \textbf{KHÔNG!} Vì \texttt{end()} là vực thẳm (vạch vôi đỏ) sau toa cuối cùng. Ở đó không có toa tàu nào cả. Nếu cố tình dùng \texttt{*}, chương trình sẽ bị lỗi (crash).
    \end{alertblock}
\end{frame}

\begin{frame}{4. Từ khóa \texttt{auto} và Băng chuyền tự động}
    Thay vì viết dài dòng \texttt{vector<string>::iterator}, ta dùng \textbf{\texttt{auto}}.
    
    \begin{itemize}
        \item \textbf{\texttt{auto}}: Chiếc kính thông minh, tự nhận diện kiểu dữ liệu.
        \item \textbf{Range-based for loop}: Biến đoàn tàu thành "Băng chuyền".
    \end{itemize}

    \begin{block}{So sánh hình ảnh}
        \begin{itemize}
            \item \textbf{Cách cũ:} Người soát vé đi bộ từng toa (\texttt{begin}, \texttt{++}, \texttt{*}, \texttt{end}).
            \item \textbf{Cách mới:} Bạn đứng yên, băng chuyền tự đẩy món đồ (\texttt{item}) đến trước mặt.
        \end{itemize}
    \end{block}
\end{frame}

\begin{frame}[fragile]{Code mẫu: Số tự động (Range-based for loop)}
    Cú pháp sạch và đẹp hơn rất nhiều:

\begin{lstlisting}
int main() {
  vector<string> doan_tau = {"Pizza", "Ga ran", "Tra sua"};

  // Dich nghia: "Voi moi 'mon an' nam trong 'doan tau'..."
  for (auto mon_an : doan_tau) {
    
    // O day 'mon_an' da la du lieu that roi!
    // Khong can dung dau * nua.
    cout << "Mon an tren bang chuyen: " << mon_an << "\n";
  }
  return 0;
}
\end{lstlisting}
\end{frame}

\begin{frame}{5. So sánh nhanh}
    \begin{table}[]
        \begin{tabular}{|l|l|l|}
        \hline
        \textbf{Đặc điểm} & \textbf{Cách cũ (Iterator)} & \textbf{Cách mới (Auto + Range)} \\ \hline
        \textbf{Hình ảnh} & Người đi bộ từng toa & Băng chuyền tự động \\ \hline
        \textbf{Độ dài} & Rất dài, dễ sai & Ngắn gọn, súc tích \\ \hline
        \textbf{Quản lý} & Phải lo \texttt{begin}, \texttt{++} & Máy tính lo hết \\ \hline
        \textbf{Kiểm soát} & Biết rõ vị trí (địa chỉ) & Chỉ biết giá trị món đồ \\ \hline
        \end{tabular}
    \end{table}

    \begin{block}{Vấn đề nhỏ}
        Khi dùng băng chuyền, bạn biết đó là "Pizza", nhưng bạn \textbf{không biết} nó nằm ở toa số mấy. Nếu muốn kiểm soát vị trí, ta phải kết hợp \texttt{auto} với vòng lặp truyền thống.
    \end{block}
\end{frame}

\begin{frame}[fragile]{6. Kết hợp: \texttt{auto} + Vòng lặp truyền thống}
    Dùng \texttt{auto} làm "thẻ tên tắc kè hoa" thay cho kiểu dữ liệu dài dòng, nhưng vẫn giữ cơ chế điều khiển thủ công.

\begin{lstlisting}
  vector<string> doan_tau = {"Pizza", "Ga ran"};

  // 'auto' tu hieu la iterator
  for (auto it = doan_tau.begin(); it != doan_tau.end(); it++) {
    
    // Van can dung (*) vi it la con tro/dia chi
    cout << "Mon an: " << *it << "\n";
  }
\end{lstlisting}

    \begin{exampleblock}{Lợi ích}
        Giúp bạn thực hiện các thao tác nâng cao: Đi bước đôi (\texttt{it += 2}), đi lùi, hoặc sửa đổi dữ liệu tại chỗ.
    \end{exampleblock}
\end{frame}

\begin{frame}[fragile]{7. Quyền năng thay đổi thực tại}
    Vì Iterator nắm giữ địa chỉ thật, khi bạn dùng \texttt{*} để mở cửa, bạn có thể \textbf{THAY ĐỔI} món đồ bên trong.

\begin{lstlisting}
  for (auto it = doan_tau.begin(); it != doan_tau.end(); it++) {
    
    // Neu thay Ga ran
    if (*it == "Ga ran") {
        // PHEP THUAT: Bien hinh!
        *it = "Com tam"; 
    }
  }
  // Ket qua: Pizza, Com tam, Tra sua...
\end{lstlisting}

    \begin{block}{Phép ẩn dụ}
        Khác với xem TV (chỉ nhìn), Iterator cho phép bạn bước vào bếp và đổi cái bánh Pizza thành Bánh Mì.
    \end{block}
\end{frame}

\begin{frame}{8. Cái bẫy \texttt{const auto}}
    Muốn tạo iterator chỉ đọc (Read-only), nhiều bạn dùng \texttt{const auto it}. Đây là sai lầm!

    \begin{alertblock}{Cái bẫy: Chân bị xích}
        \texttt{const auto it = ...} nghĩa là: "Tạo ra một iterator và \textbf{đóng băng} nó".
        \begin{itemize}
            \item \textbf{Hậu quả:} Bạn không thể thực hiện \texttt{it++} (bước đi).
            \item \textbf{Vòng lặp:} Báo lỗi ngay lập tức.
        \end{itemize}
    \end{alertblock}

    \pause

    \begin{exampleblock}{Giải pháp: \texttt{cbegin()}}
        Muốn tạo "Khách tham quan" (đi được nhưng không sửa được), hãy dùng \textbf{\texttt{cbegin()}} và \textbf{\texttt{cend()}}.
        Khi đó \texttt{auto} sẽ trở thành \texttt{const\_iterator}.
    \end{exampleblock}
\end{frame}

\begin{frame}[fragile]{Code mẫu: \texttt{cbegin()} (An toàn tuyệt đối)}
\begin{lstlisting}
  vector<string> doan_tau = {"Pizza", "Ga ran"};

  // Dung cbegin (Const Begin)
  for (auto it = doan_tau.cbegin(); it != doan_tau.cend(); it++) {
    
    // 1. Doc: OK
    cout << *it << "\n";
    
    // 2. Sua: LOI!
    // *it = "Bun dau";  <-- May tinh se bao loi ngay
  }
\end{lstlisting}
\end{frame}

\begin{frame}{9. Phép thuật: Chèn toa (\texttt{insert})}
    \begin{block}{Quy tắc Cần Cẩu Khổng Lồ}
        Lệnh \texttt{insert(it, "Món mới")} sẽ:
        \begin{enumerate}
            \item Đẩy lùi toa hiện tại và các toa sau.
            \item Thả toa mới vào \textbf{PHÍA TRƯỚC} vị trí iterator đang đứng.
        \end{enumerate}
    \end{block}

    \begin{alertblock}{Cảnh báo động đất (Iterator Invalidation)}
        Sau khi chèn, đường ray bị xê dịch. Iterator cũ (\texttt{it}) có thể bị hỏng.
        \textbf{Lời khuyên:} Cập nhật lại iterator nếu muốn dùng tiếp.
    \end{alertblock}
\end{frame}

\begin{frame}{Check-point: Thứ tự chèn}
    \begin{block}{Câu hỏi}
        Đoàn tàu: \texttt{\{"A", "B", "C"\}}. \\
        Iterator \texttt{it} đang trỏ vào \textbf{"A"}. \\
        Gọi lệnh: \texttt{vector.insert(it, "Z");}
        
        Thứ tự mới là gì?
    \end{block}

    \pause

    \begin{exampleblock}{Đáp án}
        \textbf{\{"Z", "A", "B", "C"\}} \\
        (Chèn vào \textbf{trước} A).
    \end{exampleblock}
\end{frame}

\begin{frame}[fragile]{10. Hủy diệt (\texttt{erase}) và Cú nhảy lò xo}
    Lệnh \texttt{erase(it)} xóa phần tử tại vị trí \texttt{it}.
    
    \begin{alertblock}{Quy tắc sinh tồn}
        Sau khi xóa toa tàu, iterator hiện tại sẽ rơi xuống vực.
        \texttt{erase} sẽ trả về vị trí của \textbf{người kế tiếp} (cơ chế lò xo).
        Phải luôn hứng lấy nó: \texttt{it = v.erase(it);}
    \end{alertblock}

\begin{lstlisting}
    auto it = v.begin() + 1; // Dang o B
    // Xoa B, it tu dong nhay sang C
    it = v.erase(it); 
\end{lstlisting}
\end{frame}

\begin{frame}[fragile]{11. Cú nhảy cóc tai hại (Logic Trap)}
    Khi xóa trong vòng lặp, cẩn thận với \texttt{it++} ở tiêu đề vòng lặp.

    \begin{block}{Kịch bản lỗi}
        \begin{enumerate}
            \item \texttt{erase} tự đẩy \texttt{it} sang phần tử kế tiếp.
            \item Vòng lặp \texttt{for} lại thực hiện \texttt{it++} thêm lần nữa.
            \item \textbf{Kết quả:} Bạn nhảy qua đầu một phần tử mà không kiểm tra (Double Jump).
        \end{enumerate}
    \end{block}
    
    \textbf{Code sai:}
\begin{lstlisting}
// SAI: Vua erase day, vua it++ day -> Nhay coc
for (auto it = v.begin(); it != v.end(); it++) {
    if (*it % 2 == 0) it = v.erase(it);
}
\end{lstlisting}
\end{frame}

\begin{frame}[fragile]{Giải pháp: Điều khiển thủ công}
    Đưa \texttt{it++} vào trong thân vòng lặp để kiểm soát.

\begin{lstlisting}
    // CHUAN: Bo it++ o tieu de
    for (auto it = v.begin(); it != v.end(); /* Trong */ ) {
        
        if (*it % 2 == 0) {
            // Truong hop XOA: erase tu day it sang ke tiep
            it = v.erase(it); 
        } else {
            // Truong hop KHONG XOA: Tu buoc di
            it++;
        }
    }
\end{lstlisting}
\end{frame}

\begin{frame}{Tổng kết hành trình}
    Chúng ta đã tốt nghiệp khóa học "Người soát vé"!

    \begin{enumerate}
        \item \textbf{Iterator}: Con trỏ thông minh duyệt mảng.
        \item \textbf{\texttt{begin/end}}: Điểm đầu và Vực thẳm.
        \item \textbf{\texttt{auto}}: Thẻ tên tàng hình cho code gọn.
        \item \textbf{\texttt{cbegin}}: Chế độ "Khách tham quan" (Chỉ đọc).
        \item \textbf{\texttt{insert}}: Chèn phía trước (Cẩn thận động đất).
        \item \textbf{\texttt{erase}}: Xóa và hứng lấy vị trí mới (Tránh nhảy cóc).
    \end{enumerate}

    \begin{block}{Bước tiếp theo}
        Bạn đã sẵn sàng để khám phá các thuật toán sắp xếp và tìm kiếm với Iterator chưa?
    \end{block}
\end{frame}

\end{document}