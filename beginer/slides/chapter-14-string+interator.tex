\documentclass{beamer}
\usepackage[utf8]{inputenc}
\usepackage[T5]{fontenc} % Bắt buộc để hiển thị tiếng Việt
\usepackage[vietnamese]{babel}
\usepackage{tcolorbox}
\usepackage{listings}
\usepackage{xcolor}
\usepackage{booktabs}
\usetheme{Madrid}

\definecolor{codegreen}{rgb}{0,0.6,0}
\definecolor{codegray}{rgb}{0.5,0.5,0.5}
\definecolor{codepurple}{rgb}{0.58,0,0.82}
\definecolor{backcolour}{rgb}{0.95,0.95,0.92}

\lstdefinestyle{mystyle}{
    backgroundcolor=\color{backcolour},   
    commentstyle=\color{codegreen},
    keywordstyle=\color{magenta},
    numberstyle=\tiny\color{codegray},
    stringstyle=\color{codepurple},
    basicstyle=\ttfamily\scriptsize,
    breakatwhitespace=false,         
    breaklines=true,                 
    captionpos=b,                    
    keepspaces=true,                 
    numbers=left,                    
    numbersep=4pt,                  
    showspaces=false,                
    showstringspaces=false,
    showtabs=false,                  
    tabsize=2,
    escapechar=@,
    language=C++
}

\lstset{style=mystyle}

% Thông tin slide
\title{Bí Mật Của Đoàn Tàu Ký Tự}
\subtitle{C++ String \& Iterator}
\author{Learning How to Learn}
\date{\today}

\begin{document}

% Slide Tiêu đề
\begin{frame}
    \titlepage
\end{frame}

% Slide Mục lục
\begin{frame}{Mục Lục: Lộ Trình Khám Phá}
    \begin{block}{Chương 1: Đoàn tàu và Người dẫn đường}
        Hiểu bản chất \texttt{string} là đoàn tàu và \texttt{iterator} là người soát vé.
    \end{block}

    \begin{block}{Chương 2: Hai cách để đi hết một con đường}
        Duyệt qua đoàn tàu bằng vòng lặp \texttt{while} (Dò đường) và \texttt{for} (Lập trình sẵn).
    \end{block}
    
    \begin{block}{Chương 3: Phép thuật của Thầy phù thủy}
        Sử dụng Algorithms (\texttt{sort}, \texttt{reverse}) để thay đổi trật tự đoàn tàu.
    \end{block}
\end{frame}

% --- CHƯƠNG 1 ---
\begin{frame}{Chương 1: Đoàn Tàu và Người Dẫn Đường}
    \begin{exampleblock}{Phép ẩn dụ: Đoàn tàu "String"}
        Hãy tưởng tượng từ \textbf{"HELLO"} là một đoàn tàu gồm 5 toa xe:
        \begin{itemize}
            \item Mỗi toa chở một hành khách: H - E - L - L - O.
            \item Cả đoàn tàu chính là biến \texttt{string}.
        \end{itemize}
    \end{exampleblock}

    \vspace{0.5cm}
    \textbf{Nhân vật chính: Iterator (Người Soát Vé)}
    \begin{itemize}
        \item Không phải là toa tàu, không phải hành khách.
        \item Là người \textbf{đang đứng tại} một toa và chỉ tay vào hành khách.
        \item Trong C++ gọi là: \texttt{string::iterator}.
    \end{itemize}
\end{frame}

\begin{frame}{Hai nhà ga quan trọng: .begin() và .end()}
    Để Người Soát Vé làm việc, họ cần biết điểm đầu và điểm cuối.
    
    \begin{itemize}
        \item \textbf{\texttt{.begin()}}: Người Soát Vé đứng ngay tại toa đầu tiên (chữ 'H').
    \end{itemize}

    \begin{alertblock}{Cảnh báo quan trọng về .end()}
        \textbf{\texttt{.end()}} (Vực thẳm): Đây là chỗ dễ nhầm lẫn nhất!
        \begin{itemize}
            \item \texttt{.end()} \textbf{KHÔNG PHẢI} là toa cuối cùng (chữ 'O').
            \item Nó là mặt đất \textbf{ngay sau} toa cuối cùng.
            \item Nếu đi đến \texttt{.end()}, nghĩa là đã bước ra khỏi tàu (Hết tàu!).
        \end{itemize}
    \end{alertblock}
\end{frame}

\begin{frame}[fragile]{Code minh họa: Triệu hồi Người Soát Vé}
\begin{lstlisting}
#include <iostream>
#include <string>
using namespace std;

int main() {
    // 1. Tao ra doan tau
    string doan_tau = "HELLO";

    // 2. Thue mot Nguoi Soat Ve (khai bao iterator)
    string::iterator nguoi_soat_ve;

    // 3. Cho nguoi soat ve dung o toa dau tien
    nguoi_soat_ve = doan_tau.begin();

    // 4. Hoi nguoi soat ve: "Ong dang thay ai?"
    // Dau * giong nhu doi mat nhin vao toa tau
    cout << "Hanh khach dau tien: " << *nguoi_soat_ve << endl;

    return 0;
}
\end{lstlisting}
    \small \textit{Lưu ý: Dấu * dùng để truy cập giá trị (hành khách) tại vị trí iterator đang đứng.}
\end{frame}

\begin{frame}{Trạm dừng kiểm tra (Checkpoint 1)}
    \begin{alertblock}{Câu hỏi}
        Nếu tôi ra lệnh cho \texttt{nguoi\_soat\_ve} đi đến vị trí \texttt{doan\_tau.end()}, và tôi cố tình dùng dấu \texttt{*} để hỏi xem ông ấy thấy gì (\texttt{*nguoi\_soat\_ve}), chuyện gì sẽ xảy ra?
    \end{alertblock}

    \begin{itemize}
        \item A. Ông ấy thấy chữ cái cuối cùng của đoàn tàu (chữ 'O').
        \item B. Ông ấy thấy toa tàu đầu tiên.
        \item C. Lỗi! Ông ấy đang đứng dưới đường ray (ngoài đoàn tàu) nên không thấy ai cả.
    \end{itemize}

    \pause
    \vspace{0.5cm}
    \textbf{Đáp án: C.} Chính xác! \texttt{.end()} là vực thẳm, cố nhìn vào đó sẽ gây lỗi chương trình (crash).
\end{frame}

% --- CHƯƠNG 2 ---
\begin{frame}[fragile]{Chương 2: Vũ khí bí mật AUTO}
    Thay vì viết chức danh dài dòng: \texttt{string::iterator nguoi\_soat\_ve}, chúng ta dùng \textbf{\texttt{auto}}.
    
    \begin{block}{Tại sao dùng auto?}
        \texttt{auto} giống như bộ đồ tắc kè hoa. Máy tính tự hiểu: "Ông này đứng trên tàu String, nên chắc chắn là String Iterator".
    \end{block}

    \textbf{Code cũ:} 
    \begin{verbatim}string::iterator nguoi_soat_ve = doan_tau.begin();\end{verbatim}
    
    \textbf{Code mới (Gọn hơn):} 
    \begin{verbatim}auto nguoi_soat_ve = doan_tau.begin();\end{verbatim}
\end{frame}

\begin{frame}[fragile]{Cách 1: Phong cách "Dò Đường" (WHILE Loop)}
    Giống như đi bộ cẩn thận. Vừa đi vừa tự hỏi: "Mình đã rơi khỏi tàu chưa?".

\begin{lstlisting}
    string doan_tau = "CODING";
    auto it = doan_tau.begin(); 

    cout << "Duyet bang WHILE: ";

    // "Chung nao toi chua cham den vuc tham (.end)"
    while (it != doan_tau.end()) {
        // Lam viec: In hanh khach ra
        cout << *it << " "; 

        // QUAN TRONG: Buoc sang toa tiep theo
        it++; 
    }
\end{lstlisting}
    \textbf{Lưu ý:} \texttt{it++} là bước chân. Nếu quên, iterator đứng mãi một chỗ (vòng lặp vô tận).
\end{frame}

\begin{frame}[fragile]{Cách 2: Phong cách "Lập trình sẵn" (FOR Loop)}
    Giống như đi trên băng chuyền. Gom tất cả quy tắc vào một dòng.
    
    \begin{enumerate}
        \item \textbf{Khởi động:} \texttt{auto it = doan\_tau.begin()}
        \item \textbf{Điều kiện:} \texttt{it != doan\_tau.end()}
        \item \textbf{Bước nhảy:} \texttt{it++}
    \end{enumerate}

\begin{lstlisting}
    cout << "Duyet bang FOR: ";
    
    // Gom tat ca vao mot dong
    for (auto it = doan_tau.begin(); it != doan_tau.end(); it++) {
        cout << *it << " ";
    }
\end{lstlisting}
\end{frame}

\begin{frame}{Tổng kết & So sánh}
    \begin{block}{WHILE (Đi bộ tự do)}
        Bạn phải tự nhớ để bước đi (\texttt{it++}) ở bên trong. Linh hoạt nhưng dễ quên bước chân.
    \end{block}

    \begin{block}{FOR (Băng chuyền tự động)}
        Mọi thứ (xuất phát, đích đến, bước đi) đều được cài đặt ngay từ đầu. An toàn và khó quên hơn.
    \end{block}
\end{frame}

\begin{frame}{Trạm dừng kiểm tra (Checkpoint 2)}
    \begin{alertblock}{Câu hỏi tình huống}
        Trong vòng lặp \texttt{while}, nếu lỡ tay viết nhầm: Cho \texttt{it++} (bước đi) lên \textbf{trước} câu lệnh \texttt{cout << *it} (in ra). Chuyện gì xảy ra?
    \end{alertblock}

    \begin{itemize}
        \item A. Mọi thứ vẫn bình thường.
        \item B. Bỏ qua toa đầu, và khi đến cuối sẽ lỗi vì bước ra ngoài vực thẳm rồi mới nhìn.
        \item C. Nó sẽ in ngược từ dưới lên.
    \end{itemize}

    \pause
    \vspace{0.5cm}
    \textbf{Đáp án: B.} 
    \begin{itemize}
        \item \textbf{Bỏ sót:} Bước đi rồi mới nhìn -> Toa đầu bị bỏ qua.
        \item \textbf{Tai nạn:} Bước ra \texttt{.end()} rồi mới nhìn -> Lỗi chương trình.
    \end{itemize}
\end{frame}

% --- CHƯƠNG 3 ---
\begin{frame}[fragile]{Chương 3: Phép thuật của Thầy phù thủy (Algorithms)}
    Cần thư viện: \texttt{\#include <algorithm>}
    
    \textbf{Nguyên lý hoạt động:} Luôn hỏi 2 câu:
    \begin{enumerate}
        \item "Bắt đầu làm phép từ đâu?" (\texttt{.begin()})
        \item "Dừng lại trước chỗ nào?" (\texttt{.end()})
    \end{enumerate}

    \begin{exampleblock}{Hai phép thuật phổ biến}
        \textbf{1. Sắp xếp (Sort):} Biến đoàn tàu lộn xộn thành ngăn nắp (A-Z).\\
        \texttt{sort(s.begin(), s.end());}
        
        \textbf{2. Đảo ngược (Reverse):} Quay đầu toàn bộ đoàn tàu.\\
        \texttt{reverse(s.begin(), s.end());}
    \end{exampleblock}
\end{frame}

\begin{frame}[fragile]{Code minh họa Algorithms}
\begin{lstlisting}
#include <iostream>
#include <string>
#include <algorithm> // Bat buoc de dung phep thuat
using namespace std;

int main() {
    string s = "PYTHON";
    cout << "Ban dau: " << s << endl;

    // 1. Phep thuat Sap Xep (Sort)
    sort(s.begin(), s.end());
    cout << "Sau khi sort: " << s << endl; 
    // Ket qua: HNOPTY

    // 2. Phep thuat Dao Nguoc (Reverse)
    reverse(s.begin(), s.end());
    cout << "Sau khi reverse: " << s << endl;
    // Ket qua: YTPONH

    return 0;
}
\end{lstlisting}
\end{frame}

\begin{frame}{Thử thách tốt nghiệp (Final Boss)}
    \begin{block}{Đề bài}
        Có chuỗi \texttt{string s = "hocbai";}.\\
        Muốn sắp xếp \textbf{chỉ 3 chữ cái đầu} ("hoc") thành "cho", giữ nguyên "bai". Kết quả: "chobai".
    \end{block}

    Chọn câu lệnh đúng:
    \begin{itemize}
        \item A. \texttt{sort(s.begin(), s.end());}
        \item B. \texttt{sort(s.begin(), s.begin() + 3);}
        \item C. \texttt{sort(s.begin() + 3, s.end());}
    \end{itemize}

    \pause
    \vspace{0.5cm}
    \textbf{Đáp án: B.}
    \begin{itemize}
        \item Vị trí kết thúc luôn là vị trí đứng \textbf{ngay sau} phần tử muốn tác động.
        \item Muốn xếp 3 toa đầu -> Cần chỉ vào vạch ngăn cách sau toa thứ 3 (\texttt{begin() + 3}).
    \end{itemize}
\end{frame}

\begin{frame}{Lễ Tốt Nghiệp: Tổng Kết}
    Chúc mừng bạn đã làm chủ \textbf{Đoàn Tàu Ký Tự}!
    
    \begin{block}{Hành trang đã thu thập}
        \begin{enumerate}
            \item \textbf{Iterator:} Ngón tay chỉ vị trí (Người soát vé).
            \item \textbf{Loop (For/While):} Hai phong cách đi bộ trên tàu.
            \item \textbf{Algorithms (Sort/Reverse):} Phép thuật dựa trên điểm đầu và điểm cuối.
        \end{enumerate}
    \end{block}

    \vspace{0.5cm}
    \textbf{Bước tiếp theo?}
    \begin{itemize}
        \item Thực chiến với bài toán mini?
        \item Khám phá \textbf{Vector} (Đoàn tàu vô tận)?
    \end{itemize}
\end{frame}

\end{document}