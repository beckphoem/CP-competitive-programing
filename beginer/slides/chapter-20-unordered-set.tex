\documentclass{beamer}
\usepackage[utf8]{inputenc}
\usepackage[T5]{fontenc} 
\usepackage[vietnamese]{babel}
\usepackage{tcolorbox}
\usepackage{listings}
\usepackage{xcolor}
\usepackage{booktabs}
\usetheme{Madrid}

\definecolor{codegreen}{rgb}{0,0.6,0}
\definecolor{codegray}{rgb}{0.5,0.5,0.5}
\definecolor{codepurple}{rgb}{0.58,0,0.82}
\definecolor{backcolour}{rgb}{0.95,0.95,0.92}

\lstdefinestyle{mystyle}{
    backgroundcolor=\color{backcolour},   
    commentstyle=\color{codegreen},
    keywordstyle=\color{magenta},
    numberstyle=\tiny\color{codegray},
    stringstyle=\color{codepurple},
    basicstyle=\ttfamily\scriptsize,
    breakatwhitespace=false,         
    breaklines=true,                 
    captionpos=b,                    
    keepspaces=true,                 
    numbers=left,                    
    numbersep=4pt,                  
    showspaces=false,                
    showstringspaces=false,
    showtabs=false,                  
    tabsize=2,
    escapechar=@
}

\lstset{style=mystyle}

\title[C++ unordered\_map]{Chiếc Tủ Thần Kỳ: Bí mật tìm đồ trong chớp mắt}
\subtitle{Khám phá unordered\_map trong C++17/20}
\author{Slide Learning CPP}
\date{\today}

\begin{document}

\begin{frame}
    \titlepage
\end{frame}

\begin{frame}{Lộ trình Khám phá (The Map)}
    \begin{enumerate}
        \item \textbf{Chương 1:} \texttt{unordered\_map} là gì? (Chiếc tủ có nhãn dán).
        \item \textbf{Chương 2:} Cỗ máy băm thức ăn (Hash Function).
        \item \textbf{Chương 3:} Các thao tác cơ bản (Thêm, sửa, xóa nhãn).
        \item \textbf{Chương 4:} Điểm mới trong C++17/20 (Ngăn kéo thông minh).
        \item \textbf{Chương 5:} Iterator - Ngón tay chỉ đường.
        \item \textbf{Tổng kết:} Ứng dụng trong Lập trình thi đấu (CP).
    \end{enumerate}
\end{frame}

\begin{frame}{\texttt{unordered\_map} là gì?}
    \begin{block}{Ẩn dụ: Danh bạ điện thoại}
        Nó giống như danh bạ điện thoại. \textbf{Tên bạn bè} là cái nhãn (\textbf{Key}), còn \textbf{Số điện thoại} là món đồ (\textbf{Value}). Bạn chỉ cần nhớ tên là tìm được số ngay lập tức.
    \end{block}
    
    \begin{itemize}
        \item \textbf{Mảng (vector):} Phải tìm từ đầu đến cuối (mệt và chậm).
        \item \textbf{\texttt{unordered\_map}:} Dẫn thẳng bạn tới món đồ thông qua nhãn dán.
        \item \textbf{Tại sao "Unordered"?} Ưu tiên tốc độ tìm kiếm hơn là việc xếp hàng ngay ngắn.
    \end{itemize}
\end{frame}

\begin{frame}[fragile]{Ví dụ sống động (Khai báo và Truy xuất)}
\begin{lstlisting}[language=C++]
#include <iostream>
#include <unordered_map>
#include <string>
using namespace std;

int main() {
    // Key: Ten (string), Value: Diem so (int)
    unordered_map<string, int> bang_diem;

    // 1. Cat do vao tu (Them du lieu)
    bang_diem["An"] = 9;
    bang_diem["Ba"] = 7;
    bang_diem["Chi"] = 10;

    // 2. Lay do ra (Truy xuat)
    cout << "Diem cua Chi la: " << bang_diem["Chi"] << endl;
    return 0;
}
\end{lstlisting}
\end{frame}

\begin{frame}{Chương 2: Cỗ máy băm (Hash Function)}
    \begin{exampleblock}{Bí mật của tốc độ ánh sáng}
        Làm sao để biết chính xác ngăn kéo nào chứa tên "An"?
    \end{exampleblock}
    
    \begin{enumerate}
        \item Bạn đưa nhãn \textbf{"An"} vào máy băm.
        \item Máy băm chữ thành số (ví dụ: số \textbf{5}).
        \item Chiếc tủ mở đúng ngăn số \textbf{5} để cất/lấy đồ.
    \end{enumerate}
    
    \begin{alertblock}{Lưu ý về Key}
        Nhãn dán (Key) là \textbf{duy nhất}. Nếu bạn dán đè nhãn mới lên cùng một tên, giá trị cũ sẽ bị thay thế.
    \end{alertblock}
\end{frame}

\begin{frame}[fragile]{Thao tác Sửa, Xóa và Kiểm tra}
\begin{lstlisting}[language=C++]
// Kiem tra ton tai (C++20)
if (ds_lop.contains("Lan")) { ... }

// Kiem tra ton tai (Truoc C++20)
if (ds_lop.count("Lan")) { ... }

// Sua du lieu
ds_lop["An"] = "MA_MOI";

// Xoa du lieu
ds_lop.erase("An"); 
\end{lstlisting}
\end{frame}

\begin{frame}[fragile]{Chương 4: Siêu năng lực C++17 và C++20}
    \begin{block}{C++17: Structured Bindings}
        Giúp lấy cả Nhãn và Đồ ra cùng lúc cực kỳ gọn gàng.
    \end{block}
    
\begin{lstlisting}[language=C++]
// Dung dau ngoac vuong [ten, ma]
for (auto const& [ten, ma] : ds_lop) {
    cout << "Ban " << ten << " co ma: " << ma << endl;
}
\end{lstlisting}

    \begin{block}{C++20: Contains}
        Sử dụng \texttt{.contains("Key")} thay vì dùng \texttt{.find()} hoặc \texttt{.count()}.
    \end{block}
\end{frame}

\begin{frame}{Chương 5: Iterator - Ngón tay chỉ đường}
    \begin{itemize}
        \item \textbf{Iterator (\texttt{it}):} Đóng vai trò như một con trỏ chỉ vào từng ngăn kéo.
        \item \textbf{\texttt{it->first}:} Truy cập vào cái Nhãn (\textbf{Key}).
        \item \textbf{\texttt{it->second}:} Truy cập vào món đồ (\textbf{Value}).
    \end{itemize}
    
    \begin{alertblock}{Tại sao dùng dấu \texttt{->}?}
        Vì \texttt{it} giống như con trỏ, ta dùng \texttt{->} để "chỉ tay" vào thuộc tính bên trong ngăn kéo thay vì dùng dấu chấm \texttt{.} thông thường.
    \end{alertblock}
\end{frame}

\begin{frame}{Kiểm tra kiến thức}
    \textbf{Câu hỏi:} Nếu có 2 người cùng tên "An", ta thực hiện:
    \begin{itemize}
        \item \texttt{ds\_lop["An"] = "001";}
        \item \texttt{ds\_lop["An"] = "002";}
    \end{itemize}
    Kết quả cuối cùng trong ngăn "An" là gì?
    
    \vspace{0.5cm}
    \pause
    \begin{exampleblock}{Đáp án}
        Là \textbf{"002"}. Vì Key trong \texttt{unordered\_map} là duy nhất, giá trị sau sẽ ghi đè giá trị trước.
    \end{exampleblock}
\end{frame}

\begin{frame}[fragile]{Cạm bẫy trong Lập trình thi đấu (CP)}
    \begin{alertblock}{Anti-hash Test trên Codeforces}
        Kẻ xấu có thể tạo bộ dữ liệu gây xung đột băm (Collision), khiến tốc độ giảm từ $O(1)$ xuống $O(n)$, gây lỗi \textbf{TLE}.
    \end{alertblock}
    
    \textbf{Giải pháp: Custom Hash với Random Seed}
\begin{lstlisting}[language=C++]
struct custom_hash {
    size_t operator()(uint64_t x) const {
        static const uint64_t FIXED_RANDOM = chrono::steady_clock::now().time_since_epoch().count();
        return splitmix64(x + FIXED_RANDOM);
    }
};
// Khai bao an toan
unordered_map<long long, int, custom_hash> safe_map;
\end{lstlisting}
\end{frame}

\begin{frame}{Thử thách cuối cùng: Inventory Game}
    \begin{block}{Bài toán}
        Bạn nhặt thêm 1 thanh "Kiem" trong khi đã có sẵn 2 thanh trong túi đồ (\texttt{unordered\_map<string, int> inventory}). Bạn sẽ xử lý thế nào?
    \end{block}
    \pause
    \begin{exampleblock}{Gợi ý giải pháp}
        \texttt{inventory["Kiem"] += 1;} \\
        Chiếc tủ sẽ tìm ngăn "Kiem", lấy số lượng cũ (2) cộng thêm 1 và cất lại số 3 vào ngăn đó!
    \end{exampleblock}
\end{frame}

\end{document}