\documentclass{beamer}
\usepackage[utf8]{inputenc}
\usepackage[vietnamese]{babel}
\usepackage{tcolorbox}
\usepackage{listings}
\usepackage{xcolor}

\usetheme{Madrid}
\usecolortheme{default}

% Colors for code blocks
\definecolor{codegreen}{rgb}{0,0.6,0}
\definecolor{codegray}{rgb}{0.5,0.5,0.5}
\definecolor{codepurple}{rgb}{0.58,0,0.82}
\definecolor{backcolour}{rgb}{0.95,0.95,0.92}

\lstdefinestyle{mystyle}{
    backgroundcolor=\color{backcolour},   
    commentstyle=\color{codegreen},
    keywordstyle=\color{magenta},
    numberstyle=\tiny\color{codegray},
    stringstyle=\color{codepurple},
    basicstyle=\ttfamily\scriptsize,
    breakatwhitespace=false,         
    breaklines=true,                 
    captionpos=b,                    
    keepspaces=true,                 
    numbers=left,                    
    numbersep=4pt,                  
    showspaces=false,                
    showstringspaces=false,
    showtabs=false,                  
    tabsize=2,
    escapechar=@
}

\lstset{style=mystyle}

% Meta information
\title{Bài 4: Vòng lặp While (While Loop)}
\author{CPP}
\date{}

\begin{document}

% --- Slide Tiêu đề ---
\begin{frame}
    \titlepage
\end{frame}

% --- Slide Giới thiệu ---
\begin{frame}{Lời chào và Bức tranh toàn cảnh}
  

    \vspace{0.5cm}

    \begin{alertblock}{🌟 BỨC TRANH TOÀN CẢNH: SỨC MẠNH TỰ ĐỘNG HÓA}
        Hãy tưởng tượng bạn đang xúc cát để lấp đầy một cái hố.
        \begin{itemize}
            \item \textbf{Manual (Không vòng lặp):} Tự nhủ "Xúc một xẻng. Xúc thêm một xẻng...". Rất mệt và dễ quên đếm.
            \item \textbf{Automation (Có vòng lặp):} Ra lệnh cho Robot một lần duy nhất: \textbf{"Này Robot, KHI MÀ (WHILE) cái hố chưa đầy, hãy cứ xúc cát liên tục cho ta!"}
        \end{itemize}
    \end{alertblock}
\end{frame}

% --- Slide Lộ trình ---
\begin{frame}{🗺️ Lộ trình khám phá (Roadmap)}
    Dựa trên cấu trúc bài học, chúng ta có 2 "khối" kiến thức (chunks) quan trọng:

    \vspace{0.5cm}

    \begin{itemize}
        \item \textbf{Chương 1: "Người Gác Cổng" Nghiêm Khắc}
        \begin{itemize}
            \item Cơ chế hoạt động & Tại sao gọi là "Người gác cổng".
            \item Phân tích cú pháp: `while (condition) \{ code \}`.
        \end{itemize}
        
        \vspace{0.3cm}
        
        \item \textbf{Chương 2: Nghệ thuật Đếm lùi (Ứng dụng thực tế)}
        \begin{itemize}
            \item Ví dụ "Countdown" (Đếm ngược).
            \item Do/While Loop: Sự khác biệt với While.
            \item Những cạm bẫy cần tránh (Vòng lặp vô tận).
        \end{itemize}
    \end{itemize}
\end{frame}

% --- Slide Chương 1 ---
\begin{frame}[fragile]{Chương 1: "Người Gác Cổng" Nghiêm Khắc}
    Tại sao gọi là \textbf{"Người Gác Cổng"}?
    \begin{itemize}
        \item Vòng lặp `while` làm việc rất nguyên tắc: \textbf{Kiểm tra giấy tờ trước, cho vào sau.}
        \item Nếu ngay từ đầu điều kiện đã sai, cánh cổng sẽ không bao giờ mở ra.
    \end{itemize}

    \begin{exampleblock}{Cú pháp}
\begin{lstlisting}[language=C++]
// @Điều kiện được kiểm tra ĐẦU TIÊN@
while (condition) {
  // @Khối lệnh này chỉ chạy khi condition là TRUE@
  // @code to be executed@
}
\end{lstlisting}
    \end{exampleblock}
\end{frame}

% --- Slide Chương 2: Đếm ngược ---
\begin{frame}[fragile]{Chương 2: Nghệ thuật Đếm lùi (The Countdown)}
    \textbf{Phép ẩn dụ:} Đồng hồ đếm ngược đêm Giao thừa.
    
    Hãy tưởng tượng bạn đang chờ đợi khoảnh khắc bắn pháo hoa:

    \begin{exampleblock}{Code mô phỏng}
\begin{lstlisting}[language=C++]
int countdown = 3;  // @Bắt đầu từ số 3@

while (countdown > 0) {       // @Khi nào vẫn còn thời gian (số > 0)@
  cout << countdown << "\n";  // @Hét to số đó lên!@
  countdown--;                // @Thời gian trôi đi (Giảm 1 đơn vị)@
}

cout << "Happy New Year!!\n"; // @Pháo hoa nổ tung!@
\end{lstlisting}
    \end{exampleblock}
    
    Dòng chữ "Happy New Year!!" nằm \textit{ngoài} vòng lặp nên nó được chạy ngay khi vòng lặp kết thúc.
\end{frame}

% --- Slide Phân tích chuyển động ---
\begin{frame}[fragile]{Phân tích chuyển động: Đi xuống cầu thang}
    Ở đây chúng ta dùng `countdown--` thay vì `i++`.
    
    \begin{itemize}
        \item \textbf{`--` (Decrement):} Hãy hình dung nó giống như bạn đang đi xuống cầu thang.
        \item Đang ở bậc 3 $\rightarrow$ Hét "3" $\rightarrow$ Bước xuống (còn 2).
        \item Đang ở bậc 2 $\rightarrow$ Hét "2" $\rightarrow$ Bước xuống (còn 1).
        \item Đang ở bậc 1 $\rightarrow$ Hét "1" $\rightarrow$ Bước xuống (còn 0).
        \item \textbf{Tại bậc 0:} Người gác cổng nhìn thấy số 0. Điều kiện `0 > 0` là \textbf{SAI}. Cổng đóng lại.
    \end{itemize}
\end{frame}

% --- Slide Do/While ---
\begin{frame}[fragile]{Mở rộng: Vòng lặp "Làm trước, Nghĩ sau" (Do/While)}
    Nếu `While` là "Người Gác Cổng" cẩn trọng, thì \textbf{`Do/While`} là một \textbf{"Gã Liều Lĩnh"}.
    
    \begin{block}{Triết lý của Do/While}
        \textit{"Cứ làm thử một lần đi đã, rồi tính tiếp!"}
    \end{block}

    \begin{exampleblock}{Cú pháp Do/While}
\begin{lstlisting}[language=C++]
do {
  // @Làm gì đó đi! (Code block)@
}
while (condition);
\end{lstlisting}
    \end{exampleblock}

    \textbf{Sự khác biệt cốt tử:}
    \begin{itemize}
        \item \textbf{While:} Kiểm tra vé \textbf{trước khi} vào cổng.
        \item \textbf{Do/While:} Cho phép vào chơi \textbf{ít nhất một lần}, sau đó mới kiểm tra vé ở cổng ra.
    \end{itemize}
\end{frame}

% --- Slide Kiểm tra sự thấu hiểu ---
\begin{frame}[fragile]{🧠 Kiểm tra sự thấu hiểu (Check-in)}
    Hãy giải câu đố sau để phân biệt hai loại vòng lặp:
    
    \begin{alertblock}{Câu đố}
        Giả sử tôi có biến `int i = 0`.
        \begin{enumerate}
            \item \textbf{Trường hợp A:} Dùng `while (i > 0) \{ cout << "Hello"; \}`
            \item \textbf{Trường hợp B:} Dùng `do \{ cout << "Hello"; \} while (i > 0);`
        \end{enumerate}
        \textbf{Trong mỗi trường hợp, chữ "Hello" được in ra màn hình mấy lần?}
    \end{alertblock}

    \pause
    \vspace{0.5cm}
    \textbf{Đáp án:}
    \begin{itemize}
        \item \textbf{Trường hợp A (While):} "Người gác cổng" chặn ngay từ đầu ($0 > 0$ sai). \textbf{0 lần.}
        \item \textbf{Trường hợp B (Do/While):} "Gã liều lĩnh" cho làm trước 1 lần rồi mới kiểm tra. \textbf{1 lần.}
    \end{itemize}
\end{frame}

% --- Slide Mảnh ghép cuối cùng ---
\begin{frame}[fragile]{🎁 Mảnh ghép cuối cùng: Ứng dụng thực tế}
    Khi nào dùng cái nào? Hãy học tư duy giải quyết vấn đề.

    \begin{columns}
        \begin{column}{0.48\textwidth}
            \begin{block}{1. Do/While $\rightarrow$ Nhập mật khẩu}
                Bạn luôn muốn người dùng nhập \textit{ít nhất một lần}.
\begin{lstlisting}[language=C++]
int password;
do {
  cout << "Moi nhap pass: ";
  cin >> password; 
}
while (password != 1234);
// @Sai (khác 1234) thì nhập lại!@
\end{lstlisting}
                \small \textit{Ẩn dụ: Ăn thử một miếng rồi mới quyết định ăn tiếp hay không.}
            \end{block}
        \end{column}
        
        \begin{column}{0.48\textwidth}
            \begin{block}{2. While $\rightarrow$ Kiểm tra bình xăng}
                Trước khi nổ máy, phải kiểm tra xăng.
                \begin{itemize}
                    \item Xăng hết $\rightarrow$ Không đi mét nào (0 lần).
                    \item Còn xăng $\rightarrow$ Lái xe, xăng vơi dần.
                \end{itemize}
                \vspace{0.5cm}
                \small \textit{Ẩn dụ: An toàn là trên hết. Kiểm tra trước khi hành động.}
            \end{block}
        \end{column}
    \end{columns}
\end{frame}

% --- Slide Tổng kết ---
\begin{frame}{🏁 Tổng kết hành trình}
    Chúng ta đã hoàn thành xuất sắc bài học về \textbf{Vòng lặp While}!

    \begin{enumerate}
        \item \textbf{While Loop:} Người gác cổng nghiêm khắc. Kiểm tra trước, làm sau. (Dùng khi có thể không cần làm lần nào).
        \item \textbf{Do/While Loop:} Gã liều lĩnh. Làm trước, kiểm tra sau. (Dùng khi bắt buộc phải làm ít nhất 1 lần).
        \item \textbf{Vòng lặp vô tận:} Cơn ác mộng khi quên thay đổi biến đếm (quên `i++` hoặc `i--`).
    \end{enumerate}

    \vspace{1cm}
    
    \begin{block}{Bước tiếp theo}
        Thông thường, sau khi học `While` (không biết trước số lần), chúng ta sẽ chuyển sang \textbf{`For Loop`}. Bạn có muốn tiếp tục sang bài For Loop ngay không?
    \end{block}
\end{frame}

\end{document}