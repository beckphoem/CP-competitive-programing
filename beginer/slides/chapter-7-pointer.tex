\documentclass{beamer}
\usepackage[utf8]{inputenc}
\usepackage[T5]{fontenc} % Bắt buộc để hiển thị tiếng Việt
\usepackage[vietnamese]{babel}
\usepackage{tcolorbox}
\usepackage{listings}
\usepackage{xcolor}
\usepackage{booktabs}
\usetheme{Madrid}

% Định nghĩa màu sắc cho code
\definecolor{codegreen}{rgb}{0,0.6,0}
\definecolor{codegray}{rgb}{0.5,0.5,0.5}
\definecolor{codepurple}{rgb}{0.58,0,0.82}
\definecolor{backcolour}{rgb}{0.95,0.95,0.92}

% Cấu hình hiển thị code
\lstdefinestyle{mystyle}{
    backgroundcolor=\color{backcolour},   
    commentstyle=\color{codegreen},
    keywordstyle=\color{magenta},
    numberstyle=\tiny\color{codegray},
    stringstyle=\color{codepurple},
    basicstyle=\ttfamily\scriptsize,
    breakatwhitespace=false,         
    breaklines=true,                 
    captionpos=b,                    
    keepspaces=true,                 
    numbers=left,                    
    numbersep=4pt,                  
    showspaces=false,                
    showstringspaces=false,
    showtabs=false,                  
    tabsize=2,
    escapechar=@,
    inputencoding=utf8,
    extendedchars=true,
    literate={đ}{{\dj}}1 {Đ}{{\DJ}}1 {á}{{\'a}}1 {à}{{\`a}}1 {ả}{{\'a}}1 {ã}{{\~a}}1 {ạ}{{\d{a}}}1
    {ă}{{\u{a}}}1 {ắ}{{\'a}}1 {ằ}{{\`a}}1 {ẳ}{{\'a}}1 {ẵ}{{\~a}}1 {ặ}{{\d{a}}}1
    {â}{{\^a}}1 {ấ}{{\'a}}1 {ầ}{{\`a}}1 {ẩ}{{\'a}}1 {ẫ}{{\~a}}1 {ậ}{{\d{a}}}1
    {é}{{\'e}}1 {è}{{\`e}}1 {ẻ}{{\'e}}1 {ẽ}{{\~e}}1 {ẹ}{{\d{e}}}1
    {ê}{{\^e}}1 {ế}{{\'e}}1 {ề}{{\`e}}1 {ể}{{\'e}}1 {ễ}{{\~e}}1 {ệ}{{\d{e}}}1
    {í}{{\'i}}1 {ì}{{\`i}}1 {ỉ}{{\'i}}1 {ĩ}{{\~i}}1 {ị}{{\d{i}}}1
    {ó}{{\'o}}1 {ò}{{\`o}}1 {ỏ}{{\'o}}1 {õ}{{\~o}}1 {ọ}{{\d{o}}}1
    {ô}{{\^o}}1 {ố}{{\'o}}1 {ồ}{{\`o}}1 {ổ}{{\'o}}1 {ỗ}{{\~o}}1 {ộ}{{\d{o}}}1
    {ơ}{{\'o}}1 {ờ}{{\`o}}1 {ở}{{\'o}}1 {ỡ}{{\~o}}1 {ợ}{{\d{o}}}1
    {ú}{{\'u}}1 {ù}{{\`u}}1 {ủ}{{\'u}}1 {ũ}{{\~u}}1 {ụ}{{\d{u}}}1
    {ư}{{\'u}}1 {ừ}{{\`u}}1 {ử}{{\'u}}1 {ữ}{{\~u}}1 {ự}{{\d{u}}}1
    {ý}{{\'y}}1 {ỳ}{{\`y}}1 {ỷ}{{\'y}}1 {ỹ}{{\~y}}1 {ỵ}{{\d{y}}}1
}

\lstset{style=mystyle}

\title{C++ Pointers (Con trỏ)}
\subtitle{Hành trình truy tìm kho báu trong bộ nhớ}
\author{Slide Learning CPP}
\date{\today}

\begin{document}

% --- Slide Tiêu đề ---
\frame{\titlepage}

% --- Slide Giới thiệu ---
\begin{frame}{Bức tranh toàn cảnh: Dãy tủ đồ khổng lồ}
    \begin{block}{Ẩn dụ: Bộ nhớ máy tính (RAM)}
        Tưởng tượng bộ nhớ máy tính giống như một hành lang dài vô tận chứa hàng triệu cái \textbf{tủ đồ (lockers)}.
    \end{block}

    \begin{itemize}
        \item Mỗi tủ chứa một đồ vật (Dữ liệu/Biến).
        \item Mỗi tủ có một số thứ tự sơn bên ngoài (Địa chỉ bộ nhớ).
    \end{itemize}

    \begin{alertblock}{Định nghĩa Con trỏ (Pointer)}
        Con trỏ \textbf{không phải} là cái tủ chứa đồ. Nó là một mẩu giấy ghi lại \textbf{"số thứ tự của cái tủ"} để bạn tìm thấy kho báu.
    \end{alertblock}
\end{frame}

% --- Slide Lộ trình ---
\begin{frame}{Lộ trình khám phá}
    Chúng ta sẽ đi qua 3 chương chính:
    \vspace{0.5cm}
    \begin{enumerate}
        \item \textbf{Chương 1: Bí mật của những con số (Creating Pointers)} \\
        Hiểu về "Địa chỉ nhà" và cách tạo tấm bản đồ.
        \item \textbf{Chương 2: Mở rương kho báu (Dereferencing)} \\
        Cách dùng bản đồ để lấy đồ vật ra (dấu sao \texttt{*}).
        \item \textbf{Chương 3: Phép thuật thay đổi từ xa (Modifying)} \\
        Cách thay đổi đồ vật trong tủ mà không chạm vào tủ.
    \end{enumerate}
\end{frame}

% --- CHƯƠNG 1 ---
\begin{frame}[fragile]{Chương 1: Tạo Con trỏ - Địa chỉ nhà}
    \begin{block}{Toán tử \texttt{\&} (Address-of)}
        Để hỏi "Số tủ của bạn là bao nhiêu?", ta dùng ký hiệu \texttt{\&}. Nó móc vào biến để kéo ra địa chỉ nhà.
    \end{block}

    \begin{exampleblock}{Ẩn dụ quan trọng}
        \begin{itemize}
            \item Biến (\texttt{string monAn}): Cái \textbf{Hộp} chứa Pizza.
            \item Con trỏ (\texttt{string* ptr}): Ngón tay \textbf{Chỉ đường}, giữ địa chỉ nơi cất Pizza.
        \end{itemize}
    \end{exampleblock}

\begin{lstlisting}[language=C++, caption={Ví dụ tạo con trỏ}]
string monAn = "Pizza"; // 1. Tao bien (Cai hop)

// 2. Tao Con tro (Giay ghi so tu)
// Dau * o khai bao: "Toi la mot con tro"
// Dau & o day: "Lay dia chi cua monAn dua cho toi"
string* ptr = &monAn; 

cout << monAn;  // In ra: Pizza
cout << &monAn; // In ra: 0x6dfed4 (Dia chi)
cout << ptr;    // In ra: 0x6dfed4 (Giong het dong tren)
\end{lstlisting}
\end{frame}

% --- Quiz Chương 1 ---
\begin{frame}[fragile]{Kiểm tra nhanh: Chương 1}
    \begin{alertblock}{Câu đố}
        Giả sử có dòng code:
\begin{lstlisting}[language=C++]
int tuoi = 14;
int* p = &tuoi;
cout << p;
\end{lstlisting}
        Máy tính sẽ in ra số \textbf{14} hay một \textbf{địa chỉ bộ nhớ} (ví dụ: 0x7ff...)?
    \end{alertblock}

    \pause
    \vspace{0.5cm}
    \textbf{Đáp án:}
    Chính xác! Nó sẽ in ra \textbf{địa chỉ bộ nhớ}.
    \begin{itemize}
        \item \texttt{p} chỉ là mảnh giấy ghi địa chỉ.
        \item Muốn lấy số 14, ta cần sang Chương 2.
    \end{itemize}
\end{frame}

% --- CHƯƠNG 2 ---
\begin{frame}[fragile]{Chương 2: Mở rương kho báu (Dereference)}
    \begin{block}{Cánh cửa thần kỳ: Dấu sao \texttt{*}}
        Khi đặt dấu \texttt{*} trước tên con trỏ (ví dụ: \texttt{*ptr}), nó biến thành một hành động: \textbf{"Đến địa chỉ đó và LẤY đồ vật ra!"}.
    \end{block}

    \begin{exampleblock}{Hai khuôn mặt của dấu Sao}
        \begin{itemize}
            \item Khi tạo biến (\texttt{string* ptr}): \textbf{DANH TỪ} (Tôi là con trỏ).
            \item Khi sử dụng (\texttt{*ptr}): \textbf{ĐỘNG TỪ} (Mở tủ ra lấy đồ!).
        \end{itemize}
    \end{exampleblock}

\begin{lstlisting}[language=C++, caption={Sử dụng Dereference}]
string monAn = "Pizza";
string* ptr = &monAn; 

cout << ptr;  // In ra: 0x6dfed4 (Dia chi - To giay)

// Phep thuat Dereference:
cout << *ptr; // In ra: Pizza (Gia tri - Do an)
\end{lstlisting}
\end{frame}

% --- Quiz Chương 2 ---
\begin{frame}[fragile]{Kiểm tra nhanh: Chương 2}
    \begin{alertblock}{Trắc nghiệm}
        Cho đoạn code:
\begin{lstlisting}[language=C++]
int diemSo = 10;
int* p = &diemSo;
\end{lstlisting}
        Dòng lệnh nào sẽ in ra số \textbf{10}?
        \begin{itemize}
            \item A. \texttt{cout << \&diemSo;}
            \item B. \texttt{cout << p;}
            \item C. \texttt{cout << *p;}
        \end{itemize}
    \end{alertblock}

    \pause
    \vspace{0.5cm}
    \textbf{Đáp án đúng: C (`cout << *p;`)} \\
    Bạn đã dùng chìa khóa (\texttt{*}) để mở tủ lấy điểm số ra.
\end{frame}

% --- CHƯƠNG 3 ---
\begin{frame}[fragile]{Chương 3: Phép thuật thay đổi từ xa}
    \begin{block}{Cơ chế hoạt động}
        Nếu bạn dùng chìa khóa (\texttt{*ptr}), mở tủ ra và thay Pizza bằng Hamburger, thì người giữ tủ (\texttt{monAn}) cũng sẽ thấy Hamburger.
    \end{block}

    \begin{exampleblock}{Quy tắc: Một nhà, hai cửa}
        \begin{itemize}
            \item Biến (\texttt{monAn}): Cửa chính.
            \item Con trỏ (\texttt{*ptr}): Cửa phụ.
            \item Thay đổi ở cửa phụ sẽ ảnh hưởng trực tiếp đến ngôi nhà.
        \end{itemize}
    \end{exampleblock}

\begin{lstlisting}[language=C++, caption={Thay đổi giá trị qua con trỏ}]
string monAn = "Pizza";
string* ptr = &monAn;

cout << monAn; // In ra: Pizza

// Thay doi qua con tro
*ptr = "Hamburger";

// Kiem tra lai bien goc
cout << monAn; // In ra: Hamburger (Da bi thay doi!)
\end{lstlisting}
\end{frame}

% --- Tổng kết ---
\begin{frame}{Tổng kết hành trình}
    Chúc mừng bạn đã thu thập đủ bộ công cụ Con trỏ (Pointers)!

    \begin{table}
        \centering
        \begin{tabular}{l l l}
            \toprule
            \textbf{Công cụ} & \textbf{Tên gọi} & \textbf{Chức năng} \\
            \midrule
            \texttt{\&} & Address-of & Tìm số nhà (địa chỉ) \\
            \texttt{type* ptr} & Khai báo & Tạo mẩu giấy ghi địa chỉ \\
            \texttt{*ptr} & Dereference & Mở tủ lấy/sửa đồ vật \\
            \bottomrule
        \end{tabular}
    \end{table}

    \begin{block}{Bước tiếp theo}
        Cẩn thận với "Con trỏ rỗng" (Null Pointer) - cái bẫy nguy hiểm tiếp theo!
    \end{block}
\end{frame}

% --- Thử thách cuối cùng ---
\begin{frame}[fragile]{Thử thách: Code Challenge}
    \textbf{Đề bài:}
    \begin{enumerate}
        \item Tạo biến \texttt{diemThi} bằng 9.
        \item Tạo con trỏ \texttt{p} trỏ vào \texttt{diemThi}.
        \item Dùng \texttt{p} sửa điểm thành 10.
        \item In \texttt{diemThi} ra màn hình.
    \end{enumerate}

    \pause
    \begin{exampleblock}{Lời giải tham khảo}
\begin{lstlisting}[language=C++]
int main() {
    int diemThi = 9;    // 1. Tao bien
    int* p = &diemThi;  // 2. Tao con tro
    
    *p = 10;            // 3. Sua diem (phep thuat!)
    
    cout << diemThi;    // 4. In ra: 10
    return 0;
}
\end{lstlisting}
    \end{exampleblock}
\end{frame}

\end{document}