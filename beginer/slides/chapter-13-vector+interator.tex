\documentclass{beamer}
\usepackage[utf8]{inputenc}
\usepackage[T5]{fontenc} % Bắt buộc để hiển thị tiếng Việt
\usepackage[vietnamese]{babel}
\usepackage{tcolorbox}
\usepackage{listings}
\usepackage{xcolor}
\usepackage{booktabs}
\usetheme{Madrid}

% Định nghĩa màu sắc cho code
\definecolor{codegreen}{rgb}{0,0.6,0}
\definecolor{codegray}{rgb}{0.5,0.5,0.5}
\definecolor{codepurple}{rgb}{0.58,0,0.82}
\definecolor{backcolour}{rgb}{0.95,0.95,0.92}

% Cấu hình hiển thị code
\lstdefinestyle{mystyle}{
    backgroundcolor=\color{backcolour},   
    commentstyle=\color{codegreen},
    keywordstyle=\color{magenta},
    numberstyle=\tiny\color{codegray},
    stringstyle=\color{codepurple},
    basicstyle=\ttfamily\scriptsize,
    breakatwhitespace=false,         
    breaklines=true,                 
    captionpos=b,                    
    keepspaces=true,                 
    numbers=left,                    
    numbersep=4pt,                  
    showspaces=false,                
    showstringspaces=false,
    showtabs=false,                  
    tabsize=2,
    escapechar=@
}

\lstset{style=mystyle}

% Thông tin bài giảng
\title[Bí Kíp Luyện Rồng C++]{Bí Kíp Luyện Rồng C++: Vector \& Iterator}
\subtitle{Khám phá Chiếc Túi Thần Kỳ và Người Soát Vé}
\author{Slide Learning Cpp}
\date{\today}

\begin{document}

% --- Slide Tiêu đề ---
\begin{frame}
    \titlepage
\end{frame}

% --- Lộ trình ---
\begin{frame}{Bản Đồ Kho Báu C++ (Lộ trình học tập)}
    \begin{block}{Hành trình 4 Chương}
    \begin{itemize}
        \item \textbf{Chương 1: Vector - Đoàn Tàu Co Giãn:} Tại sao cần Vector thay vì mảng thường?
        \item \textbf{Chương 2: Iterator - Người Soát Vé Tận Tụy:} Hiểu về con trỏ thông minh.
        \item \textbf{Chương 3: Hành Trình Điểm Danh:} Duyệt Vector bằng vòng lặp.
        \item \textbf{Chương 4: Phép Thuật Algorithms:} Sắp xếp và Tìm kiếm.
    \end{itemize}
    \end{block}
\end{frame}

% =========================================================================
% CHƯƠNG 1
% =========================================================================
\section{Chương 1: Vector}

\begin{frame}{Chương 1: Vector - Đoàn Tàu Co Giãn}
    \begin{columns}
        \column{0.5\textwidth}
        \begin{alertblock}{Mảng (Array) Cũ Kỹ}
            Giống như khay đựng trứng nhựa cứng.
            \begin{itemize}
                \item Cố định số lượng (VD: 6 lỗ).
                \item Muốn thêm quả thứ 7? Phải mua khay mới!
            \end{itemize}
        \end{alertblock}

        \column{0.5\textwidth}
        \begin{exampleblock}{Vector (Mảng Động)}
            Giống như Đoàn Tàu Phép Thuật.
            \begin{itemize}
                \item Tự động thêm toa khi có khách mới.
                \item Có thể dài ra vô tận.
            \end{itemize}
        \end{exampleblock}
    \end{columns}
    
    \vspace{0.5cm}
    \textbf{Cốt lõi:} Vector là danh sách có thể \textbf{tự động co giãn} kích thước.
\end{frame}

\begin{frame}[fragile]{Câu thần chú: Triệu hồi Vector}
    \begin{enumerate}
        \item \textbf{Mở sách phép (Thư viện):}
\begin{lstlisting}[language=C++]
#include <vector> 
using namespace std;
\end{lstlisting}
        
        \item \textbf{Tạo đoàn tàu (Khai báo):}
\begin{lstlisting}[language=C++]
vector<string> vu_khi; // Doan tau rong
\end{lstlisting}
        
        \item \textbf{Nối thêm toa (Push Back):}
        Đẩy món đồ vào cửa sau (`back`) của đoàn tàu.
\begin{lstlisting}[language=C++]
vu_khi.push_back("Kiem Go");   // Toa 1
vu_khi.push_back("Cung Ten");  // Toa 2 noi sau
vu_khi.push_back("Khien Sat"); // Toa 3 noi tiep
\end{lstlisting}
    \end{enumerate}
\end{frame}

\begin{frame}[fragile]{Kiểm tra nhanh (Quick Check)}
    \begin{exampleblock}{Tình huống}
        Đoàn tàu \texttt{vu\_khi} đang có 3 món: Kiếm, Cung, Khiên.
        Bạn hô thần chú:
        \begin{lstlisting}[language=C++]
vu_khi.push_back("Mu Giap");
        \end{lstlisting}
    \end{exampleblock}

    \textbf{Câu hỏi:} 
    \begin{enumerate}
        \item Món "Mũ Giáp" nằm ở vị trí nào? (Đầu, Giữa hay Cuối?)
        \item Tổng cộng đoàn tàu có bao nhiêu toa?
    \end{enumerate}

    \pause
    \begin{tcolorbox}[colback=green!5!white,colframe=green!75!black,title=Đáp án]
        1. Người mới đến luôn xếp hàng ở \textbf{Cuối tàu}.\\
        2. Tổng cộng có \textbf{4 toa}.
    \end{tcolorbox}
\end{frame}

% =========================================================================
% CHƯƠNG 2
% =========================================================================
\section{Chương 2: Iterator}

\begin{frame}{Chương 2: Iterator - Người Soát Vé Tận Tụy}
    Tại sao không dùng số thứ tự (Index) \texttt{vu\_khi[0]}?
    \begin{itemize}
        \item Index giống như dịch chuyển tức thời (nhanh nhưng dễ nhầm chỗ).
        \item \textbf{Iterator} là Người Soát Vé đi bộ từng bước.
    \end{itemize}

    \begin{block}{Cách hoạt động}
    \begin{itemize}
        \item Bắt đầu từ đầu tàu.
        \item Bước từng bước sang toa kế (`++`).
        \item Biết chính xác khi nào hết đường ray (`end()`).
    \end{itemize}
    \end{block}
\end{frame}

\begin{frame}[fragile]{Ba hành động của Người Soát Vé}
    Khai báo: \texttt{vector<string>::iterator it;}
    
    \begin{enumerate}
        \item \textbf{Bắt đầu hành trình (begin):}
\begin{lstlisting}[language=C++]
it = vu_khi.begin(); // Dung o cua toa dau tien
\end{lstlisting}
        
        \item \textbf{Mở cửa kiểm tra (*it):}
        Dùng dấu sao \texttt{*} để xem bên trong.
\begin{lstlisting}[language=C++]
cout << *it; // Hien ra "Kiem Go"
\end{lstlisting}
        
        \item \textbf{Bước sang toa kế (it++):}
\begin{lstlisting}[language=C++]
it++; // Buoc sang toa "Cung Ten"
\end{lstlisting}
    \end{enumerate}
\end{frame}

\begin{frame}{Bí ẩn về điểm kết thúc (end)}
    \begin{alertblock}{Cảnh báo quan trọng!}
        \texttt{vu\_khi.end()} \textbf{KHÔNG PHẢI} là toa cuối cùng.
    \end{alertblock}
    
    Hãy tưởng tượng:
    \begin{itemize}
        \item \texttt{begin()}: Bậc thềm toa 1.
        \item \texttt{end()}: \textbf{Mặt đất sân ga}, ngay sau khi bước xuống khỏi toa cuối.
    \end{itemize}
    
    \vspace{0.5cm}
    Khi \texttt{it == end()} nghĩa là "Chân đã chạm đất", đã đi hết tàu.
\end{frame}

\begin{frame}[fragile]{Kiểm tra nhanh (Quick Check)}
    Đoàn tàu: [Toa 1: Kiếm] [Toa 2: Cung] [Toa 3: Khiên] [Toa 4: Mũ]
    
\begin{lstlisting}[language=C++]
vector<string>::iterator it = vu_khi.begin(); // (1)
it++; // (2) Buoc 1
it++; // (3) Buoc 2
cout << *it; // (4) Mo cua
\end{lstlisting}

    \textbf{Câu hỏi:} Người soát vé đang đứng trước món đồ nào?
    
    \pause
    \begin{tcolorbox}[colback=green!5!white,colframe=green!75!black,title=Đáp án]
        Sau 2 lần bước, ông ấy đứng ở \textbf{Toa 3: Khiên Sắt}.
    \end{tcolorbox}
    
    \pause
    \textbf{Câu hỏi phụ:} Nếu bước thêm 2 bước nữa (\texttt{it++} 2 lần), ông ấy đứng ở đâu?
    \pause
    $\rightarrow$ Chạm đất (\texttt{end()}).
\end{frame}

% =========================================================================
% CHƯƠNG 3
% =========================================================================
\section{Chương 3: Duyệt Vector}

\begin{frame}[fragile]{Duyệt Vector: Cách 1 (While)}
    \textbf{Quy tắc:} "Chừng nào chân ông chưa chạm đất (\texttt{!= end()}), thì cứ kiểm tra và đi tiếp."

\begin{lstlisting}[language=C++]
vector<string>::iterator it;
it = vu_khi.begin();         // 1. Xuat phat

while (it != vu_khi.end()) { // 2. Kiem tra: Chua cham dat?
    cout << *it << endl;     // 3. Hanh dong: Doc ten
    it++;                    // 4. Di chuyen: Buoc tiep
}
\end{lstlisting}
\end{frame}

\begin{frame}[fragile]{Duyệt Vector: Cách 2 (For)}
    Gói gọn 3 bước vào 1 dòng code (Băng chuyền tự động).

\begin{lstlisting}[language=C++]
// For (Khoi hanh ; Dieu kien dung ; Buoc di)
for (it = vu_khi.begin(); it != vu_khi.end(); it++) {
    cout << *it << endl;
}
\end{lstlisting}

    \begin{itemize}
        \item \textbf{Khởi hành:} \texttt{it = begin()}
        \item \textbf{Điều kiện dừng:} \texttt{it != end()} (Chưa chạm đất)
        \item \textbf{Bước đi:} \texttt{it++}
    \end{itemize}
\end{frame}

\begin{frame}[fragile]{Thám tử tìm lỗi sai}
    Đoạn code sau có một lỗi sai kinh điển trong thế giới Iterator:

\begin{lstlisting}[language=C++]
// Tim loi sai o dong nay:
for (it = vu_khi.begin(); it < vu_khi.end(); it++) { 
    cout << *it;
}
\end{lstlisting}

    \textbf{Câu hỏi:} Tại sao dùng dấu bé hơn \texttt{<} là sai, mà phải dùng \texttt{!=}?

    \pause
    \begin{alertblock}{Giải thích}
        Trong C++, các toa tàu của một số loại container (như List, Map) nằm rải rác, không thẳng hàng nên không so sánh "bé hơn" được.
        \\
        Chúng ta chỉ quan tâm: \textbf{"Đã đến đích chưa?"} (Khác đích hay bằng đích). Do đó luôn dùng \texttt{!=}.
    \end{alertblock}
\end{frame}

% =========================================================================
% CHƯƠNG 4
% =========================================================================
\section{Chương 4: Algorithms}

\begin{frame}[fragile]{Chương 4: Phép Thuật Algorithms}
    Thay vì tự viết vòng lặp, hãy dùng thư viện \texttt{algorithm}.
    
    \begin{block}{Quy tắc chung}
        Phải chỉ rõ phạm vi tác động: Từ \textbf{Đầu tàu} đến \textbf{Cuối tàu}.
    \end{block}
    
    \textbf{1. Phép thuật Sắp xếp (Sort):}
\begin{lstlisting}[language=C++]
#include <algorithm>
// Sap xep tang dan (Be -> Lon, A -> Z)
sort(vu_khi.begin(), vu_khi.end());
\end{lstlisting}
    Kết quả: \{\texttt{"Cung", "Khiên", "Kiếm"}\}
\end{frame}

\begin{frame}[fragile]{Phép thuật Tìm kiếm (Find)}
    Thả "Chó săn phép thuật" đi tìm đồ.

\begin{lstlisting}[language=C++]
// Cu phap: find(Bat dau, Ket thuc, Vat can tim);
auto ket_qua = find(vu_khi.begin(), vu_khi.end(), "Cung Ten");

if (ket_qua != vu_khi.end()) {
    cout << "Da tim thay!";
} else {
    cout << "Khong co trong balo!";
}
\end{lstlisting}

    \begin{itemize}
        \item \textbf{Tìm thấy:} \texttt{ket\_qua} trỏ vào toa chứa đồ.
        \item \textbf{Không thấy:} \texttt{ket\_qua} chạy thẳng ra \texttt{end()}.
    \end{itemize}
\end{frame}

\begin{frame}[fragile]{Thử thách cuối cùng (The Final Boss)}
    \textbf{Đề bài:} Có danh sách điểm: \texttt{vector<int> diem = \{7, 4, 9, 2\};}
    
    \textbf{Nhiệm vụ:}
    \begin{enumerate}
        \item Sắp xếp từ thấp đến cao.
        \item In ra điểm thấp nhất và cao nhất.
    \end{enumerate}
    
    \pause
    \begin{exampleblock}{Lời giải}
\begin{lstlisting}[language=C++]
sort(diem.begin(), diem.end()); // 1. Sap xep: {2, 4, 7, 9}

// 2. Diem thap nhat (Dau tau)
cout << "Min: " << diem[0];     

// 3. Diem cao nhat (Cuoi tau - 1) HOAC dung back()
cout << "Max: " << diem.back(); 
\end{lstlisting}
    \end{exampleblock}
\end{frame}

% --- Tổng kết ---
\begin{frame}{Tổng Kết Khóa Học}
    \begin{table}[]
    \centering
    \begin{tabular}{|l|l|l|}
    \hline
    \textbf{Vũ Khí} & \textbf{Ẩn Dụ} & \textbf{Tác Dụng} \\ \hline
    Vector & Đoàn tàu co giãn & Mảng động, \texttt{push\_back} \\ \hline
    Iterator & Người soát vé & Con trỏ thông minh (\texttt{begin, end}) \\ \hline
    For/While & Băng chuyền & Duyệt qua vector \\ \hline
    Algorithm & Phép thuật & \texttt{sort}, \texttt{find} \\ \hline
    \end{tabular}
    \end{table}
    
    \vspace{1cm}
    \centering
    \Large \textbf{Bạn đã sẵn sàng để viết code chưa?}
\end{frame}

\end{document}