\documentclass{beamer}
\usepackage[utf8]{inputenc}
\usepackage[T5]{fontenc} % Bắt buộc để hiển thị tiếng Việt
\usepackage[vietnamese]{babel}
\usepackage{tcolorbox}
\usepackage{listings}
\usepackage{xcolor}
\usepackage{booktabs}
\usetheme{Madrid}

\definecolor{codegreen}{rgb}{0,0.6,0}
\definecolor{codegray}{rgb}{0.5,0.5,0.5}
\definecolor{codepurple}{rgb}{0.58,0,0.82}
\definecolor{backcolour}{rgb}{0.95,0.95,0.92}

\lstdefinestyle{mystyle}{
    backgroundcolor=\color{backcolour},   
    commentstyle=\color{codegreen},
    keywordstyle=\color{magenta},
    numberstyle=\tiny\color{codegray},
    stringstyle=\color{codepurple},
    basicstyle=\ttfamily\scriptsize,
    breakatwhitespace=false,         
    breaklines=true,                 
    captionpos=b,                    
    keepspaces=true,                 
    numbers=left,                    
    numbersep=4pt,                  
    showspaces=false,                
    showstringspaces=false,
    showtabs=false,                  
    tabsize=2,
    escapechar=@
}

\lstset{style=mystyle}

\title[Xử lý Chuỗi Codeforces]{Phân tích Chiến lược & Giải thuật Xử lý Chuỗi}
\subtitle{Phân khúc Rating 800 - 1000 trên Codeforces}
\author{Slide Learning C++}
\date{\today}

\begin{document}

\begin{frame}
    \titlepage
\end{frame}

\begin{frame}{Giới thiệu}
    \begin{block}{Tầm quan trọng của Xử lý Chuỗi}
        \begin{itemize}
            \item Phân khúc 800-1000 là giai đoạn chuyển giao từ cú pháp sang tư duy giải thuật.
            \item Chuỗi ký tự là hình thức biểu diễn thông tin tự nhiên và phổ biến nhất.
            \item Yêu cầu sự kết hợp giữa logic, quản lý bộ nhớ và cấu trúc dữ liệu cơ bản.
        \end{itemize}
    \end{block}
    \pause
    \begin{exampleblock}{Mục tiêu báo cáo}
        Cung cấp bộ công cụ tư duy (mental models) để giải quyết các bài toán String thông qua 20 ví dụ điển hình.
    \end{exampleblock}
\end{frame}

\begin{frame}{Cơ sở Lý luận và Thách thức}
    \begin{itemize}
        \item \textbf{Biểu diễn bộ nhớ:} \texttt{std::string} (C++) linh hoạt nhưng cần tối ưu; String trong Python là bất biến (immutable).
        \item \textbf{Bảng mã ASCII:} Tận dụng tính liên tục của 'a'-'z' để tính toán chỉ số: \texttt{c - 'a'}.
        \item \textbf{Độ phức tạp:} Với $N \approx 10^2 - 10^5$, các giải thuật $O(N)$ hoặc $O(N \log N)$ là lý tưởng.
    \end{itemize}
    \begin{alertblock}{Cạm bẫy}
        Lạm dụng các hàm thư viện như \texttt{find} hay \texttt{substr} bên trong vòng lặp có thể dẫn đến $O(N^2)$ và gây lỗi TLE.
    \end{alertblock}
\end{frame}

% Bài 1
\begin{frame}[fragile]{1. Way Too Long Words (71A)}
    \textbf{Link:} \url{https://codeforces.com/problemset/problem/71/A} \\
    \textbf{Rating:} 800
    
    \begin{block}{Đề bài}
        Nếu từ dài \textbf{hơn 10 ký tự}, thay thế bằng: [ký tự đầu] + [số ký tự ở giữa] + [ký tự cuối]. Ví dụ: "localization" $\rightarrow$ "l10n".
    \end{block}
    \pause
    \begin{exampleblock}{Phân tích chiến lược}
        \begin{itemize}
            \item Kiểm tra điều kiện $L > 10$.
            \item Số ký tự ở giữa: $L - 2$.
            \item Lưu ý: Xử lý bộ đệm (buffer) khi chuyển từ \texttt{cin >> n} sang \texttt{getline} hoặc \texttt{cin >> s}.
        \end{itemize}
    \end{exampleblock}
\end{frame}

% Bài 2
\begin{frame}{2. Word (59A)}
    \textbf{Link:} \url{https://codeforces.com/problemset/problem/59/A} \\
    \textbf{Rating:} 800

    \begin{block}{Đề bài}
        Chuẩn hóa từ về toàn hoa hoặc toàn thường. Nếu số chữ hoa > số chữ thường: chuyển toàn bộ thành hoa. Ngược lại: chuyển thành thường.
    \end{block}
    \pause
    \begin{exampleblock}{Phân tích chiến lược}
        \begin{itemize}
            \item Duyệt 1: Đếm \texttt{upper\_count} và \texttt{lower\_count}.
            \item Duyệt 2: Biến đổi dựa trên so sánh.
            \item Kỹ thuật: Dùng \texttt{tolower()}/\texttt{toupper()} hoặc cộng trừ 32 trong mã ASCII.
        \end{itemize}
    \end{exampleblock}
\end{frame}

% Bài 3
\begin{frame}{3. Petya and Strings (112A)}
    \textbf{Link:} \url{https://codeforces.com/problemset/problem/112/A} \\
    \textbf{Rating:} 800

    \begin{block}{Đề bài}
        So sánh thứ tự từ điển của hai chuỗi không phân biệt hoa thường. Trả về -1, 1, hoặc 0.
    \end{block}
    \pause
    \begin{exampleblock}{Phân tích chiến lược}
        \begin{itemize}
            \item \textbf{Chuẩn hóa:} Chuyển cả hai chuỗi về cùng một dạng (thường là lowercase).
            \item Sử dụng toán tử so sánh có sẵn (\texttt{<}, \texttt{>}, \texttt{==}) trên \texttt{std::string}.
        \end{itemize}
    \end{exampleblock}
\end{frame}

% Bài 4
\begin{frame}{4. Word Capitalization (281A)}
    \textbf{Link:} \url{https://codeforces.com/problemset/problem/281/A} \\
    \textbf{Rating:} 800

    \begin{block}{Đề bài}
        Viết hoa chữ cái đầu tiên của từ, các chữ cái khác giữ nguyên.
    \end{block}
    \pause
    \begin{exampleblock}{Phân tích chiến lược}
        \begin{itemize}
            \item Chỉ thao tác tại chỉ số \texttt{s[0]}.
            \item Sử dụng \texttt{toupper(s[0])}. Không cần kiểm tra nếu đã là chữ hoa vì hàm thư viện đã xử lý giúp.
        \end{itemize}
    \end{exampleblock}
\end{frame}

% Bài 5
\begin{frame}{5. Boy or Girl (236A)}
    \textbf{Link:} \url{https://codeforces.com/problemset/problem/236/A} \\
    \textbf{Rating:} 800

    \begin{block}{Đề bài}
        Đếm số ký tự riêng biệt (distinct characters). Nếu chẵn là nữ (CHAT WITH HER!), nếu lẻ là nam (IGNORE HIM!).
    \end{block}
    \pause
    \begin{exampleblock}{Phân tích chiến lược}
        \begin{itemize}
            \item \textbf{Cách 1:} Dùng \texttt{std::set<char>} để lưu và lấy \texttt{size()}.
            \item \textbf{Cách 2:} Dùng mảng tần suất \texttt{bool seen[26]}.
            \item \textbf{Cách 3:} Sắp xếp chuỗi rồi đếm các cặp \texttt{s[i] != s[i-1]}.
        \end{itemize}
    \end{exampleblock}
\end{frame}

% Bài 6
\begin{frame}{6. Translation (41A)}
    \textbf{Link:} \url{https://codeforces.com/problemset/problem/41/A} \\
    \textbf{Rating:} 800

    \begin{block}{Đề bài}
        Kiểm tra xem chuỗi $t$ có phải là phiên bản đảo ngược của chuỗi $s$ hay không.
    \end{block}
    \pause
    \begin{exampleblock}{Phân tích chiến lược}
        \begin{itemize}
            \item \textbf{Cách 1:} Dùng \texttt{std::reverse(s.begin(), s.end())} rồi so sánh $s == t$.
            \item \textbf{Cách 2:} So sánh \texttt{s[i]} với \texttt{t[n-1-i]}.
            \item \textbf{Lưu ý:} Kiểm tra độ dài trước khi so sánh chi tiết.
        \end{itemize}
    \end{exampleblock}
\end{frame}

% Bài 7
\begin{frame}{7. Anton and Danik (734A)}
    \textbf{Link:} \url{https://codeforces.com/problemset/problem/734/A} \\
    \textbf{Rating:} 800

    \begin{block}{Đề bài}
        Đếm số lần 'A' (Anton) và 'D' (Danik) xuất hiện. Ai nhiều hơn thì thắng.
    \end{block}
    \pause
    \begin{exampleblock}{Phân tích chiến lược}
        \begin{itemize}
            \item Duyệt tuyến tính (Linear Scan).
            \item Sử dụng hai biến đếm đơn giản.
        \end{itemize}
    \end{exampleblock}
\end{frame}

% Bài 8
\begin{frame}{8. Amusing Joke (141A)}
    \textbf{Link:} \url{https://codeforces.com/problemset/problem/141/A} \\
    \textbf{Rating:} 800

    \begin{block}{Đề bài}
        Cho 3 chuỗi: $A$, $B$, $C$. Kiểm tra xem các ký tự của $C$ có tạo thành đúng tập hợp ký tự của ($A + B$) hay không.
    \end{block}
    \pause
    \begin{exampleblock}{Phân tích chiến lược}
        \begin{itemize}
            \item Nối chuỗi: $D = A + B$.
            \item Sắp xếp cả $C$ và $D$. So sánh $C == D$.
            \item Hoặc sử dụng mảng tần suất (Frequency Map) để so sánh.
        \end{itemize}
    \end{exampleblock}
\end{frame}

% Bài 9
\begin{frame}{9. Pangram (520A)}
    \textbf{Link:} \url{https://codeforces.com/problemset/problem/520/A} \\
    \textbf{Rating:} 800

    \begin{block}{Đề bài}
        Kiểm tra xem chuỗi có chứa đủ 26 chữ cái Latin (không phân biệt hoa thường) hay không.
    \end{block}
    \pause
    \begin{exampleblock}{Phân tích chiến lược}
        \begin{itemize}
            \item Chuyển về lowercase.
            \item Đưa vào \texttt{std::set<char>}.
            \item Kết quả là "YES" nếu \texttt{set.size() == 26}.
        \end{itemize}
    \end{exampleblock}
\end{frame}

% Bài 10
\begin{frame}{10. Stones on the Table (266A)}
    \textbf{Link:} \url{https://codeforces.com/problemset/problem/266/A} \\
    \textbf{Rating:} 800

    \begin{block}{Đề bài}
        Tìm số đá tối thiểu cần bỏ để không có hai viên đá nào cạnh nhau cùng màu.
    \end{block}
    \pause
    \begin{exampleblock}{Phân tích chiến lược}
        \begin{itemize}
            \item \textbf{Tham lam (Greedy):} So sánh các phần tử liền kề.
            \item Nếu \texttt{s[i] == s[i+1]}, tăng biến đếm (loại bỏ 1 viên).
        \end{itemize}
    \end{exampleblock}
\end{frame}

% Bài 11
\begin{frame}{11. Football (96A)}
    \textbf{Link:} \url{https://codeforces.com/problemset/problem/96/A} \\
    \textbf{Rating:} 900

    \begin{block}{Đề bài}
        Kiểm tra xem có ít nhất 7 cầu thủ cùng đội (ký tự giống nhau) đứng liên tiếp hay không.
    \end{block}
    \pause
    \begin{exampleblock}{Phân tích chiến lược}
        \begin{itemize}
            \item Duyệt và đếm số ký tự trùng lặp liên tiếp. Reset về 1 nếu gặp ký tự khác.
            \item Hoặc dùng hàm \texttt{s.find("0000000")} và \texttt{s.find("1111111")}.
        \end{itemize}
    \end{exampleblock}
\end{frame}

% Bài 12
\begin{frame}{12. Dubstep (208A)}
    \textbf{Link:} \url{https://codeforces.com/problemset/problem/208/A} \\
    \textbf{Rating:} 900

    \begin{block}{Đề bài}
        Khôi phục bài hát gốc bằng cách loại bỏ "WUB". Các từ gốc cách nhau bằng 1 dấu cách.
    \end{block}
    \pause
    \begin{exampleblock}{Phân tích chiến lược}
        \begin{itemize}
            \item Duyệt chuỗi, nếu gặp "WUB" thì bỏ qua 3 ký tự.
            \item Dùng biến cờ hiệu (\texttt{flag}) để xử lý khoảng trắng giữa các từ, tránh khoảng trắng thừa ở đầu/cuối.
        \end{itemize}
    \end{exampleblock}
\end{frame}

% Bài 13
\begin{frame}{13. HQ9+ (133A)}
    \textbf{Link:} \url{https://codeforces.com/problemset/problem/133/A} \\
    \textbf{Rating:} 900

    \begin{block}{Đề bài}
        Kiểm tra xem chương trình HQ9+ có in ra gì không (lệnh H, Q, 9 có in, lệnh + không in).
    \end{block}
    \pause
    \begin{alertblock}{Cạm bẫy}
        Đừng cố mô phỏng lệnh '+'. Nó chỉ thay đổi biến nội bộ, không tạo ra output.
    \end{alertblock}
    \begin{exampleblock}{Phân tích chiến lược}
        Tìm sự tồn tại của ký tự 'H', 'Q', hoặc '9' trong chuỗi đầu vào.
    \end{exampleblock}
\end{frame}

% Bài 14
\begin{frame}{14. String Task (118A)}
    \textbf{Link:} \url{https://codeforces.com/problemset/problem/118/A} \\
    \textbf{Rating:} 1000

    \begin{block}{Đề bài}
        Xóa nguyên âm (bao gồm cả 'y'), viết thường các phụ âm và thêm dấu '.' trước mỗi phụ âm.
    \end{block}
    \pause
    \begin{exampleblock}{Phân tích chiến lược}
        \begin{itemize}
            \item Chuyển về lowercase trước.
            \item Duyệt từng ký tự, kiểm tra nếu không phải nguyên âm thì in ra \texttt{"." + c}.
        \end{itemize}
    \end{exampleblock}
\end{frame}

% Bài 15
\begin{frame}{15. Chat Room (58A)}
    \textbf{Link:} \url{https://codeforces.com/problemset/problem/58/A} \\
    \textbf{Rating:} 1000

    \begin{block}{Đề bài}
        Kiểm tra xem "hello" có phải là một dãy con (subsequence) của chuỗi đầu vào hay không.
    \end{block}
    \pause
    \begin{exampleblock}{Phân tích chiến lược}
        \begin{itemize}
            \item Dùng một biến chỉ mục \texttt{idx = 0} cho chuỗi mục tiêu "hello".
            \item Duyệt chuỗi input, nếu \texttt{s[i] == target[idx]} thì tăng \texttt{idx}.
            \item Nếu \texttt{idx == 5} thì kết luận YES.
        \end{itemize}
    \end{exampleblock}
\end{frame}

% Bài 16
\begin{frame}{16. cAPS lOCK (131A)}
    \textbf{Link:} \url{https://codeforces.com/problemset/problem/131/A} \\
    \textbf{Rating:} 1000

    \begin{block}{Đề bài}
        Sửa lỗi Caps Lock: Nếu từ toàn in hoa HOẶC chỉ chữ cái đầu viết thường còn lại in hoa $\rightarrow$ Đảo ngược toàn bộ trạng thái hoa/thường.
    \end{block}
    \pause
    \begin{exampleblock}{Phân tích chiến lược}
        \begin{itemize}
            \item Kiểm tra điều kiện: Mọi ký tự từ vị trí thứ 2 trở đi (\texttt{s[1...n-1]}) phải là in hoa.
            \item Nếu thỏa mãn, duyệt lại và dùng \texttt{islower}/\texttt{isupper} để đảo ngược.
        \end{itemize}
    \end{exampleblock}
\end{frame}

% Bài 17
\begin{frame}{17. Case of the Zeros and Ones (556A)}
    \textbf{Link:} \url{https://codeforces.com/problemset/problem/556/A} \\
    \textbf{Rating:} 1000

    \begin{block}{Đề bài}
        Xóa cặp "01" hoặc "10" liên tiếp. Tìm độ dài nhỏ nhất còn lại.
    \end{block}
    \pause
    \begin{exampleblock}{Quan sát Toán học}
        Mỗi lần xóa luôn mất một số 0 và một số 1. Thứ tự không quan trọng.
        \begin{itemize}
            \item Gọi $n0$ là số lượng '0', $n1$ là số lượng '1'.
            \item Số lượng cặp bị xóa: $\min(n0, n1)$.
            \item Độ dài còn lại: $|n0 - n1|$.
        \end{itemize}
    \end{exampleblock}
\end{frame}

% Bài 18
\begin{frame}{18. Keyboard (474A)}
    \textbf{Link:} \url{https://codeforces.com/problemset/problem/474/A} \\
    \textbf{Rating:} 900

    \begin{block}{Đề bài}
        Khôi phục ký tự gốc khi tay bị lệch sang trái ('L') hoặc phải ('R') trên bàn phím.
    \end{block}
    \pause
    \begin{exampleblock}{Phân tích chiến lược}
        \begin{itemize}
            \item Lưu bàn phím vào 1 chuỗi: \texttt{"qwertyuiopasdfghjkl;zxcvbnm,./"}
            \item Tìm vị trí ký tự đã gõ, rồi lấy ký tự ở vị trí \texttt{index-1} (nếu lệch 'R') hoặc \texttt{index+1} (nếu lệch 'L').
        \end{itemize}
    \end{exampleblock}
\end{frame}

% Bài 19
\begin{frame}{19. Double-ended Strings (1506C)}
    \textbf{Link:} \url{https://codeforces.com/problemset/problem/1506C} \\
    \textbf{Rating:} 1000

    \begin{block}{Đề bài}
        Tìm số lần xóa ký tự ở hai đầu ít nhất để hai chuỗi bằng nhau.
    \end{block}
    \pause
    \begin{exampleblock}{Phân tích chiến lược}
        \begin{itemize}
            \item Thực chất là tìm \textbf{Chuỗi con chung dài nhất} (Longest Common Substring - LCS).
            \item Với $N \le 20$, dùng Brute Force duyệt mọi chuỗi con của $A$ và kiểm tra trong $B$ bằng \texttt{find()}.
            \item Đáp án: $len(A) + len(B) - 2 \times len(LCS)$.
        \end{itemize}
    \end{exampleblock}
\end{frame}

% Bài 20
\begin{frame}{20. Comparison String (1837B)}
    \textbf{Link:} \url{https://codeforces.com/problemset/problem/1837/B} \\
    \textbf{Rating:} 900

    \begin{block}{Đề bài}
        Xây dựng dãy số thỏa mãn chuỗi dấu so sánh sao cho số lượng giá trị phân biệt là ít nhất.
    \end{block}
    \pause
    \begin{exampleblock}{Phân tích chiến lược}
        \begin{itemize}
            \item Quan sát: Ta chỉ cần thêm số mới khi dấu so sánh kéo dài liên tục theo một hướng ($<$ hoặc $>$).
            \item Đáp án = (Độ dài đoạn ký tự giống nhau liên tiếp dài nhất) + 1.
        \end{itemize}
    \end{exampleblock}
\end{frame}

\begin{frame}{Tổng hợp Kỹ thuật}
    \centering
    \scriptsize
    \begin{tabular}{llll}
        \toprule
        Bài toán & Rating & Kỹ thuật & Cạm bẫy \\
        \midrule
        71A & 800 & Parsing & Định dạng số \\
        112A & 800 & Chuẩn hóa & So sánh trực tiếp \\
        236A & 800 & Set/Freq Map & Đếm sai ký tự trùng \\
        96A & 900 & Scan/Find & Reset biến đếm sai \\
        58A & 1000 & Greedy Subsequence & Nhầm với Substring \\
        556A & 1000 & Math Observation & Mô phỏng gây TLE \\
        \bottomrule
    \end{tabular}
\end{frame}

\begin{frame}{Lời khuyên Chuyên gia}
    \begin{itemize}
        \item \textbf{Thành thạo STL:} \texttt{std::set}, \texttt{sort}, \texttt{reverse}, \texttt{find}.
        \item \textbf{Chuẩn hóa sớm:} Đưa về lowercase ngay sau khi đọc input.
        \item \textbf{Phân biệt rõ:} \textit{Substring} (liên tiếp) và \textit{Subsequence} (không nhất thiết liên tiếp).
        \item \textbf{Tư duy tối ưu:} Luôn tìm quy luật toán học trước khi thực hiện mô phỏng tốn kém.
    \end{itemize}
    \begin{block}{}
        \centering \textbf{Chúc các bạn thành công trên con đường CP!}
    \end{block}
\end{frame}

\end{document}