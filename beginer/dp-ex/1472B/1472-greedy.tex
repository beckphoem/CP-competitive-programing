\documentclass{beamer}
\usepackage[utf8]{inputenc}
\usepackage[T5]{fontenc} 
\usepackage[vietnamese]{babel}
\usepackage{tcolorbox}
\usepackage{listings}
\usepackage{xcolor}
\usepackage{booktabs}
\usetheme{Madrid}

\definecolor{codegreen}{rgb}{0,0.6,0}
\definecolor{codegray}{rgb}{0.5,0.5,0.5}
\definecolor{codepurple}{rgb}{0.58,0,0.82}
\definecolor{backcolour}{rgb}{0.95,0.95,0.92}

\lstdefinestyle{mystyle}{
    backgroundcolor=\color{backcolour},   
    commentstyle=\color{codegreen},
    keywordstyle=\color{magenta},
    numberstyle=\tiny\color{codegray},
    stringstyle=\color{codepurple},
    basicstyle=\ttfamily\scriptsize,
    breakatwhitespace=false,         
    breaklines=true,                 
    captionpos=b,                    
    keepspaces=true,                 
    numbers=left,                    
    numbersep=4pt,                  
    showspaces=false,                
    showstringspaces=false,
    showtabs=false,                  
    tabsize=2,
    escapechar=@
}

\lstset{style=mystyle}

\title[Fair Division]{Huấn luyện viên Tư duy Thuật toán}
\subtitle{Codeforces 1472B - Fair Division}
\author{Gemini AI}
\date{2026}

\begin{document}

\begin{frame}
  \titlepage
\end{frame}

\begin{frame}{Phẫu thuật đề bài (Briefing)}
    \begin{block}{Cốt lõi vấn đề}
        \begin{itemize}
            \item \textbf{Dữ liệu:} Túi kẹo gồm 2 loại: nặng 1g và nặng 2g.
            \item \textbf{Mục tiêu:} Chia kẹo thành 2 phần có \textbf{tổng khối lượng} bằng nhau.
            \item \textbf{Ràng buộc:} Không được cắt đôi viên kẹo.
        \end{itemize}
    \end{block}

    \begin{exampleblock}{Lộ trình tư duy}
        \begin{enumerate}
            \item Điều kiện cần (Tổng khối lượng).
            \item Khả năng lấp đầy (Sử dụng kẹo 2g và kẹo 1g).
            \item Xử lý bẫy logic.
        \end{enumerate}
    \end{exampleblock}
\end{frame}

\begin{frame}{Chunk 1: Điều kiện "Tổng chẵn"}
    \begin{itemize}
        \item Để chia đôi thành 2 phần bằng nhau ($A$ và $B$), thì mỗi phần phải nặng đúng bằng $\frac{S}{2}$.
        \item \textbf{Ẩn dụ:} Nếu có tổng 7 viên kẹo 1g, không thể chia đều mà không bẻ đôi.
    \end{itemize}

    \begin{alertblock}{Thử thách tư duy}
        Túi có: 1 viên 2g và 1 viên 1g. Tổng khối lượng là 3g. Có thể chia đều không?
    \end{alertblock}
    
    \pause
    \textbf{Đáp án:} Không! Vì 3 là số lẻ.
    \begin{tcolorbox}[colback=green!5,colframe=green!40!black,title=Điều kiện 1]
        Tổng khối lượng tất cả viên kẹo phải là một \textbf{số chẵn}.
    \end{tcolorbox}
\end{frame}

\begin{frame}{Chunk 2: "Bẫy" kẹo 2g}
    \begin{alertblock}{Trường hợp ngang trái}
        Bạn có 1 viên kẹo 2g và 0 viên kẹo 1g. Tổng là 2 (số chẵn). Có chia được không?
    \end{alertblock}

    \pause
    \textbf{Đáp án:} Không! Vì chỉ có 1 khối 2g, không thể chia cho 2 người mỗi người 1g.

    \begin{block}{Quan sát quan trọng}
        \begin{itemize}
            \item Nếu số lượng kẹo 2g là \textbf{chẵn}: Chia đều dễ dàng.
            \item Nếu số lượng kẹo 2g là \textbf{lẻ}: Dư ra một viên 2g. Cần kẹo 1g để bù đắp.
        \end{itemize}
    \end{block}

    \pause
    \begin{exampleblock}{Câu hỏi}
        Cần ít nhất bao nhiêu viên 1g để bù cho 1 viên 2g dư?
    \end{exampleblock}
    \pause
    \textbf{Chính xác:} Cần ít nhất \textbf{2 viên 1g}.
\end{frame}

\begin{frame}{Chunk 3: Tổng kết logic}
    Gọi $n_1$ là số lượng kẹo 1g, $n_2$ là số lượng kẹo 2g.
    
    \begin{itemize}
        \item \textbf{Trường hợp A:} Nếu $n_2$ chẵn. Chỉ cần $n_1$ chẵn là xong.
        \item \textbf{Trường hợp B:} Nếu $n_2$ lẻ. Cần ít nhất 2 viên 1g để cân bằng, sau đó số kẹo 1g còn lại ($n_1 - 2$) cũng phải chẵn.
    \end{itemize}

    \begin{alertblock}{Thử thách chốt hạ}
        $n_2 = 3$ và $n_1 = 1$. Tổng = 7g. Có chia được không?
    \end{alertblock}
    \pause
    \textbf{Đáp án:} Không! Vì tổng khối lượng lẻ.
\end{frame}

\begin{frame}[fragile]{Mã giả (Pseudocode)}
    \begin{lstlisting}[language=Python, caption=Logic kiểm tra điều kiện]
    Neu (Tong_khoi_luong la so le):
        In ra "NO"
    Neu (So_luong_keo_2g la so le VA So_luong_keo_1g == 0):
        In ra "NO"
    Con lai:
        In ra "YES"
    \end{lstlisting}
    
    \begin{block}{Giải thích}
        Nếu tổng chẵn và có đủ kẹo 1g để bù đắp cho sự lẻ loi của kẹo 2g (nếu có), chúng ta luôn chia được.
    \end{block}
\end{frame}

\begin{frame}{Mở rộng: Tư duy Quy hoạch động (DP)}
    \begin{block}{Bản chất DP}
        Sử dụng mảng \texttt{dp[i]} để đánh dấu xem có thể tạo ra tổng khối lượng bằng \texttt{i} hay không.
    \end{block}

    \begin{table}[]
        \centering
        \begin{tabular}{@{}lll@{}}
        \toprule
        Đặc điểm & Greedy/Math & Quy hoạch động (DP) \\ \midrule
        Tốc độ & Cực nhanh $O(1)$ & Chậm hơn $O(n \cdot \text{target})$ \\
        Độ linh hoạt & Thấp & \textbf{Rất cao} (nhiều loại tạ) \\ \bottomrule
        \end{tabular}
    \end{table}
\end{frame}

\begin{frame}[fragile]{Cấu trúc mã giả DP}
    \begin{lstlisting}[language=C++, caption=Thuật toán Dynamic Programming]
    target = tong_khoi_luong / 2
    dp[0] = True
    Cac dp[1...target] = False

    Voi moi vien keo v trong danh sach:
        Chay i tu target xuong den v:
            dp[i] = dp[i] HOAC dp[i - v]

    Ket qua: Neu dp[target] la True thi YES, nguoc lai NO
    \end{lstlisting}

    \begin{alertblock}{Câu hỏi tư duy}
        Tại sao biến \texttt{i} phải chạy \textbf{ngược} từ \texttt{target} về \texttt{v}?
    \end{alertblock}
    \pause
    \textbf{Gợi ý:} Để đảm bảo mỗi viên kẹo chỉ được sử dụng \textbf{một lần} duy nhất.
\end{frame}

\end{document}