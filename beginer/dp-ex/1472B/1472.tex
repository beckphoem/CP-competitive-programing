\documentclass{beamer}
\usepackage[utf8]{inputenc}
\usepackage[T5]{fontenc} % Bắt buộc để hiển thị tiếng Việt
\usepackage[vietnamese]{babel}
\usepackage{tcolorbox}
\usepackage{listings}
\usepackage{xcolor}
\usepackage{booktabs}
\usetheme{Madrid}

\definecolor{codegreen}{rgb}{0,0.6,0}
\definecolor{codegray}{rgb}{0.5,0.5,0.5}
\definecolor{codepurple}{rgb}{0.58,0,0.82}
\definecolor{backcolour}{rgb}{0.95,0.95,0.92}

\lstdefinestyle{mystyle}{
    backgroundcolor=\color{backcolour},   
    commentstyle=\color{codegreen},
    keywordstyle=\color{magenta},
    numberstyle=\tiny\color{codegray},
    stringstyle=\color{codepurple},
    basicstyle=\ttfamily\scriptsize,
    breakatwhitespace=false,         
    breaklines=true,                 
    captionpos=b,                    
    keepspaces=true,                 
    numbers=left,                    
    numbersep=4pt,                  
    showspaces=false,                
    showstringspaces=false,
    showtabs=false,                  
    tabsize=2,
    escapechar=@
}

\lstset{style=mystyle}

\title{Chinh phục bài toán 1472B - Fair Division}
\subtitle{Tiếp cận theo hướng Quy hoạch động (DP)}
\author{Coach "Slide Learning CPP"}
\date{\today}

\begin{document}

\begin{frame}
    \titlepage
\end{frame}

\begin{frame}{Giới thiệu}
    \begin{block}{Lời chào}
        Chào bạn! Rất vui được đồng hành cùng bạn chinh phục bài toán \textbf{1472B - Fair Division} trên Codeforces.
    \end{block}
    
    \begin{itemize}
        \item Phương pháp: Quy hoạch động (Dynamic Programming - DP).
        \item Bản chất DP: "Xây dựng kết quả lớn từ những kết quả nhỏ đã biết".
        \item Lộ trình: Chia nhỏ vấn đề thành các Micro-chunks logic.
    \end{itemize}
\end{frame}

\begin{frame}{Bước 1: Phẫu thuật đề bài (Deconstruct)}
    \begin{block}{Cốt lõi bài toán}
        \begin{itemize}
            \item Bạn có $n$ viên kẹo.
            \item Cân nặng mỗi viên: \textbf{1 unit} hoặc \textbf{2 units}.
            \item \textbf{Mục tiêu:} Chia toàn bộ kẹo thành 2 phần có tổng cân nặng bằng nhau.
        \end{itemize}
    \end{block}

    \begin{exampleblock}{Lộ trình tư duy}
        \begin{enumerate}
            \item Xác định tổng trọng lượng cần đạt được cho mỗi người.
            \item Dùng DP để kiểm tra khả năng tạo ra phần quà đó.
        \end{enumerate}
    \end{exampleblock}
\end{frame}

\begin{frame}{Bước 2: Mảnh ghép 1 - Mục tiêu của phép chia}
    Giả sử tổng cân nặng của tất cả viên kẹo là $S$. Để chia đều, mỗi người phải nhận chính xác $S/2$.

    \begin{alertblock}{Bẫy logic (Trap)}
        Nếu $S$ là một số lẻ (ví dụ $S=5$), liệu chúng ta có bao giờ chia đều được không?
    \end{alertblock}

    \begin{block}{Thử thách tư duy}
        Nếu tôi có: 1 viên kẹo nặng 2 unit và 1 viên kẹo nặng 1 unit $\rightarrow S=3$.
        Theo bạn, chúng ta có cần dùng DP để kiểm tra trường hợp này không, hay có thể kết luận ngay là "NO"? Tại sao?
    \end{block}
\end{frame}

\begin{frame}[fragile]{Mảnh ghép 2 - Cập nhật trạng thái DP}
    Khi ta nhặt thêm một viên kẹo nặng $w=2$, các trạng thái mới của mảng \texttt{dp} sẽ được cập nhật.

    \begin{alertblock}{Cảnh báo: Bẫy Knapsack}
        Nếu duyệt từ đầu đến cuối (từ 0 đến $S$), bạn có thể vô tình dùng một viên kẹo nhiều lần!
    \end{alertblock}

    \begin{block}{Thử thách tư duy}
        Để tránh việc một viên kẹo vừa làm \texttt{dp[j]} thành \texttt{true} lại tiếp tục được dùng để cập nhật \texttt{dp[j+w]}, chúng ta nên:
        \begin{itemize}
            \item \textbf{A.} Duyệt từ $S$ ngược về 0.
            \item \textbf{B.} Duyệt từ 0 tiến lên $S$.
        \end{itemize}
    \end{block}
    \pause
    \textbf{Đáp án: A.} Duyệt ngược là chìa khóa để mỗi vật phẩm chỉ được dùng đúng một lần.
\end{frame}

\begin{frame}[fragile]{Bước 3: Tổng kết Thuật toán (Mã giả)}
    \begin{lstlisting}[language=Python, caption=Thuật toán DP cho Fair Division]
Tong = tong tat ca vien kẹo
Neu Tong % 2 != 0: 
    In ra "NO"
Target = Tong / 2
dp[0] = true, tat ca dp[1...Target] = false

Voi moi vien kẹo nang 'w':
   Duyet j tu Target nguoc ve w:
      Neu dp[j-w] == true:
         dp[j] = true

Neu dp[Target] == true: 
    In ra "YES"
Con lai: 
    In ra "NO"
    \end{lstlisting}
\end{frame}

\begin{frame}{Mảnh ghép cuối: Tối ưu hóa logic}
    \begin{block}{Kiểm tra trường hợp đặc biệt}
        Giả sử có 3 viên kẹo nặng 2 ($n=3$, $S=6$, $Target=3$).
    \end{block}

    \begin{itemize}
        \item \texttt{dp[0] = true}
        \item Viên 1 ($w=2$): \texttt{dp[2] = true}
        \item Viên 2 ($w=2$): \texttt{dp[4]} (vượt Target 3 nên bỏ qua)
        \item Kết quả: \texttt{dp[3]} vẫn là \texttt{false}.
    \end{itemize}

    \begin{exampleblock}{Kết luận}
        Thuật toán DP trả về \textbf{NO} là chính xác, vì không thể tạo ra mức 3 từ các viên kẹo nặng 2. DP kiểm tra được cả "nguyên liệu" có phù hợp hay không.
    \end{exampleblock}
\end{frame}

\begin{frame}{Bước 4: Chốt hạ kiến thức}
    \begin{enumerate}
        \item \textbf{Loại trừ:} Tổng lẻ luôn là "NO".
        \item \textbf{Xây dựng:} Dùng mảng \texttt{dp} kiểm tra khả năng tạo mức $Target$.
        \item \textbf{Duyệt ngược:} Đảm bảo tính chất của bài toán Knapsack 0/1.
    \end{enumerate}

    \begin{block}{Bạn muốn làm gì tiếp theo?}
        \begin{enumerate}
            \item \textbf{Thử thách Code:} Tự viết mã C++ hoặc Python.
            \item \textbf{Tối ưu hóa:} Khám phá cách giải bằng Toán học (Greedy).
            \item \textbf{Sang bài mới:} Luyện tập bài DP khác trên Codeforces.
        \end{enumerate}
    \end{block}
\end{frame}

\end{document}