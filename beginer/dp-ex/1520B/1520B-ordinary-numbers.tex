\documentclass{beamer}
\usepackage[utf8]{inputenc}
\usepackage[T5]{fontenc} % Bắt buộc để hiển thị tiếng Việt
\usepackage[vietnamese]{babel}
\usepackage{tcolorbox}
\usepackage{listings}
\usepackage{xcolor}
\usepackage{booktabs}
\usetheme{Madrid}

\definecolor{codegreen}{rgb}{0,0.6,0}
\definecolor{codegray}{rgb}{0.5,0.5,0.5}
\definecolor{codepurple}{rgb}{0.58,0,0.82}
\definecolor{backcolour}{rgb}{0.95,0.95,0.92}

\lstdefinestyle{mystyle}{
    backgroundcolor=\color{backcolour},   
    commentstyle=\color{codegreen},
    keywordstyle=\color{magenta},
    numberstyle=\tiny\color{codegray},
    stringstyle=\color{codepurple},
    basicstyle=\ttfamily\scriptsize,
    breakatwhitespace=false,         
    breaklines=true,                 
    captionpos=b,                    
    keepspaces=true,                 
    numbers=left,                    
    numbersep=4pt,                  
    showspaces=false,                
    showstringspaces=false,
    showtabs=false,                  
    tabsize=2,
    escapechar=@
}

\lstset{style=mystyle}

\title[Ordinary Numbers]{Giải quyết bài toán Ordinary Numbers}
\subtitle{Codeforces 1520B}
\author{Algorithmic Coach}
\date{2026}

\begin{document}

\begin{frame}
  \titlepage
\end{frame}

\begin{frame}{Tiếp nhận \& Phẫu thuật (Briefing)}
    \begin{block}{Định nghĩa: Số bình thường}
        Một số được gọi là "bình thường" nếu \textbf{tất cả các chữ số của nó đều giống nhau}.
    \end{block}
    
    \begin{itemize}
        \item \textbf{Ví dụ:} $1, 2, 11, 333$ là số bình thường.
        \item \textbf{Phản ví dụ:} $12, 121$ \textbf{không} phải là số bình thường.
        \item \textbf{Nhiệm vụ:} Cho số nguyên $n$, đếm số lượng số bình thường trong khoảng từ $1$ đến $n$.
    \end{itemize}

    \begin{exampleblock}{Lộ trình tư duy}
        \begin{enumerate}
            \item Nhận diện quy luật theo từng "bậc".
            \item Công thức hóa để tính nhanh với $n \le 10^9$.
            \item Xây dựng thuật toán tối ưu.
        \end{enumerate}
    \end{exampleblock}
\end{frame}

\begin{frame}{Chunk 1: Quy luật của những "vị khách giống hệt nhau"}
    Hãy tưởng tượng các số bình thường theo từng nhóm chữ số:
    \begin{itemize}
        \item \textbf{Đoàn có 1 chữ số:} $1, 2, 3, 4, 5, 6, 7, 8, 9$ (9 số).
        \item \textbf{Đoàn có 2 chữ số:} $11, 22, 33, 44, 55, 66, 77, 88, 99$ (9 số).
        \item \textbf{Đoàn có 3 chữ số:} $111, 222, \dots, 999$ (9 số).
    \end{itemize}

    \begin{alertblock}{Bẫy tư duy}
        Mỗi "bậc" (số chữ số) luôn chỉ có tối đa 9 số bình thường, dù $n$ có lớn đến đâu.
    \end{alertblock}
\end{frame}

\begin{frame}{Thử thách tư duy số 1}
    \begin{exampleblock}{Câu hỏi}
        Nếu cho số $n = 45$, có bao nhiêu số bình thường từ $1$ đến $45$?
    \end{exampleblock}
    
    \pause
    
    \textbf{Đáp án:}
    \begin{itemize}
        \item Hàng đơn vị: $1, 2, 3, 4, 5, 6, 7, 8, 9$ (9 số).
        \item Hàng chục: $11, 22, 33, 44$ (4 số).
        \item \textbf{Tổng cộng: 13 số.}
    \end{itemize}
\end{frame}

\begin{frame}{Phát triển quy luật tổng quát}
    Giả sử xét $n = 715$:
    \begin{itemize}
        \item \textbf{Bậc 1 chữ số:} $1 \dots 9$ (9 số).
        \item \textbf{Bậc 2 chữ số:} $11 \dots 99$ (9 số).
        \item \textbf{Bậc 3 chữ số:} Các số ứng viên $111, 222, 333, 444, 555, 666, 777 \dots$
    \end{itemize}

    \begin{exampleblock}{Thử thách}
        Với $n = 715$, số bình thường lớn nhất có 3 chữ số thỏa mãn là bao nhiêu? Tổng cộng có bao nhiêu số từ $1 \dots 715$?
    \end{exampleblock}

    \pause
    \textbf{Giải đáp:}
    \begin{itemize}
        \item Số lớn nhất là $666$ (vì $777 > 715$).
        \item Tổng cộng: $9 + 9 + 6 = 24$ số.
    \end{itemize}
\end{frame}

\begin{frame}{Chunk 2: Công thức hóa "Siêu tốc"}
    Để xử lý $n$ lên tới $10^9$:
    \begin{enumerate}
        \item Đếm số lượng chữ số của $n$ (gọi là $k$).
        \item Với mỗi bậc nhỏ hơn $k$, cộng thêm $9$ số.
        \item Với bậc cuối cùng (bậc có $k$ chữ số), kiểm tra các số dạng $x, xx, xxx \dots$
    \end{enumerate}

    \begin{block}{Ví dụ với $n = 210$}
        \begin{itemize}
            \item Bậc 1 chữ số: 9 số.
            \item Bậc 2 chữ số: 9 số.
            \item Bậc 3 chữ số: Chỉ có $111 \le 210$ (1 số).
            \item \textbf{Tổng: $9 + 9 + 1 = 19$ số.}
        \end{itemize}
    \end{block}
\end{frame}

\begin{frame}{Chunk 3: Xây dựng giải pháp tổng quát}
    Xét ví dụ $n = 332$:
    \begin{enumerate}
        \item Số chữ số $k = 3$.
        \item Phần "chắc chắn có": $(3 - 1) \times 9 = 18$ số.
        \item Bậc có 3 chữ số: Kiểm tra các số $\{111, 222, 333\}$.
    \end{enumerate}

    \begin{exampleblock}{Câu hỏi}
        Trong các số $\{111, 222, 333\}$, số nào $\le 332$? Tổng kết quả là bao nhiêu?
    \end{exampleblock}

    \pause
    \textbf{Đáp án:}
    \begin{itemize}
        \item Thỏa mãn: $111, 222$ (2 số).
        \item Tổng cộng: $18 + 2 = 20$ số.
    \end{itemize}
\end{frame}

\begin{frame}[fragile]{Chunk 4: Tổng kết thuật toán (Final Logic)}
    \begin{block}{Mã giả (Pseudocode)}
        \begin{enumerate}
            \item Nhập $n$, khởi tạo \texttt{dem = 0}.
            \item Lặp \texttt{chu\_so} từ $1$ đến $9$:
            \begin{itemize}
                \item Khởi tạo \texttt{so\_tao\_ra = chu\_so}.
                \item Trong khi \texttt{so\_tao\_ra <= n}:
                \begin{itemize}
                    \item Tăng \texttt{dem}.
                    \item \texttt{so\_tao\_ra = so\_tao\_ra * 10 + chu\_so}.
                \end{itemize}
            \end{itemize}
            \item In ra \texttt{dem}.
        \end{enumerate}
    \end{block}

    \begin{exampleblock}{Ví dụ tạo số với \texttt{chu\_so = 2}}
        $2 \rightarrow (2 \times 10 + 2) = 22 \rightarrow (22 \times 10 + 2) = 222 \dots$
    \end{exampleblock}
\end{frame}

\begin{frame}{Kết luận}
    \begin{itemize}
        \item Cách tiếp cận này giúp "quét" sạch mọi số bình thường mà không cần quan tâm $n$ lớn hay nhỏ.
        \item Độ phức tạp cực thấp: Chỉ khoảng $9 \times 9$ lần lặp cho $n = 10^9$.
    \end{itemize}

    \begin{center}
        \begin{tcolorbox}[colback=blue!5,colframe=blue!75,title=Bước tiếp theo]
            Bạn có muốn xem mã nguồn chi tiết bằng C++ hay Python không, hay chúng ta sẽ thử sức với một bài toán mới?
        \end{tcolorbox}
    \end{center}
\end{frame}

\end{document}