\documentclass{beamer}
\usepackage[utf8]{inputenc}
\usepackage[T5]{fontenc} % Bắt buộc để hiển thị tiếng Việt
\usepackage[vietnamese]{babel}
\usepackage{tcolorbox}
\usepackage{listings}
\usepackage{xcolor}
\usepackage{booktabs}
\usetheme{Madrid}

\definecolor{codegreen}{rgb}{0,0.6,0}
\definecolor{codegray}{rgb}{0.5,0.5,0.5}
\definecolor{codepurple}{rgb}{0.58,0,0.82}
\definecolor{backcolour}{rgb}{0.95,0.95,0.92}

\lstdefinestyle{mystyle}{
    backgroundcolor=\color{backcolour},   
    commentstyle=\color{codegreen},
    keywordstyle=\color{magenta},
    numberstyle=\tiny\color{codegray},
    stringstyle=\color{codepurple},
    basicstyle=\ttfamily\scriptsize,
    breakatwhitespace=false,         
    breaklines=true,                 
    captionpos=b,                    
    keepspaces=true,                 
    numbers=left,                    
    numbersep=4pt,                  
    showspaces=false,                
    showstringspaces=false,
    showtabs=false,                  
    tabsize=2,
    escapechar=@
}

\lstset{style=mystyle}

\title{Beat The Odds - Codeforces 1691A}
\subtitle{Tư duy thuật toán theo phong cách "Learning How to Learn"}
\author{Coach Tư Duy Thuật Toán}
\date{\today}

\begin{document}

\begin{frame}
    \titlepage
\end{frame}

\begin{frame}{Bước 1: Phẫu thuật đề bài (Briefing)}
    \begin{block}{Cốt lõi vấn đề}
        \begin{itemize}
            \item \textbf{Dữ liệu:} Một dãy số nguyên (mảng $a$).
            \item \textbf{Yêu cầu:} Loại bỏ một số lượng \textbf{ít nhất} các con số sao cho: Với mọi cặp số nằm cạnh nhau trong dãy còn lại, \textbf{tổng của chúng phải là một số chẵn}.
        \end{itemize}
    \end{block}

    \begin{exampleblock}{Lộ trình tư duy}
        \begin{enumerate}
            \item Mảnh ghép 1: Bản chất của việc "Tổng hai số là số chẵn".
            \item Mảnh ghép 2: Hiệu ứng dây chuyền trong một dãy số.
            \item Mảnh ghép 3: Chiến thuật "Hy sinh ít nhất".
        \end{enumerate}
    \end{exampleblock}
\end{frame}

\begin{frame}{Chunk 1: Quy luật của sự "Chẵn hóa"}
    Để tổng của 2 số là một số \textbf{Chẵn}, có hai kịch bản xảy ra:
    \begin{itemize}
        \item \textbf{Chẵn + Chẵn = Chẵn} (Ví dụ: $2 + 4 = 6$)
        \item \textbf{Lẻ + Lẻ = Chẵn} (Ví dụ: $3 + 5 = 8$)
    \end{itemize}

    \begin{alertblock}{Bẫy logic}
        Nếu bạn để một số Chẵn đứng cạnh một số Lẻ ($2 + 3 = 5$), tổng của chúng sẽ là số Lẻ. Đây là điều đề bài "cấm".
    \end{alertblock}
\end{frame}

\begin{frame}{Thử thách tư duy 1}
    Giả sử chúng ta có một dãy gồm 3 số đứng cạnh nhau: $a_1, a_2, a_3$. Để cả hai cặp $(a_1, a_2)$ và $(a_2, a_3)$ đều có tổng là số chẵn, thì mối quan hệ của $a_1$ và $a_3$ phải như thế nào?
    
    \begin{itemize}
        \item \textbf{A.} $a_1$ và $a_3$ có thể khác tính chẵn lẻ.
        \item \textbf{B.} $a_1$ và $a_3$ bắt buộc phải cùng tính chẵn lẻ với nhau.
    \end{itemize}
    
    \pause
    \begin{block}{Đáp án: B}
        \textbf{Hệ quả:} Trong một dãy thỏa mãn yêu cầu, \textbf{tất cả các số trong dãy phải có cùng tính chẵn lẻ} (tất cả cùng chẵn hoặc tất cả cùng lẻ).
    \end{block}
\end{frame}

\begin{frame}{Chunk 2: Chiến thuật "Hy sinh ít nhất"}
    Bạn chỉ có 2 lựa chọn (Option) để dãy hợp lệ:
    \begin{itemize}
        \item \textbf{Option 1:} Giữ lại toàn bộ số \textbf{Chẵn}, xóa sạch số \textbf{Lẻ}.
        \item \textbf{Option 2:} Giữ lại toàn bộ số \textbf{Lẻ}, xóa sạch số \textbf{Chẵn}.
    \end{itemize}

    \begin{block}{Ẩn dụ hóa}
        Rổ trái cây gồm Cam (Chẵn) và Táo (Lẻ). Bạn muốn vứt đi \textbf{ít nhất} để trong rổ chỉ còn 1 loại quả.
    \end{block}
\end{frame}

\begin{frame}{Thử thách tư duy 2}
    Giả sử trong rổ (mảng $a$) có:
    \begin{itemize}
        \item 7 quả Cam (7 số chẵn)
        \item 3 quả Táo (3 số lẻ)
    \end{itemize}

    \textbf{Câu hỏi:} Số lượng quả \textbf{ít nhất} bạn phải vứt đi là bao nhiêu?
    
    \pause
    \begin{exampleblock}{Kết luận}
        \textbf{Kết quả là 3.} Ta sẽ cho phe "yếu thế" hơn (số lượng ít hơn) bay màu để giữ lại phe đông đảo hơn.
    \end{exampleblock}
\end{frame}

\begin{frame}{Chunk 3: Tổng kết \& Thuật toán}
    Bài toán đưa về 3 bước đơn giản:
    \begin{enumerate}
        \item \textbf{Bước 1:} Đếm xem trong dãy có bao nhiêu số \textbf{Chẵn}.
        \item \textbf{Bước 2:} Đếm xem trong dãy có bao nhiêu số \textbf{Lẻ}.
        \item \textbf{Bước 3:} Kết quả chính là \textbf{số nhỏ hơn} trong hai kết quả đếm trên.
    \end{enumerate}
\end{frame}

\begin{frame}[fragile]{Mã giả (Pseudocode)}
\begin{lstlisting}[language=Python]
Nhap vao so bo test (t)
Voi moi bo test:
    Nhap vao so luong phan tu (n)
    Bien dem_le = 0
    Bien dem_chan = 0
    
    Voi moi so x trong day:
        Neu x chia het cho 2:
            dem_chan = dem_chan + 1
        Neu khong:
            dem_le = dem_le + 1
            
    In ra gia tri nho nhat giua (dem_chan, dem_le)
\end{lstlisting}
\end{frame}

\begin{frame}{Thử thách cuối cùng}
    \begin{alertblock}{Tư duy biên (Edge Cases)}
        Nếu dãy chỉ có đúng 2 số và cả hai đều là số lẻ (ví dụ: $[3, 5]$), thì theo thuật toán trên, bạn sẽ phải xóa bao nhiêu số? Và dãy còn lại có thỏa mãn đề bài không?
    \end{alertblock}
    
    \pause
    \begin{block}{Gợi ý}
        Dãy còn lại 1 số mặc định là thỏa mãn vì không còn cặp nào kề nhau để vi phạm. Thuật toán trả về 0 (vì \texttt{dem\_chan} = 0).
    \end{block}
    
    \begin{center}
        \textbf{Chúc bạn lập trình thành công!}
    \end{center}
\end{frame}

\end{document}