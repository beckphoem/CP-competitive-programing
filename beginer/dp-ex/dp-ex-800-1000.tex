\documentclass{beamer}
\usepackage[utf8]{inputenc}
\usepackage[T5]{fontenc} % Bắt buộc để hiển thị tiếng Việt
\usepackage[vietnamese]{babel}
\usepackage{tcolorbox}
\usepackage{listings}
\usepackage{xcolor}
\usepackage{booktabs}
\usetheme{Madrid}

\definecolor{codegreen}{rgb}{0,0.6,0}
\definecolor{codegray}{rgb}{0.5,0.5,0.5}
\definecolor{codepurple}{rgb}{0.58,0,0.82}
\definecolor{backcolour}{rgb}{0.95,0.95,0.92}

\lstdefinestyle{mystyle}{
    backgroundcolor=\color{backcolour},   
    commentstyle=\color{codegreen},
    keywordstyle=\color{magenta},
    numberstyle=\tiny\color{codegray},
    stringstyle=\color{codepurple},
    basicstyle=\ttfamily\scriptsize,
    breakatwhitespace=false,         
    breaklines=true,                 
    captionpos=b,                    
    keepspaces=true,                 
    numbers=left,                    
    numbersep=4pt,                  
    showspaces=false,                
    showstringspaces=false,
    showtabs=false,                  
    tabsize=2,
    escapechar=@
}

\lstset{style=mystyle}

% Thông tin bài trình bày
\title[DP Codeforces 800-1000]{Tuyển tập 20 Bài tập Quy hoạch động trên Codeforces\\(Rating 800-1000)}
\subtitle{Phân tích Hệ thống và Hướng giải quyết}
\author{Biên soạn từ tài liệu}
\date{\today}

\begin{document}

% Slide tiêu đề
\begin{frame}
    \titlepage
\end{frame}

% Slide Mục lục/Giới thiệu
\begin{frame}{Giới thiệu chung}
    \begin{block}{Mục tiêu}
        Nghiên cứu và áp dụng thuật toán Quy hoạch động (DP) trong phạm vi xếp hạng 800-1000 trên Codeforces. Đây là bước chuyển đổi quan trọng từ tư duy lập trình cơ bản sang tư duy thuật toán chuyên sâu.
    \end{block}

    \begin{alertblock}{Đặc điểm}
        Các bài toán thường yêu cầu sự kết hợp giữa kỹ năng triển khai mã nguồn và khả năng nhận diện các cấu trúc con tối ưu.
        \begin{itemize}
            \item \textbf{Phương pháp:} Memoization hoặc Tabulation đơn giản.
            \item \textbf{Dạng bài:} DP một chiều hoặc đếm trạng thái.
        \end{itemize}
    \end{alertblock}
\end{frame}

% Slide Bảng tổng hợp
\begin{frame}{Phân loại theo mức xếp hạng}
    \begin{table}
        \centering
        \small
        \begin{tabular}{c p{4cm} p{4cm}}
            \toprule
            \textbf{Rating} & \textbf{Đặc điểm chính} & \textbf{Kỹ năng yêu cầu} \\
            \midrule
            800 & Truy hồi đơn giản, đếm trạng thái tuyến tính. & Cấu trúc vòng lặp, mảng một chiều. \\
            \midrule
            900 & Tối ưu hóa lựa chọn, mảng cộng dồn. & Tư duy tham lam kết hợp lưu trữ trạng thái. \\
            \midrule
            1000 & Bài toán cái túi đơn giản, DP trên chuỗi/số học. & Nhận diện cấu trúc con tối ưu. \\
            \bottomrule
        \end{tabular}
        \caption{Tổng hợp đặc điểm bài toán DP theo Rating}
    \end{table}
\end{frame}

% --- Bắt đầu danh sách bài tập ---

% Bài 1
\begin{frame}{1. The Way to Home (910A)}
    \begin{itemize}
        \item \textbf{Link:} \url{https://codeforces.com/problemset/problem/910/A}
        \item \textbf{Rating:} 800
    \end{itemize}
    \begin{block}{Đề bài}
        Một chú ếch ở vị trí 1 muốn đến vị trí $n$. Đường đi có các vị trí '1' (có hoa súng) hoặc '0'. Ếch nhảy tối đa $d$ đơn vị vào nơi có hoa súng. Tìm số bước tối thiểu.
    \end{block}
    \begin{exampleblock}{Hướng giải quyết}
        Bài toán tìm đường ngắn nhất trên mảng một chiều.
        \begin{itemize}
            \item Gọi $dp[i]$ là số bước tối thiểu đến vị trí $i$.
            \item Khởi tạo $dp[1] = 0$, còn lại là vô cùng.
            \item Với mỗi $i$ có hoa súng: $dp[i] = \min(dp[j]) + 1$ với mọi $j$ thỏa mãn $i-d \le j < i$ và $j$ có hoa súng.
            \item Độ phức tạp $O(N \cdot d)$ hoặc $O(N^2)$.
        \end{itemize}
    \end{exampleblock}
\end{frame}

% Bài 2
\begin{frame}{2. Subtract or Divide (1451A)}
    \begin{itemize}
        \item \textbf{Link:} \url{https://codeforces.com/problemset/problem/1451/A}
        \item \textbf{Rating:} 900
    \end{itemize}
    \begin{block}{Đề bài}
        Cho số nguyên $n$. Có thể giảm $n$ đi 1 hoặc thay $n$ bằng $n/d$ (với $d$ là ước của $n$). Tìm số bước tối thiểu để đưa $n$ về 1.
    \end{block}
    \begin{exampleblock}{Hướng giải quyết}
        Mối quan hệ giữa các số mô tả qua trạng thái $dp[i]$ (số bước tối thiểu biến $i$ thành 1).
        \begin{itemize}
            \item Cân nhắc lựa chọn tối ưu giữa trừ 1 và chia.
            \item Thực tế có thể giải bằng Toán học/Tham lam dựa trên tính chẵn lẻ, nhưng tư duy DP giúp hiểu về chuyển trạng thái.
        \end{itemize}
    \end{exampleblock}
\end{frame}

% Bài 3
\begin{frame}{3. Alex and a Rhombus (1180A)}
    \begin{itemize}
        \item \textbf{Link:} \url{https://codeforces.com/problemset/problem/1180/A}
        \item \textbf{Rating:} 800
    \end{itemize}
    \begin{block}{Đề bài}
        Tính số ô vuông trong hình thoi bậc $n$ được tạo ra bằng cách thêm các ô vào cạnh hình thoi bậc $n-1$.
    \end{block}
    \begin{exampleblock}{Hướng giải quyết}
        Minh họa cấu trúc truy hồi cơ bản:
        \begin{itemize}
            \item Gọi $f(n)$ là số ô vuông bậc $n$.
            \item Công thức truy hồi: $f(n) = f(n-1) + 4(n-1)$.
            \item Cơ sở: $f(1) = 1$.
        \end{itemize}
    \end{exampleblock}
\end{frame}

% Bài 4
\begin{frame}{4. Non-Substring Subsequence (1451B)}
    \begin{itemize}
        \item \textbf{Link:} \url{https://codeforces.com/problemset/problem/1451/B}
        \item \textbf{Rating:} 900
    \end{itemize}
    \begin{block}{Đề bài}
        Cho chuỗi $s$ và truy vấn $(l, r)$. Kiểm tra xem có tồn tại chuỗi con bằng $s[l \dots r]$ nhưng không phải là chuỗi con liên tiếp tại vị trí đó không.
    \end{block}
    \begin{exampleblock}{Hướng giải quyết}
        Lưu trữ trạng thái tồn tại của ký tự:
        \begin{itemize}
            \item Để chuỗi không liên tiếp, chỉ cần tìm xem có ký tự $s[l]$ nào ở vị trí $< l$ không, hoặc $s[r]$ nào ở vị trí $> r$ không.
            \item Sử dụng tư duy tiền xử lý (Preprocessing) giống DP để trả lời truy vấn nhanh.
        \end{itemize}
    \end{exampleblock}
\end{frame}

% Bài 5
\begin{frame}{5. Fair Division (1472B)}
    \begin{itemize}
        \item \textbf{Link:} \url{https://codeforces.com/problemset/problem/1472/B}
        \item \textbf{Rating:} 800
    \end{itemize}
    \begin{block}{Đề bài}
        Chia kẹo loại 1g và 2g thành hai phần có tổng trọng lượng bằng nhau.
    \end{block}
    \begin{exampleblock}{Hướng giải quyết}
        Biến thể của bài toán Subset Sum (Cái túi):
        \begin{itemize}
            \item Gọi $dp[w]$ là trạng thái boolean: có thể tạo ra tổng trọng lượng $w$ hay không.
            \item Do giá trị nhỏ, có thể kiểm tra tính chẵn lẻ của số lượng kẹo 1g và 2g kết hợp tổng trọng lượng.
        \end{itemize}
    \end{exampleblock}
\end{frame}

% Bài 6
\begin{frame}{6. Beat The Odds (1691A)}
    \begin{itemize}
        \item \textbf{Link:} \url{https://codeforces.com/problemset/problem/1691/A}
        \item \textbf{Rating:} 800
    \end{itemize}
    \begin{block}{Đề bài}
        Tìm số phần tử tối thiểu cần loại bỏ để tổng mọi cặp liên tiếp là chẵn.
    \end{block}
    \begin{exampleblock}{Hướng giải quyết}
        \begin{itemize}
            \item Tổng hai số là chẵn khi cả hai cùng chẵn hoặc cùng lẻ.
            \item Để mọi cặp có tổng chẵn, dãy còn lại phải toàn số chẵn hoặc toàn số lẻ.
            \item Đếm số lượng số chẵn ($n_{even}$) và số lẻ ($n_{odd}$).
            \item Kết quả: $\min(n_{even}, n_{odd})$.
        \end{itemize}
    \end{exampleblock}
\end{frame}

% Bài 7
\begin{frame}{7. Even Subset Sum Problem (1323A)}
    \begin{itemize}
        \item \textbf{Link:} \url{https://codeforces.com/problemset/problem/1323/A}
        \item \textbf{Rating:} 800
    \end{itemize}
    \begin{block}{Đề bài}
        Tìm một tập con (các chỉ số) sao cho tổng các phần tử là số chẵn.
    \end{block}
    \begin{exampleblock}{Hướng giải quyết}
        \begin{itemize}
            \item Nếu có phần tử chẵn: Chọn chính nó (tập kích thước 1).
            \item Nếu không có phần tử chẵn (toàn lẻ): Chọn 2 phần tử lẻ bất kỳ (Tổng: Lẻ + Lẻ = Chẵn).
            \item Nếu $n=1$ và phần tử đó lẻ: Vô nghiệm.
        \end{itemize}
    \end{exampleblock}
\end{frame}

% Bài 8
\begin{frame}{8. Ordinary Numbers (1520B)}
    \begin{itemize}
        \item \textbf{Link:} \url{https://codeforces.com/problemset/problem/1520/B}
        \item \textbf{Rating:} 800
    \end{itemize}
    \begin{block}{Đề bài}
        Đếm số lượng các số nguyên dương từ 1 đến $n$ có tất cả các chữ số giống nhau.
    \end{block}
    \begin{exampleblock}{Hướng giải quyết}
        Tiền đề cho Digit DP:
        \begin{itemize}
            \item Liệt kê các số có dạng 1, 2...9, 11, 22...99, 111...
            \item Đếm trực tiếp các số này thỏa mãn $\le n$.
            \item Có tổng cộng rất ít số như vậy trong phạm vi bài toán.
        \end{itemize}
    \end{exampleblock}
\end{frame}

% Bài 9
\begin{frame}{9. Squares and Cubes (1619B)}
    \begin{itemize}
        \item \textbf{Link:} \url{https://codeforces.com/problemset/problem/1619/B}
        \item \textbf{Rating:} 800
    \end{itemize}
    \begin{block}{Đề bài}
        Đếm các số từ 1 đến $n$ là số chính phương ($k^2$) hoặc lập phương ($k^3$).
    \end{block}
    \begin{exampleblock}{Hướng giải quyết}
        Sử dụng Set để tránh trùng lặp hoặc Nguyên lý bao hàm-loại trừ:
        \begin{itemize}
            \item Số lượng = (Số số $x^2 \le n$) + (Số số $x^3 \le n$) - (Số số $x^6 \le n$).
            \item $x^6$ là số vừa là chính phương vừa là lập phương.
        \end{itemize}
    \end{exampleblock}
\end{frame}

% Bài 10
\begin{frame}{10. Extremely Round (1766A)}
    \begin{itemize}
        \item \textbf{Link:} \url{https://codeforces.com/problemset/problem/1766/A}
        \item \textbf{Rating:} 800
    \end{itemize}
    \begin{block}{Đề bài}
        Đếm các số $\le n$ chỉ có đúng một chữ số khác 0 (Ví dụ: 5, 10, 900).
    \end{block}
    \begin{exampleblock}{Hướng giải quyết}
        Dạng Digit DP đơn giản hoặc liệt kê:
        \begin{itemize}
            \item Các số có dạng $d \times 10^k$ (với $d \in [1,9]$).
            \item Duyệt qua độ dài và chữ số đầu tiên để đếm.
        \end{itemize}
    \end{exampleblock}
\end{frame}

% Bài 11
\begin{frame}{11. 2-3 Moves (1716A)}
    \begin{itemize}
        \item \textbf{Link:} \url{https://codeforces.com/problemset/problem/1716/A}
        \item \textbf{Rating:} 800
    \end{itemize}
    \begin{block}{Đề bài}
        Đứng tại 0, muốn đến $n$. Có thể nhảy $+2, +3, -2, -3$. Tìm số bước tối thiểu.
    \end{block}
    \begin{exampleblock}{Hướng giải quyết}
        DP trạng thái hoặc Tham lam theo modulo:
        \begin{itemize}
            \item Nếu $n=1$: Cần 2 bước (+3, -2).
            \item Nếu $n \% 3 == 0$: $n/3$ bước.
            \item Nếu $n \% 3 == 1$: $n/3 + 1$ bước (bớt một bước +3, thay bằng hai bước +2).
            \item Nếu $n \% 3 == 2$: $n/3 + 1$ bước (thêm một bước +2).
        \end{itemize}
    \end{exampleblock}
\end{frame}

% Bài 12
\begin{frame}{12. Maximum GCD (1370A)}
    \begin{itemize}
        \item \textbf{Link:} \url{https://codeforces.com/problemset/problem/1370/A}
        \item \textbf{Rating:} 800
    \end{itemize}
    \begin{block}{Đề bài}
        Tìm $\gcd(a, b)$ lớn nhất với $1 \le a < b \le n$.
    \end{block}
    \begin{exampleblock}{Hướng giải quyết}
        Tìm giá trị tối ưu:
        \begin{itemize}
            \item Để GCD lớn nhất là $g$, ta cần ít nhất $g$ và $2g$ nằm trong khoảng $[1, n]$.
            \item Điều kiện $2g \le n \Rightarrow g \le n/2$.
            \item Đáp án luôn là $\lfloor n/2 \rfloor$.
        \end{itemize}
    \end{exampleblock}
\end{frame}

% Bài 13
\begin{frame}{13. Team Training (73B)}
    \begin{itemize}
        \item \textbf{Link:} \url{https://codeforces.com/problemset/problem/73/B}
        \item \textbf{Rating:} 800 (Lưu ý: Tên bài có thể là 519C, ID 73B là bài khác, nhưng giải theo logic Team Training)
    \end{itemize}
    \begin{block}{Đề bài}
        Có $n$ lập trình viên kinh nghiệm và $m$ người mới. Mỗi đội cần (1 kinh nghiệm, 2 mới) hoặc (2 kinh nghiệm, 1 mới). Tối đa bao nhiêu đội?
    \end{block}
    \begin{exampleblock}{Hướng giải quyết}
        Bài toán chia tài nguyên có cấu trúc tối ưu:
        \begin{itemize}
            \item Số đội tối đa không vượt quá $n$, không vượt quá $m$.
            \item Tổng số người dùng là $3 \times k$.
            \item Kết quả: $\min(n, m, (n+m)/3)$.
        \end{itemize}
    \end{exampleblock}
\end{frame}

% Bài 14
\begin{frame}{14. No Casino in the Mountains (41B)}
    \begin{itemize}
        \item \textbf{Link:} \url{https://codeforces.com/problemset/problem/41/B}
        \item \textbf{Rating:} 800
    \end{itemize}
    \begin{block}{Đề bài}
        (Bài toán Martian Dollar) Mua đô la ngày $i$ giá $a_i$ và bán ngày $j > i$ giá $a_j$ để tối đa hóa tiền.
    \end{block}
    \begin{exampleblock}{Hướng giải quyết}
        \begin{itemize}
            \item Duyệt qua từng ngày mua $i$.
            \item Tìm giá bán cao nhất trong các ngày sau $i$ (có thể dùng DP hậu tố max suffix để tối ưu, hoặc duyệt trâu vì $N$ nhỏ).
            \item Tính lợi nhuận tối đa cho từng phương án.
        \end{itemize}
    \end{exampleblock}
\end{frame}

% Bài 15
\begin{frame}{15. Minimum Varied Number (1714C)}
    \begin{itemize}
        \item \textbf{Link:} \url{https://codeforces.com/problemset/problem/1714/C}
        \item \textbf{Rating:} 800
    \end{itemize}
    \begin{block}{Đề bài}
        Tìm số nhỏ nhất có các chữ số khác nhau và tổng các chữ số bằng $s$.
    \end{block}
    \begin{exampleblock}{Hướng giải quyết}
        Tham lam/Xây dựng từ cuối lên:
        \begin{itemize}
            \item Để số nhỏ nhất, số lượng chữ số phải ít nhất (các chữ số ở hàng đơn vị phải lớn nhất có thể).
            \item Lấy các số 9, 8, 7... lần lượt trừ vào $s$ từ hàng đơn vị lên hàng cao hơn.
        \end{itemize}
    \end{exampleblock}
\end{frame}

% Bài 16
\begin{frame}{16. Tanya and Stairways (1005A)}
    \begin{itemize}
        \item \textbf{Link:} \url{https://codeforces.com/problemset/problem/1005/A}
        \item \textbf{Rating:} 800
    \end{itemize}
    \begin{block}{Đề bài}
        Đếm số cầu thang và số bậc của mỗi cầu thang dựa trên dãy số đếm bậc (1, 2, 3, 1, 2...).
    \end{block}
    \begin{exampleblock}{Hướng giải quyết}
        \begin{itemize}
            \item Mỗi khi gặp số 1, đó là bắt đầu một cầu thang mới.
            \item Số bậc của cầu thang trước đó chính là giá trị phần tử ngay trước số 1 mới (hoặc phần tử cuối cùng của dãy).
        \end{itemize}
    \end{exampleblock}
\end{frame}

% Bài 17
\begin{frame}{17. Polycarp's Pockets (1003A)}
    \begin{itemize}
        \item \textbf{Link:} \url{https://codeforces.com/problemset/problem/1003/A}
        \item \textbf{Rating:} 800
    \end{itemize}
    \begin{block}{Đề bài}
        Có $n$ đồng xu với các mệnh giá. Chia vào các túi sao cho mỗi túi không có 2 đồng cùng mệnh giá. Tìm số túi tối thiểu.
    \end{block}
    \begin{exampleblock}{Hướng giải quyết}
        \begin{itemize}
            \item Số túi tối thiểu chính bằng tần suất xuất hiện lớn nhất của một mệnh giá bất kỳ trong mảng.
            \item Dùng mảng đếm (Frequency Array).
        \end{itemize}
    \end{exampleblock}
\end{frame}

% Bài 18
\begin{frame}{18. Construct the String (1335B)}
    \begin{itemize}
        \item \textbf{Link:} \url{https://codeforces.com/problemset/problem/1335/B}
        \item \textbf{Rating:} 800
    \end{itemize}
    \begin{block}{Đề bài}
        Xây dựng chuỗi độ dài $n$, mọi chuỗi con độ dài $a$ có đúng $b$ ký tự khác nhau.
    \end{block}
    \begin{exampleblock}{Hướng giải quyết}
        Sử dụng tính chất chu kỳ:
        \begin{itemize}
            \item Tạo một mẫu chu kỳ gồm $b$ ký tự khác nhau (ví dụ: 'abc...').
            \item Lặp lại mẫu này cho đến khi đủ độ dài $n$.
            \item $S[i] = \text{'a'} + (i \pmod b)$.
        \end{itemize}
    \end{exampleblock}
\end{frame}

% Bài 20 (Note: Skipping 19 as per source)
\begin{frame}{20. Water Buying (1118A)}
    \begin{itemize}
        \item \textbf{Link:} \url{https://codeforces.com/problemset/problem/1118/A}
        \item \textbf{Rating:} 800
    \end{itemize}
    \begin{block}{Đề bài}
        Cần mua $n$ lít nước. Chai 1L giá $a$, chai 2L giá $b$. Tìm chi phí tối thiểu.
    \end{block}
    \begin{exampleblock}{Hướng giải quyết}
        Bài toán Knapsack cơ bản/Tham lam:
        \begin{itemize}
            \item So sánh giá mua 2 chai 1L ($2 \times a$) và 1 chai 2L ($b$).
            \item Nếu $2a \le b$: Mua toàn bộ bằng chai 1L.
            \item Nếu $2a > b$: Ưu tiên mua chai 2L (với $n/2$ chai), nếu lẻ thì mua thêm 1 chai 1L.
        \end{itemize}
    \end{exampleblock}
\end{frame}

% Slide Phân tích & Lộ trình
\begin{frame}{Phân tích Sâu và Chiến lược Luyện tập}
    \begin{block}{Sự hình thành tư duy}
        Ở mức 800-1000, khái niệm "trạng thái" và "truy hồi" được lồng ghép trong các bài toán triển khai (Implementation). Hiểu sâu về Prefix Sums (DP đơn giản) là bước đệm quan trọng.
    \end{block}

    \begin{alertblock}{Lộ trình đề xuất}
        \begin{enumerate}
            \item \textbf{Mức 800:} Chuyển đổi ý tưởng thành mã nguồn sạch.
            \item \textbf{Mức 900:} Vẽ cây quyết định hoặc bảng trạng thái trên giấy.
            \item \textbf{Mức 1000:} Làm quen bài toán con chồng lấn (CP-31 Sheet).
        \end{enumerate}
    \end{alertblock}
\end{frame}

\begin{frame}{Tương quan xếp hạng}
    \begin{table}
        \centering
        \small
        \begin{tabular}{l l l}
            \toprule
            \textbf{Codeforces} & \textbf{USACO} & \textbf{Trình độ tương ứng} \\
            \midrule
            800 - 1000 & Beginner / Bronze & Newbie / Pupil \\
            1100 - 1500 & Bronze / Silver & Pupil / Specialist \\
            \bottomrule
        \end{tabular}
    \end{table}
    \begin{center}
        \textit{"Quy hoạch động không phải là rào cản, mà là cánh cửa mở ra những giải pháp thông minh."}
    \end{center}
\end{frame}

\end{document}