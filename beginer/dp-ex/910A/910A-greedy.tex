\documentclass{beamer}
\usepackage[utf8]{inputenc}
\usepackage[T5]{fontenc} % Bắt buộc để hiển thị tiếng Việt
\usepackage[vietnamese]{babel}
\usepackage{tcolorbox}
\usepackage{listings}
\usepackage{xcolor}
\usepackage{booktabs}
\usetheme{Madrid}

\definecolor{codegreen}{rgb}{0,0.6,0}
\definecolor{codegray}{rgb}{0.5,0.5,0.5}
\definecolor{codepurple}{rgb}{0.58,0,0.82}
\definecolor{backcolour}{rgb}{0.95,0.95,0.92}

\lstdefinestyle{mystyle}{
    backgroundcolor=\color{backcolour},   
    commentstyle=\color{codegreen},
    keywordstyle=\color{magenta},
    numberstyle=\tiny\color{codegray},
    stringstyle=\color{codepurple},
    basicstyle=\ttfamily\scriptsize,
    breakatwhitespace=false,         
    breaklines=true,                 
    captionpos=b,                    
    keepspaces=true,                 
    numbers=left,                    
    numbersep=4pt,                  
    showspaces=false,                
    showstringspaces=false,
    showtabs=false,                  
    tabsize=2,
    escapechar=@
}

\lstset{style=mystyle}

\title[Codeforces 910A]{Huấn luyện Tư duy Thuật toán}
\subtitle{Bài 910A - The Way to Home}
\author{Slide Learning C++}
\date{\today}

\begin{document}

\begin{frame}
  \titlepage
\end{frame}

\begin{frame}{Chào mừng!}
  \begin{block}{Lời chào}
    Chào bạn! Tôi đã sẵn sàng nhập vai \textbf{Huấn luyện viên Tư duy Thuật toán} của bạn. Chúng ta sẽ cùng nhau "mổ xẻ" bài \textbf{Codeforces 910A - The Way to Home} theo đúng triết lý chia nhỏ kiến thức (Micro-chunking) để bạn nắm vững bản chất nhé.
  \end{block}
\end{frame}

\begin{frame}{Bước 1: Phẫu thuật đề bài (Deconstruct)}
  \begin{itemize}
    \item \textbf{Bối cảnh:} Bạn ở vị trí số 1, mục tiêu là đến vị trí $n$.
    \item \textbf{Địa hình:} Con đường là chuỗi ký tự '0' và '1'.
    \begin{itemize}
      \item \texttt{'1'}: Điểm an toàn (có thể đứng).
      \item \texttt{'0'}: Điểm nguy hiểm (không được đặt chân vào).
    \end{itemize}
    \item \textbf{Khả năng:} Nhảy xa tối đa $d$ đơn vị.
    \item \textbf{Nhiệm vụ:} Tìm số bước nhảy \textbf{ít nhất} hoặc báo không thể.
  \end{itemize}
  
  \begin{block}{Lộ trình tư duy}
    1. Chiến thuật nhảy tối ưu.\\
    2. Xử lý bẫy logic.\\
    3. Tổng kết thuật toán Tham lam (Greedy).
  \end{block}
\end{frame}

\begin{frame}{Mảnh ghép 1: Chiến thuật "Cú nhảy xa nhất"}
  \begin{exampleblock}{Ẩn dụ}
    Nếu bạn có thể nhảy tối đa 3 mét, và có các hòn đá ở khoảng cách 1m, 2m, và 3m. Để về đích với ít bước nhất, bạn chọn hòn đá nào?
  \end{exampleblock}
  \pause
  \begin{itemize}
    \item \textbf{Tham lam (Greedy):} Tại mỗi bước, cố gắng tiến xa nhất có thể trong phạm vi cho phép.
  \end{itemize}
  
  \begin{alertblock}{Bẫy logic}
    Không chỉ kiểm tra điểm cách đúng $d$ đơn vị. Nếu đó là '0', ta phải "lùi dần" tìm điểm '1' gần nhất với khoảng cách tối đa.
  \end{alertblock}
\end{frame}

\begin{frame}{Thử thách tư duy 1}
  \textbf{Giả sử:}
  \begin{itemize}
    \item Vị trí hiện tại: \textbf{1}, $d = 4$
    \item Con đường: \texttt{101101...} (Vị trí 1 đến 6)
  \end{itemize}
  
  \begin{table}[]
    \begin{tabular}{lcccccc}
    \toprule
    Vị trí & 1 & 2 & 3 & 4 & 5 & 6 \\ \midrule
    Trạng thái & 1 & 0 & 1 & 1 & 0 & 1 \\ \bottomrule
    \end{tabular}
  \end{table}

  \textbf{Câu hỏi:} Từ vị trí 1, bước tiếp theo bạn nhảy đến đâu?
  \pause
  \begin{block}{Đáp án}
    Nhảy đến vị trí \textbf{4}. Vì đó là điểm '1' xa nhất trong tầm với ($4-1=3 \le 4$).
  \end{block}
\end{frame}

\begin{frame}{Mảnh ghép 2: Khi nào thì "bó tay"?}
  \begin{alertblock}{Tình huống xấu nhất}
    Chuyện gì xảy ra nếu trong tầm nhảy $d$ toàn là số '0'?
  \end{alertblock}

  \textbf{Thử thách tư duy 2:} 
  Vị trí hiện tại là 1, $d = 2$, đoạn đường: \texttt{10001}
  \begin{itemize}
    \item Vị trí 1: \texttt{1} (Hiện tại)
    \item Vị trí 2, 3, 4: \texttt{0}
    \item Vị trí 5: \texttt{1} (Đích)
  \end{itemize}
  \pause
  \begin{block}{Kết quả}
    Bạn không thể nhảy đi đâu cả! Kết quả bài toán lúc này thường sẽ được yêu cầu trả về \textbf{-1}.
  \end{block}
\end{frame}

\begin{frame}[fragile]{Mảnh ghép cuối cùng: Mã giả (Pseudocode)}
  Hãy điền vào logic còn thiếu:
  \begin{lstlisting}[language=C++, style=mystyle]
1. @\texttt{vi\_tri\_hien\_tai}@ = 0
2. @\texttt{so\_buoc}@ = 0
3. Lap cho den khi @\texttt{vi\_tri\_hien\_tai}@ == @\texttt{dich}@:
   - @\texttt{xa\_nhat}@ = @\texttt{vi\_tri\_hien\_tai}@ + d
   - Neu @\texttt{xa\_nhat}@ > @\texttt{dich}@, @\texttt{xa\_nhat}@ = @\texttt{dich}@
   - Vong lap tim kiem: Tu @\texttt{xa\_nhat}@ lùi dần về @\texttt{vi\_tri\_hien\_tai}@ + 1:
     + Neu gap '1':
        * Cap nhat @\texttt{vi\_tri\_hien\_tai}@ = vi tri do
        * Tang @\texttt{so\_buoc}@ len 1
        * Thoat vong lap tim kiem
   - Neu khong tim thay '1' nao:
     + Tra ve [...] va dung chuong trinh.
  \end{lstlisting}
\end{frame}

\begin{frame}{Tổng kết}
  \begin{block}{Câu hỏi cuối cùng}
    Theo bạn, ở bước cuối cùng (nếu không tìm thấy số '1' nào), chúng ta nên trả về giá trị bao nhiêu?
  \end{block}
  \pause
  \begin{itemize}
    \item \textbf{Đáp án:} Trả về \textbf{-1}.
  \end{itemize}

  \begin{exampleblock}{Tiếp theo}
    Bạn có muốn tôi hỗ trợ chuyển logic này sang C++ hay Python không, hay bạn muốn tự mình "thử lửa" trước?
  \end{exampleblock}
\end{frame}

\end{document}