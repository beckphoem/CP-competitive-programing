\documentclass{beamer}
\usepackage[utf8]{inputenc}
\usepackage[T5]{fontenc} % Bắt buộc để hiển thị tiếng Việt
\usepackage[vietnamese]{babel}
\usepackage{tcolorbox}
\usepackage{listings}
\usepackage{xcolor}
\usepackage{booktabs}
\usetheme{Madrid}

\definecolor{codegreen}{rgb}{0,0.6,0}
\definecolor{codegray}{rgb}{0.5,0.5,0.5}
\definecolor{codepurple}{rgb}{0.58,0,0.82}
\definecolor{backcolour}{rgb}{0.95,0.95,0.92}

\lstdefinestyle{mystyle}{
    backgroundcolor=\color{backcolour},   
    commentstyle=\color{codegreen},
    keywordstyle=\color{magenta},
    numberstyle=\tiny\color{codegray},
    stringstyle=\color{codepurple},
    basicstyle=\ttfamily\scriptsize,
    breakatwhitespace=false,         
    breaklines=true,                 
    captionpos=b,                    
    keepspaces=true,                 
    numbers=left,                    
    numbersep=4pt,                  
    showspaces=false,                
    showstringspaces=false,
    showtabs=false,                  
    tabsize=2,
    escapechar=@
}

\lstset{style=mystyle}

\title{Luyện tập Map trong Competitive Programming}
\subtitle{Danh sách bài tập Codeforces Rank 800 - 1000}
\author{slide learning cpp}
\date{\today}

\begin{document}

\begin{frame}
    \titlepage
\end{frame}

\begin{frame}{Giới thiệu về Map trong CP}
    \begin{block}{Sức mạnh của Map}
        Cấu trúc dữ liệu \texttt{map} (\texttt{std::map} hoặc \texttt{std::unordered\_map}) là công cụ mạnh mẽ để:
        \begin{itemize}
            \item Đếm tần suất xuất hiện của phần tử.
            \item Quản lý cặp khóa - giá trị (Key-Value).
            \item Tối ưu hóa phép tìm kiếm.
        \end{itemize}
    \end{block}
    
    \begin{alertblock}{Lưu ý về hiệu năng}
        \begin{itemize}
            \item \texttt{std::map}: Độ phức tạp $O(\log n)$, các khóa được sắp xếp.
            \item \texttt{std::unordered\_map}: Độ phức tạp trung bình $O(1)$, không sắp xếp. Cẩn thận bị "hack" $O(n)$ trên Codeforces.
        \end{itemize}
    \end{alertblock}
\end{frame}

\begin{frame}{Bài 1: 1003A - Polycarp's Pockets}
    \textbf{Link:} \url{https://codeforces.com/problemset/problem/1003/A} \\
    \textbf{Rank:} 800
    
    \begin{block}{Dịch đề bài}
        Cho $n$ đồng xu với các mệnh giá khác nhau. Cần ít nhất bao nhiêu cái túi để không có 2 đồng xu cùng mệnh giá trong 1 túi?
    \end{block}
    
    \begin{exampleblock}{Chiến lược giải với Map}
        \begin{itemize}
            \item Dùng \texttt{map<int, int>} để đếm tần suất của từng mệnh giá.
            \item \pause \textbf{Kết quả:} Là giá trị lớn nhất trong các tần suất đã đếm.
        \end{itemize}
    \end{exampleblock}
\end{frame}

\begin{frame}{Bài 2: 1454B - Unique Bid Auction}
    \textbf{Link:} \url{https://codeforces.com/problemset/problem/1454/B} \\
    \textbf{Rank:} 800
    
    \begin{block}{Dịch đề bài}
        Tìm người đưa ra giá duy nhất và nhỏ nhất trong cuộc đấu giá.
    \end{block}
    
    \begin{exampleblock}{Chiến lược giải với Map}
        \begin{itemize}
            \item Dùng \texttt{map} lưu tần suất và vị trí (index) của mỗi giá trị.
            \item \pause \textbf{Thực hiện:} Duyệt qua \texttt{map} theo thứ tự khóa tăng dần, tìm khóa đầu tiên có tần suất bằng 1.
        \end{itemize}
    \end{exampleblock}
\end{frame}

\begin{frame}{Bài 3: 1616A - Integer Diversity}
    \textbf{Link:} \url{https://codeforces.com/problemset/problem/1616/A} \\
    \textbf{Rank:} 800
    
    \begin{block}{Dịch đề bài}
        Cho mảng số nguyên, bạn có thể đổi dấu bất kỳ số nào. Tìm số lượng phần tử phân biệt tối đa.
    \end{block}
    
    \begin{exampleblock}{Chiến lược giải với Map}
        \begin{itemize}
            \item Dùng \texttt{map} đếm trị tuyệt đối của các số.
            \item Với số $x > 0$: nếu có từ 2 số trở lên, ta lấy được cả $x$ và $-x$.
            \item Với số $0$: chỉ lấy được duy nhất một số 0.
        \end{itemize}
    \end{exampleblock}
\end{frame}

\begin{frame}{Bài 4: 236A - Boy or Girl}
    \textbf{Link:} \url{https://codeforces.com/problemset/problem/236/A} \\
    \textbf{Rank:} 800
    
    \begin{block}{Dịch đề bài}
        Đếm số lượng ký tự phân biệt trong một cái tên để xác định giới tính.
    \end{block}
    
    \begin{exampleblock}{Chiến lược giải với Map}
        \begin{itemize}
            \item Dùng \texttt{map<char, int>} để đánh dấu các ký tự khác nhau.
            \item \pause \textbf{Kết quả:} Kiểm tra tính chẵn lẻ của \texttt{map.size()}.
        \end{itemize}
    \end{exampleblock}
\end{frame}

\begin{frame}{Bài 6: 4C - Registration System}
    \textbf{Link:} \url{https://codeforces.com/problemset/problem/4/C} \\
    \textbf{Rank:} 1000
    
    \begin{alertblock}{Bài tập kinh điển}
        Quản lý việc đăng ký tên người dùng. Nếu tên đã tồn tại, thêm số thứ tự vào sau.
    \end{alertblock}
    
    \begin{exampleblock}{Chiến lược giải với Map}
        \begin{itemize}
            \item Dùng \texttt{map<string, int>} để lưu số lần mỗi tên đã xuất hiện.
            \item Nếu \texttt{map[s] == 0}, in "OK", ngược lại in \texttt{s + map[s]}.
            \item \pause Sau đó tăng \texttt{map[s]++}.
        \end{itemize}
    \end{exampleblock}
\end{frame}

\begin{frame}{Bài 13: 903C - Boxes Packing}
    \textbf{Link:} \url{https://codeforces.com/problemset/problem/903/C} \\
    \textbf{Rank:} 1000
    
    \begin{block}{Dịch đề bài}
        Hộp nhỏ có thể cho vào hộp lớn hơn. Tìm số hộp ít nhất còn lại sau khi lồng nhau.
    \end{block}
    
    \begin{exampleblock}{Chiến lược giải với Map}
        \begin{itemize}
            \item \textbf{Nhận xét:} Số hộp không thể lồng vào nhau chính là số lượng hộp có cùng kích thước nhiều nhất.
            \item \pause Dùng \texttt{map} đếm tần suất xuất hiện của mỗi kích thước hộp.
            \item \textbf{Đáp án:} Giá trị lớn nhất (max frequency) trong \texttt{map}.
        \end{itemize}
    \end{exampleblock}
\end{frame}

\begin{frame}{Bài 17: 520A - Pangram}
    \textbf{Link:} \url{https://codeforces.com/problemset/problem/520/A} \\
    \textbf{Rank:} 800
    
    \begin{block}{Dịch đề bài}
        Kiểm tra chuỗi có chứa đủ 26 chữ cái tiếng Anh (không phân biệt hoa thường) không.
    \end{block}
    
    \begin{exampleblock}{Chiến lược giải với Map}
        \begin{itemize}
            \item Chuyển toàn bộ chuỗi về chữ thường.
            \item Dùng \texttt{map<char, bool>} để đánh dấu các chữ cái đã xuất hiện.
            \item \pause Kiểm tra nếu \texttt{map.size() == 26}.
        \end{itemize}
    \end{exampleblock}
\end{frame}

\begin{frame}[fragile]{Mẫu Code cơ bản (C++)}
\begin{lstlisting}[language=C++]
#include <iostream>
#include <map>
#include <string>
using namespace std;

int main() {
    int n; cin >> n;
    map<string, int> reg_system;
    while(n--) {
        string s; cin >> s;
        if (reg_system[s] == 0) {
            cout << "OK" << endl;
        } else {
            cout << s << reg_system[s] << endl;
        }
        reg_system[s]++;
    }
    return 0;
}
\end{lstlisting}
\end{frame}

\begin{frame}{Kết luận}
    \begin{itemize}
        \item Map là công cụ không thể thiếu để xử lý các bài toán đếm và phân loại.
        \item Hãy chú ý giữa \texttt{std::map} và \texttt{std::unordered\_map} để tránh TLE hoặc bị Hack.
        \item Luyện tập thêm các bài rank 1000+ để thành thạo việc kết hợp Map với các cấu trúc khác.
    \end{itemize}
\end{frame}

\end{document}