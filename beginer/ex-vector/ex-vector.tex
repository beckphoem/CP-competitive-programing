\documentclass{beamer}
\usepackage[utf8]{inputenc}
\usepackage[T5]{fontenc} % Bắt buộc để hiển thị tiếng Việt
\usepackage[vietnamese]{babel}
\usepackage{tcolorbox}
\usepackage{listings}
\usepackage{xcolor}
\usepackage{booktabs}
\usepackage{hyperref}
\usetheme{Madrid}

\definecolor{codegreen}{rgb}{0,0.6,0}
\definecolor{codegray}{rgb}{0.5,0.5,0.5}
\definecolor{codepurple}{rgb}{0.58,0,0.82}
\definecolor{backcolour}{rgb}{0.95,0.95,0.92}

\lstdefinestyle{mystyle}{
    backgroundcolor=\color{backcolour},   
    commentstyle=\color{codegreen},
    keywordstyle=\color{magenta},
    numberstyle=\tiny\color{codegray},
    stringstyle=\color{codepurple},
    basicstyle=\ttfamily\scriptsize,
    breakatwhitespace=false,         
    breaklines=true,                 
    captionpos=b,                    
    keepspaces=true,                 
    numbers=left,                    
    numbersep=4pt,                  
    showspaces=false,                
    showstringspaces=false,
    showtabs=false,                  
    tabsize=2,
    escapechar=@
}

\lstset{style=mystyle}

\title[Codeforces Vector 800-1000]{Phân tích Chiến lược và Giải thuật}
\subtitle{Hệ thống Bài toán Vector trên Codeforces (Rating 800-1000)}
\author{Slide Learning C++}
\date{\today}

\begin{document}

\begin{frame}
  \titlepage
\end{frame}

\begin{frame}{Tổng quan}
  \begin{block}{Vai trò của Vector trong CP}
    \begin{itemize}
      \item Quản lý bộ nhớ linh hoạt (Dynamic Arrays).
      \item Tư duy về chỉ số (\texttt{indexing}) và truy cập ngẫu nhiên.
      \item Nền tảng cho các thuật toán tham lam (\texttt{greedy}) và mô phỏng.
    \end{itemize}
  \end{block}
  \begin{exampleblock}{Phân khúc Rating 800-1000}
    Tập trung vào khả năng chuyển đổi logic thành mã nguồn (\texttt{implementation}) và tư duy \texttt{ad-hoc}.
  \end{exampleblock}
\end{frame}

% --- NHÓM 1: THE CLASSICS ---

\begin{frame}{2.1. Next Round (158A)}
  \textbf{Link:} \href{https://codeforces.com/problemset/problem/158/A}{codeforces.com/problem/158/A}
  
  \begin{block}{Đề bài}
    Xác định số lượng thí sinh lọt vào vòng tiếp theo. Điều kiện: Điểm số phải dương ($>0$) và lớn hơn hoặc bằng điểm của người ở vị trí thứ $k$.
  \end{block}

  \begin{exampleblock}{Chiến thuật: Tư duy Vector}
    \begin{itemize}
      \item Chuyển đổi 1-based index sang 0-based: Ngưỡng điểm là \texttt{a[k-1]}.
      \item Duyệt qua \texttt{vector<int> a}: Kiểm tra \texttt{a[i] >= a[k-1] \&\& a[i] > 0}.
      \item \textbf{Lưu ý:} Dừng sớm nếu gặp phần tử vi phạm vì mảng đã sắp xếp giảm dần.
    \end{itemize}
  \end{exampleblock}
\end{frame}

\begin{frame}{2.2. Way Too Long Words (71A)}
  \textbf{Link:} \href{https://codeforces.com/problemset/problem/71/A}{codeforces.com/problem/71/A}
  
  \begin{block}{Đề bài}
    Viết tắt từ có độ dài $> 10$ ký tự: Giữ ký tự đầu, cuối và thay phần giữa bằng số lượng ký tự bị lược bỏ.
  \end{block}

  \begin{exampleblock}{Chiến thuật: String as Vector}
    \begin{itemize}
      \item Kiểm tra \texttt{s.length() > 10}.
      \item Sử dụng \texttt{s.front()}, \texttt{s.back()} và \texttt{s.length() - 2}.
      \item \pause \textbf{Kết quả:} \texttt{s[0] + to\_string(s.length()-2) + s.back()}.
    \end{itemize}
  \end{exampleblock}
\end{frame}

\begin{frame}{2.3. Team (231A)}
  \textbf{Link:} \href{https://codeforces.com/problemset/problem/231/A}{codeforces.com/problem/231/A}
  
  \begin{block}{Đề bài}
    Đếm số bài toán mà ít nhất 2 trong 3 người chắc chắn về lời giải.
  \end{block}

  \begin{exampleblock}{Chiến thuật: Row Sum}
    \begin{itemize}
      \item Coi mỗi bài toán là một hàng trong ma trận hoặc một tập hợp 3 số.
      \item Tính \texttt{sum = p + v + t}. Nếu \texttt{sum >= 2}, tăng biến đếm.
      \item Có thể xử lý trực tiếp (\texttt{online}) mà không cần lưu vector 2 chiều.
    \end{itemize}
  \end{exampleblock}
\end{frame}

\begin{frame}{2.4. Bit++ (282A)}
  \textbf{Link:} \href{https://codeforces.com/problemset/problem/282/A}{codeforces.com/problem/282/A}
  
  \begin{block}{Đề bài}
    Thực hiện các lệnh \texttt{++} và \texttt{--} trên biến $X$ (khởi tạo bằng 0).
  \end{block}

  \begin{exampleblock}{Chiến thuật: Pattern Matching}
    \begin{itemize}
        \item Thay vì so sánh cả chuỗi, chỉ cần kiểm tra ký tự ở giữa \texttt{s[1]}.
        \item Nếu \texttt{s[1] == '+'} thì \texttt{X++}, ngược lại \texttt{X--}.
    \end{itemize}
  \end{exampleblock}
\end{frame}

\begin{frame}{2.5. Domino Piling (50A)}
  \textbf{Link:} \href{https://codeforces.com/problemset/problem/50/A}{codeforces.com/problem/50/A}
  
  \begin{block}{Đề bài}
    Tìm số quân domino $2 \times 1$ tối đa có thể đặt vào bảng $M \times N$.
  \end{block}

  \begin{alertblock}{Tư duy toán học}
    \begin{itemize}
      \item Không cần dùng mảng/vector để mô phỏng lưới.
      \item Mỗi quân chiếm 2 đơn vị diện tích.
      \item Công thức: $\lfloor \frac{M \times N}{2} \rfloor$. Trong C++: \texttt{(m * n) / 2}.
    \end{itemize}
  \end{alertblock}
\end{frame}

% --- NHÓM 2: HÌNH HỌC & SẮP XẾP ---

\begin{frame}{3.1. Beautiful Matrix (263A)}
  \textbf{Link:} \href{https://codeforces.com/problemset/problem/263/A}{codeforces.com/problem/263/A}
  
  \begin{block}{Đề bài}
    Tìm số bước tối thiểu đưa số 1 về tâm $(2, 2)$ của ma trận $5 \times 5$.
  \end{block}

  \begin{exampleblock}{Chiến thuật: Khoảng cách Manhattan}
    \begin{itemize}
      \item Tìm tọa độ $(r, c)$ của số 1.
      \item Kết quả: $|r - 2| + |c - 2|$ (với 0-based index).
      \item Sử dụng hàm \texttt{abs()} trong thư viện \texttt{<cmath>}.
    \end{itemize}
  \end{exampleblock}
\end{frame}

\begin{frame}{3.3. Helpful Maths (339A)}
    \textbf{Link:} \href{https://codeforces.com/problemset/problem/339/A}{codeforces.com/problem/339/A}
    
    \begin{block}{Đề bài}
      Sắp xếp lại chuỗi phép cộng các số 1, 2, 3 theo thứ tự tăng dần.
    \end{block}
  
    \begin{exampleblock}{Chiến thuật: Parsing và Sorting}
      \begin{itemize}
        \item Trích xuất các số vào \texttt{vector<int> numbers}.
        \item Dùng \texttt{std::sort(numbers.begin(), numbers.end())}.
        \item In lại kèm dấu \texttt{+}. Lưu ý không in dấu dư ở cuối.
      \end{itemize}
    \end{exampleblock}
  \end{frame}

% --- NHÓM 3: MÔ PHỎNG ---

\begin{frame}{4.2. Nearly Lucky Number (110A)}
    \textbf{Link:} \href{https://codeforces.com/problemset/problem/110/A}{codeforces.com/problem/110/A}
    
    \begin{block}{Đề bài}
      Đếm số chữ số may mắn (4, 7). Kiểm tra xem \textbf{số lượng} đó có phải là số may mắn không.
    \end{block}
  
    \begin{alertblock}{Cạm bẫy}
      \begin{itemize}
        \item Dữ liệu đầu vào cực lớn: Phải đọc bằng \texttt{string}.
        \item Đếm xong mới kiểm tra \texttt{count == 4 || count == 7}.
      \end{itemize}
    \end{alertblock}
  \end{frame}

% --- NHÓM 4: NÂNG CAO ---

\begin{frame}{5.1. Numbers Box (1447B) - Rating 1000}
    \textbf{Link:} \href{https://codeforces.com/problemset/problem/1447/B}{codeforces.com/problem/1447/B}
    
    \begin{block}{Đề bài}
      Đổi dấu 2 ô kề nhau tùy ý. Tìm tổng lớn nhất có thể của ma trận.
    \end{block}
  
    \begin{exampleblock}{Tư duy kiến thiết (Constructive)}
      \begin{itemize}
        \item Dấu trừ có thể "di chuyển" tự do.
        \item Nếu số lượng số âm là \textbf{Chẵn}: Tổng = Tổng trị tuyệt đối.
        \item Nếu số lượng số âm là \textbf{Lẻ}: Phải để lại 1 số mang dấu âm. Chọn số có trị tuyệt đối nhỏ nhất (\texttt{min\_abs}).
        \item \pause \textbf{Công thức:} \texttt{total\_abs\_sum - 2 * min\_abs}.
      \end{itemize}
    \end{exampleblock}
  \end{frame}

\begin{frame}{5.8. Twins (160A) - Rating 900}
    \textbf{Link:} \href{https://codeforces.com/problemset/problem/160/A}{codeforces.com/problem/160/A}
    
    \begin{block}{Đề bài}
      Lấy ít đồng xu nhất sao cho tổng tiền của bạn lớn hơn tổng tiền còn lại.
    \end{block}
  
    \begin{exampleblock}{Chiến thuật: Greedy}
      \begin{itemize}
        \item Sắp xếp vector giảm dần: \texttt{sort(v.rbegin(), v.rend())}.
        \item Lấy dần các đồng xu lớn nhất.
        \item Dừng lại khi \texttt{my\_sum > total\_sum / 2}.
      \end{itemize}
    \end{exampleblock}
  \end{frame}

% --- TỔNG KẾT & BEST PRACTICES ---

\begin{frame}[fragile]{6.3. Cạm bẫy và Tối ưu hóa}
    \begin{alertblock}{Lưu ý kỹ thuật}
      \begin{itemize}
        \item \textbf{Integer Overflow:} Dùng \texttt{long long} cho các bài tính tổng hoặc nhân (ví dụ: Soldier and Bananas).
        \item \textbf{Indexing:} Luôn cẩn thận với lỗi lệch 1 (Off-by-one).
      \end{itemize}
    \end{alertblock}
  
    \begin{exampleblock}{Fast I/O trong C++}
  \begin{lstlisting}[language=C++]
  int main() {
      ios_base::sync_with_stdio(false);
      cin.tie(NULL);
      // Code logic here
      return 0;
  }
  \end{lstlisting}
    \end{exampleblock}
  \end{frame}

\begin{frame}{Lời kết}
  \begin{center}
    \Large \textbf{Cảm ơn bạn đã theo dõi!}
    
    \vspace{0.8cm}
    \small Hãy luyện tập thường xuyên để biến tư duy Vector thành phản xạ.
    
    \vspace{0.5cm}
    \textit{Slide created for CP Learning Path.}
  \end{center}
\end{frame}

\end{document}